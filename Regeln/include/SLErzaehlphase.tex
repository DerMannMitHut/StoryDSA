\chapter{SL-Erz"ahlphase}\label{Ch:SLErzaehlphase}\index{SL-Erzählphase}
\lettrine{E}{ine}
SL-Erz"ahlphase hat diverse Funktionen: Sie soll das Abenteuer einleiten und beenden, "Uberg"ange zwischen freiem Spiel und Konflikten schaffen und die Story voranbringen, indem der Spielleiter die M"oglichkeit hat, unwichtige Dinge, langweilige Reisen und andere uninteressante Begebenheiten einfach zu "uberspringen. Dabei soll der SL nat"urlich nicht zum Alleinunterhalter verkommen, sondern vielmehr in kurzen Worten die richtige Stimmung f"ur die kommende Szene schaffen und die Spieler m"oglichst schnell wieder ans Ruder zu lassen.

Eine gute SL-Erz"ahlphase greift zun"achst kurz den Punkt auf, an dem die letzte Szene geendet hat. Die wird "ublicherweise durch die Einleitung ``Nachdem du/ihr\dots'' begonnen. Zur Verdeutlichung hebt der SL eine Hand: Eine SL-Erz"ahlphase hat begonnen.

Dann folgt eine kurze "Uberleitung zur n"achsten Szene, in der der Spielleiter darauf eingeht, wie sich die Spielercharaktere verhalten und was vielleicht einige der weiteren Charaktere unternehmen. Eine Reise "uber mehrere Wochen ist hier genauso m"oglich wie ein kurzes Aufschauen oder ein Luft Anhalten seitens der Charaktere.

Eventuell kann er auch eine Blende zu anderen Charakteren an einer anderen Stelle machen, so z.\,B. auf die Tochter des Barons, die gerade in Ketten in irgendeinem Kerker schmort oder zum Hexenzirkel, der sich gerade im Schutze der Nacht trifft. Solche Elemente k"onnen, wenn sie geschickt eingesetzt werden, die Spannung der Geschichte erh"ohen. Andererseits versorgt man die Spieler mit Wissen, dass die Charaktere nicht haben. Das st"ort manche Spieler und sollte unbedingt mit der Gruppe abgekl"art werden.

Zuletzt kommt der Rahmen der neuen Szene: Wo findet die n"achste Szene statt? Wer ist alles daran beteiligt? Was ist das Ziel der Charaktere oder was sind die offen stehenden Optionen? Hierbei sollte der Spielleiter den Ort, soweit er den Spielern noch nicht bekannt ist, kurz umrei"sen.


