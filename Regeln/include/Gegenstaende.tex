\chapter{Gegenstände}\label{Ch:Gegenstaende}

\BN
\lettrine{I}{n} \StoryDSA spielen Gegenstände wie auch Geld nur dann eine regeltechnische Rolle, wenn sie benutzt werden, die Geschichte zum Vorteil des Helden zu beeinflussen. So könnte das alte Familienschwert besonders mächtig sein oder der Charakter hat so viel Geld, dass sein Reichtum entscheidenden Einfluss auf wichtige Verhandlungen hat.

Letztendlich ist ein besonderer Gegenstand also eine Art Vorteil, der nicht angeboren ist, sondern durch den Besitz des Gegenstandes ``eingeschaltet'' wird. Daher müssen solche Gegenstände auch mit Steigerung gekauft werden.

Bei den normalen Gegenständen geht man im Spiel davon aus, dass der Charakter die benötigten Werkzeuge für seine Talente mitbringt. So wird ein Heiler sicherlich Tücher, Bandagen und blutungshemmende Kräuter im Gepäck haben. Ein Krieger wird eine angemessene Waffe haben. Ein Magier hat einen Magierstab.

Das Ganze soll niemanden daran hindern, den normalen Warenhandel auszuspielen: So kann man auch ohne dass Geld explizit berücksichtigt wird, mit dem Wirt um ein Zimmer feilschen. Oder ein Auftraggeber kann die Helden mit einem gewissen Geldbetrag zu Verhandlungen nach Al'Anfa schicken.

In bestimmten Situationen kann es natürlich sein, dass die Helden ihre Ausrüstung explizit nicht zur Verfügung haben. Beispielsweise können sie gefangen genommen sein

\section{Kosten und Fähigkeiten}
Was können Gegenstände? In der Geschichte können Gegenstände eine Menge Fähigkeiten haben, denn auch magische Artefakte werden nach denselben Regeln abgehandelt  Spieltechnisch gesehen geben Gegenstände grundsätzlich Bonuswürfel in bestimmten Situationen oder für bestimmte Talente. Spezielle Gegenstände können aber auch andere regeltechnische Effekte haben, diese sind dann unten einzeln aufgeführt.

Man muss vier Arten von Gegenständen unterscheiden. Gegenstände werden wie Fähigkeiten mit Steigerungen bezahlt.

\subsection{Gegenstände 1. Art}
Hierunter fallen alle Arten von Gegenständen, die die typische Anwendung einzelner Talente unterstützen oder die für die Anwendung von Basistalenten benötigt werden. So kann ein Seil für das Talent Athletik (Klettern) benutzt werden und Geld dient als Unterstützung für das Talent Überreden/Überzeugen. Ein Dolch ist nötig, um Dolchkampf einsetzten zu können.

Durch die Stufe des Helden wird die Mächtigkeit der Gegenstände begrenzt. So kann ein Held erst ab Stufe 8 Gegenstände mit einem Bonus von zwei Würfeln benutzen und ab Stufe 16 Gegenstände mit drei Bonuswürfeln. Das bedeutet jetzt nicht, dass ein Held das göttliche Kletterseil erst in Stufe 16 bekommen darf. Jedoch wird es einem Erststüfler eben maximal einen Bonus von 1 geben.

\begin{tabular}[C]{lll}
  1 Bonuswürfel & 2 Steigerungen & (ab Stufe 1) \\
  2 Bonuswürfel & 6 Steigerungen & (ab Stufe 8) \\
  3 Bonuswürfel & 12 Steigerungen & (ab Stufe 16) \\
\end{tabular}

\subsection{Gegenstände 2. Art}
Einige Gegenstände ermöglichen erst das Anwenden eines Spezialtalentes. So benötigt man ein Schwert oder eine vergleichbare Waffe, wenn man das Talent Einhänder benutzen möchte. Man braucht ein Pferd, wenn man reiten will.

Ein Spieler kann die Gegenstände zweiter Art mit einem Bonuswürfel mehr erwerben, als ihm eigentlich zusteht. Ein Schwert kann also zwei Bonuswürfel geben, auch wenn der Held erst in der ersten Stufe ist. Die Kosten sind dafür ganz normal, wie für zwei Bonuswürfel üblich. Daher können Gegenstände dieser Art auch bis zu vier Bonuswürfel geben.

\begin{tabular}[C]{lll}
  1 Bonuswürfel & 2 Steigerungen & (ab Stufe 1) \\
  2 Bonuswürfel & 6 Steigerungen & (ab Stufe 1) \\
  3 Bonuswürfel & 12 Steigerungen & (ab Stufe 8) \\
  4 Bonuswürfel & 20 Steigerungen & (ab Stufe 16) \\
\end{tabular}

\subsection{Gegenstände 3. Art}
In diese Kategorie fallen alle Gegenstände, die den Helden komplett neue Handlungen ermöglichen. Hierbei handelt es sich im Normalfall um besondere oder magische Gegenstände, wie z.B. ein fliegender Besen oder eine Brille, mit der man durch Wände sehen kann.

Gegenstände der dritten Art geben einen Bonuswürfel weniger als Gegenstände erster Art.

\begin{tabular}[C]{lll}
  0 Bonuswürfel & 2 Steigerungen & (ab Stufe 1) \\
  1 Bonuswürfel & 6 Steigerungen & (ab Stufe 1) \\
  2 Bonuswürfel & 12 Steigerungen & (ab Stufe 8) \\
  3 Bonuswürfel & 20 Steigerungen & (ab Stufe 16) \\
\end{tabular}


\subsection{Gegenstände 4. Art}
Rüstungen aller Art. Alles, was einen Helden vor geistigen oder körperlichen Verletzungen schützen kann, fällt in diese Kategorie. Sie geben bei Schadenswürfen einen Bonus auf die Schadensgrenze (die normalerweise 9 für Nebenkonflikte und 5 für Hauptkonflikte beträgt).

\begin{tabular}{lll}
 +1 Bonus & 1 Steigerung  & (ab Stufe 1) \\
 +2 Bonus & 3 Steigerungen & (ab Stufe 1)  \\
 +3 Bonus & 6 Steigerungen & (ab Stufe 1)  \\
 +4 Bonus & 10 Steigerungen & (ab Stufe 5)  \\
 +5 Bonus & 15 Steigerungen & (ab Stufe 12)  \\
 +6 Bonus & 21 Steigerungen & (ab Stufe 18)  \\
\end{tabular}

\begin{design}
\subsubsection{Designanmerkungen: Kosten von Gegenständen}
In StoryDSA werden nur die regeltechnischen Auswirkungen der Gegenstände als Grundlage genommen und dann auch nicht mit Geld, sondern mit Steigerungen bezahlt. Dies dient dazu, um nicht das Problem der Preisgestaltung zwischen `realistischen' Kosten und Nutzen in der Spielwelt zu haben. Darüberhinaus ist ein Konfliktgegenstand regeltechnisch dasselbe wie ein Wissenstalent, so dass beide auch dasselbe kosten sollten.

Die Kosten innerhalb der Spielwelt, eventuelle Verhandlungen und Feilschen um Preise usw. können natürlich immer noch ausgespielt werden, haben jedoch keine weiteren Auswirkungen. So ist es auch problemlos möglich, die Charaktere große Schätze finden zu lassen, ohne dass das Spiel aus dem Ruder läuft.
\end{design}

\subsection{Gegenstände im Spiel}
Da die Gegenstände durch Steigerungen wie Fähigkeiten ``freigekauft'' werden müssen, könnte letztendlich dem Spieler komplett überlassen werden, welcher Gegenstand wie viele Bonuswürfel gibt. Das trägt aber nicht unbedingt zur Stimmung bei; je mehr Bonuswürfel ein Gegenstand gibt, umso seltener und wertvoller ist er auch. Daher sollten gerade gute Gegenstände auch wirklich als Belohnung durch den Spielleiter verliehen werden -- auch auf Aventurien gibt es nicht an allen Ecken magische Schwerter im Dutzend.

Als Richtlinie sollte gelten:

\begin{tabularx}{\textwidth}{lX}
1--3 Steigerungen & Handwerklich besonders gute Gegenstände; eventuell eine magische Verstärkung oder kleinen magischen Effekten \\
6 Steigerungen & Handwerklich hervorragende Gegenstände, zumeist mit einer magischen Verbesserung \\
10--15 Steigerungen & magisches Artefakt mit besonderen Effekten \\
20--21 Steigerungen & herausragender magischer Gegenstand, oft einzigartig \\
\end{tabularx}

Gegenstände mit bis zu drei Steigerungen können die Charaktere als Sonderanfertigung in größeren Städten erwerben. Alles, was darüber hinausgeht, sollte erspielt werden, d.h. die Helden bekommen solche Gegenstände als Belohnung für einen speziellen Auftrag oder als Fundstück in Abenteuern. Es ist auch möglich, dass Spieler einen Gegenstand mit sechs Steigerungen bei der Charaktererschaffung wählen (z.B. als Erbstück).

Insgesamt sollte der Spielleiter die Spieler hier einerseits nicht zu sehr einschränken: Kommen die Helden in Stufe 8, so wollen die Spieler auch gerne entsprechende Gegenstände bekommen. Andererseits sollten aber auch die Spieler nicht zu sehr auf ihr ``Recht'' pochen! Der Spielspaß steht im Vordergrund, und eine Queste für ein besonderes magisches Schwert ist sicherlich einer banalen Aussage ``Mein Charakter lässt sich sein Schwert mit einer Rune magisch verstärken'' vorzuziehen.

\section{Spezielle Gegenstände}
Nachdem die allgemeinen Richtlinien für Gegenstände geklärt sind, werden in diesem Abschnitt auf besondere Arten von Gegenständen genauer eingegangen. Die hier gemachten Angaben sollen vor allem als Richtlinie dienen, wie Gegenstände in StoryDSA umgesetzt werden.

\subsection{Waffen}
Waffen sind immer an bestimmte Kampftalente gekoppelt. DSA4 kennt ja einen ganzen Sack voll Kampftalenten, und viele Waffen lassen sich mit mehreren Talenten ``bedienen''. Die Anzahl der Kampftalente ist bei \StoryDSA deutlich eingeschränkt; die Zuordnung ist immer eindeutig.

\subsubsection{Nahkampf-Waffen}
\begin{description}
\item[Dolche] Basistalent. Alle Arten von Messern und Dolchen fallen in diese Kategorie. Diese Waffen sind immer Gegenstände 1. Art.
\item[Einhänder] Spezialtalent. Alles, was bei DSA4 Hiebwaffen, Schwerter, Säbel, usw. sind. Diese Waffen sind immer Gegenstände 2. Art und zeichnen sich durch höheren Schaden gegenüber Messern und Dolchen aus.
\item[Fechtwaffen] Spezialtalent, entspricht den DSA4-Fechtwaffen. Elegante, schnelle Waffen, die zwar nicht so durchschlagend sind, wie Einhänder, dafür aber auch nicht so grobschlächtig. Gegenstände 2. Art.
\item[Lanzenreiten] Spezialtalent, entspricht dem DSA4-Lanzenreiten. Die zugehörigen Waffen sind Gegenstände 1. Art -- allerdings verursacht Lanzenreiten einen automatischen Offensiverfolg (bzw. zwei Bonuswürfel in Kurzkonflikten). Gewürfelt wird auf Lanzenreiten oder Reiten, je nachdem, welcher Wert niedriger ist.
\item[Raufen] Basistalent. Schlagringe oder Würgebänder können beim Raufen eingesetzt werden. Dabei handelt es sich um ganz normale Gegenstände 1. Art.
\item[Stäbe/Speere] Spezialtalent. Speere sind Gegenstände 2. Art.
\item[Zweihänder] Spezialtalent. Alle zweihändig bedienbaren Waffen, wie z.B. Zweihandschwerter oder Infanteriewaffen fallen hier hinein. Diese Waffen sind Gegenstände 2. Art. Zweihänder bekommen einen zusätzlichen Bonuswürfel (der nicht gegen die Beschränkung durch die Stufe zählt), allerdings ist der Talentgesamtwert um 3 niedriger und es dürfen höchstens zwei Würfel defensiv eingesetzt werden.
\end{description}

Spezielle Waffen:
\begin{description}
\item[Anderthalbhänder] Diese Waffen können als Einhänder oder Zweihänder benutzt werden.
\item[Kampfstäbe und Zauberstäbe] Diese Waffen zählen nur als Gegenstände 1. Art.
\end{description}

\subsubsection{Fernkampf-Waffen}
\begin{description}
\item[Armbrust] Spezialtalent; Armbrüste sind Gegenstände 2. Art.
\item[Belagerungswaffen] Spezialtalent. Die zugehörigen Waffen sind Gegenstände 1. Art und verursachen einen automatischen Offensiverfolg. Belagerungswaffen sind nur im Angriff auf Häuser, Burgen, Städte, Schiffe usw. einsetzbar.
\item[Blasrohr] Spezialtalent. --- Der Zusammenhang zum verwendeten Gift muss noch geklärt werden! ---
\item[Bogen] Spezialtalent; Bögen sind Gegenstände 2. Art.
\item[Einfache Wurfgeschosse] Basistalent, zum Werfen von Steinen, Wurfmessern, Wurfsternen usw. Gegenstände 1. Art.
\item[Schleuder] Spezialtalent; Schleudern sind Gegenstände 2. Art.
\item[Wurfwaffen] Spezialtalent, zum Werfen von Wurfäxten, Wurfspeeren usw.; dies sind Gegenstände 2. Art.
\end{description}

\FEHLT{Zusammenhang von Schusswaffe zum Geschoss ist noch unklar}

\begin{optional}
\subsection{Optional: Waffen vereinfachen}
Die verschiedenen Waffen sind vor allem in Anlehnung an die DSA-Regeln so gestaltet und stellen darüberhinaus eine weiterer Beitrag zum Nischenschutz.

Möchte sich die Spielgruppe mit diesen Regeln noch weiter von DSA entfernen, so sollte es für den Kampf nur noch zwei Basistalente Nahkampf und Fernkampf geben. Alle Waffen zählen dann als einfache Konfliktgegenstände 1. Art.
\end{optional}

\subsection{Rüstungen}
Klassisch sind natürlich Rüstungen wie Lederrüstung, Kettenhemd oder Plattenrüstung. Im Vergleich zu DSA4 sollte eine \StoryDSA-Rüstung etwa halb so viel Rüstungsschutz bieten. Darüber hinaus gehende Schutzpunkte sollten durch hervorragende handwerkliche Qualität oder Magie erklärt werden.

Aber auch geistiger Schutz ist bei \StoryDSA im Prinzip möglich. Es gibt allerdings auf Aventurien nicht gerade eine üppige Auswahl an geistigen Rüstungen -- außer ein paar Zaubern

\subsection{alltägliche Gegenstände}

\subsection{magische Artefakte}

\EN
\lettrine[lraise=0.35]{G}{egenstände}, wie auch Geld, sind in \StoryDSA regeltechnisch völlig nebensächlich. Das soll niemanden daran hindern einen geldgierigen Charakter zu spielen oder einen adligen Spross, der in Geld nur so schwimmt. Geld in der Spielwelt ist natürlich nicht unwichtig; allein von den Regeln her macht es keinen Unterschied, wie viel Geld ein Charakter besitzt.

Um besondere Gegenstände zu erwerben, muss ein Spieler wie bei der Steigerung bzw. der Charaktererschaffung Steigerungen ausgeben. Der Erwerb solcher Gegenstände sollte natürlich plausibel in der Spielwelt begründet werden, wie z.B. ein Schatz, ein Kauf, eine Erbschaft o.ä.

Die Anzahl der Steigerungen, die ausgegeben werden müssen, hängt nur von den regeltechnischen Auswirkungen der Gegenstände ab. Die meisten Dinge fallen in den Bereich Konfliktgegenstände und geben, wenn sie in einem Konflikt benutzt werden können, bis zu drei Bonuswürfeln. Die zweite Art von Gegenständen werden als Rüstungen bezeichnet und schützen vor Schaden, d.\,h. sie geben einen Bonus auf den Schadenswurf nach einem Konflikt.

\section{Kosten}
Die Kosten für Gegenstände orientieren sich alleine an ihren spieltechnischen Auswirkungen. Wie teuer Gegenstände in Aventurien sind, kann beispielsweise im Beiheft zum Spielleiterschirm gefunden werden. Es gibt aber auch zahlreiche Quellen im Internet für Einkaufslisten.

\BN
Eine kleine Anmerkung: Normalerweise können Helden keine vier-Bonuswürfel-Gegenstände haben, bei dreien ist Schluss. Bei Waffen und magischen Gegenständen gibt es aber ausnahmen.
\EN

Bei einer Verbesserung muss nur die Differenz gezahlt werden; verliert man einen Gegenstand oder verkauft ihn, werden die ausgegebenen Steigerungen wieder frei.

\begin{beispiel}
\paragraph{Beispiel:} 1000 Dukaten können in diesem Sinne ein Gegenstand sein, der bei Bestechungen einen Bonuswürfel gibt und deswegen zwei Steigerungen kostet. Ist das Gold verbraucht, so werden die Steigerungen wieder frei.
\end{beispiel}

In einer Konfliktrunde kann ein Spieler maximal einen Gegenstand im Konflikt einsetzen. Um nicht direkt Anfängerhelden mit den tollsten Gegenständen zu haben, werden die maximalen Würfel, die ein Konfliktgegenstand geben kann, in Abhängigkeit von der Stufe gedeckelt. So sind erst ab Stufe 8 Gegenstände mit 2 Bonuswürfeln und ab Stufe 16 Gegenstände mit 3 Bonuswürfeln erlaubt.

Auch Rüstungsgegenstände haben eine begrenzte Macht. Anfangs können sie maximal 3~Punkte Rüstungsschutz geben, ab Stufe 5 sind 4~Punkte erlaubt, ab Stufe 12 dann 5 und schließlich, ab Stufe 18, auch 6~Rüstungsschutz-Punkte.

Mit der Verbesserung von Gegenständen kann im Spiel auf verschiedene Weisen umgegangen werden:
\begin{itemize}
\item Die Spieler ersetzen schlechtere Gegenstände durch bessere
\item Ein Charakter entdeckt neue Möglichkeiten für die Nutzung eines mächtigen Gegenstandes
\item In den Beschreibungen bleibt ein Gegenstand gleich -- nur die mechanische Auswirkung mit mehr Bonuspunkten bzw. einem höheren Rüstungsschutz ist anders
\end{itemize}
Von diesen drei Möglichkeiten ist keine besser oder schlechter; wahrscheinlich werden alle drei in einem Spiel angewendet werden. So könnte ein Charakter eine neue Fähigkeit seines Schwertes in einem Abenteuer entdecken, ein anderer kauft sich ein neues, besseres Schwert und der dritte behält einfach das Erbstück seines Vaters, mit dem dieser in der dritten Dämonenschlacht gekämpft hat. Dass der Spieler dann irgendwann vier Steigerungen ausgibt und das Schwert zwei statt einen Bonuswürfel gibt, merkt der Charakter ja nicht.

\section{Waffen}
Waffen nehmen eine etwas spezielle Rolle ein, da sie an bestimmte Kampftalente gekoppelt sind. Einfache Messer und Dolche werden, genau wie Stäbe, als ganz normale Konfliktgegenstände behandelt.

Ist eine Waffe jedoch an eines der Spezialtalente Einhänder, Fechtwaffen, Stäbe/Speere oder Zweihänder gekoppelt, kann sie einen Bonuswürfel mehr haben als normalerweise erlaubt ist. Das bedeutet, dass ein Anfängercharakter bereits ein 2-Bonuswürfel-Schwert haben kann und dass ab Stufe 16 bis zu 4~Bonuswürfel möglich sind.

Lanzen, von einem Pferd aus eingesetzt, und Kampfstäbe bilden eine Ausnahme. Während Kampfstäbe wie ganz normale Konfliktgegenstände behandelt werden, geben Lanzen kostenlos einen automatischen Offensiverfolg (bzw. im Kurzkonflikt zwei Bonuswürfel); für den erfolgreichen Einsatz einer Lanzen muss der Charakter aber auch die beiden Spezialtalente Lanzenreiten und Reiten beherrschen (auf den niedrigeren der beiden Werte wird beim Lanzenreiten gewürfelt). Dieser automatische Erfolg braucht nicht mit Steigerungen bezahlt zu werden und sie zählt auch nicht als Bonuswürfel o.ä. Dafür kann eine Lanze nicht mehr als 3~Bonuswürfel haben.

\begin{beispiel}
\paragraph{Beispiel:}
Ein Charakter hat ein rondrageweihtes Schwert als Belohnung bekommen. Der Spieler beschließt, dass es einen Bonuswürfel bekommen soll und gibt 2 Steigerungen dafür aus. Da es ein Schwert ist, könnte er auch 6 Steigerungen für einen zweiten Bonuswürfel ausgeben, obwohl erst in Stufe 6 ist (und somit eigentlich nur Gegenstände mit einem Bonuswürfel benutzen dürfte).
\end{beispiel}

Hiervon wiederum eine Ausnahme ist der Kampfstab, der keinen automatischen Bonuswürfel gibt und trotzdem an das Spezialtalent Stäbe/Speere gebunden ist.

\section{Rüstungen}
 Rüstungen verhindern Schaden. Normale Rüstungen -- aus Stoffe, Leder, Eisen, Holz usw. -- verhindern körperlichen Schaden. Durch Magie oder profane Vorteile kann auch geistiger Schaden verhindert werden. Ob in einer Situation eine Rüstung Schaden verhindert, muss von Fall zu Fall unterschieden werden.

Wenn eine Rüstung Schaden verhindern kann, gibt sie Bonus auf die Schadenswürfe. Bei Nebenkonflikten gibt es ja normalerweise einen Schaden, wenn der W20 mindestens eine 9 zeigt. Hat ein Charakter eine Rüstung mit einem Rüstungsschutz von 3, so gibt es erst ab einer 12 einen Schaden. Bei Hauptkonflikten verhält es sich genauso. Allerdings gibt es hier schon ein Schaden ab einer 5 + Rüstungsschutz.


\begin{optional}
\section{Optional: Geld als Ressource}

Eine Spielgruppe kann auch, wie im klassischen Spiel, Geld als Ressource beibehalten und genau mitprotokollieren, welcher Charakter wie viel Geld hat, Lebenshaltungskosten ausrechnen und jeden Kreuzer Trinkgeld in der Taverne nachhalten.

Das ist eigentlich problemlos; das Geld sollte jedoch \emph{zusätzlich} zu den Steigerungen abgezogen werden. Denn letztendlich sorgen die Steigerungen für ein ausgewogenes Spiel, in dem kein Charakter wesentliche Vorteile vor den anderen hat.
\end{optional}


