\chapter{Begriffserklärungen}\label{Ch:Begriffserklaerungen}
\lettrine{A}{ls} Ergänzung zu den Kurzregeln ist hier eine Art Glossar, welches die Regeln wiederum von einer anderen Seite beleuchtet. Sind die Kurzregeln nach Themen zusammengefasst, so werden hier die Begriffe alphabetisch geordnet erklärt. Um hier mehr als nur stichpunktartige Erklärungen zu bieten, sind viele eng verbundene Begriffe zu längeren Artikeln zusammengefasst. So gibt es beispielsweise keinen separaten Eintrag `Talentgesamtwert'; dieser ist unter `Talent' zu finden.

\begin{description}

\item[Abenteuerpunkte:]\index{Abenteuerpunkt} Für das Überstehen von Konflikten bekommt ein Charakter Abenteuerpunkt (Hauptkonflikt:~50, Nebenkonflikt:~20, Kurzkonflikt:~10). Erhaltene Abenteuerpunkte spiegeln den Ruf des Charakters wieder. Von diesen ist auch abhängig, in welcher \DEF{Stufe}\index{Stufe} sich ein Charakter befindet. Von der Stufe wiederum sind Maximalwerte für Talentwerte, Rüstung, Gegenstände usw. abhängig.

Außerdem bekommt ein Charakter für 100~Abenteuerpunkt einen \DEF{Steigerung}\index{Steigerung}. Diese kann er benutzen, um seine vorhandenen Fähigkeiten zu verbessern oder neue zu lernen (im Rahmen seiner Stufe).
 
\item[Eigenschaft:] Die Eigenschaften sind die grundlegenden geistigen und körperlichen Attribute eines Charakters. Sie legen Grundsteine für die Talente. Der Wert bewegt sich zwischen --1 und 4, beträgt zu Spielbeginn jedoch maximal 3. Die 8 Eigenschaften sind:
			\textbf{MU}t\index{Mut}\index{MU|see{Mut}}, 
			\textbf{KL}ugheit\index{Klugheit}\index{KL|see{Klugheit}}, 
			\textbf{IN}tuition\index{Intuition}\index{IN|see{Intuition}}, 
			\textbf{CH}arisma\index{Charisma}\index{CH|see{Charisma}}, 
			\textbf{GE}wandheit\index{Gewandheit}\index{GE|see{Gewandheit}}, 
			\textbf{F}inger\textbf{F}ertigkeit\index{Fingerfertigkeit}\index{FF|see{Fingerfertigkeit}}, 
			\textbf{KO}nstitution\index{Konstitution}\index{KO|see{Konstitution}}, 
			\textbf{K}"orper\textbf{K}raft\index{Körperkraft}\index{KK|see{Körperkraft}}

\item[Energien:]\index{Energie} Energien dienen in StoryDSA dazu, häufiges Wiederholen von bestimmten, gleichartigen Dingen zu beschränken. Gleichzeitig machen sie auf diese Weise die Auswirkungen von Anstrengungen und Verletzungen plastischer, da die Charaktere gewisse Zeit zur Regeneration brauchen.

Statt der DSA4-Lebensenergie werden zwei Energien getrennt verwaltet: \DEF{Lebenskraft}\index{Lebenskraft} und \DEF{Willenskraft}\index{Willenskraft}. Diese geben an, wie viele Konflikte ohne Regeneration die Charaktere überstehen. Grundsätzlich beträgt die Willenskraft MU+KL+IN+CH+6, die Lebenskraft GE+FF+KO+KK+8. Die natürliche Regenerationsgeschwindigkeit ist 1~Punkt Willenskraft pro Stunde bzw. 1~Punkt Lebenskraft pro Tag.

Eine zweite Form von Energie bildet die \DEF{Astralenergie}\index{Astralenergie}. Magiekundige können bei der Charaktererschaffung bzw. beim erstmaligen Erwerb eines Magietalentes Lebens- und Willenskraft im Verhältnis 1:2 in Astralenergie umwandeln. Steigt im Laufe des Spieles eine Eigenschaft, so können sich Magiekundige aussuchen, ob sie dafür lieber Kräfte oder Astralenergie bekommen. Die Astralenergie beschränkt die Anzahl und Stärke der Magieanwendung. Die Regenerationsgeschwindigkeit beträgt 1~Punkt pro Tag.

\item[Erzählung:]\index{Erzählung} Während der Konflikte muss der \DEF{Erzählwert}\index{Erzählwert} einer jeden Erzählung bestimmt werden. Dazu zählt man einfach die Fakten, die vom Spieler eingebracht werden. Der maximale Erzählwert beträgt 5.

Darüberhinaus gibt es \DEF{SL-Erzählphasen}\index{SL-Erzählphase} in denen der Spielleiter alleine den Übergänge zur nächsten Szene erzählt.

Siehe auch Konflikt, Würfel.

\item[Gegenstand:]\index{Gegenstand} Im Spiel wird zwischen einfachen Gegenständen, \DEF{Rüstungen}\index{Rüstung} und \DEF{Konflikt-Gegenständen}\index{Konflikt-Gegenstand} unterschieden. Mit letzteren sind solche Gegenstände gemeint, die ähnlich wie Wissenstalente bei Einsatz im Spiel Bonuswürfel ergeben. Waffen, die nur mit Spezialtalenten genutzt werden können, sind üblicherweise Konfliktgegenstände; aber auch Dietriche, ein heiliges Symbol oder eine Kletterausrüstung können Konfliktgegenstände sein. Auch der Erwerb von Konfliktgegenständen ist ähnlich zu Wissenstalenten. Im Gegensatz zum offiziellen DSA spielt \DEF{Geld}\index{Geld} in den StoryDSA-Regeln keine Rolle.

Rüstungen schützen den Charakter vor den negativen Auswirkungen eines Konfliktes, indem sie einen Bonus auf den Schadenswurf geben.

\item[Konflikt:]\index{Konflikt} Im Spiel wird zwischen Haupt-, Neben- und Kurzkonflikten unterschieden. Ein Konflikt liegt immer dann vor, wenn der Spielleiter aus dramaturgischen Gründen entscheidet, dass es nicht klar ist, ob die Charaktere ihr Ziel ohne Schwierigkeiten erreichen.

\DEF{Hauptkonflikte}\index{Hauptkonflikt} sind die entscheidenden Stellen einer Geschichte. Hier geht es um Gewinnen oder Verlieren, um Leben und Tod. \DEF{Nebenkonflikte}\index{Nebenkonflikt} sind immer noch interessante Konflikte, die aber keine entscheidende Bedeutung für den weiteren Verlauf der Geschichte bzw. das Ende haben. \DEF{Kurzkonflikte}\index{Kurzkonflikt} zuletzt sind dafür da, um relativ langweilige Konflikte, bei denen aber die Möglichkeit besteht, dass es negative Auswirkungen gibt schnell abzuhandeln.

Kurzkonflikte sind vergleichbar mit einfachen Eigenschaftsproben bei DSA4; Haupt- und Nebenkonflikte werden rundenweise ausgetragen. Dabei hat jeder Charakter eine Anzahl von \DEF{Konfliktpunkten}, die angibt, wie lange der Charakter im Konflikt durchhält. Allerdings regenerieren sich diese Konfliktpunkte sofort nach einem Konflikt wieder. Am Ende eines Konfliktes wird der \DEF{Schaden}\index{Schaden} bestimmt, indem für jede Runde und jeden verlorenen Konfliktpunkt eine Probe gegen 8 (Nebenkonflikte) bzw. 4 (Hauptkonflikte) gemacht wird.

Siehe auch Erzählung, Lebenskraft, Willenskraft, Würfel.

\item[Spielleitercharakter (SLC):]\index{Spielleitercharakter}\index{SLC|see{Spielleitercharakter}} Alle Charaktere, die nicht von Charakterspielern gelenkt werden und trotzdem in der Geschichte eine Rolle spielen, werden vom Spielleiter übernommen und Spielleitercharaktere genannt.

\item[Spieler:]\index{Spieler} Jeder, der mitspielt, also Spielleiter und Charakterspieler.

Der \DEF{Spielleiter}\index{Spielleiter} (SL\index{SL|see{Spielleiter}}), bei DSA auch \DEF{Meister}\index{Meister} genannt, verk"orpert die Welt, übernimmt die Spielleitercharaktere und lenkt den Plot. Üblicherweise bereitet er eine Geschichte vor, in dessen Rahmen die \DEF{Charakterspieler}\index{Charakterspieler} (CS\index{CS|see{Charakterspieler}}) ihre Spielercharaktere lenken.

\item[Spielercharakter (SC):]\index{Spielercharakter}\index{SC|see{Spielercharakter}} Protagonisten im Plot, werden von den Charakterspielern im gewissen Rahmen gelenkt; wird häufig auch Held\index{Held|see{Spielercharakter}} oder Abenteurer\index{Abenteurer|see{Spielercharakter}} genannt.

\item[Sonderfertigkeit:]\index{Sonderfertigkeit} Hierbei handelt es sich um eine spezielle Ausbildung. Damit sind solche wie auch DSA üblichen Sonderfertigkeiten waldkundig oder Hammerschlag gemeint, aber auch andere Ausbildungen, wie z.\,B. KGIA-Agent können hiermit modelliert werden.

Mechanisch geben Sonderfertigkeiten in Kurzkoflikten höhere Chancen auf kritische Erfolge; in Neben- und Hauptkonflikten geben Sonderfertigkeiten zusätzliche Erfolge bei besonders niedrigen Würfen. Sonderfertigkeiten können erst während des Spieles und nicht bei der Charaktererschaffung erworben werden.

\item[Talent:]\index{Talent} Ein Talent ist eine Fähigkeit eines Charakters, die durch Training verbessert werden kann. Es gibt eine feste Liste möglicher Basis-, Spezial- und Wissenstalente. Darüberhinaus gibt es noch Berufstalente. Im Gegensatz zu DSA ist die Liste der möglichen Talente deutlich verkürzt; vieles fällt jetzt unter Berufstalente und einiges wurde zusammengefasst.

\DEF{Basistalente}\index{Basistalent} sind grundlegende Fähigkeiten, ohne die ein Charakter nicht auf Abenteuer ziehen sollte, wohingegen \DEF{Spezialtalente}\index{Spezialtalent} eine besondere Ausbildung in einer für Abenteuer relevanten Fähigkeit repräsentieren. Mit \DEF{Wissenstalenten}\index{Wissenstalent} kann ein Spieler die Chancen der Anwendung eines Talentes verbessern. Je nach Wissensstand bekommt er bei Einsatz bis zu drei Bonuswürfeln. \DEF{Berufstalente}\index{Berufstalent} repräsentieren die erlernen Fähigkeiten, die nur selten in Abenteuern gebraucht werden.

Jedes Basis-, Spezial- und Berufstalent hat einen \DEF{Talentgesamtwert}\index{Talentgesamtwert}. Grundlage von Basis- und Spezialtalenten bildet jeweils die Hälfte der Summe von drei Eigenschaften. Dazu wird noch der \DEF{Talentwert}\index{Talentwert} addiert. Diese Talentgesamtwerte liegen bei Spielbeginn im Bereich 2 bis 11, können aber im laufenden Spiel bis auf 18 gesteigert werden.

Dagegen wird der Talentgesamtwert eines Berufstalentes zu Beginn unabhängig von den Eigenschaften auf 10 festgelegt. Berufstalente steigen automatisch bis maximal auf 19.

Wissenstalente haben keinen Talentgesamtwert und können auf jeweils drei Stufen gelernt werden. Die einzigen Wissenstalente, die jeder Charakter zu Beginn seiner Abenteurerkarriere hat, sind seine Muttersprache und eine Zweitsprache.

\DEF{Magische Talente}\index{Talent!magisch} sind einfach Spezialtalente, die aus den aventurischen magischen Schulen, Merkmalen oder Zauberprofessionen abgeleitet werden. Charaktere mit einem oder mehreren magischen Talenten sind \DEF{magiebegabt}\index{magiebegabt} bzw. \DEF{zauberkundig}\index{zauberkundig}.

\item[Vor- und Nachteil:]\index{Vorteil}\index{Nachteil} Ein Vorteil ist eine besondere Fähigkeit oder Charaktereigenschaft, die nicht durch die Talente erfasst wird und sich positiv für den Spieler auswirkt. Nachteile dagegen wirken sich negativ aus. Üblicherweise sind Vor- und Nachteile Eigenschaften, die nicht durch Training verbessert werden können, wie z.\,B. ein besonderer Stand oder psychische Besonderheiten.

Zu Spielbeginn hat ein Spieler Vorteile für 4~Punkte plus den Wert der Nachteile. Die Nachteile übersteigen am Anfang nicht einen Wert von 10. Vor- und Nachteile können durch kritische Erfolge und Misserfolge bei Hauptkonflikten steigen oder sinken.

\item[Würfel:]\index{Würfel} Wann immer bei StoryDSA von einem Würfel geredet wird, ist ein W20 gemeint. Damit werden, wie bei DSA auch sonst üblich, Unterwürfelproben gemacht, d.\,h. das Ergebnis eines Wurfes wird mit einem Talentwert verglichen. Ist das Ergebnis größer als der Talentwert, so ist die Probe misslungen, was im Allgemeinen schlechte Auswirkungen für den Charakter hat. Ansonsten, also wenn das Ergebnis höchstens so groß ist wie der Talentwert, ist die Probe gelungen und die Auswirkung ist für den Charakter normalerweise positiv.

In Neben- und Hauptkonflikten werden üblicherweise 5 Würfel geworfen, denn die Spieler erhalten für ihre Erzählung Würfel (entsprechend dem Erzählwert, bis zu 5). In Kurzkonflikten kommt normalerweise nur ein einzelner Würfel zum Einsatz. \DEF{Bonuswürfel}\index{Bonuswürfel} sind zusätzliche Würfel, die durch Magie, Vorteile, Wissenstalente oder Konflikt-Gegenstände dazukommen. \DEF{Maluswürfel}\index{Maluswürfel} kann es durch Magie oder Nachteile geben und müssen von der Anzahl der Würfel abgezogen werden.

Siehe auch Erzählung, Konflikt.

\end{description}

\FEHLT{Weitere Begriffe?}




