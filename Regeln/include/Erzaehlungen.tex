\chapter{Erzählungen}\label{Ch:Erzaehlungen}\index{Erzählung}
\lettrine[findent=0.3em,nindent=0.2em]{E}{rzählungen} nehmen eine zentrale Rolle im Rollenspiel ein. Insbesondere in diesem Spiel kommt es aufs Erzählen an -- stimmungsvoll ausgespielte Konflikte sollen ja schließlich im Vordergrund stehen.

Um das zu erreichen, soll einerseits jeder Spieler das Recht bekommen, im vom Spielleiter gesteckten Rahmen alles das zu erzählen, was ihm Spaß macht. Andererseits sollen aber auch die anderen Spieler die Möglichkeite haben, ihre Meinung über das Erzählte zu äußern.

\section{Prinzip der Erzählte Wahrheit}
Dieses Prinzip gilt während des gesamten Spiels (mit einer kleinen Abweichung in den Kurzkonflikten). Es besagt:
\begin{quote}
Alles, was ein Spieler erzählt, passiert, sobald er es erzählt und genau so, wie er es erzählt.
\end{quote}

Das bedeutet, es gibt kein `ich versuche, dies oder jenes zu tun', sondern nur ein klares `ich tue dies oder jenes'. Es gibt auch keine Nachfrage beim Spielleiter, ob irgendetwas vorhanden ist oder ob etwas möglich ist oder nicht. Die Erzählte Wahrheit gilt unabhängig von irgendwelchen Würfelergebnissen.

Die Kurzkonflikte weichen von dem Prinzip dahingehend ab, dass zunächst gesagt wird, was der Charakter vorhat, dann wird gewürfelt und erst danach wird im Rahmen des Würfelergebnisses erzählt.

Insbesondere soll hier darauf hingewiesen werden, dass auch in den SL-Erzählphasen das Prinzip der Erzählten Wahrheit gilt. Allerdings gibt es hier (im Gegensatz zu den Konflikten und dem freien Spiel) kein persönliches Veto-Recht (s.\,u.), so dass der Spielleiter in seinen Erzählungen den Plot in die Richtung lenken kann, die er vorgesehen hat.

\section{Das Konfliktende}
Durch das Prinzip der Erzählten Wahrheit könnte ein Spieler gleich zu Beginn eines Problems das Ende beschreiben. Das wird aber durch die \DEF{Konfliktende}-Regel\index{Konfliktende} verboten.
\begin{quote}
  Bevor nicht das Ende eines Konfliktes erreicht ist, darf niemand das Ende vorwegnehmen.
\end{quote}
Das Ende eines Konfliktes wird, wie später noch erklärt wird, durch Würfelwürfe bestimmt. Ist das Konfliktende erreicht, darf die Gewinnerseite erklären, wie das Konfliktende aussieht. Dabei hat der SL aber immer das Recht, Ergänzungen in der dann folgenden SL-Erzählphase zu machen.

Konfliktende bedeutet jedoch nicht, dass z.\,B. am Ende eines Kampfes alle Gegner getötet werden müssen. Es kann auch sein, dass ein Überlebender gefangen wird, dass die Gegner alle vertrieben wurden, usw. Am Konfliktende steht also fest, wer gewinnt. Wie genau dieser Gewinn aussieht, kann der Gewinner des Würfelduells festlegen.

\section{Kompetenzen}
Damit das Spiel nicht ganz aus dem Ruder läuft, gibt es eine klare Kompetenzregelung, was die einzelnen Spieler erzählen dürfen. Das wichtige ist dabei, dass kein Spieler seine Kompetenzen überschreiten darf. Der Spielleiter hat in den SL-Erzählphasen, das Recht, die Geschichte beliebig weiter zu erzählen. Dabei darf er die Spielercharaktere genauso einbinden wie andere Charaktere oder die Umgebung. Auch Zeitsprünge o.\,ä. sind dem Spielleiter erlaubt.

Im freien Spiel hat jeder Charakterspieler die Kompetenz, die Handlungen seines Charakters zu beschreiben und eventuelle Details der Umgebung hinzuzufügen. Der Spielleiter hat im freien Spiel die Hoheit über alle Charaktere außer den Spielercharakteren, wobei er seine Kompetenz auch an andere Spieler abgeben darf.

In Konflikten haben alle Spieler die Kompetenz, einen beliebigen Verlauf des Geschehens zu erzählen, einschließlich aller Charaktere. Einzig die Konfliktende-Regel muss eingehalten werden.

\section{Veto}
Gegen jegliche Art von Erzählung während der Konflikte, des freien Spiels und der SL-Erzählphasen hat jeder Spieler ein allgemeines Veto-Recht. Setzt ein Spieler sein Veto ein, wird das Spiel angehalten. Dann wird darüber diskutiert, was dem Spieler an der Erzählung nicht gepasst hat und wie man das Problem beheben könnte. Anschließend wird die Erzählung wiederholt und das Spiel läuft normal weiter.

Das allgemeine Veto dient als eine Art Notbremse dazu, unpassende und stimmungstötende Beschreibungen zu unterbinden. Weiterhin kann damit das verfrühte Ende eines Konfliktes oder eine Kompetenzüberschreitung gestoppt werden. Das allgemeine Veto darf nicht dazu benutzt werden, eigene Ideen in die Beschreibung von anderen mit einzuflechten sondern dient ausschließlich dazu, andere Beschreibungen mit einer guten Begründung zu stoppen. 

In Konflikten hat jeder Spieler zusätzlich noch ein persönliches Veto, was bedeutet, dass jeder Spieler praktisch die Kontrolle über seinen eigenen Charakter behält (bzw. der Spielleiter über alle weiteren Spielfiguren). Dieses kann jeder Spieler einsetzen, um Beschreibungen zu stoppen, die ihm einfach nicht für diesen Charakter passend erscheinen. Für ein persönliches Veto ist eine Begründung wie `stimmungstötend' oder `Kompetenzüberschreitung' nicht nötig. Das Missfallen einer Beschreibung reicht schon.

Wichtig ist, dass die Vetos auch eingesetzt werden, d.\,h. ein Spieler sollte nicht etwas einfach so dulden, obwohl er ein Veto einsetzen könnte. Wenn das Veto auch wirklich benutzt wird, funktioniert die Veto-Regel nach einiger Zeit durch nicht-Anwendung, d.\,h. die Spieler werden eine bestimmte Tatsache, bei der sie sicher sind, ein Veto zu erhalten, einfach nicht erzählen.

\section{Erzählwert}
In Haupt- und Nebenkonflikten (nicht aber in Kurzkonflikten) wird der Fortgang des Konfliktes häppchenweise erzählt. Nach jedem Stück Erzählung wird der \DEF{Erzählwert}\index{Erzählwert} durch den SL festgelgt. Für jedes eingebrachte Faktum gibt es einen Würfel, bis maximal 5 Würfel. Dabei ist egal, ob die Fakten stimmig sind, gut zusammenpassen o.\,ä. Die Qualität spielt also keine Rolle, es zählt nur die Quantität. Was genau als Faktum zählt, liegt beim SL. Als Richtlinie gilt, dass jede Aktion und jede nähere Beschreibung mit einem Würfel belohnt werden soll.

Ist eine Beschreibung allzu abwegig und damit nicht hinnehmbar, so sollte ein Veto eingelegt werden. Für qualitativ besonders gelungene Beschreibungen können die Charakterspieler Erzählmarken (s.\,u.) vergeben.

Die Erzählwert-Regel dient dazu, die Menge der Erzählungen der Spieler zu steuern. Da ein einzelner Punkt leicht zu bekommen ist, kann man davon ausgehen, dass jede Erzählung immer den vollen Punktewert bekommt.

\subsection{Beispiele}
\begin{beispiel}
\begin{itemize}
\item ``Ich schlage mit meinem Schwert.'' gibt 1 Würfel.
\item ``Ich schlage mit meinem Schwert in einem weiten Bogen.'' gibt 2 Würfel
\item ``Ich schlage mit meinem Schwert in einem weiten Bogen in Richtung seines Halses.'' gibt 3 Würfel
\item ``Ich schlage mit meinem Schwert in einem weiten Bogen in Richtung seines Halses. Er duckt sich darunter weg.'' gibt 4 Würfel
\item ``Ich schlage mit meinem Schwert in einem weiten Bogen in Richtung seines Halses. Er duckt sich darunter weg, so dass mein Schwert nur in den Balken fährt.'' gibt 5 Würfel
\end{itemize}
\end{beispiel}

\begin{optional}
\section{Optional: Anderes Erzählwert-Maximum}

Im Laufe des Spiel kann sich herausstellen, dass das Maximum von 5 Würfeln für die Gruppe zu hoch oder zu niedrig ist. Es kann natürlich problemlos geändert werden, jedoch sollte der SL dann beachten, dass sich dadurch die Schwierigkeit von Neben- und Hauptkonflikten ändert, d.\,h. evtl. müssen dann auch die Konfliktpunkte der Konfliktgegner angepasst werden. Dabei gilt natürlich: Je höher das Erzählwert-Maximum ist, umso einfacher sind die Konflikte.
\end{optional}

\begin{optional}
\section{Optional: Ohne Erzählwert}

Möchten die Spieler selber nicht so viel erzählen, so kann man die Erzählwert-Regel auch einfach weglassen und stattdessen \emph{jegliche} Aktion mit fünf Würfeln belohnen, unabhängig von der Menge der Beschreibung.

Eine Anregung, gute Erzählungen zu liefern, wird damit immer noch durch die Erzählmarken und durch die Konfliktstruktur gegeben. Aber ich empfehle ausdrücklich, dem Erzählwert eine Chance zu geben und ihn zumindest einen Spielabend lang zu testen. Auch bislang sehr ruhige Spieler wachsen erfahrungsgemäß bei solchen Regeln oft über sich hinaus.
\end{optional}

\section{Erzählmarken}
Darüberhinaus kann jeder Charakterspieler pro Spielabend zwei \DEF{Erzählmarken}\index{Erzählmarke} für seiner Meinung nach besonders stimmungsvolle Beschreibungen an einen anderen Charakterspieler vergeben. Das kommt oft in Haupt- oder Nebenkonflikten vor, kann auch für ein besonders gelungenes freies Spiel oder eine gute Beschreibung für einen Kurzkonflikt sein. Für eine Beschreibung kann ein Charakterspieler höchstens eine Erzählmarke bekommen. Wollen mehrere Spieler gleichzeitig für eine Beschreibung eine Marke vergeben, so müssen sie sich einigen, wer sie vergibt. Unabhängig von der Länge der Erzählung beträgt der Erzählwert für etwas, wofür eine Erzählmarke vergeben wurde, immer dem Maximum.

Wichtig ist, dass diese Marke bei einem späteren Konflikt eingesetzt werden kann; pro Konflikt kann höchstens eine Marke eingesetzt werden. Eine eingesetzte Marke gibt einen Bonuswürfel für die gesamte Dauer eines Konfliktes und zwar unabhängig von der eingesetzten Fähigkeit.

Für einen laufenden Konflikt ist die Marke wertlos. Außerdem kann niemand mehr als zwei Erzählmarken haben. Weitere vergebene Marken verfallen einfach. Marken, die an einem Spielabend nicht vergeben wurden, verfallen ebenfalls.

Erzählmarken können gut durch Pokerchips dargestellt werden, wobei streng zwischen noch nicht vergebenen und bekommenen Marken unterschieden werden muss -- am besten durch unterschiedliche Farben.

\begin{design}
\subsubsection{Designanmerkung: Das Zusammenspiel der Erzählregeln}\label{DesignAnmerkungenZusammenspielErzaehlregeln}

Durch die gegebenen Regeln -- Prinzip der Erzählten Wahrheit, Konfliktende, Veto, Erzählwert und Erzählmarken -- wird eine Qualitätssicherung in Sachen Erzählung erreicht.

Das Prinzip der Erzählten Wahrheit gibt jedem Spieler die Möglichkeit, seine Vorstellung von Aventurien und seinem Charakter im Spiel auch wirklich umzusetzen und nicht durch irgendwen oder irgendwelche verteilten Punkte aktiv dauernd daran gehindert zu werden. Damit das aber nicht ausufert, gibt es die Konfliktende- und die Veto-Regel. Erstere verhindert, dass dem SL der Plot aus der Hand genommen wird, letztere sichert, dass kein Spieler für ihn unterträgliches Spiel dulden muss.

Durch den Erzählwert wird die Quantität der Erzählung, durch die Erzählmarken die Qualität der Erzählung gesichert. Während der Erzählwert ein Minimum an Erzählung vorgibt, können die Mitspieler durch die Erzählmarken klare Signale setzen, welche Art von Beschreibung sie gerne haben wollen. Durch die Kombination aus Erzählwert und Erzählmarken wird aber auch indirekt ein Maximum an Beschreibung vorgegeben: Zu langatmige, selbstdarstellerische Beschreibungen, die deutlich den maximalen Erzählwert überschreiten, werden wohl kaum durch Erzählmarken gewürdigt. Umgekehrt können besonders einfallsreiche, sehr kurze Beschreibungen zu einer Erzählmarke führen und damit ggf. auch zum maximalen Erzählwert aufgestockt werden.
\end{design}

\begin{optional}
\section{Optional: Dem SL die Zügel aus der Hand nehmen}\label{Optional:ZuegelAusDerHand1}

Mit ein paar regeltechnischen Handgriffen ist es auch möglich, dem SL die Zügel aus der Hand zu nehmen. Dazu erweitert man einfach das persönliche Veto der Charakterspieler auch auf die SL-Erzählphase. Die Erzählung des Konfliktendes legt man völlig frei in die Hand des Gewinners -- dieser kann dann erzählen, wie die Geschichte weitergeht und die nächste SL-Erzählphase beginnen. Sollten die Spieler den Konflikt gewinnen, so müssen sie auch nicht den Erfolg der Charaktere erzählen. Vielmehr kann das Ende auch ein Misserfolg der Charaktere darstellen.

Weiterhin gibt man auch dem SL zwei Erzählmarken zum Verteilen und die CS können auch dem SL Erzählmarken geben. Erzählmarken sollen nicht mehr unbedingt für gutes Erzählen vergeben werden, sondern für gute Ideen im Allgemeinen. Außerdem müssen diese dann \emph{sobald möglich} (also im laufenden oder im nächsten Konflikt) eingesetzt werden.
\end{optional}


\section{Tipps für gute Erzählungen}\label{TippsGuteErz}
Erzählungen während der Konflikte haben nur geringe Auswirkungen auf den Ausgang. Einzig das Talent, auf welches gewürfelt wird, wird durch die Beschreibung bestimmt. Der Rest dient nur dazu, Würfel und Erzählmarken zu bekommen. Damit ist man darin, was man jetzt tatsächlich erzählt, sehr frei.

Es ist also kein Risiko dabei, zu erzählen, wie der eigene Held in Bedrängnis gerät und wie es ihm gelingt, sich aus diser Lage durch ein schwieriges Manöver wieder zu befreien. Wichtig ist nur, dass die Beschreibung zum Spielstil der Gruppe und zur Situation passen und die erzählte Geschichte bereichern. Die Würfelwürfe sind davon absichtlich losgelöst, damit nicht eine spannende Stelle zur Sicherheit lieber langweiliger beschrieben wird. Trotzdem haben natürlich die Fähigkeiten der Helden erheblichen Einfluss auf den Konfliktausgang. Veto-Recht und Erzählmarken stellen die Qualität der Erzählung sicher (vgl. hierzu auch die Design-Anmerkungen auf Seite~\pageref{DesignAnmerkungenZusammenspielErzaehlregeln}).

Durch den Erzählwert neigen manche Spieler dazu, eine reine Aufzählung von Adjektiven zu machen, um die fünf Würfel voll zu kriegen. Dem sollen die folgenden Tipps entgegenwirken. Klarerweise können sie nicht alle gleichzeitig angewedet werden, sondern sollen für interessante Beschreibungen abwechselnd benutzt werden.

\begin{description}
  \pagebreak[3]
  \item[Perspektiven:]~
  \begin{itemize}
    \item Ich-Perspektive
    \item Perspektive des Konfliktgegners
    \item Perspektive einer dritten Person
    \item Zoom auf ein Detail
    \item Zeitlupe
  \end{itemize}
  \pagebreak[3]
  \item[Inhalte:]~
  \begin{itemize}
    \item Aktion des Charakters oder eines Konfliktgegners
    \item Misslungene Aktion des Charakters oder eines Konfliktgegners 
    \item Einbeziehung der Umgebung und anderer Konfliktteilnehmer
  \end{itemize}
  \pagebreak[3]
  \item[Art und Weise:]~
  \begin{itemize}
    \item Besonders schnell oder langsam sprechen
    \item Besonders laut oder leise sprechen
    \item absichtliche kurze Pause
  \end{itemize}
\end{description}

Dabei sollte die Beschreibung natürlich immer zur Stimmung der Situation und des Konfliktes passen. Klar, wenn der Charakter heimlich an einer Wache vorbeischleicht, wird man kaum absichtlich laut oder hektisch reden -- höchstens um spannende Akzente zu setzen.

\begin{beispiel}
Die folgenden Beispiele haben jeweils einen Erzählwert von mindestens~5:
\begin{itemize}
  \item {Der Fußboden knackt leise, als Harro ganz vorsichtig einen Schritt nach vorne mache. Erschreckt bleibt er stehen. [hörbares Einatmen, kurzes Luftanhalten] Hat die Wache ihn gehört?}

  \item Ich springe vor, schlage mit meinem Schwert dem Ork mit einem `Schack' den Kopf von den Schultern. Dabei rammt mir ein anderer von links sein Knie in die Seite. Ich breche röchelnd zusammen.

  \item {Der Goblin sieht aus dem Augenwinkel einen Lichtblitz -- doch zu spät. Ein Pfeil steckt plötzlich in seinem Hals, Blut sickert langsam aus der Einstichwunde. Langsam und ohne einen Laut von sich zu geben bricht der Goblin zusammen.}

  \item {Von einem Stuhl in den Bauch gerammt rutscht Alrik auf mich zu. Mit einem Hechtsprung bringe ich mich in Sicherheit, reiße den Schürhaken aus dem Feuer und gehe bedrohlich auf den nächstbesten Schläger zu.}

  \item Der Wirt schaut mich an: `Was willst du? Informationen über den roten Alrik? Unter zwei Dukaten läuft da gar nichts.' Ich krame in meinem Geldbeutel herum und ziehe eine goldene Münze hervor.

  \item Schon seit Stunden irren wir durch den Wald. Der Regen ist so stark, dass selbst das dichte Blätterdach keinen Schutz mehr bietet. Den Blick immer nach unten gerichtet, um nicht die deutlichen Schleifspuren zu verlieren, die die Räuber hinterlassen haben.

  \item {Travianes rechte Hand krallt sich am Felsvorsprung fest. Jede Muskel ist angespannt, die Adern des Handrückens treten bläulich hervor. Sie keucht -- um Haaresbreite wäre sie abgestürzt. Doch jetzt, da sie sich wieder gefangen hat, schiebt sie sich wieder weiter nach oben.}
\end{itemize}
\end{beispiel}


