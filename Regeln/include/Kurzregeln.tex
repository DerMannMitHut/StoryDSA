\chapter{Kurzregeln}
Die Kurzregeln sollen dazu dienen, einerseits einen schnellen Überblick zu bekommen und andererseits als Möglichkeit, um bei Fragen die Regeln in einer kompakten Version nachschlagen zu können. Zur besseren Zuordnung und Übersicht sind die Überschriften innerhalb der Kurzregeln genauso benannt wie die Kapitel, die zusammengefasst werden. Optionalregeln, Beispiele und Designanmerkungen werden in diesem Abschnitt nicht aufgeführt.

\section[Charaktererschaffung]{Charaktererschaffung (ab Seite~\pageref{Ch:Charaktererschaffung})}
Die Charaktererschaffung in Stichpunkten:
\begin{itemize}
	\item Charaktere sollten immer gemeinsam erschaffen werden
	\item Wähle zunächst Rasse, Kultur und Profession
	\item Verfasse eine Hintergrundgeschichte mit maximal 50 Wörtern, wobei der Name des Charakters nicht mitgezählt wird
	\item Beantworte folgende Fragen:
	\begin{enumerate}
		\item Warum macht es dir Spaß, diesen Charakter zu spielen?
		\item Was ist das wichtigste Wesen im Leben deines Charakters oder warum?
		\item Nenne zwei oder drei Leidenschaften deines Charakters (Verpflichtung, Liebe, Wut, Angst).
		\item Nenne zwei oder drei Überzeugungen oder Prinzipien deines Charakters.
	\end{enumerate}
	\item Vor- und Nachteile: 4 Vorteilspunkte, bis zu 10 weitere durch Nachteile
	\begin{itemize}
		\item 1 Talentpunkt Bonus pro Vorteilspunkt
		\item 1/2/3 Bonuswürfel: 2/6/12 Vorteilspunkte
		\item 1/2/3/4 Rüstungsschutz Bonus: 1/3/6/10 Vorteilspunkte
		\item 2 Talentpunkte Malus pro Vorteilspunkt
		\item 1/2/3 Bonuswürfel: 1/3/6 Vorteilspunkte
		\item 1/2/3 Rüstungsschutz Malus: 1/2/3 Vorteilspunkte
		\item sonstige Vor- oder Nachteile: Je nach Definition
	\end{itemize}
	\item Eigenschaften: 23 Punkte verteilen, pro Eigenschaft --1 bis 3; eine Eigenschaft auf 3, zwei weitere auf mindestens 2

Kosten:
\begin{tabular}[C]{cccccccc}
  \bf MU\index{Mut}%
& \bf KL\index{Klugheit}%
& \bf IN\index{Intuition}%
& \bf CH\index{Charisma}%
& \bf GE\index{Gewandtheit}%
& \bf FF\index{Fingerfertigkeit}%
& \bf KO\index{Konstitution}%
& \bf KK\index{K"orperkraft}%
\\
3 & 2 & 5 & 2 & 3 & 4 & 1 & 3 \\
\end{tabular}

Mindest-/Höchstwerte Umrechnung:
\begin{tabular}[C]{lccccc}
  \bfseries DSA4 & 8--9 & 10--11 & 12--13 & 14--15 & ab 16 \\
  \bfseries \StoryDSA & $-1$ & 0 & 1 & 2 & 3 \\
\end{tabular}

Mindest-/Höchstwerte durch Bonus/Maluspunkte:
\begin{tabular}[C]{lcccc}
  \bfseries DSA4 & $+1$ & $+2$ & $+3$ & ab $+4$ \\
  \bfseries \StoryDSA & mind. $0$ & mind. $1$ & mind. $2$ & mind. $3$ \\[\medskipamount]
  \bfseries DSA4 & $-1$ & $-2$ & $-3$ & ab $-4$ \\
  \bfseries \StoryDSA & max. $2$ & max. $1$ & max. $0$ & max. $-1$ \\
\end{tabular}

	\item automatische Talente:
		\begin{itemize}
			\item Expertenwissen (2) Muttersprache
			\item Grundwissen (1) Zweitsprache
		\end{itemize}

	\item Talente: 50 Steigerungen verteilen
	
	\begin{tabular}[C]{ll}
		1 Talentwert & 1 Steigerung (max. 6) \\
		Aktivierung Spezialtalent & 1 Steigerung \\
		Wissentalent 1 Bonuswürfel & 2 Steigerungen \\
		Wissentalent 2 Bonuswürfel & 6 Steigerungen \\
		Gegenstand: 1 Bonuswürfel & 2 Steigerungen \\
		Gegenstand: 2 Bonuswürfel & 6 Steigerungen \\
		Rüstung: 1 Rüstungsschutz & 1 Steigerung \\
		Rüstung: 2 Rüstungsschutz & 3 Steigerungen \\
		Rüstung: 3 Rüstungsschutz & 6 Steigerungen \\
		magisches Talent & wie Spezialtalent \\
		magischer Effekt & 1 Steigerung (max. Talentgesamtwert) \\
	\end{tabular}
	
	\item Talentgesamtwert = Hälfte Eigenschaftssumme zuzügliche Talentwert
	
	\item Konfliktpunkte: 3
	
	\item Willenskraft: MU+KL+IN+CH+6
	
	Lebenskraft: GE+FF+KO+KK+8
	
	\item Astralpunkt (bei einem magischen Talent): 2~Astralpunkt für 1~Kraftpunkt (beliebig)
\end{itemize}

\section[Struktur]{Struktur (ab Seite~\pageref{Ch:Struktur})}
Das Spiel unterteilt sich in SL-Erzählphasen, freies Spiel, Kurz-, Neben- und Hauptkonflikte. Dabei liegt immer eine SL-Erzählphase zwischen zwei anderen. Der Spielleiter verdeutlicht die Einleitung der unterschiedlichen Phasen durch Schlüsselworte:
\begin{description}
\item[Freies Spiel:] ``Was wollt ihr tun?''

\item[unerwarteter Konflikt:] ``\dots als/und plötzlich\dots''

\item[absehbarer Konflikt:] ``Es kommt zu Schwierigkeiten.''

\item[ungeplanter Konflikt:] ``\dots nicht so einfach \dots, wie du/ihr es gedacht hast/habt.''

\item[Einleitung eines Konfliktes:] ``Dein/Euer Ziel ist es, \dots''

\item[SL-Erzählphase:] ``Nachdem du/ihr \dots''. Zur Verdeutlichung, dass in eine SL-Erz"ahlphase "ubergeleitet wird, kann der SL zus"atzlich die Hand heben.
\end{description}


\section[Erzählungen]{Erzählungen (ab Seite~\pageref{Ch:Erzaehlungen})}
Grundsätzlich gilt das Prinzip der erzählten Wahrheit. Das bedeutet, dass etwas in der Spielwelt eintritt, sobald ein Spieler es erzählt, unabhängig von irgendwelchen Würfelergebnissen. Außerdem gilt in Konflikten die Konflikt-Ende-Regel, d.\,h. das das Konfliktende durch eine Erzählung nicht vorweg genommen werden darf.

Die Kompetenzverteilung ist klassisch: Jeder Charakterspieler kontrolliert seinen Charakter, der Spielleiter den Rest. In Neben- und Hauptkonflikten dürfen alle beteiligten Charaktere in eine Erzählung mit eingebunden werden.

Um einen vernünftiges Spiel zu gewährleisten gibt es zwei Vetos. Das allgemeine Veto ist eine Notbremse und dient dazu, stimmungstötende Beschreibungen zu unterbinden. Es darf jederzeit von jedem Spieler angewendet werden. Das persönliche Veto betrifft die eigene Kompetenz: Betrifft die Beschreibung eines Mitspielers direkt den eigenen Kompetenzbereich, darf das sofort und ohne weitere Begründung gestoppt werden.

In Neben- und Hauptkonflikten wird darüber hinaus für jede Erzählung gezählt, wie viele Fakten ein Spieler einbringt (Erzählwert). Dafür bekommt er Würfel, maximal jedoch 5. 

Außerdem können sich Spieler gegenseitig Erzählmarken geben. Davon erhält jeder Spieler am Beginn jedes Spielabends zwei. Spielmarken werden für besonders gute Beschreibungen vergeben. Eine von einem anderen Mitspieler bekommene Marke kann in einem späteren Konflikt eingesetzt werden -- dafür bekommt man einen Bonuswürfel zusätzlich für die gesamte Konfliktdauer. Mehr als eine Erzählmarke darf pro Konflikt und Charakter nicht eingesetzt werden.


\section[SL-Erzählphase]{SL-Erzählphase (ab Seite~\pageref{Ch:SLErzaehlphase})}
In einer solchen Phase erzählt der Spielleiter einen Übergang zur nächsten Phase. Die Charakterspieler haben hier nur das kleine Veto.


\section[Freies Spiel]{Freies Spiel (ab Seite~\pageref{Ch:FreiesSpiel})}
Die Spieler können ihre Charaktere frei ausspielen. Im freien Spiel wird nicht gewürfelt, außer Richtungsentscheidungen passiert nichts spielrelevantes.


\section[Konflikte]{Konflikte (ab Seite~\pageref{Ch:Konflikte})}
Ob ein Talent in einem Konflikt relevant ist, entscheidet im Zweifelsfall der Spielleiter. Genauso verhält es sich mit Sonderfertigkeiten, Wissenstalenten und Konfliktgegenständen. Sonderfertigkeiten ermöglichen sogenannte meisterliche Würfe, wohingegen Wissenstalente und Konfliktgegenstände Bonuswürfel geben. Ob ein Talent relevant ist, hängt auch von der konkreten Situation innerhalb eines Konfliktes ab und kann sich im Laufe eines Neben- oder Hauptkonfliktes auch ändern.

Eine Besonderheit im Gegensatz zum offiziellen DSA ist das explizite Auformulieren des Zieles der Charaktere, also was sie während des Konfliktes erreichen wollen. Dabei sollte die Methode möglichst nicht im Ziel vorkommen. In Kurz- und Nebenkonflikten wird dieses Ziel oft (unter Schwierigkeiten und mit einem Folgekonflikt) trotzdem erreicht. In Hauptkonflikten ist der Ausgang völlig offen.

Schaden wird nach einem Neben- oder Hauptkonflikt verteilt. Dabei würfelt der Spieler für jede Runde und jeden verlorenen Konfliktpunkt 1W20 gegen 8 (Nebenkonflikte) bzw. 4 (Hauptkonflikte), jeder Würfel, der mehr zeigt, bedeutet einen Schadenspunkt. Ob der Schaden körperlich oder geistig ist, geht aus dem Zusammenhang hervor.

Durch Kurzkonflikte bekommt ein Charakter nur dann Schaden, wenn der Spielleiter dies vor dem Kurzkonflikt angekündigt hat.

Konflikte können nur wiederholt werden, wenn die Umstände, die zum Ziel führen sollen, andere sind und muss von den äußeren Umständen her erlaubt sein. Im Zweifelsfall entscheidet der Spielleiter.

\section[Kurzkonflikt]{Kurzkonflikt (ab Seite~\pageref{Sec:Kurzkonflikt})}
Darstellung von kurzen oder langweiligen Konflikten. Auch gut ausspielbare Konflikte können hier in Frage kommen. An Kurzkonflikten ist immer nur ein Charakter beteiligt.

Der Ablauf in Stichpunkten:
\begin{enumerate}
\item Festlegen des Konfliktziels und eventueller Schadensfolgen beim Misserfolg
\item Wahl des (relevanten) Talentes, Wissenstalentes, Gegenstandes und der Sonderfertigkeit. Eventueller Bonus (3) oder Malus (3 oder 6) für die äußeren Umstände.
\item Eventueller Abbruch des Konfliktes durch den Spieler
\item Je nach Bonuswürfel ein oder mehrere W20 gegen Talentgesamtwert; der niedrigste Würfel zählt. Ist die Probe gelungen und der niedrigste Würfel eine 1 (bzw. 2 oder 4, je nach Sonderfertigkeit), so ist das Ergebnis ein kritischer Erfolg. Ist die Probe misslungen und der höchste Würfel eine 20, so ist das Ergebnis ein kritischer Misserfolg.
\item Interpretation des Ergebnisses durch den Spielleiter und Ausspielen der Konfliktfolgen. Bei einem kritischen Misserfolg wird der eventuelle Schaden verdreifacht.
\end{enumerate}

Für Hilfe von anderen können diese vor dem entsprechenden Kurzkonflikt eigene Kurzkonflikte zur Hilfe machen. Gelingen diese, wird der Kurzkonflikt um 3 erleichtert. Misslingt die Hilfe, passiert nichts. Misslingt die Hilfe kritisch, so erschwert das den Kurzkonflikt um 3.

\section[Nebenkonflikt]{Nebenkonflikt (ab Seite~\pageref{Sec:Nebenkonflikt})}
Ausführliche Darstellung von Konflikten. Alle Charaktere können beteiligt werden.

Der Ablauf in Stichpunkten:
\begin{enumerate}
\item Festlegen des Konfliktziels und eventueller Schadensfolgen beim Misserfolg
\item Konfliktpunkte regenerieren
\item Festlegen der Konfliktpunkte des Nebenkonfliktes durch den SL (Richtlinie: Fünffache der Anzahl der beteiligten Charaktere)
\item Festlegen des automatischen Offensiv- und Defensivergebnisses durch den SL (Richtlinie für Anfängercharaktere: 1/0)
\item Rundenbeginn: Jeder Charakterspieler erzählt, der Spielleiter nicht
\item Festlegung der Erzählwerte und Verteilung entsprechend vieler Würfel
\item Aufteilen der Würfel in Offensiv- und Defensivwürfel und Würfeln gegen einen passenden Talentgesamtwert. Durch Sonderfertigkeiten gibt es bei einer 1--2 bzw. 1--4 einen meisterlichen Wurf und damit einen zusätzlichen Erfolg.
\item Verrechnung der Erfolge und Minderung der Konfliktpunkte auf allen Seiten
\item Freiwilliges oder unfreiwilliges (keine Konfliktpunkte übrig) Ausscheiden von Charakteren
\item Wiederholung der Runde, solange der Spielleiter und mindestens ein Spieler noch Konfliktpunkte haben
\item Auswürfeln von Schaden und Interpretation des Ergebnisses: Hat der SL alle Konfliktpunkte verloren, haben die Spieler gewonnen und das Ziel wird erreicht. Hat der SL noch Konfliktpunkte übrig, so interpretiert er das Ergebnis.

Darüberhinaus gibt es noch persönliche Konfliktfolgen (alle Konfliktpunkte verloren = kritischer Misserfolg, mehr als die Hälfte verloren = Misserfolg, höchstens die Hälfte verloren = Erfolg, alle Konfliktpunkte übrig = kritischer Erfolg)
\end{enumerate}

\section[Hauptkonflikt]{Hauptkonflikt (ab Seite~\pageref{Sec:Hauptkonflikt})}
Im Prinzip wie Nebenkonflikte, nur dass der Spielleiter mit (mindestens) einem eigenen Charakter beteiligt ist und zusammen mit den Spielern würfelt bzw. erzählt. Haupt- und Nebenkonflikte können auch gemischt werden; insgesamt zählt das dann als Hauptkonflikt. Reihum Festlegung des Zieles des eigenen Angriffs.

Die persönlichen Konfliktfolgen sind deutlicher:
\begin{itemize}
\item alle Konfliktpunkte verloren = kritischer Misserfolg: Verlust eines Vorteilspunktes oder Gewinn eines Nachteilspunktes
\item mehr als die Hälfte verloren = Misserfolg
\item höchstens die Hälfte verloren = Erfolg
\item alle Konfliktpunkte übrig = kritischer Erfolg: Gewinn eines Vorteilspunktes oder Verlust eines Nachteilspunktes
\end{itemize}

Muss der Spielleiter mehrere Hauptkonfliktgegener lenken, so erzählt er trotzdem pro Runde nur einmal. Der erzielte Erzählwert zählt dann jeweils für alle Gegner.

\section[Schaden, Heilung und Tod]{Schaden, Heilung und Tod (ab Seite~\pageref{Ch:SchadenHeilungUndTod})}

Körperlicher und geistiger Schaden werden getrennt verrechnet; der Schaden wird nicht von der Lebens- bzw. Willenskraft abgezogen, sondern als Summe notiert. Er kann auch die Kräfte übersteigen.
\begin{tabular}[C]{ll}
	\bf Schaden \boldmath  & \bf Malus \\
	$\ge$ 1/2 Kraft & 3 \\
	$\ge$ 3/4 Kraft & 6 \\
	$\ge$ Kraft  & --- \\
\end{tabular}

\begin{description}
\item[Natürliche Heilung:] 1 körperlichen Schaden pro Tag, 1 geistigen Schaden pro Stunde Ruhe
\item[Erste Hilfe:] Kurzkonflikt zur Heilung eines frischen Schadenspunktes, bei kritischem Misserfolg 1 zusätzlicher Schaden. Dauer: 5-10 Minuten, nur ein Versuch pro Patient
\item[Langzeitbehandlung:] (Wunden) einmal pro Tag ein Kurzkonflikt, heilt alle 12 Stunden einen Punkt, bei kritischem Misserfolg keine Heilung für den Tag
\item[Aufbauendes Gespräch:] einstündiges Gespräch, halbiert den geistigen Schaden, bei kritischem Misserfolg keine Heilung für die Stunde
\item[Magie:] 1 Schaden heilen =  1 Astralpunkt
\end{description}

Der Tod tritt nur ein, wenn das Leben des Charakters in einem Konflikt auf dem Spiel steht.


\section[Gegenstände]{Gegenstände (ab Seite~\pageref{Ch:Gegenstaende})}
Geld hat keine spieltechnischen Auswirkungen. Auswirkungen von Gegenständen werden wie Talente mit Steigerungen gekauft:

\begin{tabular}[C]{rll}
  Konfliktgegenstand & 1 Bonuswürfel & 2 Steigerungen \\
                     & 2 Bonuswürfel & 6 Steigerungen \\
                     & 3 Bonuswürfel & 12 Steigerungen \\[\medskipamount]
  Rüstung            & +1 Bonus & 1 Steigerung \\
                     & +2 Bonus & 3 Steigerungen \\
                     & +3 Bonus & 6 Steigerungen \\
                     & +4 Bonus & 10 Steigerungen \\
                     & +5 Bonus & 15 Steigerungen \\
                     & +6 Bonus & 21 Steigerungen \\
\end{tabular}

Waffen geben je nach Typ noch evtl. einen automatischen Bonuswürfel/Erfolg:
\begin{itemize}
\item Verwendung mit einem Kampf-Spezialtalent (außer Lanzenreiten, Belagerungswaffen, Blasrohr): 1 Bonuswürfel
\item Lanzenreiten: 1 automatischer Offensiverfolg (bzw. im Kurzkonflikt 2 Bonuswürfel)
\item Belagerungswaffen: keine automatischen Erfolge oder Bonuswürfel
\item Blasrohr: Je nach Gifteinsatz
\end{itemize}


\section[Magie]{Magie (ab Seite~\pageref{Ch:Magie})}
Magie im freien Spiel ist kostenlos, Magie in Konflikten und zur Heilung (siehe dort) kostet AsP. Kostentabelle:
  \begin{tabular}[C]{|rlrl|}
  \hline
    +0~Würfel & 1~AsP & je 2 Würfel & \\
    +1~W"urfel & 2~AsP & weitergeben & +1~AsP \\ 
    +2~W"urfel & 4~AsP & & \\
    +3~W"urfel & 6~AsP &--1~Würfel & +3~AsP\\
    +4~W"urfel & 9~AsP & --2~Würfel & +9~AsP\\ 
    +5~W"urfel & 12~AsP &--3~Würfel & +15~AsP\\
    +6~W"urfel & 16~AsP &--4~Würfel & +24~AsP \\
    +7~W"urfel & 20~AsP &--5~Würfel & +33~AsP \\
    +8~W"urfel & 25~AsP && \\
    +9~W"urfel & 30~AsP && \\
  \hline
  \end{tabular}
  
\section[Geweihte]{Geweihte (ab Seite~\pageref{Ch:Geweihte})}
\begin{itemize}
	\item Mirakelprobe: Literurgien (KL/IN/CH)
	\item Karmapunkte (KaP): Karma (MU/IN/CH)
	\item Regeneration von Karmapunkten: Etwas für den Gott tun gibt 2 KaP, maximal einmal pro Tag
\end{itemize}

Vier mögliche Effekte:
\begin{description}
\item[Segen] (separater Kurzkonflikt) Wiederholung von Würfelwürfen für einen anderen Charakter. Dieser darf nicht bereits gesegnet sein. Verfällt, sobald sich der Charakter ungöttlich verhält.

\emph{Kosten:} Eine Wiederholung = 2~KaP, zwei Wiederholungen = 4~KaP, drei Wiederholungen = 6~KaP, vier Wiederholungen = 9~KaP usw. (dasselbe wie die AsP-Kosten für zusätzliche Würfel)

\item[Schutz] (separater Kurzkonflikt) Die Hälfte (aufgerundet) des erlittenen KP geht auf den Geweihten als KaP-Verlust über. Beendet auf Wunsch, falls KP auf 0, falls KaP auf 0 oder falls einer etwas ungöttliches macht.

\emph{Kosten:} Das doppelte der verhinderten KP.

\item[Stoßgebet] (innerhalb eines Konfliktes) Kauf von zusätzlichen automatischen Erfolgen.

\emph{Kosten:} Wie Zusatzwürfel der Zauberkundigen.

\item[Wunder] (innerhalb eines Konfliktes) Bewirkt ein Wunder; zieht direkt KP ab.

\emph{Kosten:} Gesamte restliche KaP plus evtl. Erzählmarken. Pro drei KaP verliert ein anderer Konfliktteilneher 1~KP; pro eingesetzter Erzählmarke 2~KP.
\end{description}


