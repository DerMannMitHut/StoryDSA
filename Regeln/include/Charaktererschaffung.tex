\chapter{Charaktererschaffung}\label{Ch:Charaktererschaffung}\index{Charaktererschaffung}
\lettrine{D}{ie} Charaktererschaffung ist wesentlich einfacher und schneller als bei DSA\,4. Das liegt vor allem an den entschlackten Regeln und den wenigen Ausnahmen. Dennoch sollte man einige Zeit zur Charaktererstellung einplanen und sorgfältig vorgehen, da das Spiel nicht für nur eine einzige Sitzung gedacht ist. Es bietet sich beispielsweise an, in der ersten Spielsitzung die Charaktere gemeinsam zu erschaffen und dann in der zweiten Spielsitzung mit dem ersten Abenteuer zu beginnen.

Eventuell wird ein solcher Charakter über lange Zeit fortgeführt -- vor allem auch die Menge an AP, die für den Aufstieg bis zur 21. Stufe benötigt wird, ist nicht zu unterschätzen und kann einige Jahre Spielzeit in Anspruch nehmen.

Bei der Charaktererschaffung ist es mehr als nur empfehlenswert, wenn alle Spieler die Charaktere zusammen entwerfen. Es sollte aber nicht so sein, dass sich alle treffen und dann trotzdem jeder für sich seinen Charakter entwirft. Vielmehr sollten sich die Spieler gegenseitig mit Ideen unterstützen und auch sagen, was sie für keine so gute Idee halten. Dabei sollten die Spieler immer darauf achten, dass ihre Charaktere auch zusammen passen. Sie sollen ja schließlich gemeinsam auf Abenteuer ausziehen und sich nicht gegenseitig bekämpfen.

Dabei können kleinere Konflikte natürlich erwünscht sein und das Spiel interessanter werden lassen. Aber gerade wenn Charaktere mit sehr unterschiedlichen Weltansichten in einer Gruppe zusammen gespielt werden sollen, muss den Spielern von Anfang an klar sein, dass sie die Charaktere so spielen müssen, dass diese auch mal Fünfe gerade sein lassen und über ihre eigenen Ideen und Prinzipien hinwegsehen, um das Wohl und den Zusammenhalt der Gruppe nicht zu gefährden.

Ab Seite~\pageref{Ch:Beispielcharaktere} sind einige frisch erschaffene Charaktere als Beispiele aufgelistet. Diese können einfach so zum Spielen benutzt werden oder als Vorlage dienen. Für Spieler, die noch keine Erfahrung mit \StoryDSA haben, lohnt sich auf jeden Fall vor der ersten Charaktererschaffung ein Blick in diese Seiten, um einen Eindruck zu gewinnen, wie typischerweise frisch gebackene Abenteurer aussehen.

\begin{optional}
\section{Optional: Konfliktreiche Charaktergruppe}

Bei der Charaktererschaffung kann auch abgesprochen werden, dass die Charaktere nicht wie üblich als eine Gruppe zusammenarbeiten, sondern dass sie entgegengesetzte Ziele verfolgen und bereit sind, diese auch gegeneinander durchsetzen. So entsteht der größte Teil der Geschichte aus der Interaktion der Charaktere. Diese Spielweise hat dann nicht mehr viel mit dem klassischen DSA gemeinsam: Als Abenteurergruppe in die Welt zu ziehen um Abenteuer zu erleben wird mit solchen Charakteren fast sicher zu einem frustrierenden Spiel führen, da die Kontroversen innerhalb der Gruppe das gemeinsame Abenteuer immer überschatten. Die Gruppe sollte sich klar für oder gegen gemeinsame Abenteuer entscheiden.
\end{optional}



\section{Rasse, Kultur und Profession}\index{Rasse}\index{Kultur}\index{Profession}
Wie üblich muss der Spieler Rasse, Kultur und Profession wählen. Außer gewisse Mindestwerte für die Eigenschaften und ein paar vorgeschriebene Vor- und Nachteile hat die Wahl keine Auswirkung für die Spielwerte.





\section{Hintergrundgeschichte}\index{Hintergrundgeschichte}
Vor der Festlegung von den Spielwerten sollte sich der Spieler darüber klar werden, was er für einen Charakter erschaffen möchte. Jeder Spieler muss in maximal fünfzig Worten eine Kurzbeschreibung seines Charakters abliefern. Enthalten sein sollte auf jeden Fall der Name des Helden, der nicht mitgezählt werden muss.

Für jeden Spielabend, den der Held überstanden hat, soll die Hintergrundgeschichte um maximal fünfzehn Worte verlängert werden. Dabei dürfen neue Sätze eingefügt, aber auch alte verlängert werden. Nicht verbrauchte Worte verfallen nicht.

\begin{design}
\subsubsection{Designanmerkung: Hintergrundgeschichte}
Sinn ist es, die ganze Sache auf ein paar wesentliche Sätze einzuschränken. Klar möchten viele Spieler mehr schreiben, es soll auch niemanden dran gehindert werden, eine zusätzliche Charaktergeschichte zu machen. Aber die Erfahrung zeight, dass eine so verkürzte Geschichte dazu zwingt, sich auf das Wesentliche zu konzentrieren und sich entscheiden müssen, was jetzt Wesentlich ist. Umgekehrt ist auch niemand mit 50 Worten überfordert, so dass sich auch niemand darum drücken wird.
\end{design}



\section{Fragen an den Spieler}
Jeder Spieler beantwortet folgende Fragen mit ein paar Worten:
\begin{enumerate}
  \item Warum macht es dir Spaß, diesen Charakter zu spielen?

  \item Was ist das wichtigste Wesen im Leben deines Charakters und warum?

  \item Nenne zwei oder drei Leidenschaften deines Charakters.

  \item Nenne zwei oder drei Überzeugungen oder Prinzipien deines Charakters.
\end{enumerate}

Um die Beantwortung etwas zu erleichtern, hier ein paar Anregungen und Hilfen:
\begin{enumerate}
  \item Es gibt verschiedene Gründe, warum ein Spieler einen Charakter spielt. Eine wichtiges Kriterium ist die Suche nach einer unbesetzten Nische (``Uns fehlt noch ein Dieb.''). Darüberhinaus ist natürlich der allgemeine Spaß am Spiel wichtig, aber auch das steht nicht zur Debatte. Hier sollte der Spieler die Frage beantworten, warum möchte der CS gerade diesen Charakter spielen? Ist es vielleicht die Fähigkeit zur Magie? Eine Vorliebe für Elfen? Möchte er mal so richtig den Gegnern einen in den Arsch treten? Oder ist es das Geheimnisvolle, das seinen Charakter umgibt?

  \item Das Naheliegenste ist hier sicherlich ein Blutsverwandter oder eine Liebe, vielleicht der Ehepartner. Aber auch Haustiere, Freunde oder verstorbene Personen sind hier als Antwort erlaubt. Allerdings ist `wichtig' ja nicht unbedingt gleich `beliebt', d.\,h. ein ewiger Rivale, eine Respektsperson oder ein Feind können genauso wichtige Wesen sein. Hier sollte aber auf jeden Fall ein konkretes Wesen bezeichnet werden.

  \item Leidenschaften können in vier Bereiche unterteilt werden: Verpflichtungen, Liebe, Wut und Angst. Leidenschaften müssen nicht rational begründbar sein, d.\,h. Leidenschaften können gegen die Intuition oder Logik stehen.

  Eine Verpflichtung ist etwas, von dem der Charakter überzeugt ist, dass er entsprechend handeln muss. Dabei muss er diese Verpflichtung nicht mögen, doch hier stellt der Charakter die Verpflichtung über seine Emotionen. Beispiele sind Schwüre, die Verbundenheit mit der Familie oder auch die Ehre eines Kriegers.

  Liebe dagegen ist eine starke Emotionale Bindung. Er handelt selbstlos um seine Liebe zu verteidigen. Das Objekt der Liebe muss nicht eine andere Person sein, der Charakter kann auch sein Heimatdorf, seine Religion oder auch seine Kriegerehre lieben. Auch ist es nicht nötig, dass seine Liebe erwiedert wird.
  
  Wut endet oft in einer unkontrollierten Handlung, die sich gegen den Wutauslöser richtet. Wut kann alles mögliche auslösen: Handlungen gegen bestimmte Überzeugungen des Charakters, Leute, die nicht grüßen, usw.

  Angst kommt in verschiedenen Aspekten daher. Ein paar Beispiele sind Hilflosigkeit (z.\,B. Gefangennahme, Großbrand, Krieg), Isolation (z.\,B. eingesperrt sein, allein unter Fremden), Gewalt (z.\,B. Folter, Hunde, Straßenräuber), Übernatürliches (z.\,B. Beherrschungsmagie, Dämonen, Zorn der Götter) oder auch Identitätsverlust (z.\,B. badoc).

  \item Charaktere können beispielsweise von ihrer Religion, persönlichen Prinzipien, der vorherrschenden Weltordnung usw. überzeugt sein. Sie nehmen diese Überzeugung als gegeben und unveränderbar hin und handeln entsprechend. Einen von etwas überzeugten Charakter umzustimmen dürfte schwierig werden; selbst handfeste Gegenbeweise versucht er zunächt innerhalb seines Weltbildes zu erklären. Genau wie Leidenschaften müssen Überzeugungen nicht rational begründet werden.
\end{enumerate}

\begin{design}
\subsubsection{Designanmerkung: Fragen an den Spieler}
Die Fragen an den Spieler dienen dazu, dem Spielleiter mitzuteilen, welche Art von Geschichte ein Charakterspieler mit seinem Charakter erleben möchte. Zusammen mit der Hintergrundgeschichte sind sie für den Spielleiter die größte Hilfe, um ein interessantes Abenteuer zu gestalten. Daher sollte jeder Spieler diese Dinge mit Bedacht angeben. Zudem sollten die Antworten auch von Zeit zu Zeit überprüft werden, da sich der Geschmack des Spielers im Laufe der Zeit ändert.

Dass sich mit einer Neubeantwortung der Fragen auch der Charakter ändern kann, ist nicht schlimm. Charaktere leben und erleben Dinge, andere Personen werden wichtiger, durch die Erfahrung kann Wut oder Furcht durch andere Dinge ausgelöst werden. Auch ehemalige Überzeugungen können sich als falsch herausgestellt haben oder können hinter neuen Überzeugungen zurückfallen und unwichtiger werden.
\end{design}

Der Unterschied von Leidenschaften und Überzeugungen ist gerade im Bereich Verpflichtungen etwas fließend. Der Unterschied ist, dass Leidenschaften die Gefühle des Charakters ansprechen, Überzeugungen nicht. Letzteren ist sich der Charakter bewusst: Er weiß einfach mit Bestimmtheit, dass sich einige Dinge auf der Welt so und nicht anders verhalten. Bei den Leidenschaften wie Angst oder Wut kann es sogar sein, dass sie bislang noch nie in Erscheinung getreten sind.

\subsection{Beispiel}\label{BeispielCharaktere}
Eine Gruppe besteht aus folgenden vier Charakteren:

\subsubsection{Tharam Löwenprank, Rondrapriester}
Die Gruppe profitiert von diesem Charakter, weil er in sich sowohl Kämpferpotential als auch gesellschaftliche Fähigkeiten vereint. Seine Stellung als Geweihter öffnet ihm Zugang zu manchen verbotenen Orten und gibt ihm einige Sonderrechte.
 
\begin{enumerate}
  \item Warum macht es dir Spaß, diesen Charakter zu spielen?

Tharam ist ein guter Kämpfer mit dem Rondrakamm, und es ist cool, wie er sich mit wenig Rüstung und nur mit diesem Zweihänder einem Duell stellt. Ich mag es auch, wie er mit seiner Beharrlichkeit/Sturheit auch gesellschaftliche Probleme durchsteht.

  \item Was ist das wichtigste Wesen im Leben deines Charakters und warum?

Seine Mutter, die Oberin des Rondra-Tempels von Trallop, in dem Tharam geboren wurde und aufgewachsen ist. Sie trug ihm auf, er solle einige Jahre auf große Reise gehen, Erfahrung sammeln, seinen Glauben auf die Probe stellen. An den Berichten von seinen Taten wird sie messen, ob Tharam ein würdiger Nachfolger für sie ist.

  \item Nenne zwei oder drei Leidenschaften deines Charakters.

  \begin{itemize}
    \item  Er führt das Leben eines Rondra-Priesters, inklusive der Pflichten ggü. seiner Kirche.

    \item  Tharam schwärmt für seine Abenteuer-Gefährtin Ziliane (s.u.).

    \item Er kann die Aufforderung zu einem Duell nicht abschlagen. (Bodowius (s.u.) hat ihm schon etliche Male erzählt, wie gefährlich dieses Gelübde ist)
  \end{itemize}

  \item Nenne zwei oder drei Überzeugungen oder Prinzipien deines Charakters.

    \begin{itemize}
      \item Wer auf die Zwölfgötter vertraut, dem wird es im Leben nicht schlecht ergehen.
      \item Seine Mutter sagte ihm, er solle nicht lügen, denn das ist nicht rondra-gefällig.
    \end{itemize}

\end{enumerate}



\subsubsection{Ziliane Gassennebel, Streunerin}
Sie ergänzt die Runde mit ihren Fähigkeiten in den Diebestalenten. Statt besonders hoher Kampftalente setzt sie mehr auf Gassenwissen, Verhandeln und Überzeugen. Ihre prakmatische Einstellung holt die anderen Gruppenmitglieder immer wieder zurück auf den Boden. Tharams Annäherungsversuche findet sie irgendwie niedlich, nimmt diese aber nicht ernst.

\begin{enumerate}
  \item  Warum macht es dir Spaß, diesen Charakter zu spielen? 

Ich liebe es mit Ziliane SLCs an die Wand zu reden und so zu bekommen, was sie erreichen will. Zudem macht es mir Spass, dass Ziliane Geld von SLCs stiehlt, ohne dass die anderen SCs dieses bemerken.

  \item Was ist das wichtigste Wesen im Leben deines Charakters und warum?

Sie ist sich selbst am wichtigsten. Doch danach kommt schon die Gruppe. Sie lernte früh, als sie noch auf den Strassen Havenas lebte, selbstständig zu sein. Doch seit sie die Stadt mit der Gruppe verließ, ist diese kleine Gemeinschaft ihre neue Heimat.
 
  \item Nenne zwei oder drei Leidenschaften deines Charakters.

  \begin{itemize}
    \item Leuten das Wort im Mund herumdrehen und sie an die Wand reden.
    \item Stehlen wie eine Elster (Denidara (s.u.) hat sie allerdings schonmal beim Klauen beobachtet; Bodowius (s.u.) ist recht klug, und auch wenn er noch nie was sah, er kann 1 und 1 zusammenziehen).
    \item Sie hasst es alleine zu sein, sie braucht Freunde und Gefährten um sich. 
  \end{itemize}

  \item Nenne zwei oder drei Überzeugungen oder Prinzipien deines Charakters.

  \begin{itemize}
    \item Was man gefunden hat, das darf man auch behalten!
    \item Lügen ist nichts schlimmes. Jeder lügt mal ab und zu!
  \end{itemize}
\end{enumerate}

\subsubsection{Bodowius Feuerstab, 
Magier des Seminars der elfischen Verständigung und natürlichen Heilung zu Donnerbach}
Er hilft der Abenteuererrunde mit seiner Magie, seiner Heilkunst und seinem breiten Hintergrundwissen in unterschiedlichen Bereichen. Zudem beherrscht er diverse Fremdsprachen, u.a. das Elfische.

\begin{enumerate}
  \item Warum macht es dir Spaß, diesen Charakter zu spielen? 

Ich mag es mit ihm die Geheimnisse und Mysterien Aventuriens zu entdecken, mit ihm dorthin zu gehen, wo noch nie ein Mensch seinen Fuss gesetzt hat. Ausserdem stehe ich voll auf das aventurische Magie-System, ich hole gerne das Letzte aus seinen Sprüchen raus.

  \item Was ist das wichtigste Wesen im Leben deines Charakters und warum? 

Seine große Liebe ist Galinde, die noch in Donnerbach wohnt. Sie ist durch eine Vergiftung erblindet. Er sucht ein Heilmittel für ihr Leiden, dann will er zurückkehren und sie endlich heiraten.

  \item Nenne zwei oder drei Leidenschaften deines Charakters. 

  \begin{itemize}
    \item Wenn Unschuldige und Wehrlose verletzt werden treibt das ihn zur Raserei.
    \item Er will niemanden verletzten. Lieber blendet, verwirrt oder lähmt er Gegner.
    \item Geheimnisse entdecken, Unbekanntes erforschen (hier unterstützt ihn Denidara (s.u.) sehr häufig)
  \end{itemize}

  \item Nenne zwei oder drei Überzeugungen oder Prinzipien deines Charakters.

  \begin{itemize}
    \item Schwarze Magie anzuwenden ist eine unverzeihliche Sünde 
    \item Wissen ist Macht (Ziliane ist manchmal der Meinung, Bodowius würde seine Nase all zu in Bücher stecken anstatt zu handeln)
    \item Die Feder ist mächtiger als das Schwert (das versucht er Tharam immer zu beweisen)
  \end{itemize}

\end{enumerate}

\subsubsection{Denidara Silberlaub, Auelfen-Bognerin}
Sie ergänzt die Runde von Helden mit ihren Fähigkeiten als Waldläuferin und als Bogenschützin. Und ein Nicht-Mensch muss ja immer dabei sein, um eine aventurische Gruppe abzurunden. Zudem beherrscht sie ein kleines Repoirtoire an aussergewöhnlicher Magie, das in der Menschenwelt so gut wie unbekannt ist.

\begin{enumerate}
  \item Warum macht es dir Spaß, diesen Charakter zu spielen?

Ich mag das Exotische an ihr, dass sie aus einer völlig fremden Kultur kommt, finde ich sehr spannend, und wie sie auf die fremden Länder, vor allem das Mittelreich, reagieren wird.
 
  \item Was ist das wichtigste Wesen im Leben deines Charakters und warum?

Das wird dann wohl ihr Bruder sein, der in den heimatlichen Auwäldern auf sie wartet. Es brach ihm fast das Herz, als sie in einem leichten ``badoc'' Anfall von Abenteuerlust nach Donnerbach ging, wo sie Bodowius traf. Denidara vermeint selbst heute noch, fern der Heimat, die traurigen, abendlichen Gesänge ihres Bruders zu hören.
 
  \item Nenne zwei oder drei Leidenschaften deines Charakters. 

  \begin{itemize}
    \item Sie liebt die unberührte Natur, im Wald fühlt sie zuhause. (Ziliane findet Wälder ja völlig uninteressant, und Denidara wird nicht müde sie vom Gegenteil zu überzeugen)
    \item Vor den große, überbevölkerten, schmutzigen Metropolen des Mittelreichs graust es ihr furchtbar.
  \end{itemize}

  \item Nenne zwei oder drei Üerzeugungen oder Prinzipien deines Charakters.

  \begin{itemize}
    \item Die Schändung der Natur ist ein furchtbares Verbrechen.
    \item Es gibt die Götter zwar, aber sie sind nicht wichtig (sie diskutiert häufig mit Tharim).
    \item Eine Bitte um Hilfe schlägt sie nie ab. Sie ist überzeugt, dass Selbstlosigkeit sich später immer auszahlt (und selbst wenn es Jahrhunderte dauern sollte).
  \end{itemize}
\end{enumerate}







\section{Vor- und Nachteile}\label{VorUndNachteile}\index{Vorteil}\index{Nachteil}
Vor- und Nachteile sind Fähigkeiten und Merkmale, die nicht ohne weiteres erlernt werden können sondern durch die Vorgeschichte oder besondere Ereignisse begründet sind. Im Spiel können sie daher nicht durch Abenteuerpunkte gelernt werden, sondern werden durch den Spielleiter als Konfliktfolge (oder als Abenteuerfolge) verliehen oder gestrichen.

Jeder Charakter bekommt bei der Erschaffung 4~Vorteilspunkte, die er mindestens für Vorteile ausgeben muss. Darüber hinaus kann er für bis zu 10 Punkte weitere Vorteile wählen, die er durch Nachteile von mindestens demselben Wert ausgleichen muss. Insgesamt kann ein Charakter so Vorteile im Wert von 14 Punkten bekommen.

Grundsätzlich gibt es verschiedene Typen von Vor- und Nachteilen:
\begin{description}
  \item[Vorteil gibt Bonus auf bestimmte Talente:] Diese Bonuspunkte werden einfach mit dem Talentewert verrechnet. Insgesamt steigen dadurch die Talentwerte nicht über den für die Klasse erlaubten Wert, bei Anfängercharakteren also 6 (später bis zu 12).  Verliert der Held den Vorteil, müssen die Bonuspunkte wieder abgezogen werden. Die benötigen Vorteilspunkte sind gerade die Anzahl der Bonuspunkte.  Muss das Talent noch aktiviert werden, so kostet dies einen weiteren Vorteilspunkt.
  
  \item[Vorteil gibt Bonuswürfel in bestimmten Situationen:] Sollte der Vorteil in der Situation nützlich sein, gibt er Bonuswürfel im Konflikt. Die benötigten Vorteilspunkte entsprechen den Wissenstalenten: Ein Bonuswürfel kostet 2~Vorteilspunkte, zwei Bonuswürfel kosten 6~Vorteilspunkte und drei Bonuswürfel 12.

  \item[Vorteil gibt Rüstungsschutz:] Wie eine echte (körperliche oder geistige) Rüstung. 1~RS: 1~Punkt, 2~RS: 3~Punkte, 3~RS: 6~Punkte, 4~RS: 10~Punkte 
  
  \item[Sonstige Vorteile:]  Vorteile, die sich nicht in obiges Schema pressen lassen, brauchen entweder Sonderregelungen oder sollten nicht für Helden gewählt werden dürfen. Einige Beispiele für Vorteile, darunter auch für sonstige Vorteile, sind weiter unten aufgelistet.

  \item[Nachteil senkt bestimmte Talente:] Diese Maluspunkte werden einfach mit dem Talentwert verrechnet  -- trotzdem gilt für Talentwerte die Untergrenze von 0. Um diese nicht zu unterschreiten, muss der Charakterspieler Maluspunkte durch Steigern des Wertes ausgleichen. Verliert der Charakter im Laufe des Spiels des Nachteils, so werden die Maluspunkte wieder addiert. Nachteile dieser Art geben einen Vorteilspunkt pro Maluspunkte. Auch Spezialtalente können durch Nachteile gesenkt werden; diese müssen aber dann vom Spieler auch aktiviert werden.

  \item[Nachteil gibt Maluswürfel:]  In bestimmten Situationen bekommt der Held Nachteile. Dafür werden dann eine bestimmte Anzahl an Würfeln nach dem Festlegen des Erzählwertes festgelegt. Der Wert für die Maluswürfel:  1~Maluswürfel: 1~Punkte; 2~Maluswürfel: 3~Punkte; 3~Maluswürfel: 6~Punkte.

  \item[Nachteil senkt Rüstungsschutz:] Abzug vom Rüstungsschutz; auch unter 0. --1~RS: 1~Punkt, --2~RS: 2~Punkte, --3~RS: 3~Punkte (weiter senken geht nicht). Achtung: Hat die Spielfigur negativen Rüstungsschutz, so bekommt sie bei einem Treffer entsprechend viel Bonus-Schaden!

  \item[Sonstige Nachteile:] Je nach Definition; geben je nach Beschreibung Vorteilspunkte.
\end{description}

Grundsätzlich gibt es keine Einschränkung, wie genau Vor- und Nachteile aussehen. Die folgende Liste ist an die original DSA4-Liste der Vor- und Nachteile angelehnt und gibt einige Beispiele, wie diese umgesetzt werden können\footnote{Zunächst soll es bei dieser (unvollständigen) Liste bleiben; wenn jemand viel Lust hat, kann er gerne weitere Vorteile genauer ausarbeiten und mir schicken.}.

\subsubsection{Talentvorteile}
\begin{description}
  \item[Akademische Ausbildung:] Grundwissen in 2--4 Wissenstalenten und Etikette; Kosten: 2 Wissenstalente: 6~Vorteilspunkte, 3 Wissenstalente 8~Vorteilspunkte, 4 Wissenstalente 10~Vorteilspunkte.
  \item[Balance:] Bonuspunkte auf Körperbeherrschung; der erste Punkt in Balance aktiviert Körperbeherrschung
  \item[Schlangenmensch:]  Je 1 Bonuspunkt auf Waffenloser Kampf, Körperbeherrschung, Schleichen, Sich verstecken, Fesseln/Entfesseln; Kosten: 5~Vorteilspunkte
\end{description}

\subsubsection{Bonuswürfel-Vorteile}
\begin{description}
  \item[Ausdauernd:] 1 Zusatzwürfel, wenn körperliche Ausdauer gefragt ist; Kosten: 2~Vorteilspunkte
  \item[Kampfrausch:] 1 Zusatzwürfel; allerdings bei Aktivierung maximal einen defensiven Würfel benutzen; Kosten: 2~Vorteilspunkte
  \item[Dämmerungssicht:] 1 Zusatzwürfel im Halbdunkeln, wenn es auf Sicht ankommt; Kosten: 2~Vorteilspunkte
  \item[Kriegerbrief:] 1 Zusatzwürfel bei Verhandlungen mit Gesetzestreuen; Kosten: 2~Vorteilspunkte
  \item[Gefahreninstinkt:] 1 bzw. 2 Zusatzwürfel bei bislang nicht erkannten Gefahr; Kosten: 2 bzw. 6~Vorteilspunkte
  \item[Zwergennase:] 1 Zusatzwürfel beim Suchen von Geheimtüren und Schätzen; Kosten: 2~Vorteilspunkte
\end{description}

\subsubsection{Sonstige Vorteile}
\begin{description}
  \item[Prophezeien:] bei Bedarf kann der SL eine Vision geben; Kosten: 2~Vorteilspunkte
  \item[Vom Schicksal begünstigt:] Kosten: 2~Vorteilspunkte
\end{description}
\begin{design}
\subsubsection{Designanmerkung: Wieso sind Gaben keine Talente?}
 Gaben sind bei DSA4 regeltechnisch nur Zusatztalente mit besonderen Anwendungen. In StoryDSA wäre das sicherlich auch möglich, allerdings handelt es sich dabei meist um Unterstüzungs-Talente. So sollte z.B. Gefahreninstinkt einen Vorteil geben, wenn die Schwierigkeit plötzlich auftritt, ohne dass sie erkannt werden konnte; bei einem Angriff von hinten merkt der Held dies und kann ihn noch rechtzeitig abwehren. Damit müssten diese Vorteile analog zu Wissenstalenten gehandhabt werden. Daher erscheint es mir sinnvoll, diese einfach als Würfelbonus zu modellieren.

Prophezeien dagegen unterstützt nicht andere Fertigkeiten, sondern ist eine eigenständige Fähigkeit. Darüberhinaus wird in DSA4 auch nur verdeckt darauf gewürfelt, d.\,h. der SL kann frei entscheiden, ob er dem Spieler eine Vision schickt oder nicht. Das sollte er auch besser tun, denn nicht immer ist eine Vision für die Spieler hilfreich und für die Story sinnvoll. Als sonstiger Vorteil kann der Spielleiter gezielt Visionen vorbereiten und den Spielern als SL-Erzählphase, Freies Spiel oder sogar als Konflikt präsentieren.
\end{design}

\begin{optional}
\section{Optional: Prophezeien-Regel}
Der oben angegebene Vorteil Prophezeien ist sehr schwammig und baut darauf, dass der Spielleiter entsprechende Prophezeiungen vorbereitet. Die hier vorgeschlagene optionale Prophezeien-Regeln dienen dazu, von der Spielleiter-Willkür abzurücken und den Vorteil ``Prophezeien'' verlässlicher zu gestalten und letztendlich einen Bonuswürfel-Vorteil daraus zu machen.

Der Vorteil kann zu jeder Zeit vom Spieler aktiviert werden, indem er seine Spielfigur entsprechende Handlungen durchführen lässt wie Kaffeesatz-Lesen, Sterndeutung oder Trancezustände. Dann sagt er an, zu welchem Bereich er sich eine Prophezeiung erhofft. Der Bereich sollte räumlich und zeitlich begrenzt sein, also z.B. ``Was erwartet uns in der verlassenen Zwergenstadt, wenn wir sie morgen betreten?'' oder ``Wie wird die Kaiserin auf unsere Vorschläge reagieren?''

Die Prohpezeiung selbst wird nicht ausformuliert, stattdessen erhält der Prophet einen Bonuswürfel für die Dauer eines Konfliktes, an dem er im vorher festgelegten Bereich teilnimmt. Er kann also die Prophezeiung für seine Ziele gewinnbringend einsetzen. Der Inhalt der Prophezeiung wird also erst nachträglich genau festgelegt.

Eine Prophezeiung muss sich auf die unmittelbare Zukunft beziehen und gleichzeitig an einen bestimmten Ort bzw. an eine bestimmte Person gebunden sein. Vermeiden die Helden nach der Prophezeiung den Kontakt und wenden sich ersteinmal etwas anderem zu, so verfällt der Bonuswürfel. Auch kann nicht mehr als eine Prophezeiung gleichzeitig ``aktiv'' sein.
\end{optional}



\subsubsection{Talentnachteile}
\begin{description}
  \item[]
  \item[]
  \item[]
\end{description}

\subsubsection{Maluswürfel-Nachteile}
\begin{description}
  \item[Unangenehme Auffälligkeiten:] Ein oder zwei Maluswürfel durch schlechtes oder ungewöhnliches Aussehen in entsprechenden Konflikten (genaue Beschreibung bleibt dem Spieler überlassen; Beispiele finden sich unter den Nachteilen Albino, Sprachfehler, Unansehnliches oder Wiederwärtiges Aussehen, usw.); Gewinn: 2, 6 oder 12~Vorteilspunkte
  \item[Prinzipientreue:] Einen Maluswürfel, solange der Charakter gegen seine Prinzipien verstößt; Gewinn: 2~Vorteilspunkte
  \item[]
\end{description}

\subsubsection{Sonstige Nachteile}
\begin{description}
  \item[Gesucht:] 
  \item[]
  \item[]
\end{description}



\section{Eigenschaften}\index{Eigenschaft}
Eigenschaften sind grundlegende Fähigkeiten des Charakters, die kein spezielles Fachwissen voraussetzen. Jeweils drei Eigenschafen (oder zwei, davon dann aber eine doppelt) gehen in die Talentgesamtwerte bei Konflikten ein.

Zu Spielbeginn darf jeder Spieler 23~Punkte auf die 8~Eigenschaften seines Helden verteilen. Dabei sind Werte im Bereich $-1$ bis $3$ erlaubt; im Spielverlauf können die Eigenschaften bis maximal $4$ gesteigert werden. Dabei sind die Kosten abhängig von der Eigenschaft:

\begin{tabular}[C]{cccccccc}
  \bf MU\index{Mut}%
& \bf KL\index{Klugheit}%
& \bf IN\index{Intuition}%
& \bf CH\index{Charisma}%
& \bf GE\index{Gewandtheit}%
& \bf FF\index{Fingerfertigkeit}%
& \bf KO\index{Konstitution}%
& \bf KK\index{Körperkraft}%
\\
3 & 2 & 5 & 2 & 3 & 4 & 1 & 3 \\
\end{tabular}

Durch eine $-1$ in der entsprechenden Eigenschaft bekommt man natürlich entsprechend viele Punkte gutgeschrieben.

Bei der Charaktererschaffung muss jeder Spieler eine Eigenschaft auf 3 und mindestens zwei weitere auf 2 oder 3 setzen. Die restlichen Punkte werden dann auf die übrigen Eigenschaften verteilt.

\begin{beispiel}
Eine beispielhafte Auswahl von 8 der insgesamt 8477 Wahlmöglichkeiten:

\begin{itemize}
\item MU:3 KL:0 IN:--1 CH:1 GE:1 FF:2 KO:3 KK:1
\item MU:3 KL:3 IN:--1 CH:1 GE:0 FF:1 KO:1 KK:2
\item MU:--1 KL:1 IN:3 CH:1 GE:1 FF:--1 KO:2 KK:2
\item MU:0 KL:3 IN:1 CH:3 GE:0 FF:0 KO:3 KK:1
\item MU:0 KL:0 IN:0 CH:3 GE:3 FF:2 KO:0 KK:0 
\item MU:1 KL:--1 IN:--1 CH:1 GE:2 FF:3 KO:1 KK:2
\item MU:2 KL:--1 IN:1 CH:--1 GE:1 FF:1 KO:3 KK:2
\item MU:--1 KL:1 IN:1 CH:--1 GE:2 FF:1 KO:2 KK:3 
\end{itemize}
\end{beispiel}
Mindest- oder Höchstwerte, die durch Rasse, Kultur oder Profession vorgegeben werden, müssen umgerechnet werden.

\begin{tabular}[C]{lccccc}
  \bfseries DSA4 & 8--9 & 10--11 & 12--13 & 14--15 & ab 16 \\
  \bfseries \StoryDSA & $-1$ & 0 & 1 & 2 & 3 \\
\end{tabular}

\begin{beispiel}
   \paragraph{Beispiel:}
Ein Fleischer (Handwerker) hat die Voraussetzung KK mindesten 12. Umgerechnet bedeutet das hier eine Voraussetzung von KK mindestens 1.
\end{beispiel}

In DSA4 vergebene Bonuspunkte für Eigenschaften werden hier zu Mindestwerten, Maluspunkte werden zu Höchstwerten. Dabei müssen zunächst alle Boni und Mali miteinander verrechnet werden.

\begin{tabular}[C]{lcccc}
  \bfseries DSA4 & $+1$ & $+2$ & $+3$ & ab $+4$ \\
  \bfseries \StoryDSA & mind. $0$ & mind. $1$ & mind. $2$ & mind. $3$ \\[\medskipamount]
  \bfseries DSA4 & $-1$ & $-2$ & $-3$ & ab $-4$ \\
  \bfseries \StoryDSA & max. $2$ & max. $1$ & max. $0$ & max. $-1$ \\
\end{tabular}

Sollten Boni und Mali Mindes- oder Höchstwerte ergeben, die den anderen Mindest- oder Höchstwerten widersprechen, so ist die gewünschte Rasse/Kultur/Profession-Kombination nicht möglich.

\begin{beispiel}
   \paragraph{Beispiel:}
   Orks haben unter anderem Eigenschaftmodifikation KL$-2$ und die Kultur Ferkina eine Modifikation von KL$-1$. Zusammen macht das dann also für einen Ork mit Kultur Ferkina KL$-3$, umgerechnet also KL höchstens $0$. Ein Schiffsbauer aber hat die Voraussetzungen KL~13 und IN~11, das ergibt also KL mindestens $1$ und IN mindestens $0$, daher ist die (zugegeben recht exotische Kombination) Ork/Ferkina/Schiffsbauer nicht erlaubt.
\end{beispiel}

Diese Mindes- und Höchstwerte gelten allerdings nur bei der Heldenerschaffung. Im weiteren Spielverlauf (z.\,B. beim Steigern des Charakters) gelten diese Schranken nicht mehr.


\begin{design}
\subsubsection{Designanmerkung: Eigenschaften}
Vielleicht wird sich manch einer fragen: Wozu überhaupt Eigenschaften? Es werden keine Proben darauf gewürfelt. In \StoryDSA gibt es zwei Gründe:
\begin{enumerate}
  \item Bessere Kompatibilität mit dem Original-DSA
  \item Spezialisierung der Charaktere, denn hohe Eigenschaftswerte regen an, auch aus diesem Bereich Talente auszuwählen
\end{enumerate}
Den ersten Punkt muss man nicht näher erläutern. Wo aber ist der Vorteil von spezialisierten Charakteren? Es bilden sich Nischen, die von den einzelnen Charakteren abgedeckt werden, so dass Spieler und Charaktere im Spiel aufeinander angewiesen sind und daher ganz natürlich zusammenarbeiten -- denn nur gemeinsam lassen sich die vielfältigen Bedrohungen überwinden.

Die verschiedenen Gewichtungen der Eigenschaften entstehen durch ihren Nutzen. So ist eine hohe Intuition viel nützlicher als eine hohe Konstitution, weswegen Intuition mehr wert ist also auch mehr kostet. Dadurch wird kein Spieler durch die Wahl seiner Eigenschaftswerte ernsthaft gegenüber den anderen Spielern benachteiligt.
\end{design}


\BN
\subsection{Eigenschaftsgrenzen}

Wie bereits beschrieben, wirken sich die DSA-Eigenschafts-Boni und Mali als Mindes- bzw. Höchstwerte aus. In folgenden Tabellen werden die Grenzen für die Eigenschaften zusammengefasst:

\subsubsection{Rasse}
\begin{tabular}{ll}
Mittelländer & --- \\
Tulamiden & --- \\
Thorwaler & MU$\ge 0$, KO$\ge 0$, KK$\ge 0$ \\
Norbarden & CH$\ge 0$ \\
Trollzacker & MU$\ge 0$, KL$\le 2$, KO$\ge 0$, KK$\ge 0$ \\
Waldmenschen & CH$\ge 0$, GE$\ge 0$, KO$\ge 0$, KK$\le 2$ \\
Utulus & GE$\ge 0$, KO$\ge 0$ \\
Auelfen & KL$\le 2$, IN$\ge 0$, GE$\ge 1$, KK$\le 2$ \\
Waldelfen & KL$\le 2$, IN$\ge 1$, GE$\ge 1$, KK$\le 2$ \\
Firnelfen & KL$\le 2$, IN$\ge 0$, GE$\ge 1$, KO$\ge 0$, KK$\le 2$ \\
Halbelfen & GE$\ge 0$, KK$\le 2$ \\
Erzzwerge & FF$\ge 0$, GE$\le 2$, KO$\ge 1$, KK$\ge 0$ \\
Hügelzwerge & FF$\ge 0$, GE$\le 2$, KO$\ge 1$, KK$\ge 0$ \\
Brillantzwerge & FF$\ge 0$, GE$\le 2$, KO$\ge 1$, KK$\ge 0$ \\
Ambosszwerge & FF$\ge 0$, GE$\ge 0$, KO$\ge 1$, KK$\ge 1$ \\
Wilde Zwerge & FF$\ge 0$, GE$\le 2$, KO$\ge 1$, KK$\ge 1$ \\
Orkmänner & MU$\ge 1$, KL$\le 1$, CH$\le 1$, FF$\le 2$, KO$\ge 1$, KK$\ge 1$ \\
Orkfrauen & KL$\le 1$, CH$\le 1$, FF$\le 2$, KO$\ge 0$, KK$\ge 0$ \\
Halborks & MU$\ge 0$, KL$\le 2$, CH$\le 1$, KO$\ge 0$, KK$\ge 0$ \\
Goblins & MU$\le 2$, KO$\le 1$, KK$\ge 1$, GE$\ge 1$, KK$\le 2$ \\
Achaz & MU$\le 2$,  IN$\ge 1$, FF$\ge 0$, GE$\ge 0$, KO$\ge 0$, KK$\le 2$ \\
Orkland-Achaz & MU$\le 2$, IN$\ge 1$, FF$\ge 0$, GE$\ge 0$, KK$\le 2$ \\
\end{tabular}

\subsubsection{Kultur}

Falls es für eine Eigenschaft sowohl eine Rassen- als auch eine Kulturgrenze gibt, so müssen diese miteinander verrechnet werden. Aus folgender Tabelle kann man das Ergebnis ablesen:

\begin{tabular}{c|cccc}
        & $\ge 0$ & $\ge 1$ & $\le 2$ & $\le 1$ \\\hline
$\ge 0$ & $\ge 1$ & $\ge 2$ & ---     & $\le 2$ \\
$\ge 1$ &         & $\ge 3$ & $\ge 0$ & ---     \\
$\le 2$ &         &         & $\le 1$ & $\le 0$ \\
$\le 1$ &         &         &         & $\le -1$ \\
\end{tabular}

In der folgenden Tabelle sind nur die Kulturen aufgeführt, für die Eigenschaftsgrenzen gelten.

\begin{tabular}{ll}
Amazonenburg & KO$\ge 0$ \\
Novadi & MU$\ge 0$ \\
Ferkina & MU$\ge 1$, KL$\le 2$, KO$\ge 0$ \\
Zahori & CH$\ge 0$ \\
Gjarlskerland & KO$\ge 0$ \\
Fjarninger & MU$\ge 0$, KL$\le 2$, KO$\ge 0$, KK$\ge 0$ \\
Maraskan & IN$\ge 0$ \\
Trollzacken & KO$\ge 0$ \\
Erzzwerge & KL$\ge 0$, IN$\le 2$ \\
Hügelzwerg & IN$\ge 0$ \\
Brobim & MU$\ge 0$, FF$\le 2$ \\
Orkland-Ork (Ergoch) & MU$\le 1$, KO$\ge 0$ \\
Orkland-Ork (Drasdech) & FF$\ge 0$ \\
Orkland-Ork (Khurkach) & MU$\ge 0$ \\
Orkland-Ork (Okwach) & MU$\ge 0$, KL$\ge 0$\\
Orkland-Ork (Tscharshai) & KL$\ge 0$\\
Orkland-Ork (Zholochai) & MU$\ge 0$\\
Svellttal-Besatzer (Ergoch) & MU$\le 1$, KO$\ge 0$\\
Svellttal-Besatzer (Drasdech) & FF$\ge 0$\\
Svellttal-Besatzer (Khurkach) & MU$\ge 0$\\
Svellttal-Besatzer (Okwach) & MU$\ge 0$, KL$\ge 0$\\
Goblinstamm & MU$\le 2$, KO$\ge 0$ \\
Goblinbande & MU$\le 2$ \\
Festumer Ghetto & MU$\le 2$, KL$\ge 0$ \\
\end{tabular}


\EN




\section{Talente}
  Talente werden in \DEF{Basistalente}\index{Basistalent}, \DEF{Spezialtalente}\index{Spezialtalent} und \DEF{Wissenstalente}\index{Wissenstalent} unterschieden. Auf die Basistalente hat jeder Charakter Zugriff; Spezialtalente dürfen nur dann benutzt werden, wenn der Held sich mit dem Thema längere Zeit beschäftigt hat (regeltechnisch bedeutet das, das diese Talente \DEF{aktiviert}\index{Talent!aktiviert} werden müssen, was eine Steigerung kostet). Die \DEF{Talentwerte}\index{Talentwert} geben an, wie viel sich ein Charakter mit dem Gebiet auseinandergesetzt hat. Sie bewegen sich diese im Bereich $0$ bis $12$. Wissenstalente dagegen werden deutlich grober modelliert: sie gibt es nur in den Abstufungen 0 (Allgemeinwissen), 1 (Grundwissen), 2 (Experte) und 3 (Koryphäe). Wissenstalente selber werden nicht geprobt; sie geben Bonuswürfel auf andere Talente, falls das Wissen bei einem Konflikt weiterhilft.

\begin{beispiel}
   \paragraph{Beispiel:}
So könnte beispielsweise Tierkunde im Kampf gegen Tiere genauso gut helfen wie bei  der Spurensuche oder in einer Diskussion mit einem Gelehrten, der dazu überredet werden soll, den Helden zu helfen und sich durch das Wissen seines Gegenübers beeindrucken lässt.
\end{beispiel}

\begin{design}
\subsubsection{Designanmerkung: Wissenstalente}

Wenn in einem Plot den Spielern eine wichtige Information fehlt, muss der Spielleiter dafür sorgen, dass die Helden diese bekommen. Oft soll diese Jagd auch noch spannend gestaltet werden, so dass der SL davon ausgeht, dass keiner der Charaktere dieses Wissen hat. Daher wird er in solchen Situationen den Spielern nicht erlauben, durch einen einfachen Wurf auf ein passendes Wissenstalent an die Information zu kommen. Soll keine Informationssuche betrieben werden, muss der SL dafür sorgen, dass die Helden irgendwie anders an die Infos kommen, d.\,h. auch hier ist ein Wurf eher unangebracht.

Also kann man eigentlich Wissenstalente nur für Informationen brauchen, die die Helden nicht oder nicht dringend benötigen. Das ist aber eine eher langweilige Anwendung, für die niemand einen Neben- oder Hauptkonflikt machen würde.

Wozu sind also Wissenstalente gut? Sie dienen dazu, in bestimmten Situationen einen Vorteil zu ziehen. Ein Archäologe erkennt einen bearbeiteten Stein auf einen Blick, ein Mensch, der eine fremde Sprache gut beherrscht, erntet bei den Muttersprachlern Anerkennung. D.\,h. ein Wissenstalent unterstützt den Helden dabei, einen Konflikt zu gewinnen. Daher die Sonderbehandlung.
\end{design}
\medskip


Zur Charaktererschaffung hat jeder Spieler 50~Steigerungen, die er zur Steigerung der Talentwerte und zum Kaufen von Ausrüstung benutzen muss. Dabei kostet:
\begin{enumerate}
  \item[a)] 1 Talentwert = 1 Steigerung (maximal auf 6)
  \item[b)] Aktivierung eines Spezialtalentes = 1 Steigerung
  \item[c)] Wissenstalent von 0 auf 1 = 2 Steigerungen \\
  Wissenstalent von 1 auf 2 = 4 Steigerungen
  \item[d)] Konfliktgegenstand, ein Würfel: 2 Steigerungen 
  \item[e)] Rüstungsgegenstand, 1 RS: 1 Steigerung \\
                Rüstungsgegenstand, 2 RS: 3 Steigerungen \\
                Rüstungsgegenstand, 3 RS: 6 Steigerungen 
\end{enumerate}

 Beim Kauf von Talentwerten ist zu beachten, dass eventuelle Talent-Nachteile durch Steigerungen ausgeglichen werden müssen. Negative Talentwerte sind nicht erlaubt.

In der folgenden Talentliste sind alle Basistalente mit einem Stern ($\star$) gekennzeichnet. Sie werden zu Beginn auf einem Talentwert von $0$ festgesetzt. Die Spezialtalente sind mit einem Spiegelstrich (--) gekennzeichnet und müssen erst aktiviert werden, bevor sie gesteigert werden können. Die Wissenstalente hingegen sind am Kringel ($\circ$) zu erkennen. Jeder Held hat in allen Wissenstalenten erstmal automatisch Allgemeinwissen (also $0$, d.\,h. er bekommt keinen Bonuswürfel).

Über die hier aufgeführten Talente hinaus gibt es noch \DEF{Berufstalente}\index{Berufstalent}. Das sind die Talente, die der Charakter von seinem ursprünglichen Beruf her können müsste. Beispiele sind Bierbrauen, Töpfern oder auch Bogenbau. Pro Berufstalent muss der Spieler bei der Charakterereschaffung einen Punkt investieren -- das Talent muss sich aus der Vergangenheit des Charakters begründen. Berufstalente haben keinen Talentwert und auch keine zugeordneten Eigenschaften. Diese Talente wird man im Spiel nur selten anwenden können.

\begin{itemize}
\item Nahkampf-Talente:
\begin{itemize}
\BTalent Dolche (MU/GE/FF)
\STalent Einhänder\footnote{Schwerter, Hiebwaffen, Flegel} (MU/GE/KK)
\STalent Fechtwaffen (MU/GE/FF)
\STalent Lanzenreiten (MU/GE/KK)
\BTalent Raufen (MU/GE/KK)
\STalent Stäbe/Speere (MU/GE/KK)
\STalent Zweihänder\footnotemark[\value{footnote}] (MU/GE/KK)
\end{itemize}

\pagebreak[3]
\item Fernkampf-Talente:
\begin{itemize}
\STalent Armbrust {(IN/FF/FF)}
\STalent Belagerungswaffen (IN/FF/KK)
\STalent Blasrohr (IN/FF/FF)
\STalent Bogen (IN/FF/KK)
\BTalent Einfache Wurfgeschosse\footnote{Wurfmesser und improvisierte Dinge wie Steine} (IN/FF/KK)
\STalent Schleuder {(IN/FF/FF)}
\STalent Wurfwaffen\footnote{Wurfbeile, Speere, Diskus usw.} (IN/FF/KK)
\end{itemize}

\item Körperliche Talente:
\begin{itemize}
\BTalent Athletik\footnote{``Kraft-Ausdauer-Sport'': Klettern, Rennen, Springen, Kraftakte usw.} (GE/KO/KK)
\BTalent {Körperbeherrschung\footnote{``Körpergefühl'': Balancieren, Schleichen, Akrobatik, usw.} (MU/GE/KK)}
\STalent Reiten (CH/GE/KK)
\STalent Schwimmen (GE/KO/KK)
\BTalent Selbstbeherrschung (MU/KO/KK)
\BTalent Sich verstecken (MU/IN/GE)
\BTalent Sinnenschärfe (KL/IN/IN)
\STalent Taschendiebstahl (MU/IN/FF)
\end{itemize}

\item Gesellschaftstalente:
\begin{itemize}
\STalent Betören/Galanterie (IN/CH/CH)
\STalent Gassenwissen (KL/IN/CH)
\STalent Schauspielerei (MU/KL/CH)
\BTalent Überreden/Überzeugen (MU/IN/CH)
\STalent Zechen\footnote{Alkohol vertragen, aber auch das gesellige Beisammensein, wie z.\,B. Trinkspiele und coole Sprüche} (IN/CH/KO)
\end{itemize}

\item Naturtalente:
\begin{itemize}
\BTalent Fährtensuche (KL/IN/KO)
\BTalent Orientierung (KL/IN/IN)
\BTalent Wildnisleben (IN/GE/KO)
\end{itemize}

\item Handwerkstalente:
\begin{itemize}
\STalent Alchimie (MU/KL/FF)
\BTalent Bastelei\footnote{Alles, woran Helden rumfummeln könnten und was nicht wirklich zu einem Handwerk gehört.} (IN/FF/FF)
\STalent Boote fahren (GE/KO/KK)
\STalent Fahrzeug lenken (IN/CH/FF)
\STalent Fesseln/Entfesseln (FF/GE/KK)
\STalent Heilkunde Gift (MU/KL/IN)
\STalent Heilkunde Krankheit (MU/KL/CH)
\STalent Heilkunde Seele (KL/IN/CH)
\BTalent Heilkunde Wunden (KL/CH/FF)
\STalent Schlösser knacken (IN/FF/FF)
\STalent Fallen entschärfen (IN/FF/FF)
\end{itemize}

\item Wissenstalente:
\begin{itemize}
\WTalent Etikette
\WTalent Geographie
\WTalent Geschichtswissen
\WTalent Gesteinskunde
\WTalent Götter/Kulte
\WTalent Heraldik/Staatskunde
\WTalent Magiekunde
\WTalent Mechanik
\WTalent Menschenkenntnis
\WTalent Pflanzenkunde
\WTalent Rechtskunde
\WTalent Sagen/Legenden
\WTalent Schätzen
\WTalent Lesen/Schreiben
\WTalent (Sprachen)
\WTalent Sternkunde
\WTalent Tierkunde
\end{itemize}

\item Gaben (vgl. Vor- und Nachteile, Seite~\pageref{VorUndNachteile})
\end{itemize}

\begin{design}
\subsubsection*{Designanmerkung: Kostenberechnung}
Wenn die Kosten für Talentpunkte eins zu eins sind, warum sind dann Eigenschaften nicht auch alle gleich teuer? Und warum steigen die Kosten für Bonuswürfel?

Bei den Eigenschaften liegt es daran, dass manche Eigenschaften in den Talenten häufiger gebraucht werden als andere. So kommt Intuition in vielen Talenten vor, Konstitution aber wenig. Da kein Spieler bzw. Charakter bevorzugt werden soll, sind wichtigere Eigenschaften teurer als unwichtige. So ist der Grundstock für den Talentgesamtwert von der Verteilung der Punkte fast unabhängig.

Bei den Bonuswürfeln kommt es darauf an, wie hoch ein passendes Talent gesteigert ist. Je höher der Talentgesamtwert ist, umso mehr ist ein Bonuswürfel wert. Außerdem neigen Spieler berechtigterweise dazu, Wissenstalente oder Gegenstände zu steigern, die sie für Talenten mit hohem Gesamtwert benutzen können, d.\,h. bei guten Talenten haben sie eher Bonuswürfel als bei schlechten. Das wird durch die unterschiedlichen Kosten ausgeglichen.
\end{design}



\subsection{Talentgesamtwerte}

Der \DEF{Talentgesamtwert}\index{Talentgesamtwert}, auf den dann im Spiel auch gewüfelt wird, berechnet sich aus der Hälfte der Summe der Eigenschaften und dem Talentwert.
\begin{align*}
  \text{Talentgesamtwert} &= \text{Talentwert} + \frac{\text{Eig.1}+\text{Eig.2}+\text{Eig.3}}{2}
\end{align*}

Um eine Vorstellung der Fähigkeiten zu bekommen, hier eine grobe Einordnung der Talentgesamtwerte:
\begin{tabular}[C]{|*6c|}
\hline
 $<1$ & $1$--$4$ & $5$--$8$ & $9$--$12$ & $13$--$16$ & $>16$ \\
 ahnungslos & ungeübt & geübt & Geselle & Meister & Koryphäe \\
\hline
\end{tabular}

Für die Spielercharaktere gilt allerdings ein Mindest-Talentgesamtwert von 5, d.h. wenn nach obiger Formel der Wert 4 oder weniger beträgt, so wird als Talentgesamtwert trotzdem einfach 5 genommen (es sei denn, der Charakter hat einen Nachteil, der ein bestimmtes Talent senkt; dann wird auch der Mindest-Talentgesamtwert entsprechend gesenkt).

Spezialtalente, die nicht aktiviert wurden, haben keinen Talentwert und keinen Talentgesamtwert. Daher kommt der Mindest-Talentgesamtwert nicht zur Anwendung. Solche Talente können in Konflikten nicht benutzt werden.

\begin{beispiel}
\paragraph{Beispiel:}

Ein Charakter hat MU:~2, KL:~3, IN:~1, CH:~3, GE:~--1, FF:~1, KO:~2, KK:~--1. Darüberhinaus hat er den Nachteil ``Unfähigkeit im Dolchkampf'', der den Talentwert Dolche um 3 Punkte senkt. Diese Punkte muss er mit einem Talentwert 3 ausgleichen.
\begin{itemize}
\item Dolche (MU/GE/FF), Talentwert 3: $\text{Talentgesamtwert} = (2-1+1)/2 + 3 -3 = 1 + 3 -3 = 1$. Da dieser Talentgesamtwert kleiner als 5 ist, müsste er auf 5 festgesetzt werden; wegen des Nachteils wird er aber auf $5-3=2$ gesetzt.
\item Zechen (IN/CH/KO), Talentwert 6: $\text{Talentgesamtwert} = (1+3+2)/2 + 6 = 3 + 6 = 9$.
\item Fesseln/Entfesseln (FF/GE/KK), Talentwert 3: $\text{Talentgesamtwert} = (1-1-1) + 3 = 0 + 3 = 3$. Hier wird also der Talentgesamtwert auf 5 gesetzt.
\item Heilkunde Wunden (KL/CH/FF), Talentwert 6: $\text{Talentgesamtwert} = (3+3+1)/2 + 6 = 4 + 6 = 10$.
\end{itemize}
\end{beispiel}



Da Berufstalente keine Talentwerte oder Eigenschaften haben, werden Talentgesamtwerte für diese Talente auf eine andere Art und Weise ermittelt. Der Talentgesamtwert beginnt bei Erwerb des Talentes auf 10 und steigt immer bei erreichen einer ungeraden Stufe um 1. Bei zu Spielbeginn genommenen Berufstalenten also 10 in Stufe 1 oder 2, ab der dritten Stufe 11, ab der fünften 12, ab der siebten 13, usw., bis in der 19. Stufe der Wert auf 19 steigt. Erwirbt ein Charakter ein Berufstalent z.\,B. auf Stufe 4, beginnt es bei 10 und steigt in Stufe~5 auf 2, in Stufe~7 auf 3 usw.

\begin{design}
\subsubsection{Designanmerkung: Talentgesamtwert}

\paragraph{Warum nicht einfach Eigenschaft + Talentwert?} Das hat zwei Gründe:

\emph{Erstens} bekommt man dann Probleme damit, dass dann am Anfang die Charaktere ziemlich schlecht sind und in Konflikten kaum eine Probe gelingt. Alternativ müsste man dann mit höheren Talentwerten beginnen, dann ist aber die Steigerungsspanne kleiner als auf die im Moment gewählte Weise -- denn schließlich sollen die Talentwerte auch in höheren Stufen noch was bringen.

\emph{Zweitens} werden dann einige Eigenschaften untergehen, wie z.\,B. Mut. Wann wird der SL mal eine Probe auf Körperbeherrschung+Mut verlangen? Meist wird Gewandtheit oder Körperkraft im Vordergrund stehen. Oder anders gefragt: Welches Talent würde man im Zusammenhang mit Mut testen wollen? Durch die Festlegung kommt auch eine Eigenschaft wie Mut stärker zur spieltechnischen Geltung.

\paragraph{Wozu der Mindest-Talentgesamtwert?} Auch hier gibt es zwei Gründe:

\emph{Erstens} werden die Spieler angeregt, bei der Charaktererschaffung die Steigerungen nicht gleichmäßig auf viele Talente zu verteilen, sondern sich auf einige Spezialgebiete zu konzentrieren.

\emph{Zweitens} sind die Erzählrechte in Neben- und Hauptkonflikte dann nicht so eingeschränkt, was mehr Erzählmöglichkeiten eröffnet. 
\end{design}




\section{Sonderfertigkeiten}
Sonderfertigkeiten sind herausragende Kenntnisse des Charakters in einem bestimmten Anwendungsgebiet. Sie können dazu dienen, in einem Konflikt Vorteile zu erlangen. Beispiele von Sonderfertigkeiten sind alle DSA4-Sonderfertigkeiten wie Waldkundig, Ausweichen, Hammerschlag oder gezielter Stich, aber auch Einbrechen kann hier als Sonderfertigkeit gewählt werden.

Gemeinsam ist allen Sonderfertigkeiten, dass es Fähigkeiten sind, die mehrere Talente betreffen und einen Vorteil geben. So könnte beispielsweise Hammerschlag einen Vorteil in Einhand- und Zweihand-Flegel, -Schwerter und -Hiebwaffen geben.

Sonderfertigkeiten können nicht bei der Charaktererschaffung, sondern erst im Laufe des Spieles, erworben werden.



\section{Sprachen und Schriften}
Sprachen und Schriften werden durch Wissenstalente dargestellt. Jede Sprache hat ihr eigens Wissenstalent, Lesen/Schreiben ist ein einziges Wissenstalent.

\subsection{Sprachen}
\begin{description}
\item[Allgemeinwissen (0):] Kenntnis, dass es die Sprache gibt; eventuell einzelne Phrasen (z.\,B. `Ich liebe dich!' oder ein Schimpfwort)
\item[Grundwissen (1):] Nicht fließend sondern stockend und gebrochen, aber Verständigung möglich
\item[Expertenwissen (2):] Fließend, normale Muttersprach-Niveau
\item[Koryphäe (3):] Besondere Kenntnisse über Grammatik, Ausdrucksformen und Wissen um die Wahl der Wörter zu besonderen Gelegenheiten
\end{description}

\subsection{Lesen/Schreiben}
Hierfür gibt es nur ein einziges Wissenstalent für alle Sprachen.
\begin{description}
\item[Allgemeinwissen (0):] Erkennen der Schriftzeichen der eigenen Muttersprache, Schreiben des eigenen Namens in der Muttersprache
\item[Grundwissen (1):] Lesen und Schreiben in der Muttersprache
\item[Expertenwissen (2):] Kenntnisses des Alphabets der Muttersprache und zwei weitere Alphabete, zu denen er mindestens eine Sprache beherrscht
\item[Koryphäe (3):] Jegliche Schriften zu seinen gesprochenen Sprachen
\end{description}

\subsection{Startcharaktere}
Zu Beginn des Spieles bekommt jeder Charakter zwei automatische Sprachtalente.
\begin{description}
\item[Muttersprache] Expertenwissen
\item[Fremdsprache] Grundwissen
\end{description}
Dabei sollte darauf geachtet werden, dass alle Helden mindestens eine Sprache gemeinsam sprechen. Am besten ist dies Garethi oder Tulamidya, damit sich die Helden mit ihrer Umwelt verständigen können. Lesen und Schreiben kann dagegen kein Charakter automatisch (d.\,h. der Charakter beherrscht nur das Allgemeinwissen). Natürlich können auch Sprachen und Lesen/Schreiben mit den Steigerungen zu Anfang oder mit Vorteilen bei der Charaktererschaffung gesteigert werden.


\section{Konfliktpunkte}
Jeder Charakter startet seine Abenteurer-Laufbahn mit drei Konfliktpunkten. Die Konfliktpunkten geben an, wie lange ein Charakter in einem Konflikt bleiben kann. Sie steigen im Laufe des Spiels an.



\pagebreak[3]
\section{Lebens- und Willenskraft}
 Das, was bei DSA4 Lebensenergie ist, wird in StoryDSA durch Lebens- und Willenskraft ersetzt. Dabei handelt es sich um geistige und körperliche Lebensenergie. Diese Werte steht für Zähigkeit und Widerstandskraft und für die Fähigkeit, Verletzungen ohne Schwierigkeiten hinzunehmen. Lebens- und Willenskraft helfen im Spiel dabei, mehrere Konflikte ohne Regenerationsphasen zu überstehen.

Die Werte lassen sich direkt aus den Eigenschaften ableiten. Es gilt:
\begin{align*}
	\text{Lebenskraft} &= \text{GE + FF + KO + KK + 8} \\
	\text{Willenskraft} &= \text{MU + KL + IN + CH + 6} \\
\end{align*}



\section{Magie}
\subsection{Voll- und Halbzauberer}

Nicht jeder Charakter beherrscht von Anfang an Magie. Jeder Spieler entscheidet, ob sein Held magisch begabt ist. Es ist im Gegensatz zu DSA4 in \StoryDSA auch möglich, im späteren Spiel eine Magiebegabung zu erlangen.

Dazu muss ein Charakter nur ein magisches Talent erlernen und steigern. Dabei handelt es sich um ein Spezialtalent, welches nach der magischen Ausrichtung benannt ist, also z.B. ``Haindruidische Magie'' oder ``Andergaster Kampfmagie''. Die zugehörigen Eingeschaften sind auf jeden Fall KL/IN/CH.

Sobald das magische Talent aktiviert ist, beherrscht der Charakter Magie und kann magische Effekte erlernen. Ein magischer Effekt entspricht einem Zauberspruch oder einem Ritual von DSA. Die Kosten betragen eine Steigerung pro Effekt; dabei kann ein magiebegabter Held nie mehr magische Effekte erlernen, wie sein Talentgesamtwert des magischen Talentes.

Möchte ein Spieler, dass der Held mehr magische Effekte beherrscht, kann er seinem Charakter auch mehr als ein magisches Talent geben. Die magischen Effekte müssen allerdings eindeutig einem Talent zugeordnet sein, damit klar ist, auf welches Talent gewürfelt werden muss.

Um später im Spiel Zauber zu wirken, benötigt der magiekundige Held außerdem noch Astralenergie. Dazu kann der Spieler Lebens- und Willenskraft in Astralenergie umwandeln -- der Erwerb kostet keine zusätzlichen Steigerungen.

Die Umwandlung geht nur in die eine Richtung und nur, wenn der Charakter zum ersten Mal Magie erlernt (also im Normalfall bei der Charaktererschaffung). Es ist unmöglich, Astralenergie wieder in Lebens- und Willenskraft zu verwandeln oder später einfach eine der Kräfte zu senken und dafür Astralenergie zu steigern. Stattdessen wird im späteren Verlauf die Astralenergie beim Stufenanstieg gesteigert.

Das Umwandlungsverhältnis ist 1:2, d.\,h. für einen Punkt Lebens- oder Willenskraft bekommt der Charakter zwei Punkte Astralenergie. Maximal kann ein Magiekundiger so viele Punkte umwandeln, wie der höchste Talentgesamtwert in einem Magie-Talent beträgt. Dem Spieler bleibt überlassen, wie viel Lebens- oder Willenskraft umgewandelt wird.

\begin{beispiel}
\paragraph{Beispiel:} Ganator, ein Bühnenmagier und Schausteller, soll für seine Tricks ein paar Punkte Astralenergie bekommen. Er hat ein magisches Talent Bühnenmagie mit einem Talentgesamtwert von 8; maximal kann er also 8 Kraftpunkte in 16 Astralenergie umwandeln. Der Spieler entscheidet, dass 6 Punkte für Ganator ausreichen und möchte wandelt daher 1~Willenskraft und 2~Lebenskraft in 6~Astralenergie um.
\end{beispiel}

Ein Charakter, der hauptsächlich auf Magie baut, wie z.\,B. ein Magier, ein Druide oder eine Hexe, sollte möglichst das Magietalent ausmaximieren (d.\,h. der Spieler sollte es auf einen Talentgesamtwert von zehn oder mehr bringen) und so viel Kraftpunkte wie möglich in Astralenergie umwandeln.

\subsection{Viertelzauberer}
Viertelzauberer sind anders als Voll- oder Halbzauberer. Durch die intuitive Anwendung von Magie stehen ihnen mit Meisterhandwerk und Schutzgeist Spielarten der Magie offen, zu denen andere keinen Zugriff haben. Daher muss sich ein Spieler entscheiden, ob er seinen Charakter zum Viertelzauberer oder einen normalen Zauberer macht. Eine einmal getroffene Entscheidung kann nicht rückgängig gemacht werden.

Um Viertelzauberer zu werden, braucht der Charakter das Spezialtalent ``Magiedilletant (KL/IN/CH)''. Genau wie Halb- und Vollzauberer muss er zusätzlich magische Effekte kaufen; jeder Effekt kostet zwei Steigerungen, ist als teurer als für Voll- und Halbzauberer. Die Anzahl der magischen Effekte ist durch die Hälfte des Talentgesamtwertes im Talent ``Magiedilletant'' beschränkt. Er kann zwischen folgenden Effekten wählen:

\begin{itemize}
\item Übernatürliche Begabungen. Der Dilletant wählt wie ein normaler Magier Zaubersprüche, die er allerdings intuitiv einsetzt. Insgesamt dürfen nicht mehr als fünf gewählt werden.
\item Meisterhandwerk. Der Dilletant wählt Talente, die er genau wie Magie durch den Einsatz von Astralenergie verstärken kann. Insgesamt dürfen nicht mehr als fünf gewählt werden.
\item Schutzgeist (Segen). Der Viertelzauberer wird von einem Schutzgeist gesegnet und kann Würfelwürfe wiederholen.
\item Schutzgeist (Schutz). Der Viertelzauberer wird von einem Schutzgeist geschützt und kann Konfliktpunkt-Verlust in AsP-Verlust umwandeln.
\end{itemize}

Für den Einsatz von Magie bekommt jeder Magiedilletant die Hälfte seines Talentgesamtwertes an Astralenergie geschenkt. Es ist nicht möglich, mehr Astralpunkte zu bekommen (außer durch das Steigern des Talentes).


\pagebreak[3]
\section{Tipps zur Charaktererstellung}
Im Spiel kann es ärgerlich sein, wenn ein Charakter wegen einer ungüstigen Wahl von Fertigkeiten und Eigenschaften schlechter da steht als andere Charaktere. Daher sollten folgende Tipps beachtet werden:
\begin{itemize}
  \item Die Charaktere sollten von allen Spielern \emph{gemeinsam} erstellt werden. Auf diese Weise bekommt man Charaktere, die aufeinander abgestimmt sind -- so können z.\,B. kleinere Streitigkeiten, die etwas Farbe ins freie Spiel bringen, hier `geplant' werden. Ideal ist es, wenn die Spieler sich bei der Charaktererschaffung gegenseitig helfen und Tipps geben.
  \item Die Charaktere sollten insgesamt ein breites Spektrum an Fähigkeiten abdecken -- da in vielen Abenteuern Kampf nicht vermeidbar ist, sollte zumindest die Hälfte der Charaktere brauchbare Talentgesamtwerte (d.\,h. zumindest 8) in einem Kampftalent vorweisen.
  \item Die Eigenschaften sollten auf die Talente abgestimmt sein. Eine gute Möglichkeit ist, man überlegt sich zuerst sein Charakterkonzept, wählt dann die Talente und überlegt danach erst, wie die Punkte auf die Eigenschaften zu verteilen sind. Dabei gilt: Je häufiger eine Eigenschaft vorkommt, umso besser sollte der Wert der Eigenschaft sein.
  \item Die Eigenschaften der gewählten Talente sollten nicht zu verschieden sein! Am besten, man sucht sie so aus, dass mindestens zwei oder drei Eigenschaften gar nicht vorkommen.
  \item Am Ende der Charaktererschaffung sollte der Talentgesamtwert von zumindest fünf Talenten auf 8 oder höher liegen.
\end{itemize}

Eine Garantie für einen Helden mit guten Werten kann nur sehr schwer in Generierungsregeln gefasst werden. Das folgende, relativ einfache Verfahren zeigt Einsteigern einen Weg auf, nicht allzu ungünstige Charaktere zu entwerfen.
\begin{enumerate}
  \item Zunächst werden drei Haupteigenschaften gewählt. Diese Haupteigenschaften müssen zumindest auf 2 gesetzt werden, mindestens einen davon sogar auf 3. Die restlichen Punkte werden auf die anderen fünf Eigenschaften verteilt.
  
  \item Als zweites sollten die wichtigen Gegenstände des Charakters ausgewählt werden. Jeder Gegenstand kostet 2 Steigerungen und gibt einen Bonuswürfel im Konflikt. Rüstungen kosten 1, 3 bzw. 6 Steigerungen, je nach Rüstungsschutz. Maximal sollten 8 Punkte für Ausrüstung ausgegeben werden.

  \item Danach werden insgesamt sechs Basis- oder Spezialtalente ausgewählt. Sie bekommen einen Talentwert von 6, so dass der Talentgesamtwert bei diesen Talenten 8 oder mehr beträgt. Dabei ist zu beachten, dass Spezialtalente durch die Aktivierung 7 Steigerungen kosten.
  
  Von diesen Talenten dürfen bis zu zwei auch Wissenstalente sein, die auf Expertenwissen (2) gesetzt werden.

  \item Die restlichen Punkte sollten auf die restlichen Talente verteilt werden. Dabei ist eine Konzentration auf relativ wenige Talente einer breiten Verteilung vorzuziehen. Dabei maximal 2 Berufstalente.
\end{enumerate}


