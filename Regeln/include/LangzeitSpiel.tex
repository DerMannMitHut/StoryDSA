\chapter{Langzeit-Spiel}\label{Ch:LangzeitSpiel}
\lettrine{D}{ieses} Kapitel beschäftigt sich mit dem Spielspaß über mehrere Abenteuer hinweg. Warum sollte man ein und denselben Charakter viele Spielabende lang spielen wollen? Dazu muss der Charakter eine gewisse Entwicklung durchmachen. Da das langfristige Ziel in diesem Spiel ist, dass die Charaktere zu den berühmtesten Helden in Aventurien aufsteigen, muss die Charakterentwicklung auch in diese Richtung gehen.

Das bedeutet jetzt nicht, dass nicht andere Entwicklungen auch stattfinden dürfen: Klarerweise darf jeder Held Freunde gewinnen und seine Persönlichkeit mit der Zeit verändern. Diese Art von Veränderungen werden aber nicht durch das System unterstützt und bringen lediglich mehr Farbe ins Spiel.

\section{Abenteuerpunkte und Stufen}
F"ur die Steigerung m"ussen die Helden \DEF{Abenteuerpunkte}\index{Abenteuerpunkt} (AP)\index{AP|see{Abenteuerpunkt}} erwerben. Die Abenteuerpunkte entsprechen dabei genau denen aus DSA. Im Unterschied zum normalen DSA steigt die Stufe jedoch durch das Bekommen und nicht erst duch das Ausgeben der AP, denn die AP stehen ja f"ur den Ruf, den die Helden erworben haben.

\begin{table}
\begin{tabular}[C]{l*{6}c}
\bf Stufe & 1   & 2   & 3   & 4   & 5    \\
\bf AP    & 0   & 100 & 300 & 600 & 1000 \\[\medskipamount]
\bf Stufe & 6   & 7   & 8   & 9   & 10  \\
\bf AP    & 1500& 2100& 2800& 3600& 4500\\[\medskipamount]
\bf Stufe & 11  & 12  & 13  & 14  & 15  \\
\bf AP    & 5500& 6600& 7800& 9100&10500\\[\medskipamount]
\bf Stufe & 16  & 17  & 18  & 19  & 20   & 21 \\
\bf AP    & 12000&13600&15300&17100&19000& 21000
\end{tabular}
\medskip
\hrule
\end{table}

Die Stufe gibt an, wie bekannt ein Held \emph{aufgrund seiner Heldentaten} ist. Solange die Helden nicht "uber die zehnte Stufe hinaus sind, k"onnen sie nicht damit rechnen, dass sie au"serhalb ihres `Wirkungsbereiches' erkannt werden. Gerade am Anfang gilt der Ruf nur f"ur den Heimatort bzw. den Startort.
\begin{description}
\item[Stufe 1--2:] Die Helden werden als ganz normale B"urger angesehen und haben keinen besonderen Ruf.
\item[Stufe 3--5:] Durch ihre zweifelhaften T"atigkeiten (kein ernsthafter Beruf bzw. stecken ihre Nasen in Dinge, die ehrbare Leute nichts angehen) bekommen die Helden den Ruf als Herumtreiber, Spinner und Tr"aumer. Man beginnt, "uber sie zu reden.
\item[Stufe 6--8:] Kopfsch"utteln wandelt sich in Anerkennung. Die Helden sind in den Augen der einfachen Bev"olkerung Leute, die vieles von ihren Reisen in die weite Welt berichten k"onnen. Der Ruf dringt aber nicht aus dem direkten Wirkungskreis der Helden heraus, d.\,h. sie sind nur dort bekannt, wo sie auch aufgetreten sind.
\item[Stufe 9--11:] Erste Bardenlieder machen die Runde. Die Lieder und Geschichten dringen in die Nachbarorte vor, so dass auch Leute der Region, die die Helden noch nie gesehen haben, von ihnen geh"ort haben k"onnten.
\item[Stufe 12--14:] Der gute Ruf weitet sich aus. Es gibt ganze Gegenden, die die Helden mit Namen und ihre Taten kennen. Die Helden k"onnen davon ausgehen, dass ihre Geburtsst"adte beginnen, sich mit ihren zu r"uhmen.
\item[Stufe 15--17:] Aus den Lokalhelden sind Landeshelden geworden. Im Mittelreich entspricht das den einzelnen Provinzen, ansonsten dem gesamten Land. Auch in den Grenznahen Gebieten der Nachbarl"ander erz"ahlt man sich von den Heldentaten der Charaktere.
\item[Stufe 18--20:] Der Ruf weitet sich auf die Nachbarl"ander aus. Aufgrund der Lieder und Beschreibungen werden sie in halb Aventurien erkannt. Eventuell ist der Ruf der Helden in den Nachbarl"andern nicht so gut, wie sie erwarten (z.\,B. aus Neid auf das andere Land).
\item[Stufe 21:] Die Bekanntheit erreicht auch die hintersten Winkel Aventuriens. H"ochstens in den abgelegensten Gebieten des Dschungels oder der Eisw"uste leben Personen, die noch nichts von den Helden geh"ort haben. Eine h"ohere Bekanntheit kann man nicht erlangen, das Spiel endet und die Helden setzen sich zur Ruhe.
\end{description}

\section{Vergabe von AP}
Die Helden sollten f"ur bestandene Konflikte AP bekommen. 
Als Richtlinie sollten 10~AP f"ur einen Kurzkonflikt, 20~AP pro Nebenkonflikt und 50~AP f"ur einen Hauptkonflikt vergeben werden.
Verlorene Konflikte geben den Helden keine AP; bei gemischten Konflikten bzw. Konflikten mit mehreren Hauptkonfliktgegnern wird jeder Konfliktgegner einzeln bewertet.

\section{Abenteuer}
Aus diesen Interpretationen der Stufe ergeben sich nat"urlich einige Vorgaben an die Abenteuer. Die ersten Abenteuer drehen sich um die kleinen Dinge des Lebens: Vieh wird vergiftet, die Helden helfen bei der Aufkl"arung von "ortlichen Verbrechen oder erledingen einen Botendienst.

Spielen die ersten Abenteuer noch weitgehend in derselben Gegend, so kommen die Helden im Verlauf des Langzeit-Spieles immer weiter "uber die Grenzen ihres Dorfes hinaus. Durch l"anger werdende Reisen kommen die Helden nicht mehr so oft in ihren Heimatort. Sp"atestens ab Stufe 12 "andern sich dann die Aufgaben der Helden: Sie werden bei gr"o"seren Problemen aktiv um Hilfe gebeten, Boten werden geschickt, um die Helden zu holen. Die politische Lage wird f"ur die Helden wichtiger, die kleinen Aufgaben der einfachen Bev"olkerung treten in den Hintergrund.

Die Abenteuerplanung ist eine der wichtigsten Aufgaben f"ur den Spielleiter: Er muss \emph{angemessene} Aufgaben für die Helden finden. Dabei kann es sich natürlich um selbst gemachte und an die Gruppe angepasste Abenteuer, aber natürlich auch um Kaufabenteuer handeln. Eine Richtschnur zur Erstellung interessanter Abenteuer befindet sich im Kapitel Spiel-Leiten, ab Seite~\pageref{Ch:Spielleiten}.


\section{Steigerung}\label{Steigerungstabelle}\label{Steigerung}

\begin{tabular}[C]{l*{10}c}
\bf Stufe              & 1 & 2 & 3 & 4 & 5 & 6 & 7 & 8 & 9 & 10 \\
\bf max. Talentwert    & 6 & 6 & 7 & 7 & 7 & 8 & 8 & 8 & 8 & 9  \\
\bf max. Wissentalent  & 2 & 2 & 2 & 2 & 2 & 2 & 2 & 2 & 2 & 2  \\
\bf max. Gegenstand    & 1 & 1 & 1 & 1 & 1 & 1 & 1 & 2 & 2 & 2  \\
\bf max. Rüstung       & 3 & 3 & 3 & 3 & 4 & 4 & 4 & 4 & 4 & 4  \\
\bf Konfliktpunkte     & 3 & 3 & 3 & 4 & 4 & 4 & 4 & 4 & 5 & 5  \\
\bf Berufstalente      &   &   & +1&   & +1&   & +1&   & +1&    \\
\\
\bf Stufe              & 11 & 12 & 13 & 14 & 15 & 16 & 17 & 18 & 19 & 20 \\
\bf max. Talentwert    & 9  & 9  & 10 & 10 & 10 & 10 & 11 & 11 & 11 & 12 \\
\bf max. Wissentalent  & 3  & 3  & 3  & 3  & 3  & 3  & 3  & 3  & 3  & 3  \\
\bf max. Gegenstand    & 2  & 2  & 2  & 2  & 2  & 3  & 3  & 3  & 3  & 3  \\
\bf max. Rüstung       & 4  & 5  & 5  & 5  & 5  & 5  & 5  & 6  & 6  & 6  \\
\bf Konfliktpunkte     & 5  & 5  & 5  & 6  & 6  & 6  & 6  & 6  & 7  & 7  \\
\bf Berufstalente      & +1 &    & +1 &    & +1 &    & +1 &    & +1 &    \\
\end{tabular}

Nicht zweimal hintereinander dasselbe steigern! {1 Steigerung = 100 AP}

KP: Anzahl Konfliktpunkte pro Konflikt

max. TaW: maximaler Talent- oder Zauberfertigkeitswert

Dabei zählen nicht die verbrauchten, sondern die \emph{bekommenen} AP.

\pagebreak[3]
\subsection{Talente}

1 Talentwert steigern: {1 Steigerung}

Spezialtalent aktivieren (auf 0 bringen): {1 Steigerung}

\textbf{Wissenstalent}: 
\begin{tabular}[C]{ccccc}
\bf Wissenstalent & 0 & 1 & 2 & 3 \\
\bf 1 Punkt & 0 & 2 & 4 & 6 \\
\bf Summe & 0 & 2 & 6 & 12 \\
\end{tabular}

Berufstalent nachträglich lernen: 1 Steigerung (beginnt bei 10, steigt alle auf ungeraden Stufen automatisch um 1)

Ein zusätzlicher magischer Effekt kostet 1~Steigerung; dabei darf die Anzahl der Effekte den Talentgesamtwert nicht überschreiten


\subsection{Sonderfertigkeiten}
Beliebiger (stimmungsvoller) Name, dazu mindestens drei Talente auswählen. Dies ermöglicht für diese Talente meisterliche Offensivwürfe oder meisterliche Defensivwürfe bei 1--2 auf W20. Der Spieler muss sich festlegen: entweder ist seine Sonderfertigkeit offensiv oder defensiv. Es ist nicht möglich, dass eine Sonderfertigkeit für einige Talente offensiv und für andere defensiv wirkt.

Zugehörige Talentgesamtwerte müssen mindestens 10 betragen!

Kosten: 1 Steigerung pro Talent

Sonderfertigkeiten k"onnen für die gleichen Kosten einmal `aufgestockt' werden, dann bekommt man schon bei 1--4 einen meisterlichen Erfolg.

Zugehörige Talentgesamtwerte müssen mindestens 15 betragen!

\subsection{Eigenschaften}
Das vierfache des Wertes der Eigenschaft, also:
\begin{tabular}[C]{cccc}
  \bf MU & \bf KL & \bf IN & \bf CH \\
  $4\times3=12$ & $4\times2=8$ & $4\times5=20$ & $4\times2=8$ \\[\medskipamount]
  \bf GE & \bf FF & \bf KO & \bf KK \\
  $4\times3=12$ & $4\times4=16$ & $4\times1=4$ & $4\times3=12$ \\
\end{tabular}

Zudem gibt es entsprechend der Eigenschaft einen zusätzlichen Punkt Willens- bzw. Lebenskraft (MU, KL, IN oder CH gibt einen Punkt Willenskraft; GE, FF, KO oder KK einen Punkt Lebenskraft). Magiekundige können stattdessen auch 2~Punkte Astralenergie wählen.


\begin{optional}
\section{Optional: Ver"anderte Aufstiegsgeschwindigkeit}

Wie viele Abenteuerpunkte der SL f"ur Konflikte vergibt ist reine Geschmackssache. M"ochten die Spieler lieber schneller aufsteigen, so kann sich die Gruppe auf einen h"oheren AP-Wert einigen; geht der Aufstieg zu schnell, so ist auch eine geringere AP-Verteilung m"oglich.

Die zweite Stellschraube f"ur die Aufstiegsgeschwindigkeit ist die Umrechnung Abenteuerpunkte in Steigerungen. Da es maximale Talentwert pro Stufe gibt, hat eine Erh"ohung der verteilten Steigerung (also z.\,B. ein Umrechnungsfaktion 1~Steigerung=50~AP) zur Folge, dass sich die Helden wahrscheinlich die Eigenschaften schneller maximieren und bei der Talentauswahl mehr in die Breite gehen. W"ahrend die Anzahl der Abenteuerpunkte pro Spielabend problemlos ver"andert werden kann, sollte die Umrechung der AP in Steigerung nur mit Vorsicht ver"andert werden.

Es gibt aber noch eine dritte Option: Ein Einstieg ins Spiel mit besseren Charakteren. Das kann z.\,B. dadurch geschehen, dass man am Anfang einfach mehr Punkte f"ur Eigenschaften und Talente vergibt. Auch dann zwingt man die Charaktere aufgrund der Deckelung zu einer breiteren Ausbildung der F"ahigkeiten. Die andere M"oglichkeit ist, nicht in Stufe~1 mit 0~AP zu beginnen, sondern direkt in einer h"oheren Stufe einzusteigen. Dazu empfielt es sich, die Helden wirklich bis in die gew"unschte Stufe hochzusteigern.
\end{optional}

\label{EndeSpielregeln}



