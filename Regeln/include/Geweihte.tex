\chapter{Geweihte}\label{Ch:Geweihte}

\begin{itemize}
	\item Mit diesen Geweihten-Regeln werden erstmal nur die üblichen Heldengötter modelliert. Andere übernatürliche Wesen (Dämonen, der Namenlose) kommen später noch gesondert an die Reihe. Rasthullah-Gläubige und andere Geweihte, die von ihrem Gott keine Wunder bekommen, brauchen die Talente Liturgien und Karma nicht. Für Kult-Handlungen kann Liturgien von solchen Geweihten als Berufstalent genommen werden.
	\item Geweihte haben zwei Zusatztalente: 1. Liturgien (KL/IN/CH) und 2. Karma (MU/IN/CH)
	\item Der Talentgesamtwert von Karma gibt die zur Verfügung stehenden Karmapunkte an (KaP)
	\item Das Talent Liturgien wird zur Mirakelprobe verwendet
	\item Grundsätzlich stehen einem Geweihten vier Möglichkeiten offen: Segen, Stoßgebet, Schutz und Wunder. Alle diese müssen innerhalb der Religion interpretiert und beschrieben werden.
	\item Das Durchführen einer Segen- oder Schutz-Liturgie mit Spieltechnisch relevanten Auswirkungen wird als Kurzkonflikt außerhalb von anderen Konflikten gestaltet. Einen Bonus von +3 gibt es an geweihten Orten oder bei besonders gründlicher Vorbereitung, --3 gibt es für Zeitdruck, besonders hohen Stress oder spezielle unheilige Orte, --6 für eine Kombination daraus.
	
	Das Misslingen einer Segens- oder Schutzliturgie kostet keine KaP. Wichtig ist allerdings, dass auch hier die üblichen Wiederholungs-Regeln gelten, d.h. eine misslungenes Liturgie-Ritual kann nicht einfach wiederholt werden.
	
	\item Die Liturgien Stoßgebet und Wunder werden innerhalb von Konflikten eingesetzt; normalerweise in Neben- und Hauptkonflikten, aber auch Kurzkonflikte sind möglich. Hier benutzt der Geweihte sein Liturgie-Talent für den Wurf.
	
	\item Für die Regeneration von Karmapunkten muss sich der Geweihte für seinen Glauben einsetzen (z.B. einen Gottestdienst abhalten, Leute bekehren oder Opfer darbringen). Für eine solche Handlung bekommt der Geweihte zwei KaP zurück. Mehr als eine Handlung dieser Art pro Tag auf der Spielwelt hat keine weiteren Auswirkungen.
\end{itemize}

\begin{description}
\item[Segen]
Segen bedeutet, dass der Gott einen anderen Charakter in Zukunft schützen möge, so lange dieser nicht gegen die Regeln des Gottes verstößt. Der Geweihte verleiht dazu eine beliebige Anzahl an Wiederholungen, die der Geschützte einsetzen kann, um misslungene Würfelwürfe zu ignorieren und erneut zu werfen.

Diese Wiederholungen verfallen nicht von alleine; sie können nur verbraucht werden oder gehen durch den Verstoß gegen die Regeln des Gottes verloren. Erzählt der Spieler eines geschützten Charakters, wie dieser etwas `falsches' tut, so muss er von Spielleiter gewarnt werden. Er kann die Handlung dann wieder zurücknehmen und etwas anderes machen, oder er lässt die Wiederholungen verfallen.

Der Geweihte kann nur einen Charakter segnen, der nicht bereits gesegnet ist (von derselben Gottheit oder einer anderen ist hierbei irrelevant) und auch nicht Geweihter einer anderen Gottheit ist.

\emph{Kosten:} Eine Wiederholung = 2~KaP, zwei Wiederholungen = 4~KaP, drei Wiederholungen = 6~KaP, vier Wiederholungen = 9~KaP usw. (dasselbe wie die AsP-Kosten für zusätzliche Würfel)

\item[Schutz]
Der Geweihte bietet einem Freund Schutz, indem er in jeder Konfliktrunde einen Anteil KP-Verlust als KaP-Verluste übernimmt. Der Gott schafft eine Verbindung zwischen den beiden. Die Verbindung kann entweder willentlich gelöst werden oder wird abgebrochen, wenn einer der beiden keine KP mehr hat, bewusstlos wird, stirbt usw. oder wenn einer der beiden etwas ungöttliches macht oder die KaP auf 0 sinken.

Der Geweihte kann immer nur einen Freund schützen. Er kann keine zweite Verbindung aufbauen, ohne vorher die erste abzubrechen.

Verliert der geschützte Charakter Konfliktpunkte, so braucht er nur die Hälfte dieser Punkte (abgerundet) zu streichen. Der Geweihte verliert das doppelte der abgehaltenen Konfliktpunkte an Karmapunkten.

\item[Stoßgebet]
Der Geweihte betet zu seinem Gott, der ihm dann hilft (entweder offen durch ein kleines Wunder oder versteckt, indem dem Geweihten eine Handlung besonders gut gelingt). Spieltechnisch kauft sich der Geweihte zusätzliche automatische Erfolge. Die Kosten betragen dasselbe, wie zusätzliche Würfel an AsP für magiekundige Helden.

\item[Wunder]
Der Geweihte setzt seine \emph{gesamte verbliebene karmale Kraft} ein, um ein großes Wunder zu erflehen. Dadurch kann er in einem Konflikt direkt Konfilktpunkte in Neben- oder Hauptkonflikten reduzieren (z.B. gegen einen bestimmten Hauptkonfliktgegener). Die Menge der Konfliktpunkte ist von den übrigen KaP abhängig: Pro 3~KaP verursacht er 1~KP Verlust. Außerdem kann er Erzählmarken für weitere KaP-Verluste ausgeben: Pro Erzählmarke verursacht er weitere 2~KP Verlust.
\end{description}



