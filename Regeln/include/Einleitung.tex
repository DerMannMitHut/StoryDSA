\chapter{Einleitung}

\lettrine[findent=1em,nindent=-0.4em]{W}{ie} schon öfters festgestellt wurde (u.\,A. ``Wie spielt man DSA richtig?'' von Dominic W"asch oder auch ``Etherisches Gefl"uster 55'' von Katharina Pietsch, aber auch Unmengen an Hausregelvorschl"agen deuten darauf hin), sind die DSA-Regeln dysfunktional, d.\,h. die Regeln geben nicht wirklich vor, wie man DSA spielt. Daher haben sich die unterschiedlichsten Spielstile entwickelt, die zueinander mehr oder weniger kompatibel sind.

Im Folgenden soll ein Regelwerk präsentiert werden, dass einen Spielstil\index{Spielstil} unterstützt, mit dem man die Kaufabenteuer gut nachspielen kann. Das bedeutet:
\begin{itemize}
  \item Der Plot steht im Vordergrund; er wird vom Spielleiter gelenkt
  \item Die restlichen Spieler lenken die Charaktere im vorgegebenen Rahmen
  \item Das persönliche Schicksal der Helden, wie z.\,B. ein großer persönlicher Verlust steht nicht im Vordergrund und wird nicht weiter unterstützt
  \item Stattdessen steht die Entwicklung vom Niemand zu einem der großen Helden Aventuriens im Vordergrund
  \item Genaue Simulation der Welt tritt hinter diese Geschichte -- Spannung vor Realismus
\end{itemize}

\FEHLT{Muss natürlich noch ausführlicher}

\textbf{Anmerkung:} Gegen"uber der Vorg"angerversion inhaltlich neue oder ver"anderte Teile sind auf diese Weise neu gekennzeichnet.

\section{Einfl"usse}
Beeinflusst sind diese Regeln von mehreren Seiten. Als wichtigstes ist die Rollenspieltheorie zu nennen, speziell der Theorie- und Design-Bereich im ehemaligen GroFaFo (jetzt Tanelorn) \footnote{{http://tanelorn.net/}} und die Forge\footnote{{http://www.indie-rpgs.com/}} bzw. die Forge-Diaspora\footnote{Ein guter Einstiegspunkt ist {http://rpgtheoryreview.blogspot.com/}}. Dadurch habe ich die Vorz"uge von koh"arentem Design kennengelernt: Die Spielregeln sollen eine Anleitung dazu sein, wie man das Spiel spielen soll.

Einige wichtige Vertreter dieser Art von Spielen sind \emph{Sorcerer}\footnote{{http://www.sorcerer-rpg.com/}} (von Ron Edwards), \emph{Dogs in the Vineyard}\footnote{{http://www.septemberquestion.org/lumpley/dogs.html}} (von Vincent Baker) und \emph{Primetime Adventures}\footnote{{http://www.dog-eared-designs.com/games.html}} (von Matt Wilson). Allen diesen Rollenspielen ist eines eigen: Mit ihnen kann man eine bestimmte Art von Spiel spielen; andere Arten von Spielen sind damit nicht ohne weiteres m"oglich. Sind bei klassischen Spielen die Regeln durchaus austauschbar (vielleicht mag ein Spieler lieber W6-W"urfelpools und ein anderer W20-Additionsw"urfe, aber im Prinzip sind sich die Regeln "ahnlich), so geben diese Spiele nicht nur vor, welche Werte ein Charakter hat und wie man eine Probe w"urfelt, sondern auch \emph{wie} das Spiel zu Spielen ist.

Das ist es, was ich mit dieser storyorientierten Version von \emph{Das Schwarze Auge} erreichen will. Das Spiel ist f"ur eine bestimmte Art zu spielen gedacht, wie ich sie bereits vorhin vorgestellt habe.

Den gr"o"sten Einfluss auf die Regeln hat sicherlich \emph{Wushu}\footnote{{http://bayn.org/wushu/wushu-open.html}} (von Dan Bayn), von dem ich eine Menge einfach "ubernommen habe. Aber auch \emph{The Shadow of Yesterday}\footnote{{http://www.anvilwerks.com/?The-Shadow-of-Yesterday}} (von Clinton R. Nixon) und die \emph{DSA Saphir Edition}\footnote{http://www.alveran.org/index.php?id=236} (von Andreas John, Mark Wachholz, Torsten Basedow und Florian Sachtleben) sind nicht spurlos gewesen.

Weitere Texte, ohne die \StoryDSA keinesfalls so w"are, wie es jetzt ist, sind: \emph{The Conflict Web} und \emph{Flag Framing} aus dem Blog \emph{deep in the game}\footnote{leider nicht mehr online (war: http://bankuei.blogspot.com/)} (von Chris `Bankuei' Chinn), die Spielhilfe \emph{Heldentaten und Schurkenpl"ane}\footnote{http://www.alveran.org/index.php?id=246} (von Ulrich Lang), das Blog \emph{Story Entertainment}\footnote{http://storyentertainment.blogspot.com/} (von Norbert G. Matausch), das Blog \emph{1of3's}\footnote{http://1of3.blogspot.com/} (von Stefan Koch) und das Blog \emph{Wilde Lande}\footnote{http://wildelande.myblog.de/} (von Frank `Lord Verminaard' Tarcikowski).

Dar"uber hinaus habe ich einige Dinge Holger `Purzel' M"uller zu verdanken: Neben konkreten Beitr"agen wie die Rahmengrafik oder die Beispiel-Abenteurergruppe war und ist er immer ein guter Freund und Diskussionsparter mit hervorragenden Ideen und dem richtigen kritischen Blick. 

Ein großer Dank geht au"serdem an das (mittlerweile leider geschlossene) Teest"ubchen im Wolkenturm\footnote{http://www.wolkenturm.de/} (von Tyll Zybura und Katharina Pietsch), in dem ich bis Dezember~2006 mit meinen etwas anderen DSA-Ideen immer einen Platz zur Diskussion gefunden habe.

\section{Fragen, Anmerkungen, Lob und Tadel}
Sollten Fragen oder Anmerkungen auftauchen, k"onnen diese im Forum vom Metst"ubchen\footnote{http://www.metstuebchen.de/} ge"au"sert werden. Dort im Schriftenverzeichnis werden auch eventuelle Neuigkeiten, Abenteuervorschl"age, Charakterbl"atter usw. zum Download angeboten. Eine Anmeldung im Forum ist nicht erforderlich.

\section{Umgang mit optionalen Regeln}\index{Regel!optional}
In den Regeln werden immer wieder auch ein paar Varianten vorgestellt, die optionalen Regeln. Sie werden durch \emph{kursive Schrift} hervorgehoben und k"onnen beim ersten Lesen getrost "ubersprungen werden. Die optionalen Regeln dienen dazu, das Spiel zu driften, d.\,h. den Fokus des Spiels etwas zu verschieben. So wird durch diese Regelvarianten das Spiel nicht komplizierter, schwieriger oder profihafter sondern \emph{anders}.

Ich empfehle daher, das Spiel erstmal ohne irgendwelche optionalen Regeln zu spielen und erst nach ein paar Sitzungen dar"uber nachzudenken. Wichtig ist, dass die Gruppe gemeinsam entscheidet, welche optionalen Regeln benutzt werden. Es ist nat"urlich auch m"oglich, sich eigene Hausregeln zu "uberlegen. Ich bin immer an Ideen anderer interessiert und werde sicherlich interessante Regelvorschl"age gerne in diesen Text "ubernehmen.

\section{Beispiele und Designanmerkungen}\index{Beispiel}\index{Designanmerkung}
Auch Beispiele und Anmerkungen zum Design sind im Regeltext immer wieder zu finden. Mit beidem versuche ich, die Regeln n"aher zu erl"autern. Dabei sollen die Beispiele illustrieren, wie ein bestimmter Regeltext gemeint ist; die Designanmerkungen sollen erkl"aren, warum eine Regel so ist und nicht anders.

Dabei k"onnen insbesondere die Designanmerkungen beim Lesen problemlos "ubersprungen werden. Sie sind zum Verst"andnis der Regeln eigentlich nicht notwendig. Vielmehr dienen diese Texte dazu, vielleicht umst"andlich erscheinende Regeln zu rechtfertigen und Spieler davor zu bewahren, Hausregeln auszuprobieren, die sich bereits als ung"unstig herausgestellt haben.

Aber auch die Beispiele sind nicht unbedingt zum Verst"andnis n"otig. Es gibt keine Beispiele, die sich durch das ganze Regelwerk ziehen. Sollten Erkl"arungen klar genug sein, dass man sie auch ohne Beispiel versteht, so kann man problemlos zum n"achsten Abschnitt gehen und dann das Beispiel, was dort zu finden ist, wieder lesen.


\section{Zur Benutzung dieses Textes}
Ich gehe davon aus, dass alle Leser dieses Textes wissen, was ein Rollenspiel ist und speziell das Rollenspiel \emph{Das Schwarze Auge} (DSA) von Ulisses Spiele kennen. Alle Bezüge zu DSA verweisen immer auf die vierte Regelausgabe.

Der Ansatz von StoryDSA unterscheidet sich erheblich von DSA. Es handelt sich nicht nur um ein Abschleifen überflüssiger Regeln oder eine Sammlung von Hausregeln; vielmehr ist das Spielgefühl von StoryDSA ein völlig anderes. Alle tragen gemeinsam zum Spielerlebnis bei, die Charakterspieler können und müssen viele Dinge frei ausschmücken. Auf der anderen Seite ist das Spiel wesentlich strukturierter als DSA und gibt so dem Spielleiter die Möglichkeit, die Geschichte viel stärker zu lenken, ohne heimlich hinter dem Meisterschirm die Würfel zu drehen.

Aus diesem Grund habe ich viel Wert auf Beispiele und ausführliche Erklärungen gelegt. Diese sind es dann, die den ersten Teil des Textes bis Seite \pageref{EndeSpielregeln} in die Länge ziehen. Wer also eine ausführliche Regelerklärung haben möchte, liest den Text also einfach von vorne nach hinten durch. Beim ersten Lesen können die optionalen Regeln einfach übersprungen werden.

Der zweite Teil ist eine spezielle Anleitung für das Ausfüllen der speziellen Rolle am Tisch: Die Rolle des Spielleiters. Der Spielleiter muss den Spielabend vorbereiten und den Rahmen für die Geschichte vorgeben. Ideal aber nicht notwendig ist es, wenn alle Spieler auch diesen Teil lesen, denn dann können die Charakterspieler besser einschätzen und verstehen, was der Spielleiter gerade macht. Es werden keinerlei geheime Informationen über die Welt Aventurien gegeben.

Der dritte und letzte Teil umfasst die letzten Kapitel und gibt eine schnelle Übersicht über die Regeln des ersten Teiles. Er dient vor allem als kompaktes Nachschlagewerk, kann aber auch von Spielern benutzt werden, die sich einen kurzen Überblick über die Ideen und Strukturen von StoryDSA verschaffen wollen. Wer also einen kurzen Blick auf StoryDSA werfen möchte, kann dies hier tun und findet ein paar Beispielcharaktere, Kurzregeln und ein ausführliches Glossar.


\section{Die goldene Regel}\index{Regel!golden}
Im Gegensatz zu den originalen DSA-Regeln gilt hier: Die Regeln gelten für alle -- auch für den Spielleiter. Dieser bekommt alle Vollmachten, die er für das erfolgreiche Leiten eines Spieles benötigt.  Die Regeln sollten zu keinem Zeitpunkt ohne Absprachen außer Kraft gesetzt werden.

Natürlich bedeutet das nicht, dass ich keine Hausregeln möchte. Im Gegenteil: Wer gute Ideen für Veränderungen und Verbesserungen hat, ist gerne gelesen. Dazu bietet sich natürlich das Metstübchen an\footnote{http://metstuebchen.de/}.

\section{Rechnerisches}
In manchen Fällen müssen Zahlen dividiert werden. Das Ergebnis wird immer zur nächsten ganzen Zahl gerundet, d.h. wenn beispielsweise 5 durch 3 geteilt werden soll, ist das Ergebnis 2 (1,666 wird aufgerundet). Wird 5 durch 4 geteilt, ist das Ergebnis 1 (1,25 wird abgerundet). Kniffelig ist eigentlich nur der Fall, wenn genau ein ,5 als Ergenis herauskommt. Dann wird bei StoryDSA immer zur geraden Zahl gerundet.

\begin{beispiel}
\paragraph{Beispiele:} Die folgenden Ergebnisse sind gerundet. $-3:2=-2$. $-2:2=-1$. $-1:2=0$. $0:2=0$. $1:2=0$. $2:2=1$. $3:2=2$. $4:2=2$. $5:2=2$. $6:2=3$. $7:2=4$. $8:2=4$. $9:2=4$. $10:2=5$. $11:2=6$. $12:2=6$.
\end{beispiel}
