\chapter{Freies Spiel}\label{Ch:FreiesSpiel}\index{freies Spiel}
\lettrine{I}{m} freien Spiel können die Spieler nach Herzenslust ihre Charaktere ausspielen. Der SL hat zuvor durch eine SL-Erzählphase dazu die Rahmenbedingungen, wie Ort und anwesende Charaktere, geschaffen. Dann leitet er das freie Spiel mit den Schl"usselworten ``Was wollt ihr machen?'' ein.

Freies Spiel bringt die Story zwar nicht direkt voran, hilft aber, den gemeinsamen Vorstellungsraum auszuschmücken und den Spielern ein besseres `Gefühl' für ihren Charakter zu geben. Auch sollen im freien Spiel Planungen für das weitere Vorgehen gemacht werden oder auch Entscheidungen getroffen werden, wohin sich die Charaktere als nächstes wenden wollen. Hier können die Spieler auch durch ihre Gespräche und Handlungen dem SL Hinweise geben, was sie sich für ihre Charaktere wünschen.

Wichtig: Freies Spiel beinhaltet niemals Konflikte. Natürlich dürfen sich Charaktere im freien Spiel auch eins reinhauen o.\,ä. Diese `Konflikte' werden aber -- wenn sich die Spieler nicht einigen können -- durch den SL aufgelöst und haben keinerlei regeltechnischen Einfluss (auf Charakterwerte usw.). Der SL sollte in solchen Fällen aufgrund des Spielspaßes und der Charakterwerte entscheiden.

Der Spielleiter hat das Recht, das freie Spiel jederzeit zu unterbrechen. Dann kann er mit einer SL-Erzählphase z.\,B. zu einem Konflikt "uberleiten. Das sollte er insbesondere dann machen, wenn die Spieler im freien Spiel versuchen, die Story voran zu treiben und an eine Stelle gelangen, an der ein Konflikt vorgesehen war. Auch kann der SL damit verhindern, dass die Spieler etwas unternehmen, was im Plot nicht vorgesehen ist und kann sie mit einer SL-Erzählphase wieder zur"uck ``auf den rechten Weg'' bringen. Meist wird der SL aber allzu ausschweifende (und auf die Dauer langweilige) Charakter- bzw. Spielerdiskussionen unterbrechen um die Geschichte weiter zu f"uhren.


