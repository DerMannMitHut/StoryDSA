\chapter{Magie}\label{Ch:Magie}\index{Magie}


\lettrine{A}{uch} in Sachen Magie verfolgt \StoryDSA einen erzählerischen Ansatz. Das bedeutet, dass die Zauber und die Auswirkungen vom Spieler beschrieben werden und spielmechanisch im Prinzip nichts anderes passiert als bei der Anwendung eines einfachen Talentes. Trotzdem soll Magie etwas besonderes sein und in gewisser Weise übernatürliche Fähigkeiten erlauben. Dazu wird die bereits aus dem klassischen DSA bekannte Astralenergie genutzt.

\BN
\section{Magie im freien Spiel}
Im freien Spiel kann ein Zauberkundiger problemlos beliebig viel Magie anwenden. Voraussetzung ist dabei wie immer im freien Spiel, dass diese Magie keinen entscheidenden Einfluss auf die Geschichte hat. Das freie Spiel dient ja dazu, die Charaktere auszuspielen, Informationen an die Spieler weiterzugeben und Richtungsentscheidungen zu fällen. Daher kostet auch die Anwendung von Magie im freien Spiel keine Astralpunkte, so dass gerade Elfenspieler die alltägliche Anwendung von Magie ausspielen können, ohne Angst zu haben, dass ihnen im entscheidenden Augenblick dann Astralpunkte fehlen.

Ob die Magie wirkt und welche genauen Auswirkungen sie hat, legt der Spieler einfach während des Spieles fest. Die Effekte, die ein Charakter erzielen kann, muss der Held natürlich auch beherrschen.

Alle Kosten und spieltechnischen Effekte, die in den folgenden Regeln angegeben sind, treten also nur während Konflikten in Kraft.

\section{Spruchmagie}
Spruchmagie ist die Art der Zauberei, die fast alle magisch begabten Wesen in der Welt des Schwarzen Auges miteinander verbindet. Der Zauberkundige kann einen bestimmten magischen Effekt hervorrufen, den sich der Spieler zuvor aus der langen Liste der Zaubersprüche ausgewählt hat.
\EN

\subsection{Talente und Spruchauswahl}\index{Talent}\index{Spruch}\index{Zauberspruch}
Jeder Zauberkundige hat mindestens ein magisches Talent, das nach der Repräsentation oder Schule der Magie benannt ist. So verfügt ein Auelf über das Talent \emph{elfische Magie}, ein Magier aus Punin über \emph{Magier: Punin} und ein Haindruide über \emph{Magie der Haindruiden}. Die zugehörigen Eigenschaften sind für alle magischen Talente KL/IN/CH.

Mit einem magischen Talent kann ein Charakter aber noch keinen Zauber wirken. Die Anzahl der Zauber bzw. magischen Effekte wird durch den Talentgesamtwert begrenzt. Jeder Effekt kostet eine Steigerung. Möchte ein Spieler, dass sein Charakter mehr Effekte beherrscht, so muss er ein zweites magisches Talent steigern und die Effekte für dieses Talent kaufen.

Es ist auch problemlos möglich, erst im Laufe der Abenteurerkarriere ein magisches Talent zu lernen. Wenn ein Charakter das erste magische Talent aktiviert, kann er sofort wie bei der Charaktererschaffung bis zum Talentgesamtwert seine Lebens- und Willenskraft im Verhältnis 1:2 in Astralenergie umwandeln (dabei gilt hier natürlich auch der Mindest-Talentgesamtwert von 5).


\subsection{Neben- und Hauptkonflikte}
In Neben- und Hauptkonflikten kann das magische Talent grundsätzlich genauso eingebracht werden, wie jedes andere Talent auch. Bei der Beschreibung muss sich der Spieler natürlich an der Spruchauswahl orientieren. Gewürfelt wird dann auf das zugehörige magische Talent. Jede Anwendung von Magie kostet 1~AsP.

Mit Magie können aber auch besondere, spieltechnische Effekte erzielt werden. Dadurch wird das Spielen eines magisch begabten Charakters etwas taktischer als eines nicht-magischen Charakters, denn ein Zauberkundiger kann zwar durch seine Magie kurzfristiger besser sein, ist aber ohne Magie im Allgemeinen schlechter als ein vergleichbarer Nicht-Zauberkundiger.

Für Astralpunkte kann der Spieler eines Zauberkundigen zusätzliche Würfel kaufen, auch über das Erzählmaximum von 5 hinaus. Die Kosten können aus Tabelle~\ref{TabelleASPKosten} abgelesen werden. Diese zusätzlichen Würfel kann der Spieler auch an andere Charaktere weitergeben. Das kostet pro zwei weitergegebene Würfel einen AsP.

Die letzte Möglichkeit ist, einem Hauptkonfliktgegner Würfel wegzunehmen. Die Kosten hierfür betragen das dreifache der Kosten, zusätzliche Würfel zu kaufen.

\begin{table}
  \begin{tabular}[C]{|rlrl|}
  \hline
    +0~Würfel & 1~AsP & je 2 Würfel & \\
    +1~W"urfel & 2~AsP & weitergeben & +1~AsP \\ 
    +2~W"urfel & 4~AsP & & \\
    +3~W"urfel & 6~AsP &--1~Würfel & +3~AsP\\
    +4~W"urfel & 9~AsP & --2~Würfel & +9~AsP\\ 
    +5~W"urfel & 12~AsP &--3~Würfel & +15~AsP\\
    +6~W"urfel & 16~AsP &--4~Würfel & +24~AsP \\
    +7~W"urfel & 20~AsP &--5~Würfel & +33~AsP \\
    +8~W"urfel & 25~AsP && \\
    +9~W"urfel & 30~AsP && \\
  \hline
  \end{tabular}
  \caption{ASP-Kosten für die Anwendung von Magie}
  \label{TabelleASPKosten}
\end{table}

\BN Für die Beschreibung ist entscheidend: Auch längerfristige Sprüche geben nur einen einmaligen Würfelvorteil, dauern aber eben längere Zeit an. So wirkt der DSA4-Spruch PLUMBUMBARUM 5 Kampfrunden lang. Beispielsweise beschreibt der Spieler des Magiers, wie sein Held diesen Spruch auf einen Gegner anwendet. Er investiert 10~AsP und senkt die Würfel des Gegners um 2. In der Runde darauf greift der Spieler des Kriegers diese Beschreibung auf und erzählt, wie sein Krieger den gelähmten angreift. Einen besonderen Würfelvorteil erhält er dadurch allerdings nicht mehr.\EN

\subsection{Kurzkonflikte}
In Kurzkonflikten wird Magie genauso gehandhabt, wie in den anderen Konflikten auch. Für den Einsatz eines Magie-Talentes muss der Zauberkundige mindestens einen Astralpunkt ausgeben. Für die üblichen Kosten können weitere Würfel gekauft werden. Diese zählen wie die üblichen Bonuswürfel (vgl. den entsprechenden Abschnitt ab Seite~\pageref{Bonuswuerfel}). Darüberhinaus besteht für einen Zauberkundigen die Möglichkeit, durch die Weitergabe von Würfeln andere Helden in Kurzkonflikten zu unterstützen.

\BN
\section{Rituale}
\EN

\BN
\section{Magische Artefakte}
Bei magischen Artefakten sind zwei Dinge entscheidend: Erstens die Anwendung im Spiel und zweitens die Herstellung. Über den Einsatz von magischen Gegenständen steht mehr im Kapitel Gegenstände, ab Seite~\pageref{Ch:Gegenstaende}. Hier soll es nur um die Herstellung magischer Artefakte gehen.

\subsection{Herstellung magischer Artefakte}
\EN

\section{Astrale Regeneration}
Regeneration der AsP: 1 Punkt pro 12 Stunden nicht Zaubern. Vorteil Astrale Regeneration: 50\,\% Chance (auf W20 1 bis 10), einen weiteren AsP zu regenerieren
\section{Besonderheiten magischer Berufe}

\BN
\subsection{Magier}
\EN

\BN
\subsection{Hexe}
\EN

\subsection{Magiedilletanten}
\BN
Die intuitive Magie der Dilletanten lässt sich nicht mit geschulter Magie kombinieren. Dilletanten sind in der Auswahl und Mächtigkeit ihrer Magie eingeschränkt, bekommen dafür aber die Astralenergie geschenkt, d.h. sie müssen sie nicht gegen Lebens- und Willenskraft tauschen.

Um Magiedilletant zu sein, muss ein Spieler lediglich das Spezialtalent ``Maiedilletant (KL/IN/CH)'' wählen. Insgesamt kann ein Magiedilletant maximal so viele Fähigkeiten wählen, wie die Hälfte seines Talentgesamtwertes in Magiedilletant beträgt. Jeder Effekt kostet 2 Steigerungen. Die Möglichkeiten, die für einen Dilletanten offen stehen, sind die folgenden:

\begin{itemize}
\item Übernatürliche Begabungen. Der Dilletant wählt wie ein normaler Zauberkundiger Sprüche, die er intuitiv einsetzt. Insgesamt dürfen nicht mehr als fünf gewählt werden. Jeder Spruch kostet 2 Steigerungen.
\item Meisterhandwerk. Der Dilletant wählt Talente, für die er genau wie bei Spruchmagie durch den Einsatz von Astralenergie Zusatzwürfel kaufen kann. Maximal kann ein Dilletant fünf Talente wählen; jedes Talent kostet 2 Steigerungen.
\item Schutzgeist (Segen). Der Viertelzauberer kann Würfelwürfe wiederholen. Dazu muss er pro Wiederholungswurf 2 Astralpunkte bezahlen. Sollen mehrere Würfe gleichzeitig wiederholt werden, kostet das 4, 6, 9 usw. (wie bei einfacher Magie für zusätzliche Würfel). Diese Fähigkeit kostet 2 Steigerungen.
\item Schutzgeist (Schutz). Der Viertelzauberer kann Konfliktpunkt-Verlust in AsP-Verlust umwandeln. Pro verhindertem Konfliktpunkt kostet das 2 AsP; pro Konfliktrunde können so nicht mehr als die Hälfte der Punkte (aufgerundet) verhindert werden. Diese Fähigkeit kostet 2 Steigerungen.
\end{itemize}

Ein Magiedilletant bekommt die Hälfte seines Talentgesamtwertes an Astralenergie. Wenn der Spieler das Talent steigert, so steigt die Astralenergie entsprechend.
\EN


