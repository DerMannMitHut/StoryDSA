\chapter{Beispielcharaktere}\label{Ch:Beispielcharaktere}

\FEHLT{Die Beispielcharaktere sind nicht auf dem neuesten Stand}

\lettrine{I}{n} diesem Kapitel werden einige Beispielcharaktere aufgelistet. An ihnen kann sich ein Spieler bei der Erschaffung seines eigenen Charakters orientieren, sie können einfach zum sofort losspielen benutzt werden oder als Vorlage für Spielleitercharaktere dienen.

Jeder Charakter wird ausführlich dargestellt, wie er als frisch erschaffener Abenteurer beginnt.  Es werden allerdings nur die Basis- und Spezialtalente aufgeführt, die vom Standard abweichen, d.\,h. bei allen nicht aufgeführten Basistalente ist der Talentgesamtwert 5. Nicht aufgeführte Spezialtalente sind nicht aktiviert und können daher nicht eingebracht werden.

Darüberhinaus wird noch angegeben, wie sich die Charakterwerte zu Beginn der Stufen 6, 12 und 18 verändert haben (d.\,h. es stehen für die Steigerung insgesamt 15, 66 und 153 Steigerungen zur Verfügung); die Ergänzungen zur Vorgeschichte werden nicht aufgeführt. Zur besseren Übersicht beginnt jede Charakterbeschreibung auf einer neuen Seite. Das Inhaltsverzeichnis auf der nächsten Seite dient dazu, schnell bestimmte Charaktere anhand von Rasse, Herkunft und Profession zu finden.

\begin{tabular}{lrrlrl}
Talente       & 11 & 21 & (+10) & 35 & (+14) \\
Eigenschaften &  0 & 26 & (+26) & 67 & (+41) \\
Konfl.Geg.    &  0 &  4 & (+4)  & 14 & (+10) \\
Rüstung       &  4 &  9 & (+5)  & 21 & (+12) \\
Sonderf.      &  0 &  6 & (+6)  & 16 & (+10) \\
\hline
Summe & 15 & 66 & (+51) & 153 & (+87) \\
\end{tabular}

\newpage\section*{Beispielcharaktere -- Inhaltsverzeichnis}
\noindent\begin{footnotesize}
%
%
\begin{tabularx}{0.30\textwidth}[t]{|lR|}
\hlx{hv}
\bf Rasse & \bf \makebox[0pt][r]{Seite} \\
\hlx{vhv}
Halbelf
			& \pageref{GauklerGareth} \\
Mittelländer 
			& \pageref{SchwertgeselleHavena} \\
			& \pageref{TaugenichtsPunin} \\
			& \pageref{RitterBaliho} \\
Tulamide 
			& \pageref{StreunerAlAnfa} \\
\hlx{vh}
\end{tabularx}
%
%
\hfill
%
%
\begin{tabularx}{0.30\textwidth}[t]{|lR|}
\hlx{hv}
\bf Herkunft & \bf \makebox[0pt][r]{Seite} \\
\hlx{vhv}
Al'Anfa     & \pageref{StreunerAlAnfa} \\
Baliho      & \pageref{RitterBaliho} \\
Gareth      & \pageref{GauklerGareth} \\
Havena      & \pageref{SchwertgeselleHavena} \\
Punin       & \pageref{TaugenichtsPunin} \\
\hlx{vh}
\end{tabularx}
%
%
\hfill
%
%
\begin{tabularx}{0.30\textwidth}[t]{|lR|}
\hlx{hv}
\bf Profession & \bf \makebox[0pt][r]{Seite} \\
\hlx{vhv}
Gaukler & \pageref{GauklerGareth} \\
Ritter & \pageref{RitterBaliho} \\
Schwertgeselle & \pageref{SchwertgeselleHavena} \\
Streuner     & \pageref{StreunerAlAnfa} \\
Taugenichts & \pageref{TaugenichtsPunin} \\
\hlx{vh}
\end{tabularx}
%
%
\end{footnotesize}

\newpage\section{Eillyn Collen}\label{SchwertgeselleHavena}
\subsection{Profession und Herkunft}
Schwertgesellin nach Uinin, aus Havena

\subsection{Charaktergeschichte}
Eillyn wuchs als drittes Kind einer Händler-Familie in Havena auf. Auf den Geschäftsreisen, die sie zusammen mit ihrer Mutter machte, bekam sie die ersten Kontakte zu Schwertgesellen. Sie lernte die unterschiedlichen Stile unterscheiden und wollte nichts sehnlicher, als selbst Schwertgesellin zu werden. Heute ist sie der Stolz ihrer Mutter.

\subsection{Wichtigstes Wesen}
Scanlail ni Uinun, die Gründerin der Havener Kampfschule. Sie kommt knapp vor ihrer Mutter. Scanlail ist Eillyns Vorbild, so wie sie möchte sie auch werden und in einer bedeutenden Stadt eine Kampfschule eröffnen, die den Collener Stil lehrt.

\subsection{Leidenschaften}
Liebe: Zum Kampf und zum Schwert. Verpflichtung: Schwergesellen-Kodex. Angst: Reden vor einem Publikum.

\subsection{Überzeugungen/Prinzipien}
Albernia gehört nicht zum Mittelreich und Rohaja ist eine Tyrannin, die diesen Jast Grosam unterstüzt. Feen und Elfen bergen mehr Geheimnisse, als die Menschen je ergründen können, wobei der Genuss thorwalschen Schnaps in der Lage sein könnte, ein Tor in die Feenwelt aufzustoßen.

\subsection{Vor- und Nachteile}
\begin{description}
\item[Ausbildung: Schwertgesellin] Grundwissen Etikette, Einhandwaffen+2, Grundwissen Tulamidisch (6~Vorteilspunkte)
\item[Gefahreninstinkt:] 1 Bonuswürfel (2~Vorteilspunkte)
\item[Ehrlichkeit:] --2 auf Überreden (+1~Vorteilspunkte)
\item[Aufrichtiger Kampf:] --2 Würfel bei Hinterhältigkeiten (+3~Vorteilspunkte)
\end{description}

\subsection{Eigenschaften}
Mut:~2, KL:~0, IN:~2, CH:~--1, GE:~3, FF:~--1, KO:~1, KK~1

\subsection{Talente}
\begin{description}
\item[Raufen (MU/GE/KK):] 6+3=9
\item[Einfache Wurfgeschosse (IN/FF/KK):] 6+1=7
\item[Athletik (GE/KO/KK):] 6+3=9
\item[Körperbeherrschung (MU/GE/KK):] 6+3=9
\item[Sich verstecken (MU/IN/GE):] 6+4=10
\item[Überreden/Überzeugen (MU/IN/CH):] --2+0+2=0, also 3
\item[Wildnisleben (IN/GE/KO):] 3+3=6
\item[Einhandwaffen (MU/GE/KK):] 4+2+3=9
\item[Schwimmen (GE/KO/KK):] 0+3=3, also 5
\item[Etikette (Wissenstalent):] 1 Bonuswürfel
\item[Sprache Tulamidya (Wissenstalent):] 1 Bonuswürfel
\item[Lesen/Schreiben (Wissenstalent):] 1 Bonuswürfel
\item[Muttersprache Garethi:] 2 Bonuswürfel
\item[Zweitsprache Oloarkh:] 1 Bonuswürfel
\end{description}

\subsection{Weiteres}
\begin{description}
\item[Willenskraft:] 9
\item[Lebenskraft:] 12
\item[Konfliktpunkte:] 3
\end{description}

\subsection{Konfliktgegenstände und Rüstung}
\begin{description}
\item[Gegenstand:] Schwert und Schild, aus der Hausschmiede der Havener Schule (2~Bonuswürfel)
\item[Rüstung:] Lederrüstung aus der Hausschmiede der Schule (3~Rüstungsschutz)
\end{description}

\subsection{Geänderte Werte in Stufe 6}
\begin{description}
\item[Athletik (GE/KO/KK):] 8+3=11
\item[Körperbeherrschung (MU/GE/KK):] 8+3=11
\item[Wildnisleben (IN/GE/KO):] 8+3=11
\item[Einhandwaffen (MU/GE/KK):] 6+2+3=11
\item[Rüstung:] Metallverstärkte Lederrüstung, Geschenk zur Belohnung (4~Rüstungsschutz)
\item[Konfliktpunkte:] 4
\end{description}


\subsection{Geänderte Werte in Stufe 12}
\begin{description}
\item[Eigenschaften:] Mut:~2, KL:~0, IN:~2, CH:~--1, GE:~4, FF:~--1, KO:~1, KK~2
\item[Raufen (MU/GE/KK):] 7+4=11
\item[Einfache Wurfgeschosse (IN/FF/KK):] 6+2=8
\item[Athletik (GE/KO/KK):] 9+4=13
\item[Körperbeherrschung (MU/GE/KK):] 9+4=13
\item[Sich verstecken (MU/IN/GE):] 9+4=13
\item[Wildnisleben (IN/GE/KO):] 9+4=13
\item[Einhandwaffen (MU/GE/KK):] 7+2+4=13
\item[Schwimmen (GE/KO/KK):] 6+4=10
\item[Gegenstand:] Schwert und Schild; gefunden in einer Zwergenmine (3~Bonuswürfel)
\item[Rüstung:] Metallverstärkte Lederrüstung, Geschenk zur Belohnung + Schutzamulett (5~Rüstungsschutz)
\item[offensive Sonderfertigkeiten:] Wuchtschlag (Raufen, Einhandwaffen)
\item[Lebenskraft:] 14
\item[Konfliktpunkte:] 5
\end{description}

\subsection{Geänderte Werte in Stufe 18}
\begin{description}
\item[Eigenschaften:] Mut:~2, KL:~0, IN:~2, CH:~--1, GE:~4, FF:~--1, KO:~4, KK~4
\item[Raufen (MU/GE/KK):] 11+5=16
\item[Einfache Wurfgeschosse (IN/FF/KK):] 6+3=9
\item[Athletik (GE/KO/KK):] 11+6=17
\item[Körperbeherrschung (MU/GE/KK):] 11+5=16
\item[Sich verstecken (MU/IN/GE):] 11+4=15
\item[Wildnisleben (IN/GE/KO):] 11+5=16
\item[Einhandwaffen (MU/GE/KK):] 9+2+5=16
\item[Schwimmen (GE/KO/KK):] 11+6=17
\item[Anatomie (Wissenstalent):] 2 Bonuswürfel
\item[Gegenstand:] Legendäres Schwert 'Feindhacker' (4~Bonuswürfel)
\item[Rüstung:] Metallverstärkte Lederrüstung, Geschenk zur Belohnung + Schutzamulett (verstärkt) (6~Rüstungsschutz)
\item[offensive Sonderfertigkeiten:] Hammerschlag (Raufen, Einhandwaffen, aufgestockt), Waldkundig (Athletik, Körperbeherrschung, Wildnisleben, aufgestockt)
\item[defensive Sonderfertigkeiten:] Meisterparade (Raufen, Einhandwaffen, aufgestockt)
\item[Lebenskraft:] 19
\item[Konfliktpunkte:] 6
\end{description}



\newpage\section{Gissa}\label{StreunerAlAnfa}
\subsection{Profession und Herkunft}
Tulamidische Streunerin aus Al'Anfa

\subsection{Charaktergeschichte}
Gissa, ein Findelkind, hatte das Glück von Schlossermeister Hrubusan nicht als Sklave verkauft worden zu sein. Noch während ihrer Lehre wurde sie von einer Einbrecherbande verführt, geheime Nachschlüssen zu fertigen und zu verkaufen. Schlussendlich wurde sie von Meister Hrubusan auf die Straße gesetzt und hat gelernt, sich mit ihrem Können durchzuschlagen.

\subsection{Wichtigstes Wesen}
Schlossermeister Hrubusan. Sie war bei ihm wie bei einem Vater aufgewachsen. Mittlerweile tut es ihr leid, sein Vertrauen ausgenutzt zu haben.

\subsection{Leidenschaften}
Gissa ist (mittlerweile) eine ehrliche Haut, die sich für ihre Vergangenheit schämt. Ihr große Hilfsbereitschaft wurde von der Einbrecherbande benutzt. Außerdem hat sie große Angst vor Spinnen.

\subsection{Überzeugungen/Prinzipien}
Trotz ihres Reinfalls mit den Einbrechern ist sie davon überzeugt, dass man jedem helfen sollte, der in Not ist, denn dann wird einem das Schicksal gut gesonnen sein, wenn man selbst in Not ist. Zudem heiligt ein guter Zweck alle Mittel.

\subsection{Vor- und Nachteile}
\begin{description}
\item[Abgebrochene Schlosserausbildung:] Lesen/Schreiben, Mechanik, Talent Schlösser knacken aktivieren und Berufstalent Schlosser (6~Vorteilspunkte)
\item[Ruhe und Gelassenheit:] +1 Würfel (2 Vorteilspunkte)
\item[Angst vor Spinnen:] 1 Maluswürfel (+2 Vorteilspunkte)
\item[Stadtkind:] 1 Maluswürfel bei direktem Naturzusammenhang (+2 Vorteilspunkte)

\end{description}

\subsection{Eigenschaften}
Mut:~0, KL:~--1, IN:~1, CH:~3, GE:~2, FF:~3, KO:~--1, KK~--1

\subsection{Talente}
\begin{description}
\item[Dolche (MU/GE/FF):] 5+3=8
\item[Armbrust (IN/FF/FF):] 6+4=10
\item[Gassenwissen (KL/IN/CH):] 6+2=8
\item[Sinnenschärfe (KL/IN/IN):] 6+1=7
\item[Überreden/Überzeugen (MU/IN/CH):] 6+2=9
\item[Bastelei (IN/FF/FF):] 6+2=8
\item[Schlösser knacken (IN/FF/FF):] 6+4=10
\item[Fallen entschärfen (IN/FF/FF):] 6+4=10
\item[Lesen/Schreiben (Wissenstalent):] 1 Bonuswürfel
\item[Muttersprache (Wissenstalent):] 2 Bonuswürfel
\item[Zweitsprache (Wissenstalent):] 1 Bonuswürfel
\item[Menschenkenntnis (Wissenstalent):] 1 Bonuswürfel
\item[Schlosser (Berufstalent):] 10
\end{description}

\subsection{Konfliktgegenstände und Rüstung}
\begin{description}
\item[Gegenstand:] Dietrich-Set (1~Bonuswürfel)
\item[Gegenstand:] Werkzeug-Kasten (1~Bonuswürfel)
\end{description}

\subsection{Veränderungen in Stufe 6}
(+15 Steigerungen, max. TaW 8, max. Wissen 2, max. Gegenstand 1, max. Rüstung 4, Berufstalent +2)

\subsection{Veränderungen in Stufe 12}
(+51 Steigerungen, max. TaW 9, max. Wissen 3, max. Gegenstand 2, max. Rüstung 5, Berufstalent +5)

\subsection{Veränderungen in Stufe 18}
(+87 Steigerungen, max. TaW 11, max. Wissen 3, max. Gegenstand 3, max. Rüstung 6, Berufstalent +8)




\newpage\section{Selindio da Vanya}\label{TaugenichtsPunin}
\subsection{Profession und Herkunft}
Taugenichts aus Punin

\subsection{Charaktergeschichte}
Punin. Stadt der Städte, Schmelztigel der Kulturen, Perle am Yaquir. Und Hort der Langeweile, auch Academie der Hohen Magie genannt. Welch Freude, als ich ihr entkam! Vater sprach von großer Schmach und Schande, ich von der Befreiung einer Last, die ich nie tragen konnte. Welt -- ich komme! Vater -- auf Nimmerwiedersehen.

\subsection{Wichtigstes Wesen}
Ich. Was sollte es wichtigeres geben?

\subsection{Leidenschaften}
Liebt die persönliche Freiheit. Hat Angst, seinem Vater unter die Augen zu treten.

\subsection{Überzeugungen/Prinzipien}
Magie ist stinklangweilig und wirkt nur auf Nicht-Eingeweihte schwierig. Vorurteile stimmen eigentlich niemals. Zu lange an einem Ort ist langweilig.

\subsection{Vor- und Nachteile}
\begin{description}
\item[Gefahreninstinkt:] 1~Bonuswürfel bei unvorhersehbaren Gefahren (2~Vorteilspunkte)
\item[Adlige Abstammung:] 1~Bonuswürfel bei Verhandlungen im Adel (2~Vorteilspunkte)
\end{description}

\subsection{Eigenschaften}
Mut:~2, KL:~1, IN:~2, CH:~3, GE:~1, FF:~--1, KO:~0, KK~0

\subsection{Talente}
\begin{description}
\item[Raufen (MU/GE/KK):] 6+2=8
\item[Überreden/Überzeugen (MU/IN/CH):] 6+4=10
\item[Betören/Galanterie (IN/CH/CH):] 6+4=10
\item[Gassenwissen (KL/IN/CH):] 6+3=9
\item[Schaspielerei (MU/KL/CH):] 6+4=10
\item[Heilkunde Seele (KL/IN/CH):] 6+3=9
\item[Etikette (Wissenstalent):] 1 Bonuswürfel
\item[Menschenkenntnis (Wissenstalent):] 2 Bonuswürfel
\item[Lesen/Schreiben (Wissenstalent):] 1 Bonuswürfel
\item[Geschichtskunde (Wissenstalent):] 1 Bonuswürfel
\item[Sagen/Legenden (Wissenstalent):] 1 Bonuswürfel
\end{description}

\subsection{Konfliktgegenstände und Rüstung}
keine


\subsection{Veränderungen in Stufe 6}
(+15 Steigerungen, max. TaW 8, max. Wissen 2, max. Gegenstand 1, max. Rüstung 4, Berufstalent +2)

\subsection{Veränderungen in Stufe 12}
(+51 Steigerungen, max. TaW 9, max. Wissen 3, max. Gegenstand 2, max. Rüstung 5, Berufstalent +5)

\subsection{Veränderungen in Stufe 18}
(+87 Steigerungen, max. TaW 11, max. Wissen 3, max. Gegenstand 3, max. Rüstung 6, Berufstalent +8)

\newpage\section{Winobert Wackernagel}\label{GauklerGareth}
\subsection{Profession und Herkunft}
Der wunderbare Wino, halbelfischer Gaukler aus Gareth

\subsection{Charaktergeschichte}
Winos elfischer Vater starb in Tobrien auf der Flucht vor Galottas Schergen; damals war er noch ein Kind. In Gareth schloss sich seine Mutter der Gauklertruppe `Spektakulatius' an, bei der er als Artist und Jongleur auftrat. Aufgrund eines Wahrsagerspruches lehnte Wino ab, mit einer spektakulären Hochseilnummer aufzutreten und musste die Gauklertruppe verlassen.

\subsection{Wichtigstes Wesen}
Seine Ratte Alrik. Zu der hat er Vertrauen.

\subsection{Leidenschaften}
Zirkus und Gaukeleien sind seine Leidenschaft. Er hasst die schwarzen Lande und deren Auswüchse. Er möchte berühmt werden und hören, wie die Barden über ihn dichten.

\subsection{Überzeugungen/Prinzipien}
Seit seines Wahrsagerspruches glaubt Wino, dass er bei seinem nächsten Hochseilakt sterben wird. Zeige, was du kannst, dann erinnern sich die Leute an dich.


\subsection{Vor- und Nachteile}
\begin{description}
\item[Viertelzauberer:] Gauklermagie Aktivierung und +2 (3~Vorteilspunkte)
\item[Gutaussehend:] 1 Bonuswürfel (+2~Vorteilspunkte)
\item[Kann kein Blut sehen:] Heilkunde Wunden --1 (+1~Vorteilspunkt)
\end{description}

\subsection{Eigenschaften}
Mut:~2, KL:~--1, IN:~2, CH:~--1, GE:~3, FF:~--1, KO:~0, KK~2

\subsection{Talente}
\begin{description}
\item[Raufen (MU/GE/KK):] 6+4=10
\item[Athletik (GE/KO/KK):] 6+3=9
\item[Körperbeherrschung (MU/GE/KK):] 6+4=10
\item[Selbstbeherrschung (MU/KO/KK):] 6+2=8
\item[Sich verstecken (MU/IN/GE):] 6+4=10
\item[Heilkunde Wunden (KL/CH/FF):] --1+--1=--2, also 4
\item[Gauklermagie (MU/IN/CH):] 2+6+2=10

\textbf{Magische Effekte:} Flamme, kleinen Gegenstand schweben lassen, Gegenstand unsichtbar machen

\item[Reiten (CH/GE/KK):] 3+2=5
\item[Schwimmen (GE/KO/KK):] 3+3=6
\item[Jonglieren (Berufstalent):] 10
\item[Muttersprache: Garethi (Wissenstalent):] 2 Bonuswürfel
\item[Sprache: Elfisch (Wissenstalent):] 1 Bonuswürfel
\end{description}

\subsection{Konfliktgegenstände und Rüstung}
\begin{description}
\item[Gegenstand:] Jonglier-Ausrüstung (Bälle, Tücher, Keulen, Papierblume, \dots) (1~Bonuswürfel)
\end{description}

\subsection{Veränderungen in Stufe 6}
(+15 Steigerungen, max. TaW 8, max. Wissen 2, max. Gegenstand 1, max. Rüstung 4, Berufstalent +2)

\subsection{Veränderungen in Stufe 12}
(+51 Steigerungen, max. TaW 9, max. Wissen 3, max. Gegenstand 2, max. Rüstung 5, Berufstalent +5)

\subsection{Veränderungen in Stufe 18}
(+87 Steigerungen, max. TaW 11, max. Wissen 3, max. Gegenstand 3, max. Rüstung 6, Berufstalent +8)



\newpage\section{Wittmar Almund von Edeneichen}\label{RitterBaliho}
\subsection{Profession und Herkunft}
Ritter aus Baliho

\subsection{Charaktergeschichte}
Witmar Aldmund von Edeneichen, ein untadeliger Ritter aus Weiden, hat noch nie etwas getan, für das er sich schämen müsste. Stolz hält Witmar die Tugenden des Rittertums hoch: Tapfer beschützt Aldmund die Hilflosen, bekämpft ruchlose Verbrecher mit dem Schwert und er spricht nur die Wahrheit. Doch Witmar weiß noch nicht, dass sein Stand nur erschwindelt ist.

\subsection{Wichtigstes Wesen}
Friedwind von Hohenacker ist die Verkörperung all dessen, was Witmar unter einem schlechten Retter versteht. Schon in ihren Knappenzeit waren sie Kontrahenden. Friedwind besiegte auf einem Turnier Witmar mit der Lanze.

\subsection{Leidenschaften}
Liebe zu den Tugenden des Ritters: Beschütze die Hilflosen. Sprich nur die Wahrheit. Sei tapfer.

\subsection{Überzeugungen/Prinzipien}
Seit zehn Generationen sind meine Vorfahren Freiherren und Ritter von Edeneichen. Der Lanzengang ist die edelste Art seinen Streit zwischen auszutragen. Geld ist unwichtig.

\subsection{Vor- und Nachteile}
\begin{description}
\item[Ritterbrief:] 1 Bonuswürfel (2~Vorteilspunkte)
\item[Ausbildung:] Grundwissen Etikette, Rechtskunde, Schrift (Garethi), Heraldik (8~Vorteilspunkte)
\item[Stand Freiherr:] 1 Bonuswürfel (2~Vorteilspunkte)
\item[Ehrlichkeit:] --3 auf Überreden (+3~Vorteilspunkte)
\item[Prinzipientreue:] 1 Maluswürfel (+2~Vorteilspunkte)
\item[Nicht heimlich:] --3 Verstecken (+3~Vorteilspunkte)
\end{description}

\subsection{Eigenschaften}
Mut:~2, KL:~--1, IN:~0, CH:~2, GE:~3, FF:~--1, KO:~1, KK~3

\subsection{Talente}
\begin{description}
\item[Athletik (GE/KO/KK):] 6+4=10
\item[Selbstbeherrschung (MU/KO/KK):] 6+3=9
\item[Sich verstecken (MU/IN/GE):] --3+0+3=0, also 2
\item[Überreden/Überzeugen (MU/IN/CH):] --3+0+2=--1, also 2
\item[Einhandwaffen (MU/GE/KK):] 6+4=10
\item[Lanzenreiten (MU/GE/KK):] 6+4=10
\item[Reiten (CH/GE/KK):] 6+4=10
\item[Galanterie (IN/CH/CH):] 2+6=8
\item[Rechtskunde (Wissenstalent):] 1 Bonuswürfel
\item[Etikette (Wissenstalent):] 1 Bonuswürfel
\item[Schrift: Garethi (Wissenstalent):] 1 Bonuswürfel
\item[Heraldik/Staatskunde (Wissenstalent):] 1 Bonuswürfel
\item[Muttersprache: Garethi (Wissenstalent):] 1 Bonuswürfel
\item[Sprache: Bosparano (Wissenstalent):] 1 Bonuswürfel
\end{description}

\subsection{Konfliktgegenstände und Rüstung}
\begin{description}
\item[Gegenstand:] Pferd, prächtiges Tier, teurer Sattel (1~Bonuswürfel)
\item[Gegenstand:] Gutes Schwert (1~Bonuswürfel)
\item[Rüstung:] Kettenhemd, Helm, Panzerhandschuhe, 3~RS (3~Rüstungsschutz)
\end{description}

\subsection{Veränderungen in Stufe 6}
(+15 Steigerungen, max. TaW 8, max. Wissen 2, max. Gegenstand 1, max. Rüstung 4, Berufstalent +2)

\subsection{Veränderungen in Stufe 12}
(+51 Steigerungen, max. TaW 9, max. Wissen 3, max. Gegenstand 2, max. Rüstung 5, Berufstalent +5)

\subsection{Veränderungen in Stufe 18}
(+87 Steigerungen, max. TaW 11, max. Wissen 3, max. Gegenstand 3, max. Rüstung 6, Berufstalent +8)




\newpage\section{Charaktername}\label{Charaktername}
\subsection{Profession und Herkunft}

\subsection{Charaktergeschichte}

\subsection{Wichtigstes Wesen}

\subsection{Leidenschaften}

\subsection{Überzeugungen/Prinzipien}

\subsection{Vor- und Nachteile}
\begin{description}
\item[Vorteil:] Beschreibung (2~Vorteilspunkte)
\item[Nachteil:] Beschreibung (+2~Vorteilspunkte)
\end{description}

\subsection{Eigenschaften}
Mut:~1, KL:~1, IN:~1, CH:~1, GE:~1, FF:~1, KO:~1, KK~1

\subsection{Talente}
\begin{description}
\item[Talent (AA/BB/CC):] ?+?=?
\item[Talent (Wissenstalent):] ? Bonuswürfel
\item[Talent (Berufstalent):] 10
\end{description}

\subsection{Konfliktgegenstände und Rüstung}
\begin{description}
\item[Gegenstand:] Beschreibung (1~Bonuswürfel)
\item[Rüstung:] Beschreibung (?~Rüstungsschutz)
\end{description}

\subsection{Veränderungen in Stufe 6}
(+15 Steigerungen, max. TaW 8, max. Wissen 2, max. Gegenstand 1, max. Rüstung 4, Berufstalent +2)

\subsection{Veränderungen in Stufe 12}
(+51 Steigerungen, max. TaW 9, max. Wissen 3, max. Gegenstand 2, max. Rüstung 5, Berufstalent +5)

\subsection{Veränderungen in Stufe 18}
(+87 Steigerungen, max. TaW 11, max. Wissen 3, max. Gegenstand 3, max. Rüstung 6, Berufstalent +8)




