\chapter{Konflikte}\label{Ch:Konflikte}\index{Konflikt}
\lettrine[findent=1em,lraise=0.1]{K}{onflikte} sind das Salz in der Suppe eines Rollenspiels und sollten neben freiem Rollenspiel auch den Großteil einer Sitzung ausmachen. Mit Konflikten wird die Geschichte vorangetrieben, für die Helden steht etwas auf dem Spiel. Daher wird durch Konflikte das Spiel spannend.

Mit Konflikten ist jetzt nicht gemeint, dass es sich dabei zwangsweise um einen Kampf handelt. Konflikte sind alle Situationen, in denen (objektiv gesehen) nicht von vorne herein klar ist, ob das gewünschte Ergebnis eintritt. Also praktisch alles, wofür man normalerweise bei DSA einen Würfelwurf macht, ist ein Konflikt.

Ein Konflikt kann das Umgehen einer Wache sein (durch Schleichen, Bestechung oder auch durch von hinten niederstechen), es kann eine Verfolgungsjagd sein oder auch das Suchen von Informationen in einer Bibliothek. Aber natürlich ist auch ein schnöder Kampf ein Konflikt.

Bei Konflikten sollte der Spielleiter immer nur das Ziel vor Augen haben, nie aber die Methode. Im Beispiel oben habe ich ja auch angedeutet: eine Wache kann auf die verschiedensten Arten umgangen werden, wichtig ist für die Geschichte am Ende nur, ob das Ziel erreicht wurde oder nicht.

Außerdem muss sich der Spielleiter über die Rolle, die ein anstehender Konflikt spielt, Gedanken machen. Ist es ein Konflikt, der auf dem Weg zum Abenteuerende steht? Ist es wichtig, dass die Helden das Ziel erreichen? Wenn ja, dann sollte er den Konflikt als Nebenkonflikt modellieren. Ist hingegen der Konflikt ein Scheideweg für die Protagonisten oder ist sowohl das Gewinnen als auch das Verlieren eines Konfliktes interessante Optionen, sollte der Meister daraus einen Hauptkonflikt machen.

\section{Relevante Talente, Sonderfertigkeiten und Gegenstände}
In einem Konflikt kann im Prinzip jegliche Art von Talent benutzt werden. Dennoch gibt es Talente, die für einen Konflikt relevant sind und andere, die es nicht sind. Im Zweifelsfall entscheidet der Spielleiter, ob ein Talent in einem Konflikt relevant ist. Nur, wenn der Charakterspieler ein relevantes Talent einsetzt, darf er für den Konfliktausgang würfeln.

\begin{beispiel}
Beispiele für relevante Talente:
\begin{itemize}
  \item Eine Mauer erklimmen: Athletik oder Körperbeherrschung
  \item Der hübschen Wirtstochter Informationen entlocken: Betören, Überreden, Zechen, Tanzen, Singen
  \item Die Orkbande töten: alle Arten von Kampftalenten
\end{itemize}
\end{beispiel}

Zu den Talenten kann der Spieler eventuell noch passende Sonderfertigkeiten wählen. Pro Talent darf der Spieler aber nicht mehr als je eine offensive und eine defensive Sonderfertigkeit (das können auch aufgestockte Sonderfertigkeiten sein) einbringen. Sonderfertigkeiten geben immer die Möglichkeit zu einem meisterlichen Wurf.

Auch ist die Verwendung von Gegenständen erlaubt und gibt zusätzliche Würfel oder sogar automatische Erfoge. Üblicherweise kann aktiv nur ein Gegenstand eingesetzt werden (z.\,B. eine Waffe). Wirken Gegenstände passiv, so können diese zusätzlich benutzt werden (z.\,B. eine Rüstung oder ein Schild).

Um die Boni von Sonderfertigkeiten oder Gegenständen zu bekommen, müssen diese in den Beschreibungen mit einfließen. Dabei reicht es aber, wenn der Spieler das nur implizit erwähnt. Beispielsweise kann ein Charakter in der ersten Runde sein Schwert ziehen. Wenn der Spieler dann in der nächsten Runde beschreibt, wie der Charakter zuschlägt, kann man davon ausgehen, dass er das mit dem gezogenen Schwert tut.

Natürlich muss ein Charakter nicht immer während des gesamten Konfliktes dasselbe Talent mit denselben Sonderfertigkeiten und denselben Gegenständen einbringen. Das hängt immer von der Beschreibung ab, die die Spieler liefern. Die Nutzung von mehreren Talenten gleichzeitig ist jedoch nicht möglich.

Andererseits sind auch nicht unbedingt während des gesamten Konfliktes immer dieselben Talente relevant: Sollte z.\,B. eine Diskussion kippen und in einer Schlägerei enden, so ist Überreden/Überzeugen anfangs noch interessant, nachdem aber erst einmal die Fäuste fliegen ist es nicht mehr sehr angebracht. Wie immer liegt das im Ermessen des Spielleiters.

\section{Ziele}
Im normalen DSA-Spiel bleiben die Ziele, zu denen ein Konflikt aus Sicht der Helden führen soll, meist unerwähnt. Als Beispiel betrachte ich einmal einen typischen Grund für eine Probe, wie sie normalerweise bei DSA gemacht wird: `Schaffe ich es, über die Mauer zu klettern?' Klar: Es wird auf Athletik gewürfelt und je nach Probenausgang ist die Aktion gelungen oder auch nicht (oder sie ist, wenn der Meister es so entscheidet, nur halb gelungen, auf das Ergebnis kommt es hier aber nicht an). Ist die Probe misslungen, so versucht es der Spieler entweder nochmal oder er probiert was anderes. Z.\,B. die Frage: `Schaffe ich es, mit meiner Spitzhacke ein Loch in die Mauer zu machen, so dass ich da durch passe?'

Und genau das ist der Punkt: Eigentlich geht es doch normalerweise gar nicht darum, über die Mauer zu klettern. Normalerweise geht es darum, die Mauer zu überwinden, und dabei spielt die Methode nur eine untergeordnete Rolle. Im normalen DSA-Spiel wird immer Methode und Ziel miteinander verknüpft. Im Beispiel ist das (ungenannte) Ziel: `Schaffe ich es, die Mauer zu überwinden?' Und diese Frage wird zunächst fest mit der Methode `Klettern', dann im zweiten Versuch mit der Methode `Hacken' verbunden. Es gibt natürlich noch viele andere Methoden und das normale DSA-Spiel geht davon aus, dass der Spieler alle diese Methoden hintereinander ausprobiert, bis eine Methode gelingt.

In diesem Spiel steht aber zunächst immer die Frage nach dem eigentlichen Ziel im Vordergrund. Was will der Spieler bzw. sein Held erreichen? Wo will er wirklich ankommen? Und genau das wird als \DEF{Ziel}\index{Ziel}\index{Konfliktziel|see{Ziel}} bezeichnet. Manchmal ist auch das Gelingen einer Methode das Ziel (z.\,B. ein Schwertkampf auf einem Ritterturnier; da geht es darum den Gegner mit dem Schwert zu besiegen), aber das ist eher die Ausnahme.

Während des Konfliktes werden (wenn es nicht ein Kurzkonflikt ist) mehrere Würfe gemacht und durch die Erzählte Wahrheit kann jeder Spieler den Weg versuchen, den geeigneten Weg zur Hindernisüberwindung zu wählen. Daher ist es wichtig, wirklich das Ziel und nicht vor allem die Methode vor Augen zu haben.

\section{Schaden}\label{Schaden}\index{Schaden}
Während der Konflikte kann ein Charakter \DEF{Schaden}\index{Schaden} bekommen. Wie viel Schaden genau vergeben wird, wird bei den einzelnen Konfliktarten beschrieben. Dabei ist mit Schaden nicht ausschließlich eine Wunde gemeint -- Schaden kann es auch durch Gespräche, Stress u.\,ä. geben. Grundsätzlich gibt es zwei Arten von Schaden: \DEF{Körperlichen}\index{Schaden!körperlich} und \DEF{geistigen Schaden}\index{Schaden!geistig}.

Der Schaden wird immer nach Ablauf eines Konfliktes bestimmt. Dabei gilt: Je länger ein Konflikt andauert, umso größer ist im Mittel der verursachte Schaden. Für jede Runde, die ein Konflikt andauert, und jeden verlorenen Konfliktpunkt wird 1W20 geworfen. Jeder Würfel, der 9 oder mehr (Nebenkonflikte\index{Nebenkonflikt}) bzw. 5 oder mehr (Hauptkonflikte\index{Hauptkonflikt}) zeigt, ergibt einen Schadenspunkt.

Kurzkonflikte sind hier relativ ungefährlich. Nur, wenn Schaden auch wirklich auf dem Spiel steht, können Charaktere durch einen verlorenen Konflikt Schaden bekommen. 

Welche Art von Schaden gemacht wird, geht aus dem Zusammenhang hervor. Körperlicher Schaden\index{Schaden!körperlich} sind z.\,B. Verletzungen oder Schmerzen durch äußere Gewalteinwirkung. Geistiger Schaden\index{Schaden!geistig} hingegen kann sich in Ungeduld, Unaufmerksamkeit, Schamgefühl, Angst usw. äußern. Im Zweifelsfall legt der Spielleiter die Art des Schadens fest.

\section{Konfliktausgang}\index{Konfliktausgang}
Gewonnene Konflikte haben (außer evtl. Schaden) niemals negative Auswirkungen für die Helden. Ist ein Konflikt gewonnen, heißt das, das die Helden ihr Ziel\index{Ziel} erreicht haben.

Bei verlorenen Konflikten müssen Haupt-, Nebenkonflikte und Kurzkonflikte unterschieden werden. In Kurz-\index{Kurzkonflikt} und Nebenkonflikten\index{Nebenkonflikt} sollten folgende zwei Regeln beachtet werden:
\begin{enumerate}
\item[\#1:] Die beteiligten Charaktere versagen nicht auf ganzer Linie.
\item[\#2:] Der Misserfolg sollte einen weiteren Konflikt nach sich ziehen.
\end{enumerate}
Das bedeutet jetzt nicht, dass die Helden in Nebenkonflikten nicht versagen können. Es ist problemlos möglich, dass sie ihr Ziel nicht erreichen -- das kommt ganz auf den SL und die geplante Geschichte an. Jedoch sollen die Charaktere in Nebenkonflikten nicht entgültig verlieren und es muss eine plausible Möglichkeit geben, wie die Geschichte weiter verläuft. Aber: Misserfolge sollen den Helden größere Probleme einbringen, wenn möglich \DEF{Folgekonflikte}\index{Folgekonflikt}. Dabei handelt es sich einfach um Konflikte, die sich daraus ergeben, dass ein Konflikt verloren wurde. Folgekonflikte sind niemals wichtiger als ihre Ursache, d.\,h. ein Folgekonflikt zu einem Nebenkonflikt ist entweder wieder ein Nebenkonflikt oder ein Kurzkonflikt, aber kein Hauptkonflikt.

Misslungene Hauptkonflikte\index{Hauptkonflikt} bedeuten einen ernsthaften Verlust auf der Seite der Charaktere. Das kann der Tod eines Helden sein, aber auch einfach das Scheitern an der gestellten Aufgabe. So beendet ein verlorener Hauptkonflikt üblicherweise nicht das gesamte Spiele. Der Ruhm der Helden wird jedoch nicht zunehmen, so dass es für diesen Teil des Spieles dann keine Abenteuerpunkte gibt.

\section{Wiederholung von verlorenen Konflikten}\index{Wiederholung}

Als Spieler könnte man auf die Idee kommen, einen verlorenen Konflikt zu wiederholen um ein verfehltes Ziel doch noch zu erreichen. Das ist aber nicht so ohne weiteres möglich. Ein verlorener Konflikt bedeutet, dass der Charakter alles versucht hat, sein Ziel zu erreichen. Es ist nicht so, dass nur der erste Versuch danebengegangen ist, sondern dass alle Möglichkeiten ausgereizt wurden.

Daher muss sich an den Umständen, die zum Ziel führen, etwas wesentlich anders sein. So kann der Held z.\,B. Unterstützung durch seine Kameraden bekommen, der Konflikt findet an einem völlig anderen Ort statt oder es hat sich etwas anderes grundlegend an den Voraussetzungen geändert.

Außerdem muss eine Wiederholung auch von den äußeren Umständen her erlaubt sein. So bedeutet ein verlorener Konflikt für einen Verfolgten zumeist, dass er eingeholt wurde. Dass man diesen Konflikt dann auf keinen Fall wiederholen kann (höchstens durch einen erneuten Fluchtversuch) versteht sich von selbst.

\begin{beispiel}
\textbf{Beispiele:}
\begin{itemize}
  \item Ein zweiter Athletik-Konflikt (bei demselben Hindernis) steht einem Helden z.\,B. dann zu, wenn er es besser ausgerüstet erneut versucht oder wenn sich der Athletik-Talentgesamtwert in der Zwischenzeit verbessert hat.

  \item Um eine Person doch noch zur Herausgabe eines wichtigen Gegenstandes zu bewegen, kehrt der Held mit seinen Freunden zur Unterstüzung zurück.
\end{itemize}
\end{beispiel}

\begin{design}
\subsubsection{Designanmerkung: Konfliktwiderholungen}
Warum kann man nicht einfach einen Konflikt nochmal auswürfeln? Das hat zwei Gründe:
\begin{enumerate}
  \item Die Konflikte werden dadurch für die erzählte Geschichte bedeutend. Ein verlorer Konflikt hat Auswirkungen, ein Konflikt ist immer ein wichtiges Ereignis. Umgekehrt wird so der Spielleiter dazu gezwungen, sich zu überlegen, ob tatsächlich ein Konflikt vorliegt oder ob es nur darum geht, den Spieler nochmal würfeln zu lassen. Vor allem auch die Auswirkungen eines Konfliktes müssen überlegt sein, denn wie langweilig wäre ein verlorener Konflikt, der keine Auswirkungen hat?
  \item Würfelorgien werden vermieden. Oft passiert es im klassischen Spiel, dass die Spieler z.\,B. eine Suchen-Probe so oft wiederholen, bis sie etwas gefunden haben. Da wäre es doch besser von Anfang an zu beschreiben, wie der Charakter den gesuchten Gegenstand findet, ohne das Spiel durch eine Probe zu unterbrechen.
\end{enumerate}
\end{design}



\section{Kurzkonflikt}\index{Kurzkonflikt}\label{Sec:Kurzkonflikt}
Kurzkonflikte sind noch `unwichtiger' als Nebenkonflikte, d.\,h. im wesentlichen sind es Schwierigkeiten, die überwunden werden sollten oder keine entscheidende Auswirkung auf die Geschichte haben. Geeignet für Kurzkonflikte sind Situationen, die schnell wieder vorbei sind oder solche, die zwar länger dauern aber in ihrem Verlauf prinzipiell nicht besonders interessant sind. Eine weitere Anwendung ergibt sich in Situationen, die gut ausgespielt werden können und bei denen viel Würfelei nur stört. In jedem Fall sollte nur ein Spielercharakter am Konflikt beteiligt sein bzw. jeder SC muss die Situation individuell meistern.

\begin{design}
\subsubsection{Designanmerkung: Wozu Kurzkonflikte?}
Kurzkonflikte sind von den Regeln her vor allem für die Anwendung von Heilungstalenten vorgesehen. Das Wiederherstellen der Spielercharaktere ist ein für DSA typischer Vorgang und passt auch gut zu den Abenteuern, die üblicherweise gespielt werden. Trotzdem wären Nebenkonflikte für die Heilung zu aufwändig und nähmen zu viel Spielzeit ein. Die Heilungs-Kurzkonflikte stellen außerdem sicher, dass es eine Nische `Heiler' gibt, die von einem Charakter besetzt werden kann.

Aus ähnlichen Gründen kann der Spielleiter Kurzkonflikte im Spiel einsetzen: Das ausführliche Ausspielen eines Konfliktes zu einem bestimmten Talent könnte im Spiel als zu langweilig empfunden werden. Wenn ein Spieler das Talent trotzdem gesteigert hat, kann der Spielleiter dies über Kurzkonflikte wichtig machen, ohne das im Spiel auszubreiten.

Eine weitere Möglichkeit für Kurzkonflikte ist, sie als Weiche für verschiedene Wege zu benutzen. So könnte der einfachere Weg den Charakteren dann offenstehen, wenn ein bestimmter Kurzkonflikt gewonnen wurde. Eine typisches Beispiel hierfür ist Fährtensuchen: Gelingt der Kurzkonflikt, so gelingt die Verfolgung ohne Probleme, misslingt dagegen die Probe, so folgt ein Kampf, weil die Charaktere von der Ideallinie abweichen und in einen Hinterhalt geraten.
\end{design}

Ob ein relativ unwichtiger Konflikt als Kurz- oder Nebenkonflikt ausgespielt wird, entscheidet der SL, wobei er dabei Rücksicht auf die Wünsche der Spieler nehmen soll.

\begin{beispiel}
\paragraph{Einige Beispiele:}
\begin{description}
\item[Kurze Handlungen:] Taschendiebstahl, sich verstecken, etwas werfen, jemanden hinterrücks erstechen
\item[Langweilige Handlungen:] Klettern, auf Lauer liegen, die Bibliothek durchsuchen
\item[Gut ausspielbar:] Streitgespräche, Bestechungen
\end{description}
Dabei kann natürlich etwas, was hier als `kurz' oder `langweilig' beschrieben ist, in der konkreten Situation lang oder besonders interessant sein (z.\,B. wenn der Taschendieb einen für die Story wichtigen Gegenstand klauen will). Aber in vielen Fällen lohnt in diesen Punkten das Ausspielen als Kurzkonflikt.
\medskip
\end{beispiel}

Wichtig ist bei Kurzkonflikten, dass die Aktivität immer vom Helden ausgeht. Der Held muss etwas schaffen, der SL würfelt in solchen Situationen nie. Wenn sich die Helden versuchen, an einem Wachmann vorbeizuschleichen: Die Charakterspieler müssen Schleichen würfeln. Wenn sich ein Räuber versucht, an einem Helden vorbeizuschleichen: Die Charakterspieler müssen Sinnenschärfe würfeln.


\subsection{Ablauf}\label{KurzkonfliktAblauf}\index{Kurzkonflikt!Ablauf}

\begin{enumerate}
  \item Zuerst wird vom SL (bzw. von allen Spielern gemeinsam) das Konfliktziel festgelegt. Also: Was soll erreicht werden? Dabei soll das \emph{Ergebnis} im Vordergrund stehen, nicht, auf welche Weise es der Charakter erreichen soll. Weichen die möglichen Folgen eines verlorenen Kurzkonfliktes vom Nichterreichen des Zieles ab, so muss dies hier auch vom Spielleiter angesagt werden. Eines der wichtigsten Beispiele hierfür ist, dass der Charakter durch den Ausgang eines Kurzkonfliktes Schaden nehmen könnte.

  \item Dann wählt der Spieler aus, mit welchem Talent oder Zauber der SC das Ziel erreichen soll; dazu legt der SL dann die Schwierigkeit fest. Normalerweise sind alle Aufgaben `normal schwierig'. Nur, wenn die äußeren Umstände widrig sind, kann es einen Schwierigkeit von $+3$ oder $+6$ (besonders widrig) geben. Besonders günstige Umstände können die Probe auch vereinfachen (Schwierigkeit $-3$). Die Schwierigkeit soll alleine für die Umstände, nicht für das Ziel an sich vergeben werden.

  \begin{beispiel}
    \paragraph{Beispiel:} Eine steile Wand zu erklimmen erfordert eine Athletik-Probe. Das ganze ohne Seil oder im stürmischen Regen ergibt eine Athletik-Probe $+3$. Ohne Seil im stürmischen Regen dagegen ist $+6$.
  \end{beispiel}

  Konflikte, von denen der SL meint, sie seien für den Charakter nicht zu gewinnen, sind automatisch verloren (automatischer Misserfolg). Sollte die Schwierigkeit nicht $0$ betragen oder sogar die Probe unmöglich zu schaffen sein, so muss der SL dies ansagen.

  \item Wenn dem Spieler die Vorgaben vom SL zu risikoreich erscheinen, kann er sich noch entscheiden, ein anderes Talent zu benutzen (und hoffen, dass der Konflikt damit einfacher zu gewinnen ist) oder dem Problem ganz aus dem Weg gehen. Das Ergebnis wird in diesem Fall vom SL festgelegt. Dieser \emph{kann} auch entscheiden, dass das aus dem Weg gehen keine Option für den Charakter darstellt.
  
  \item\label{KKAbschnittWuerfeln} Jetzt wird mit W20 gewürfelt. Hat der Spieler mehr als einen Würfel zur Verfügung {(vgl. auch den Abschnitt ``Bonuswürfel'' ab Seite \pageref{Bonuswuerfel})}, zählt für das grundsätzliche Gelingen oder Misslingen erstmal nur das niedrigste Ergebnis. Dazu wird jetzt die Schwierigkeit addiert. Ist das Ergebnis höchstens der Talentgesamtwert, so ist die \DEF{Probe gelungen}\index{Probe!gelungen}, der Charakter hat einen \DEF{Erfolg}\index{Erfolg}. Ist das Ergebnis über dem Talentgesamtwert, ist die \DEF{Probe misslungen}\index{Probe!misslungen}, der Charakter hat einen \DEF{Misserfolg}\index{Misserfolg}.
  
  Wenn das Ergebnis gelungen ist \emph{und} das kleinste Würfelergebnis höchstens 1 zeigt, so könnte ein kritischer Erfolg vorliegen. Sollte der Charakter über eine passende Sonderfertigkeit verfügen, so reicht es aus, wenn das Würfelergebnis im Bereich 1--2 (bzw. 1--4 bei einer aufgestockten Sonderfertigkeit) liegt. In diesem Fall ist der \DEF{Erfolg kritisch}\index{Erfolg!kritisch}.

  Wenn das Eregebnis misslungen ist \emph{und} das größte Ergebnis 20 zeigt, so liegt ein \DEF{kritischer Misserfolg}\index{Misserfolg!kritisch} vor.

  \item Als letztes geht es ans Erzählen des Konfliktausgangs. Hierzu gibt es weitere Informationen im nächsten Abschnitt.
\end{enumerate}

\subsection{Konfliktfolgen}
 Das genaue Ende des Konfliktes legt der Spielleiter fest: Entweder, er erzählt es selbst, oder er erklärt seine Idee und spielt es dann zusammen mit den Spielern aus. Haben die Spieler gewonnen, legt der Spielleiter positive Folgen fest. Haben sie dagegen verloren, legt er die Folgen des Scheiterns fest. Wie schon geschrieben, dürfen die Charaktere aufgrund eines Kurzkonfliktes nicht auf ganzer Linie verlieren, auch wenn ein verlorener Konflikt natürlich negative Folgen haben sollte.

Der Spielleiter kann natürlich die Ausschmückung des Konfliktendes auch dem Spieler überlassen. Dabei sollte er aber darauf achten, dass ihm das Ruder der Geschichte nicht aus der Hand gleitet.

Als Richtlinie für Schaden und andere Konfliktfolgen soll folgende Liste gelten:

\begin{description}
  \item[kritischer Misserfolg] Es passieren katastrophale oder lustige Dinge -- hier kann der Spielleiter seiner Phantasie freien Lauf lassen, um ein besonders lustiges, spannendes oder dramatisches Probenende zu erzählen. Es kann sein, dass das Konfliktziel dennoch mit Mühe und Not erreicht wird. In diesem Fall folgt normalerweise ein knackiger Folgekonflikt.
  
  Vorher angesagter Schaden wird verdreifacht.

  \item[Misserfolg] Der Spielleiter erzählt den Konfliktausgang. Dabei wird das Konfliktziel häufig dennoch erreicht, jedoch kommt es zu Problemen. Normalerweise folgt auf einen Misserfolg, bei dem der Charakter das Konfliktziel dennoch erreicht, ein weiterer Konflikt. Der Spielleiter kann auch entscheiden, dass das Konfliktziel nicht erreicht wird.
  
  Vorher angesagter Schaden tritt ein.

  \item[Erfolg] Das bedeutet, dass der Konflikt erfolgreich gemeistert wurde, das Ziel wurde ohne größere Probleme erreicht.

  \item[kritischer Erfolg] Der Charakter erreicht sein Ziel grandios. Der Spielleiter darf besonders spektakulär ausschmücken, wie der Charakter alle Schwierigkeiten überwindet.
\end{description}

\subsection{Beispiele}

\begin{beispiel}
\paragraph{Beispiel 1:} Der SL beschreibt (SL-Erzählphase), wie die Flucht des Helden durch eine ihm unbekannte Stadt plötzlich in einer Sackgasse vor einer Mauer endet. Es kommt zum Kurzkonflikt: Kann der Held die Mauer überwinden?

Damit ist Punkt~1 des Ablaufes (vgl. Seite~\pageref{KurzkonfliktAblauf}) schon abgehandelt -- auch ohne, dass ein Wort gewechselt werden muss. Es ist klar, das der Charakter weiter fliehen will, da die ihn verfolgende Übermacht zu groß ist. Der Spieler sagt: ``Ich versuche, über die Mauer zu klettern''. Da es keine widrigen Umstände gibt, beträgt die Schwierigkeit 0 und der SL sagt nur ``Ok, wenn die Probe misslingt, bekommst du 2 Punkte Schaden. würfle.'' (Punkt 2) Die Punkte 3 und 4 werden dann auch schnell abgehandelt: Der Spieler hat als offensive Sonderfertigkeit `Einbrechen' gewählt, was ihm einen meisterlichen Wurf auf Athletik, Schlösser knacken, Körperbeherrschung und Fallen entschärfen ermöglicht; in Klettern hat er einen Gesamtwert von 12. Er darf also einen W20 benutzen und würfelt eine 20, also misslungen, da die Zahl größer als 12 ist. Da der Würfel aber sogar eine 20 zeigt, könnte das einen kritischen Misserfolg nach sich ziehen. Also nachwürfeln: 12. Knapp gelungen, also war es nicht kritisch, trotzdem aber misslungen.

So erzählt der SL (Punkt 6): ``Beim Versuch, die bestimmt drei Schritt hohe Mauer zu überwinden, rutscht du ab und schürfst dir deinen rechten Arm auf (d.\,h. der erhaltene Schaden ist körperlich). Die Verfolger biegen um die Ecke und sind schon in Sichtweite, als es dir unter Schmerzen endlich gelingt, dich über die Mauer zu ziehen. Du hetzt weiter und siehst dich um: Auch die Verfolger haben es über die Mauer geschafft!''

Jetzt schließt sich ein Folgekonflikt an.

\paragraph{Beispiel 2:} Die Streunerin Adessa der Gruppe soll sich in der zwielichtigen Gesellschaft umhören, wo sie die Diebesbande 'Goldener Handschuh' finden kann. Im freien Spiel entscheiden die Spielerinnen, dass sie sich in einer Taverne im Hafenviertel umhören soll. Sie betritt also die Taverne und trifft dort auf den schmierigen Wilbur. Das Ziel der Streunerin ist klar: ``Herausfinden, wo der Goldene Handschuh zu finden ist''.

``Den betöre ich.'', sagt die Spielerin. Da Wilbur fett, hässlich und schmierig ist hat er schon lange keine Frau mehr gehabt, also entscheidet der SL, dass dies vereinfachende Umstände sind (Schwierigkeit $-3$). Adessa hat in Betören/Galanterie einen Gesamtwert von 10 und kann keine Sonderfertigkeit einbringen. Sie würfelt also 1W20: 3. Das Ergebnis ist damit 0 (wegen der Schwierigkeit), also könnte der Erfolg sogar kritisch sein. Erneuter Wurf: 12, macht 9, also ist die Probe kritisch gelungen!

Da die Probe gelungen ist, wird Adessa die Information durch ihre Betörungskünste bekommen. Durch ihren grandiosen Erfolg schafft sie es sogar, Wilbur dann die versprochene Bettpartie wieder auszuschlagen, ohne dass dieser sauer auf sie wird.

\paragraph{Beispiel 3:} Die Spielgruppe ist im freien Spiel, das Nachtlager ist gerade aufgeschlagen worden. Die Helden beschließen, jagen zu gehen. Da es sich hierbei nicht um einen Konflikt handelt, der die Story irgendwie weiterbringt, beschreibt der SL oder der Spieler (je nach Präferenz der Gruppe) kurz den Jagderfolg des Jägers und fährt dann mit dem Konflikt der Nachtwache fort. Es kommt zum Konflikt: Schafft es der gerade wachhabende Held, den sich anschleichenden Räuber zu bemerken?
\end{beispiel}

\subsection{Gegenseitige Hilfe}
Möchte ein Charakter einen anderen bei der Ausführung seines Kurzkonfliktes unterstützen, so muss er dies ansagen, \emph{bevor} der Spieler würfelt, aber nachdem sich der Spieler festgelegt hat, welches Talent er benutzen will (also im Ablauf direkt vor Abschnitt~\ref{KKAbschnittWuerfeln}). Der Spieler muss beschreiben, auf welche Weise sein Charakter helfen möchte.

Der ursprüngliche Kurzkonflikt wird dann unterbrochen und ein neuer Kurzkonflikt zur Hilfe wird begonnen, mit allen Konsequenzen für den Helfenden. Geligt der Kurzkonflikt, so wird der ursprüngliche Konflikt um 3 erleichtert. Misslingt die Hilfe, so passiert nichts, misslingt die Hilfe jedoch kritisch, so wird der ursprüngliche Konflikt um 3 erschwert.

Auf diese Weise kann einem Charaker theoretisch von beliebig vielen anderen geholfen werden. Die Hilfe muss allerdings plausibel erklärbar sein. Als Richtlinie soll hier aber festgehalten werden, dass wenn zwei oder mehr Charaktere einen weiteren bei einem Kurzkonflikt unterstützen wollen, sollte besser ein Nebenkonflikt ausgetragen werden.

\begin{optional}
\section{Optional: Dem SL die Zügel aus der Hand nehmen}
Wie schon auf Seite~\pageref{Optional:ZuegelAusDerHand1} beschrieben, kann durch ein paar Kniffe dem Spielleiter die Kontrolle entzogen werden. Dann wird aus \StoryDSA ein Rollenspiel, bei dem alle Spieler gemeinsam eine Geschichte entwickeln. Um das zu erreichen, wird das Erzählrecht für das Konfliktende in die Hand des Gewinners gelegt. Gewinnen dann also die Spieler, dürfen sie völlig frei entscheiden, wie der Konflikt endet und die Geschichte so nach eigenen Ideen fortsetzen.
\end{optional}

\begin{optional}
\section{Optional: Schaden in Kurzkonflikten}

Manche Gruppen wollen vielleicht eine größer regeltechnische Nähe von Kurz- und Nebenkonflikten. Das Schadensrisiko ist ja nach den Standardregeln in Kurzkonflikten stark vermindert, denn in Nebenkonflikten ist es so, dass sogar nach kritischen Erfolgen der Charakter Schaden davontragen könnte (siehe Seite~\pageref{Sec:Nebenkonflikt}ff).

Um dies abzubilden kann man am Konfliktende Schaden wie in Nebenkonflikten auswürfeln (eine 9 oder mehr verursacht 1 Punkt Schaden). Die Anzahl der W20 ist vom Konfliktausgang abhängig:
\begin{tabular}[C]{|ll|}
\hline
kritischer Erfolg & 3W20 \\
Erfolg & 5W20 \\
Misserfolg & 5W20+Konfliktpunkte/2 \\
kritischer Misserfolg & 5W20+Konfliktpunkte \\
\hline
\end{tabular}
Ein Charakter, der 5 Konfliktpunkte hat, würde also bei einem Misserfolg 8W20 werfen und bei einem kritischen Misserfolg 10W20, um den Schaden zu bestimmen.

Wegen des höheren Risikos sollten dann auch für Neben- und Kurzkonflikte gleich viele Abenteuerpunkte vergeben werden.
\end{optional}







\section{Nebenkonflikt}\label{Sec:Nebenkonflikt}
Nebenkonflikte stellen Schwierigkeiten dar, die überwunden werden. Bei Nebenkonflikten können sich verschiedene Charaktere gegenseitig unterstützen.

Geeignet für Nebenkonflikte sind vor allem Action-Sequenzen, angefangen bei Kämpfen gegen eine Orkbande über die Flucht aus einem einstürzenden Gebäude bis hin zu einer wilden Verfolgungsjagd. Grundsätzlich ist aber jede Art von Konflikt geeignet, die sich in einer Art `Filmsequenz' darstellen lässt. Dabei kann man auch die unterschiedlichstens Zeiteinteilungen für Runden zulässig: Ob die Szene in Zeitlupe oder Zeitraffer abläuft, hängt ganz von der Gesamtdauer und der Art der Erzählung ab. So kann eine tagelange Verfolgung von Spuren, quer durch unwegsames Gelände mit einem einzigen Nebenkonflikt abgehandelt werden. Ein anderer Nebenkonflikt könnte das Stehlen von Schlüsseln einer schlafenden Wache sein -- ein Vorgang, der nur wenige Sekunden in Anspruch nimmt.

\begin{beispiel}
Einige Beispiele:
\begin{itemize}
  \item Kämpfe gegen Schergen des Bösen
  \item Verfolgungsjagden
  \item Taschendiebstahl eines wichtigen Gegenstandes
  \item Angemessenes Benehmen auf dem Hofball
  \item Durchquerung eines reißenden Flusses
\end{itemize}
\end{beispiel}

Die Anzahl von Gegnern, die Länge des zurückzulegenden Weges, usw, sollte in einem Nebenkonflikt niemals genau festgelegt sein. Der Konflikt ist genau dann vorbei, wenn die Konfliktpunkte des Spielleiters alle verbraucht sind. Wenn z.\,B. die Helden von einer Goblinbande aufgehalten werden, so sollte der Spielleiter möglichst nichts genaueres sagen als ``etwa ein Dutzend''. Der Vorteil ist, dass dann das Ende nicht genau festgelegt ist, so dass auch dann noch eine vernünftige Beschreibung möglich ist, wenn das Ende sehr plötzlich kommt oder unerwartet lange auf sich warten lässt.

\subsection{Ablauf}
\begin{enumerate}
  \item Als erstes wird vom SL (bzw. von allen Spielern gemeinsam) das Konfliktziel festgelegt. Also: Was soll erreicht werden? Dabei soll das \emph{Ergebnis} im Vordergrund stehen, nicht, auf welche Weise es der Charakter erreichen soll.

  \item {Jeder Spieler bekommt seine Konfliktpunkte (anfangs 3, in höheren Stufen bis zu 7, vgl. Abschnitt ``Steigerung'' ab Seite \pageref{Steigerung}).}
  
  \item Außerdem legt der SL die \DEF{Konfliktpunkte}\index{Konfliktpunkte} fest. Am Anfang ist
  \begin{align*}
    \text{SL Konfliktpunkte} &= 5 \times \text{Anzahl beteilige Helden}
  \end{align*}
  eine gute Richtlinie. Mit etwas Erfahrung kann der Spielleiter dies dann an seine Gruppe anpassen (vgl. auch die Designanmerkung ``Schaden und Konfliktpunkte'').

  Um den Spielern eine Erzählhilfe für den Fortgang des Konfliktes zu geben, werden alle Konfliktpunkte, auch die des SL, offen gezeigt.

  \item Dann braucht der Konflikt noch ein automatisches \DEF{Offensivergebnis}\index{Offensivergebnis} und ein automatisches \DEF{Defensivergebnis}\index{Defensivergebnis}, die wiederum vom SL festgelegt werden. Je höher diese sind, umso gefährlicher ist der Konflikt. Der SL sollte die Werte vor allem auch in Hinsicht der Fähigkeiten der Charaktere wählen. {Offensivergebnisse sollten im Bereich $1$ bis $5$ liegen, Defensivergebnisse im Bereich $0$ bis $4$ (in den oberen Stufen kann das auch überschritten werden -- der Spielleiter sollte die Werte immer so angleichen, dass auch Nebenkonflikte für die Spieler spannend bleiben).}

  Dann wird der Konflikt rundenweise ausgetragen.

  \item\label{NKNaechsteRunde} Die beteiligten CS erzählen reihum, was im Konflikt passiert. Dabei darf, laut Konfliktende-Regel, das Ende des Konfliktes nicht vorweg genommen werden. Es ist aber möglich und erwünscht, dass der Spieler beschreibt, wie er selber Nachteile erleidet -- darüberhinaus kann er auch andere Charaktere mit einbeziehen {(vgl. Abschnitt ``Tipps für gute Erzählungen'', ab Seite~\pageref{TippsGuteErz})}. Desweiteren sollte mit einbezogen werden, wie viele Konfliktpunkte der Charakterspieler noch übrig hat. {Aus der Beschreibung sollte hervorgehen, welches Talent der Spieler einbringt.}

  Auch in Nebenkonflikten kann es natürlich zu Wortwechseln kommen. Dann darf der Angesprochene (z.\,B. der SL, aber auch andere CS) außer der Reihe antworten.

  Der SL sollte darauf achten, dass der Spieler, der eine Runde beginnt, von Runde zu Runde wechselt.
  
  \item Nach jeder Erzählung wird der \DEF{Erzählwert}\index{Erzählwert} bestimmt, der Spieler erhält entsprechend viele W20 {(wenn das eingebrachte Talent auch relevant ist)}. Zum Erzählwert werden jedoch nur die eigenen Aussagen hinzugerechnet, d.\,h. im Zwiegespräch mit einem anderen Charakter bekommt der Spieler nicht für die Aussagen des anderen irgendwelche Würfel. Für die Erzählung gibt es mindestens ein und maximal fünf W20.

  \item Haben alle beteiligten CS etwas beigetragen und Würfel erhalten, müssen sie die Würfel in \DEF{Offensivwürfel}\index{Offensivwürfel} und \DEF{Defensivwürfel}\index{Defensivwürfel} aufteilen und anschließend würfeln. Jeder Würfel, der höchstens den Talentgesamtwert (inklusive aller Boni) zeigt, zählt als gelungen, ansonsten als nicht gelungen. 
  
  Konnte der Spieler auch Sonderfertigkeiten einbringen, sind hier meisterliche Würfe möglich; ohne Sonderfertigkeiten zählt keiner der Würfel als meisterlich. Maximal kann ein Spieler eine offensive und eine defensive Sonderfertigkeit einbringen, beide können auch aufgestockt sein.

  Meisterliche Würfe treten bei einem Ergebnis 1 oder 2 ein. Bei einer aufgestockten Sonderfertigkeit zählt bereits eine 1--4 als ein meisterlicher Wurf.
  
  Die Anzahl der gelungenen offensiven Würfe gibt das \DEF{Offensivergebnis}\index{Offensivergebnis} des CS, die Anzahl der gelungenen defensiven Würfe das \DEF{Defensivergebnis}\index{Defensivergebnis}. Jeder meisterliche Wurf erhöht das Offensiv- bzw. Defensivergebnis um einen weiteren Punkt, d.\,h. meisterliche Würfe zählen doppelt.

  \item Liegt das Offensivergebnis eines CS über dem Defensivergebnis des Konfliktes, werden die Konfliktpunkte des SL um die Differenz gemindert.
  
  Liegt das Defensivergebnis eines CS unter dem Offensivergebnis des Konfliktes, werden die Konfliktpunkte des CS um die Differenz gemindert.

  \item Möchte ein Spieler aus dem Konflikt aussteigen, so kann er dies tun, sofern er noch Konfliktpunkte übrig hat. Im Spiel gibt der Held den Konflikt auf. Dies wird in jedem Fall für den Charakter als persönlicher Misserfolg gewertet.

  Ein Spieler, dessen Konfliktpunkte auf 0 gesunken sind, scheidet automatisch aus dem Konflikt aus (persönlicher kritischer Misserfolg).

  \item Solange noch der SL und mindestens einer der CS weiterhin am Konflikt beteiligt sind, wird die nächste Konfliktrunde ausgetragen (es geht weiter bei~\ref{NKNaechsteRunde}). Ist aber der SL oder alle CS aus dem Konflikt ausgeschieden, so endet der Konflikt.

  \item Am Ende würfelt jeder Spieler den Schaden für diesen Konflikt aus. Dazu wirft er pro Runde, die der Charakter am Konflikt beteiligt war und für jeden verlorenen Konfliktpunkt einen W20. Für jeden Würfel, der mindestens eine 9 zeigt, bekommt der Charakter einen Punkt Schaden. 

  Außerdem muss festgelegt werden, ob der Schaden körperlich oder geistig ist. Die Art geht aus dem Zusammenhang hervor und wird im Zweifelsfall vom Spielleiter festgelegt. Jegliche Verletzung oder Erschöpfung durch körperliche Anstrengung verursacht körperlichen Schaden, wohingegen Reden, Konzentration usw. geistigen Schaden. 

\end{enumerate}

\begin{design}
\subsubsection{Designanmerkung: Schaden und Konfliktpunkte}

Vielleicht erscheint es etwas merkwürdig, dass der Schaden im Wesentlichen von der Dauer des Konfliktes und nicht so sehr von der Anzahl der verlorenen Konfliktpunkte abhängt. Der Grund ist, die Spieler zu animieren, Konflikte möglichst schnell zu beenden und nicht so sehr auf Sicherheit zu spielen und so die Konflikte in die Länge zu ziehen.

Die Konfliktlänge -- und damit die Gefährlichkeit -- kann der Spielleiter über die Konfliktpunkte regeln. Als Richtlinie gilt, dass ein Konflikt etwa fünf Runden lang dauern sollte. Daher ist die Empfehlung für den Anfang gerade $\text{SL Konfliktpunkte} = 5 \times \text{Anzahl beteilige Helden}$, da man in der ersten Stufe etwa mit einem Konfliktpunkt-Verlust von 1 pro Runde und Held rechnen muss.

In höheren Stufen kann man nur schwer eine pauschale Empfehlung geben. Der Spielleiter entwickelt mit der Zeit ein sehr gutes Gefühl für die anzusetzenden Konfliktpunkte und sollte sich darauf verlassen.
\end{design}

\subsection{Ergebnis}
Hier muss unterschieden werden zwischen dem Gesamtergebnis des Konfliktes und den Einzelergebnissen der Spieler.

\begin{enumerate}
\item Für die Gruppe gilt:
\begin{description}
\item[SL hat keinen Konfliktpunkt übrig:] Die Charakterspieler gewinnen den Konflikt.\footnote{Es kann passieren, dass am Ende des Konfliktes weder Charakterspieler noch Spielleiter Konfliktpunkte übrig haben. In diesem Fall haben auch die Spieler gewonnen.} Insgesamt ist das Konfliktziel erreicht; es gibt keine weiteren Schwierigkeiten. Obwohl der Konflikt insgesamt gewonnen wurde, kann es sein, dass einzelne Charaktere einen persönlichen Misserfolg (oder kritischen Misserfolg) erlitten haben.
  \item[SL hat noch Konfliktpunkte übrig:] Die Charakterspieler verlieren den Konflikt. Das Konfliktziel wurde (im Normalfall) zwar erreicht -- jedoch ist mit einem Folgekonflikt zu rechnen. Klarerweise haben alle Helden einen persönlichen Misserfolg (oder sogar kritischen Misserfolg) erlitten.
\end{description}

 Der Spielleiter erzählt das Konfliktende. Gewinnen die Spieler, so ist das Ergebnis positiv für die Helden, verlieren die Spieler, ist es negativ. Aber auch hier gilt wiederum:
Die Charaktere könnten das Konfliktziel dennoch erreichen (wenn auch mit Schwierigkeiten und unter Verlusten) oder das Konfliktziel verfehlen, wobei dabei nicht ein verfrühtes und unbefriedigendes Ende der Geschichte eintreten darf.

\item Für jeden einzelnen Spieler gilt:
\begin{description}
  \item[Alle Konfliktpunkte verloren:] (persönlicher kritischer Misserfolg) Dem Charakter ist ein erhebliches Missgeschick passiert. Er bekommt meist noch einen Folgekonflikt.
  \item[Mehr als die Hälfte der Konfliktpunkt verloren:] (persönlicher Misserfolg) Dem Charakter ist ein kleines Missgeschick passiert, das aber, wenn die Gruppe insgesamt gewonnen hat, von den anderen Charakteren aber ausgeglichen wurde.
  \item[Höchstens die Hälfte der Konfliktpunkte verloren:] (persönlicher Erfolg) keine speziellen Vor- oder Nachteile.
  \item[Alle Konfliktpunkte übrig:] (persönlicher kritischer Erfolg) Dem Charakter ist ein spektakulärer Erfolg geglückt.
\end{description}
\end{enumerate}


\subsection{Beispiel}
Eine kleine Anmerkung vorweg: Um die Beispiele nicht ausarten zu lassen, sind die Konfliktpunkte eher niedrig gewählt.

\begin{beispiel}
In der SL-Phase beschreibt der Spielleiter, wie die Helden mit einem Fuhrwerk durch einen Wald reisen, \emph{als plötzlich} eine Gruppe Goblins aus dem Busch springt. Der Anführer ruft einen goblinischen Kampfschrei und stürzt sich auf den erstbesten Helden.

Das Konfliktziel dürfte in dieser Siuation klar sein: Abwehren der Goblins. Wie viele Goblins es sind wird nicht festgelegt. Die Heldengruppe besteht aus zwei Stufe-1-Helden. Jeder Spieler bekommt also 3~Konfliktpunkte. Der Spielleiter legt die Konfliktpunkte auf 6 fest. Außerdem setzt er das Offensivergebnis auf 1 und das Defensivergebnis auf 0 fest. Das sagt er auch laut an und legt einen Würfel mit der 6 nach oben als Konfliktpunkt-Anzeiger hin. So wissen die Spieler, wie weit sie vom Konfliktende entfernt sind und können das in ihre Erzählungen mit einfließen lassen.

Nun wird der Kampf Rundenweise abgehandelt:

\begin{description}
\item[Runde 1:] Der Spieler des zwergischen Söldners beschreibt: ``Ha, ich springe vom Wagen, greife mir dabei meinen Zwergenschlägel und schlage dem erstbesten Goblin den Schädel ein. Dann reiße ich meine Waffe über den Kopf um einen Säbelhieb abzufangen.'' Diese Beschreibung ist 5 Würfel wert (vom Wagen springen, Hammer greifen, erstbester Goblin, Schädel einschlagen, Waffe hochreißen, Säbelhieb abfangen wären eigentlich sogar 6 Fakten). Der große Kriegshammer als Zweihandwaffe gibt einen Offensivwürfel extra, also nimmt sich der Spieler 5W20.

Dann beschreibt die Spielerin der thorwalschen Piratin: ``Einer der Goblins sieht die Wurfaxt von Ragna auf sich zufliegen. Er duckt sich darunter weg -- die Wurfaxt fliegt weiter und bleibt in einem Baumstamm stecken. Dann stürmt er vor, um die Piratin auf seinem Speer aufzuspießen.'' Macht dann also auch 5W20 (Wurfaxt fliegt, Goblin duckt sich, Axt steckt im Baumstamm, Goblin stürmt vor, Speer aufspießen).

Nachdem beide Charakterspieler erzählt haben, müssen die Spieler ihre Würfel auf Offensive und Defensive aufteilen. Der Spieler des Söldners hat einen Talentgesamtwert von 9. Durch seinen Kriegshammer muss er von seinen 6~Würfeln mindestens einen in die Offensive liegen, er entscheidet sich für drei Offensiv- und drei Defensivwürfel.

Die Spielerin der thorwalschen Piratin hat nur einen Talentgesamtwert von $8$. Trotzdem ist sie mutig und nimmt nur einen Würfel für die Defensive; sie hofft, den Konflikt damit direkt in der ersten Runde zu beenden.

Nun wird gewürfelt. Beide Spieler rollen gleichzeitig ihre Würfel. Zunächst zum zwergischen Söldner: Die Offensiv-Würfel zeigen 9, 9, 15; die Defensivwürfel 14, 4, 1. Das macht zwei gelungene Offensivwürfe (die Neunen) und zwei gelungene Defensivwürfe (4 und 1). Das Offensiv- und Defensivergebnis ist also jeweils 2. Die thorwalsche Piratin würfelt als Offensive 19, 15, 3, 4; der Defensiv-Würfel zeigt eine 9. Das macht ein Offensivergebnis von~$2$ (die 3 und die 4) und ein Defensivergebnis von~$0$.

Beide Würfelergebnisse werden jetzt mit Offensiv- und Defensivergebnis des Konfliktes verglichen. Da das Defensivergebnis des Konfliktes $0$ beträgt, verliert der Spielleiter sowohl durch den Söldner als auch durch die Piratin jeweils $2$ Konfliktpunkte. Umgekehrt verursacht der Konflikt jeweils einen Schaden bei Söldner und Throwalerin. Da der Söldner aber ein Defensivergebnis von $2$ hat, kann er den Schaden abwenden. Die Thorwalerin
verliert einen Konfliktpunkt.

Der Stand nach der ersten Runde ist also wie folgt:
\begin{itemize}
\item Spielleiter: 2 Konfliktpunkte
\item Söldner: 3 Konfliktpunkte
\item Thorwalerin: 2 Konfliktpunkte
\end{itemize}

\item[Runde 2:] Diesmal beginnt die Thorwalerin: ``Ich reiße dem anstürmenden Goblin den Speer aus der Hand und spieße ihn auf seine eigene Waffe. Dann ziehe ich meine Axt und stelle mich erwartungsvoll dem Goblinanführer entgegen, der auf mich zustürmt.'' 5 Würfel.

Dann der Söldner: ``Nachdem ich noch drei weitere Goblins niedergeknüppelt habe, sehe ich, wie der Goblinanführer mit erhobenem Speer vor der Piratin steht. Die Speerspitze blinkt im Sonnenlicht und fährt in Ragnas Schulter. Ich springe vor und ramme dem Goblin meinen Schlägel in den Bauch, so dass er ächzend zusammensinkt. Dabei bricht mit einem lauten Knacken der Speer in zwei Teile.'' Ragnas Spielerin findet die Erzählung besonders gelungen und vergibt dafür spontan eine Erzählmarke. Daher hat diese Erzählung auf jeden Fall einen Wert von 5.

Es wird gewürfelt: Da die Spielerin der Thorwalerin auch die Axt in die Erzählung mit eingebaut hat, gestattet der Spielleiter, dass sie auf Einhand-Hiebwaffen würfeln darf -- sie hat darin eine 9 als Talentgesamtwert. Die Würfel zeigen offensiv 20, 19 und defensiv 14, 10, 2, macht also gerade mal ein Defensivergebnis von $1$, so dass sie keinen weiteren Konfliktpunkt verliert.

Der Spieler des Söldners wirft offensiv 4, 6, 16 defensiv 9, 17, 20. Er hat also ein Offensivergebnis von 2 und ein Defensivergebnis von 1. Damit bekommt der SL insgesamt 2 Konfliktpunkte abgezogen -- umgekehrt verlieren die Spieler keine Punkte.

Der Stand nach der zweiten Runde ist also wie folgt:
\begin{itemize}
\item Spielleiter: 0 Konfliktpunkte
\item Söldner: 3 Konfliktpunkte
\item Thorwalerin: 2 Konfliktpunkte
\end{itemize}
\end{description}

Die Spieler gewinnen damit den Konflikt und dürfen das Ende erzählen. Die Spieler erzählen gemeinsam, dass die Goblins sehen, wie ihr Anführer in sich zusammensinkt und daraufhin fliehen. Außerdem wollen sie den Anführer liegen lassen und einfach weiterziehen. Der Anführer lebt zwar noch, allerdings gehen die Spieler davon aus, dass die Bande zurückkehrt und ihm hilft.

Trotzdem muss noch der Schaden für die Charaktere erwürfelt werden. Der Konflikt hat 2~Runden lang gedauert und die Thorwalerin hat einen Konfliktpunkt verloren; das ergibt 2~Schadenswürfel für den Söldner (1, 4) und 3 für die Thorwalerin (3, 6, 12). Damit bekommt sie einen Schadenspunkt, da ein Würfel 9 oder mehr zeigt. Der Schaden ist klarerweise körperlich (Stich in die Schulter). Hätte die Söldnerin keinen Schaden erwürfelt, so wäre der Stich nur ein Kratzer gewesen.
\end{beispiel}

\begin{optional}
\section{Optional: Schaden schon während der Konflikte}
Die Schadensauswirkungen sind normalerweise erst nach dem Konflikt spürbar. Wenn dies als zu unrealistisch empfunden wird, können die Auswirkungen auch sofort spürbar gemacht werden. Soll diese Regel benutzt werden, so sollte sie natürlich auch in Hauptkonflikten eingeführt werden. Dann muss aber auch Willens- oder Lebenskraft und Schaden für Hauptkonfliktgegner eingeführt werden (vgl. Hauptkonflikte), was dann insgesamt zu einem erhöhten Rechenaufwand während der Konflikte führt.
\end{optional}

\subsection{Beispiel}
\begin{beispiel}
Der Thorwaler Rune Runesson und Nargrim Sohn des Ischgrim, ein diebischer Hügelzwerg, wollen im Haus eines Händlers, bei dem sie zu Gast sind, unbemerkt ins Büro eindringen. Rune soll Wache schieben, während Nargrim sich am Schloss zu schaffen macht. Beide sind bereits Stufe 7 und haben jeweils 4~Konfliktpunte. Nargrim hat sogar eine passende Sonderfertigkeit: Einbrechen (meisterliche Offensivwürfe für Athletik, Schlösser knacken, Körperbeherrschung und Fallen entschärfen möglich). Außerdem hat Nargrim Mechanik~1 (Grundwissen in Mechanik) und bekommt daher beim Schlösser knacken einen zusätzlichen Würfel. Der SL legt die Konfliktpunkte auf 13, das automatische Offensivergebnis auf 2 und das Defensivergebnis auf 0 fest.

Die Beschreibungen sind jeweils für 5 Würfel gut genug; daher wird im jetzt folgenden Beispiel auf das Zählen der Fakten verzichtet. Außerdem wird die Optionalregel ``Schaden schon während der Konflikte'' angewendet.

\begin{description}
\item[Runde 1:]
Nargrims Spieler beginnt: ``Ich hocke mich hin, um das Schloss erstmal unter die Lupe zu nehmen. Hm\dots das Schloss scheint recht neu zu sein. Hoffentlich habe ich das richtige Werkzeug dabei\dots Ich krame in meinem Diebeswerkzeug.''

Runes Spieler: ``Währenddessen geht Rune zur Tür am Ende des Ganges. Sie öffnet leider nicht in den Gang hinein. Daher öffnet Rune die Tür, geht hindurch und wartet vor der halb angelehnten Tür, um Nargrim rechtzeitig Bescheid geben zu können.''

Nargrim hat in Schlösser knacken einen Talentgesamtwert von 12, der Spieler teilt seine 6W20 in 3 offensive und 3 defensive Würfel auf; er wirft 16/20/2 offensiv und 1/6/6 defensiv, ergibt ein Offensivergebnis von 2 (ein meisterlicher Wurf) und 3 defensive gelungene Würfe (kein meisterlicher Wurf, da seine SF Einbrechen nur meisterliche Offensivwürfe erlaubt). Damit kann er beide Offensivpunkte des SL abwehren, verursacht aber 2 Konfliktpunkte Verlust beim SL.

Rune würfelt gegen seinen Talentgesamtwert von 11 in Sinennschärfe. Er ist vorsichtig und teilt seine Würfel in 2 offensive und 3 defensive auf und würfelt 2/14 und 18/1/9, hat also 1 offensiven und 2 defensive gelungene Würfe. Damit bekommt auch er keinen Abzug seiner Konfliktpunkte und verursacht 1 Konfliktpunkt Verlust beim SL.

Dann würfeln beide Charakterspieler Schaden aus: Nargrims Spieler würfelt eine 14, Runes Spieler eine 7. Schaden gibt es (wie immer in Nebenkonflikten) ab einer 9, also einen Punkt Schaden für Nargrim.

Stand der Konfliktpunkte nach der ersten Runde: SL 10, Nargrim 4 (1 Punkt geistiger Schaden), Rune 4.

\item[Runde 2:] 
Runes Spieler: ``Ich schaue gelangweilt den Gang entlang. Auf dem Boden liegt ein Teppich mit einem roten, exotischen Muster -- wahrscheinlich tulamidisch. An den Wänden hängen irgendwelche Bilder.''

Nargrims Spieler: `` `Ha, da ist einer, der müsste passen.' Nargrim nimmt sich einen Dietrich und setzt ihn vorsichtig am Schloss an. Er versucht, ihn in das Schloss zu schieben, doch er passt nicht. `Mist.' ''

Beide bleiben bei ihrer Würfelaufteilung. Rune wirft 11/10 und 10/4/2 (Offensivergebnis 2, Defensivergebnis 3), Nargrim dagegen 6/10/7 und 18/4/15 (Offensivergebnis 3, Defensivergebnis 1). Macht insgesamt 5 Konfliktpunkte Verlust für den SL und 1 für Nargrim.
Nargrims Spieler muss also jetzt zwei Schadenswürfe machen (eine Runde ist vergangen und Nargrim hat einen Konfliktpunkt verloren). Sie ergeben mit 11 und 19 jeweils einen Punkt Schaden. Runes Spieler würfelt mit dem Schadenswürfel eine 14, was auch einen Punkt geistigen Schaden ergibt.

Stand nach der zweiten Runde: SL 5, Nargrim 3 (3~Punkte geistiger Schaden), Rune 4 (1~Punkt geistiger Schaden).

\item[Runde 3:]
Nargrims Spieler: ``Nargrim wird langsam nervös, Schweiß rinnt ihm von der Stirn. Dann greift er zu einem kleinen Schraubendreher: `Wenn es nicht anders geht, dann eben hiermit.' Dann setzt er das Werkzeug am Schloss an.''

Runes Spieler: ``Habe ich da was gehört? Am Ende des Ganges, hinter einer Tür, waren doch Schritte, oder? Ich schleiche vorsichtig Richtung Tür und lege mein Ohr daran um zu lauschen.''

Nargrim versucht es jetzt offensiver mit 4/2 und Rune steigt auf 3/2 um. Runes Spieler kann aussuchen, ob er lieber auf Schleichen oder wieder auf Sinnenschärfe würfeln möchte -- beides ist nach seiner Beschreibung sinnvoll. Da sein besserer Wert aber Sinnenschärfe ist, bleibt er dabei. Nargrims Spieler wirft 4/13/12/7 und 16/8 (Offensivergebnis 3, Defensivergebnis 1), Runes Spieler dagegen 13/16/11 und 19/13 (Offensivergebnis 1, Defensivergebnis 0). Macht also zwei Konfliktpunkte Verlust für Rune und einen weiteren für Nargrim. Der SL bekommt allerdings 4 Konfliktpunkte Abzug.

Die Schadenswürfel ergeben bei Nargrim 1 und 11, bei Rune 6, 6 und 15. Damit bekommen Nargrim und Rune jeweils 1~Schaden.

Damit ergibt sich folgender Stand nach der dritten Runde: SL 1, Nargrim 2 (4~Punkte geistiger Schaden), Rune~2 (2~Punkte geistiger Schaden).

\item[Runde 4:] 
Runes Spieler: `` `Tatsächlich Schritte!' denkt sich Rune und huscht leise zu Nargrim. (geflüstert zu Nargrims Spieler gewandt) `Hey, du musst dich beeilen! Da kommt jemand!' '' Nargrims Spieler antwortet (auch flüsternd): `` `Ich tu' was ich kann!' '' Runes Spieler wieder: ``Draußen hört man die Tür aufgehen.''

Nargrims Spieler: ``Hektisch prokelt Nargrim mit dem Werkzeugt im Schloss herum. Als Rune plötzlich hinter ihm steht, zuckt er zusammen, rutscht ab und macht einen leichten Kratzer in das Holz der Tür. Mist. Nargrim wird immer hektischer.''

Weder Nargrims noch Runes Spieler möchten noch einen Konfliktpunkt verlieren (sonst haben sie einen persönlichen Misserfolg erlitten), außerdem hat der SL nur noch einen Konfliktpunkt. Daher setzen beide 4 Würfel auf die Defensive. Runes Spieler würfelt: 8 und 17/15/19/19, nur der offensive Wurf ist gelungen. Nargrim dagegen würfelt fulminante 5/2 und 3/5/19/5, was ein Offensiv- und Defensivergebnis von jeweils drei bedeutet (die zwei ist ja ein meisterlicher Wurf). Der SL und Rune verlieren ihre letzten Konfliktpunkte. Nur Nargrims Spieler bleibt mit 2 Konfliktpunkten übrig.
Zum letzten Mal wird Schaden gewürfelt: Runes Spieler wirft 5, 16, 19 (2 Schaden), Nargrims Spieler eine 4 (keinen Schaden).

Damit ist das Endergebnis: SL 0, Nargrim 2 (4 Punkte geistigen Schaden), Rune 0 (auch 4~Punkte geistigen Schaden).
\end{description}

Das Ende des Konfliktes darf der Spieler von Nargrim erzählen: Ein Erfolg für die Spieler. Er beschreibt kurz, wie Nargrim Rune wieder zur Flurtür schickt und das Schloss dann mit einem Knacken nachgibt. Nargrim huscht in den Raum und schließt die Tür leise hinter sich.

Nun werden die regeltechnischen Konfliktfolgen festgestellt: Nargrim hat einen persönlichen Erfolg erzielt (hat gerade einmal die Hälfte der Konfliktpunkte verloren) und damit sein Ziel erreicht. Rune dagegen hat einen persönlichen kritischen Misserfolg erlitten. Daher folgt nun eine kurze SL-Erzählphase, in der der Spielleiter zu einem Folgekonflikt für Rune überleitet: Er muss erreichen, dass der gerade aufgetauchte Hausdiener keine weiteren Fragen mehr stellt, warum sich Rune so in der Nähe des Büros herumtreibt und wo Nargrim abgeblieben ist. Allerdings will der SL dies nur als Kurzkonflikt abhandeln, um dann die Aufmerksamkeit wieder mehr auf Nargrim zu lenken, der ja währenddessen das Büro durchsucht.
\end{beispiel}

\begin{optional}
\section{Optional: SL-Nebenkonflikt-Erzählungen}
Es ist problemlos möglich, auch dem Spielleiter während der Nebenkonflikte Erzählungen zu gestatten. Er würfelt dann zwar nicht, kann aber trotzdem, am Besten nachdem alle Spieler etwas gesagt haben, kurz die Situation zusammenfassen und/oder ein Stück weit das Verhalten der Nebenkonflikt-Gegner lenken.

Diese Regel bietet sich insbesondere bei Gruppen an, deren Charakterspieler eher auf die Darstellung ihres eigenen Helden abzielen oder und dem Spielleiter eine zusätzliche Möglichkeit zu geben, auf den Nebenkonflikt Einfluss zu nehmen.

Auch möglich aber nicht ratsam ist es, den Spielern zu verbieten, das Verhalten der Nebenkonfliktgegner zu beschreiben. Denn oftmals müssen für gute Ideen der Spieler die Nebenkonfliktgegner entsprechend reagieren, wohingegen das genaue Verhalten für den Ausgang im Allgemeinen keine Rolle spielt.
\end{optional}


\begin{optional}
\section{Optional: Eingeschränkte Erzählrechte}

Es gibt Spieler, die lieber nur ihren eigenen Charakter beschreiben, und andere Figuren lieber komplett außen vorlassen. Das ist im Prinzip zwar möglich, führt oftmals aber zu langweiligeren Darstellungen der Konflikte. Worauf aber verzichtet werden kann ist die Erlaubnis, auch andere Spielercharaktere in die eigene Beschreibung mit einzubeziehen, so dass jeder Spieler die Handlung des eigenen Charakters und die Rekationen der SLCs beschreibt. Dies fördert die Beziehung des Spielers zum eigenen Charakter.

Wird diese Optionalregel für Nebenkonflikte benutzt, sollte sie natürlich auch entsprechend für Hauptkonflikte gelten.
\end{optional}





\section{Hauptkonflikt}\label{Sec:Hauptkonflikt}
\subsection{Einsatzmöglichkeiten}
Hauptkonflikte sind die Konflikte, für die sich die Helden ins Abenteuer stürzen. Sie sind die wichtigen Meilensteine auf dem Weg zum Ziel. Auch der letze Konflikt, also der `Endkampf' sollte immer ein Hauptkonflikt sein.

Als Hauptkonflikt ist prinzipiell \emph{jeder} Konflikt möglich. Da auch der SL in den Erzählvorgang mit eingebunden ist, können auch Argumente und Diskussionen umgesetzt werden. Andererseits muss hier auch gesagt werden, dass die Stärke des Systems bei aufregenden Kampfszenen oder anderen spannenden Handlungen liegt. Gespräche können durch die strenge Verteilung der Erzählrechte vor allem bei größeren Gruppen leicht holprig werden. Aber da die meisten Höhepunkte von DSA-Abenteuern keine sozialen Konflikte sind, wird es hier kaum Schwierigkeiten geben.

Das entscheidende Merkmal von Hauptkonflikten ist, dass sie ergebnisoffen sind. Das bedeutet, dass der Spielleiter Hauptkonflikte nur dann einsetzen sollte, wenn sowohl ein Sieg der SCs als auch eine Niederlage die geplante Geschichte nicht völlig aus der Bahn wirft. Insbesondere sind sie also als Endkonflikte interessant, denn am Ende der Geschichte kann diese nicht mehr vor die Wand gefahren werden!

Auch bei Hauptkonflikten gilt: Man kann die unterschiedlichsten Zeiteinteilungen für Runden benutzen. Eine genauere Ausführung hierzu findet man bei den Nebenkonflikten.

\begin{beispiel}
Einige Beispiele:
\begin{itemize}
  \item Endkampf gegen den Bösewicht
  \item Duell in einem Turnier
  \item Eine Gerichtsverhandlung
\end{itemize}
\end{beispiel}

\subsection{Ablauf}
Zunächst wird hier der Ablauf eines Hauptkonfliktes gegen \emph{einen einzelnen}  Hauptkonfliktgegener beschrieben. Erweiterungen auf mehrere Konfliktgegener oder Mischungen mit Nebenkonflikten werden anschließend beschrieben.
\begin{enumerate}
  \item Hauptkonflikte sind üblicherweise bereits im Vorfeld durch den SL geplant. Im Normalfall stellt der Hauptkonflikt einen Konflikt mit einem einzelnen Gegner dar (z.\,B. Mensch, Monster, Dämon). Der SL soll sich dabei, ähnlich wie die CS, in den Konflikt einbringen. Daher hat auch ein Hauptkonflikt Werte, die einem Helden ähneln. Der SL legt für den Hauptkonflikt fest:
\begin{itemize}
\item Ziel, das die Helden erreichen wollen
\item Folgen, die eintreten, wenn die Helden ihr Ziel nicht erreichen
\item Konfliktpunkte des Hauptkonfliktes
\item Talentwerte für passende Talente
\item Sonderfertigkeiten und Gegenstände, die vom SL angewendet werden können
\end{itemize}

\begin{design}
\subsubsection{Designanmerkung: Werteverteilung im Hauptkonflikt}
Es ist schwierig, pauschale Angaben zu machen, wie gut oder schlecht ein Hauptkonfliktgegner zu gestalten ist. Als Fausregel gilt: Die Spieler sollen das Gefühl bekommen, dass Hauptkonfliktgegner ernsthafte Gegner, aber keine unüberwindbaren Hindernisse sind.

Um \DEF{menschliche Hauptkonfliktgegner}\index{Hauptkonfliktgegner!menschlich} (oder elfische, zwergische, orkische usw.) glaubhaft darzustellen, sollten sie ähnlich gut wie die Helden sein, vielleicht sogar etwas besser. Damit diese Art von Hauptkonflikten spannend wird, sollten die Helden auf mehrere Hauptkonfliktgegner gleichzeitig treffen oder ein Hauptkonflikt und ein Nebenkonflikt gemischt werden.

Als Richtlinie für den Talentgesamtwert des Haupttalentes sollte der maximale Talentgesamtwert dienen, den die Helden dem Hauptkonflikt entgegensetzen können. Auch die Anzahl der Konfliktpunkte sollte in etwa dieser Höhe entsprechen, vielleicht ein Punkt mehr. Auch ist es möglich, einem Hauptkonfliktgegner Wissenstalente, Sonderfertigkeiten oder auch Vor- und Nachteile zu geben.

\DEF{Nichtmenschliche Hauptkonflikte}\index{Hauptkonfliktgegner!nichtmenschlich} (z.\,B. Dämonen oder andere, monströse Wesenheiten) oder auch rein nicht-materielle Gegner (z.\,B. die entscheidende Schlichtung eines Streites) können auch mit deutlich anderen Werten gut dargestellt werden. Hier reicht dann unter Umständen ein einziger, gut ausgestatteter Hauptkonfliktgegner. Es sei ausdrücklich darauf hingewiesen, dass es auch bei Hauptkonflikten keine Obergrenze für Konfliktpunkte auf Seiten des Spielleiters gibt. Ein Hauptkonfliktgegner kann z.\,B. mit 15 Konfliktpunkten, einem Talentgesamtwert von 12 und einem automatischen Offensiverfolg ausgestattet werden, was für eine Gruppe von vier Erststüflern ein erhebliches Hindernis darstellen sollte.
\end{design}


  Um den Spielern eine Erzählhilfe für den Fortgang des Konfliktes zu geben, werden alle Konfliktpunkte, auch die des SL, offen gezeigt.

  \item {Wie beim Nebenkonflikt bekommt dann jeder Spieler seine Konfliktpunkte (anfangs 3, in höheren Stufen bis zu 7).} Dann wird der Konflikt rundenweise ausgetragen.

  \item\label{HKNaechsteRunde} Die beteiligten Spieler (also auch der Spielleiter) erzählen reihum, was im Konflikt passiert. Dabei darf, laut Konfliktende-Regel, das Ende des Konfliktes nicht vorweg genommen werden. Es ist aber möglich und erwünscht, dass der Spieler beschreibt, wie er selber Nachteile erleidet -- darüberhinaus kann er auch andere Charaktere mit einbeziehen {(vgl. Abschnitt ``Tipps für gute Erzählungen'', ab Seite~\pageref{TippsGuteErz})}. Desweiteren sollte mit einbezogen werden, wie viele Konfliktpunkte der Charakterspieler noch übrig hat.

  Wortwechsel in Hauptkonflikten werden ähnlich wie die in den Nebenkonflikten gehandhabt. Der Angesprochene (z.\,B. der SL, aber auch andere CS) darf außer der Reihe antworten. 
  
  Der SL sollte darauf achten, dass der Spieler, der eine Runde beginnt, von Runde zu Runde wechselt.
  
  \item Nach jeder Erzählung wird der \DEF{Erzählwert}\index{Erzählwert} bestimmt, der Spieler erhält entsprechend viele W20. Zum Erzählwert werden jedoch nur die eigenen Aussagen hinzugerechnet, d.\,h. im Zwiegespräch mit einem anderen Charakter bekommt der Spieler nicht für die Aussagen des anderen irgendwelche Würfel. Für die Erzählung gibt es mindestens einen und maximal fünf W20.

  Hinzu kommen irgendwelche Bonuswürfel durch Gegenstände, Wissenstalente oder Magie.
Wenn ein Spieler Sonderfertigkeiten einbrigt, sind hier meisterliche Würfe möglich. Maximal kann ein Spieler zwei Sonderfertigkeiten einbringen eine offensive und eine defensive Sonderfertigkeit. Das können auch aufgestockte Sonderfertigkeiten sein.

  Meisterliche Würfe treten mit einer einfachen Sonderfertigkeit bei einer gewürfelten 1 oder 2 ein. Ist die Sonderfertigkeit aufgestockt, gibt es bereits bei einer 1--4 einen meisterlichen Wurf.
  
  Der Spielleiter muss sich direkt nach seinem Erzählbeitrag entscheiden, gegen welchen beteiligten SC sich seine Offensive richtet (aber nur einen). Dabei muss die Erzählung wie üblich nichts damit zu tun haben, d.\,h. der SL könnte z.\,B. ausschließlich erzählen, wie sein Hauptkonfliktgegner von einem SC bedrängt wird und gegen einen anderen Spieler, dessen Held mit der Erzählung nichts zu tun hatte, offensiv werden.

  \item Haben alle beteiligten Spieler etwas beigetragen und haben Würfel erhalten, müssen sie die Würfel in Offensivwürfel und Defensiv-Würfel aufteilen und würfeln. Im Gegensatz zum Nebenkonflikt würfelt hier auch der Spielleiter.
  
  Jeder Würfel, der höchstens den Talentgesamtwert (inklusive aller Boni) zeigt, zählt als gelungener Wurf, ansonsten als misslungener Wurf.
  Die Anzahl der gelungenen Offensivwürfe gibt das \DEF{Offensivergebnis}\index{Offensivergebnis} des CS, die Anzahl der gelungenen Defensivwürfe das \DEF{Defensivergebnis}\index{Defensivergebnis}. Jeder meisterliche Wurf erhöht das Offensiv- bzw. Defensivergebnis um einen weiteren Punkt, d.\,h. meisterliche Würfe zählen wie zwei gelungene Würfe.

  \item Nun vergleicht der SL sein Defensivergebnis mit der Summe der Offensivergebnisse aller CS. Umgekehrt vergleicht der CS, gegen den sich die Offensive des SLs richtet, sein Defensivergebnis mit dem Offensivergebnis des Spielleiters. Wenn das Offensivergebnis das Defensivergebnis übertrifft, wird die Differenz an Konfliktpunkten abgezogen.

  \item Möchte ein Spieler aus dem Konflikt aussteigen, so kann er dies tun, sofern er noch Konfliktpunkte übrig hat. Im Spiel gibt der Held den Konflikt auf. Dies wird in jedem Fall für den Charakter als persönlicher Misserfolg gewertet.

  Ein Spieler, dessen Konfliktpunkte auf 0 gesunken sind, scheidet automatisch aus dem Konflikt aus (persönlicher kritischer Misserfolg).

  \item Solange noch der SL und mindestens einer der CS weiterhin am Konflikt beteiligt sind, wird die nächste Konfliktrunde ausgetragen (es geht weiter bei~\ref{HKNaechsteRunde}). Ist aber der SL oder alle CS aus dem Konflikt ausgeschieden, so endet der Konflikt.

  \item Am Ende würfelt jeder Spieler den Schaden für diesen Konflikt aus. Dazu wirft er pro Runde, die der Charakter am Konflikt beteiligt war und für jeden verlorenen Konfliktpunkt einen W20. Für jeden Würfel, der mindestens eine 5 zeigt, bekommt der Charakter einen Punkt Schaden. 

  Außerdem muss festgelegt werden, ob der Schaden körperlich oder geistig ist. Die Art geht aus dem Zusammenhang hervor und wird im Zweifelsfall vom Spielleiter festgelegt. Jegliche Verletzung oder Erschöpfung durch körperliche Anstrengung verursacht körperlichen Schaden, wohingegen Reden, Konzentration usw. geistigen Schaden. 

\end{enumerate}

\subsection{Ergebnis}
Hier muss unterschieden werden zwischen dem Gesamtergebnis des Konfliktes und den Einzelergebnissen der Spieler.

\begin{enumerate}
\item Für die Gruppe gilt:
\begin{description}
\item[SL hat keinen Konfliktpunkt übrig:] Die Charakterspieler gewinnen den Konflikt.\footnote{Es kann passieren, dass am Ende des Konfliktes weder Charakterspieler noch Spielleiter Konfliktpunkte übrig haben. In diesem Fall haben auch die Spieler gewonnen.} Insgesamt ist das Konfliktziel erreicht; es gibt keine weiteren Schwierigkeiten. Obwohl der Konflikt insgesamt gewonnen wurde, kann es sein, dass einzelne Charaktere einen persönlichen Misserfolg (oder kritischen Misserfolg) erlitten haben.
  \item[SL hat noch Konfliktpunkte übrig:] Die Charakterspieler verlieren den Konflikt. Das Konfliktziel wurde (im Normalfall) \emph{nicht} erreicht -- die Spieler haben verloren. Klarerweise haben alle Helden einen persönlichen Misserfolg (oder sogar kritischen Misserfolg) erlitten.
\end{description}


 Auch hier gilt wieder: Über das Ende des Konfliktes entscheidet der Spielleiter.  Gewinnen die Spieler den Konflikt, erreichen die Helden das Konfliktziel. Verlieren die Spieler den Konflikt, so erzählt der Spielleiter, wie sie (üblicherweise) das Konfliktziel tatsächlich nicht erreicht haben.

\item Für jeden einzelnen Spieler gilt (wenn er zu Konfliktbeginn drei oder mehr Konfliktpunkte hatte):
\begin{description}
  \item[Alle Konfliktpunkte verloren:] (persönlicher kritischer Misserfolg) Dem Charakter ist ein erhebliches Missgeschick passiert. Die Folge ist eine kleine dauerhafte (negative) Veränderung des Charakters im Wert von einem Vorteilspunkt (d.\,h. Gewinn eines Nachteils oder Verlust eines Vorteils).
  
  Dabei kann auch ein Nachteil `begonnen' werden: Gewinnt ein Charakter eine unangenehme Auffälligkeit wie z.\,B. eine Narbe, so wäre das ein Nachteil im Wert von vier Vorteilspunkten. Von diesen vier Punkten bekommt der Charakter erstmal nur einen. Der angefangene Nachteil hat noch keine Auswirkungen. Diese kommen erst dann zum Tragen, wenn auf diese Weise vier Punkte zusammengekommen sind. (Genauso kann auch der Verlust eines Vorteils begonnen werden; der Vorteil ist dann entgültig weg, wenn der letzte Punkt verschwunden ist).
  \item[Mehr als die Hälfte der Konfliktpunkt verloren:] (persönlicher Misserfolg) Dem Charakter ist ein Missgeschick passiert, das aber keine regeltechnischen Auswirkungen hat.
  \item[Höchstens die Hälfte der Konfliktpunkte verloren:] (persönlicher Erfolg) keine speziellen Vor- oder Nachteile.
  \item[alle Konfliktpunkte übrig:] (persönlicher kritischer Erfolg) Dem Charakter ist ein spektakulärer Erfolg geglückt; neben dem erfolgreich erreichten Ziel kommt es zu einer dauerhaften (positive) Veränderung des Charakters, z.\,B. Verlust einer schlechten Eigenschaft oder der Gewinn eines kleinen Vorteils (z.\,B. eine Verbindung) im Wert von einem Vorteilspunkt.
\end{description}
\end{enumerate}

\subsection{Mehrere Hauptkonfliktgegner}
Es ist auch möglich, dass der SL mehrere Hauptkonfliktgegner in einem Hauptkonflikt vorsieht. So können dann z.\,B. mehrere Individuen dargestellt werden (z.\,B. in einer Gerichtsverhandlung der Angeklagte und sein Verteidiger oder auch ganz klassisch der Magier und sein persönlicher Leibwächter).

Ist dies der Fall, so ist der SL trotzdem nur einmal pro Konfliktrunde an der Reihe -- der erreichte Erzählwert gilt dann für alle Hauptkonfliktgegner; damit wird verhindert, dass ein Hauptkonflikt zu einer SL-One-Man-Show verkommt. Trotzdem sollte der SL versuchen (evtl. abwechselnd) auf die verschiedenen SLC einzugehen. Umgekehrt sind hier die CS mehr als sonst ermuntert, auch passende Aktionen der SLC zu beschreiben.

Die Würfe des SL werden für jeden Hauptkonfliktgegner einzeln durchgeführt. Am besten ist, wenn er genügend Würfel in verschiedenen Farben zur Verfügung hat; ansonsten können die Würfel des SL auch mehrmal nacheinander gerollt werden. Dabei sucht der SL natürlich für jeden Hauptkonfliktgegner einen Helden als `Opfer' aus.

Genau wie sonst der SL müssen die CS direkt nach ihrer Erzählung entscheiden, gegen welchen Konfliktgegner sich die Offensivwürfel richten. Die Würfel dürfen nicht zwischen verschiedenen Konfliktgegnern aufgeteilt werden. Umgekehrt können natürlich die SC auch von mehreren Konfliktgegnern bedrängt werden. Um keine Konfliktpunkte zu verlieren, darf die Summe der Offensiverfolge, die gegen den Helden gerichtet sind, die Defensiverfolge nicht übertreffen.

Hier und bei den kombinierten Konflikten ist es wichtig, dass der SL darauf achtet, dass die Reihenfolge, in der alle Spieler (auch er selber) an der Reihe ist, rotiert, da die Spieler, die als letztes das Ziel ihrer Offensivwürfel ansagen müssen, im Vorteil sind.

\subsection{Kombinierte Konflikte}
Es können Haupt- und Nebenkonflikte kombiniert werden. Dabei kann es mehrere Hauptkonfliktgegner geben, aber immer nur einen Nebenkonfliktgegner. Sollten es sich z.\,B. um verschiedene Gruppen von Nebenkonfliktgegnern handeln, so werden sie einfach zusammengefasst (Addieren von Konfliktpunkten, Offensivergebnissen und Defensivergebnissen).

Der SL übernimmt nur die Rolle der Hauptkonfliktgegner. Wie bei mehreren Hauptkonfliktgegnern auch muss hier auch jeder CS seinen Angriff gegen ein Ziel richten; das sollte aus der Erzählung hervorgehen.

Der Nebenkonfliktgegner verursacht in jeder Runde auf jeden Charakter Schaden; das Defensivergebnis gilt für jeden Charakter, der dem Nebenkonfliktgegner Schaden machen will, separat (wie ein normaler Nebenkonflikt eben).

In Sachen Schaden zählt ein kombinierter Konflikt wie ein Hauptkonflikt, d.\,h. jeder Würfel, der eine 5~oder mehr zeigt, verursacht einen Schadenspunkt beim Charakter.

\begin{optional}
\section{Optional: Beschreibungen ausgeschiedener Spieler}

Man kann auch Spieler, die aus dem Konflikt herausgefallen sind, weiterhin an der Beschreibung des Konfliktes beteiligen. Im Rahmen dessen, was passiert ist, können sie entweder das Verhalten ihrer ausgeschiedenen Charaktere oder auch das Verhalten von Nebenkonfliktgegnern weiter beschreiben, allerdings ohne dafür Würfel zu bekommen. Ihre Erzählungen haben also noch Einfluss auf das Geschehen, tragen jedoch nicht zur Beendigung des Konfliktes bei.

Dabei gelten für sie natürlich weiterhin alle Erzählregeln -- der Erzählwert braucht allerdings natürlich nicht bestimmt zu werden.
\end{optional}

\begin{optional}
\section{Optional: Konfliktpunkte bringen Glück}
Wenn die Spieler zusätzliche Erzählmöglichkeiten bekommen sollen, ist es möglich, die Konfliktpunkte zusätzlich für besonders glückliche oder unglaubliche Ereignisse einzusetzen. So könnte beispielsweise ein Spieler mit Glück unter der Zellenbank den gesuchten Schlüssel für den Ausbruch finden oder sich mit einem Sprung aus dem Fenster auf das zufällig darunter stehende Pferd vor einem Angriff retten. Dafür muss der Spieler nur einen Konfliktpunkt abgeben.

Welcher Art das Glück ist bzw. wie spektakulär die Aktionen sein dürfen, müssen die Spieler vor Spielbeginn gemeinsam beschließen.
\end{optional}

\begin{optional}
\section{Optional: Offene Erzählreihenfolge}
Gerade in sozialen Konflikten ist es schöner, auf die strenge Erzählreihenfolge zu verzichten. Eine mit den Regeln gut vertraute Gruppe kann sich daher dazu entschließen, ohne feste Reihenfolge in den Konflikten zu beschreiben. Die Spieler können sich so erzählerische Bälle zuwerfen. Der Spielleiter unterbricht dann ab und zu die Erzählung, wenn alle Spieler genug Erzählpunkte gesammelt haben und lässt für eine Runde würfeln.
\end{optional}

\subsection{Beispiel}
Die Heldengruppe trifft auf den Schurken des Abenteuers, Junker Gritton. Er will mit seinen Wachen die Helden festnehmen -- umgekehrt wollen die Helden Gritton dingfest machen, um ihn an die örtliche Gerichtsbarkeit zu übergeben.

Der Spielleiter entscheidet auf einen gemischten Konflikt: Als Hauptkonfliktgegener gibt er Gritton einen Talentgesamtwert von 12 und vier Konfliktpunkte, während die Charaktere der Spieler nur drei Konfliktpunkte haben. Die Wachen werden parallel dazu als Nebenkonflikt dargestellt mit einen Offensivergebnis von 1, einem Defensivergebnis von 0 und 10 Konfliktpunkten.

Die Heldengruppe besteht aus drei Charakteren: einem Weidener Ritter, einer Gauklerin der Truppe ``Feuerball'' aus Gareth und einer Söldnerin aus Elenvina. Ort ist die Halle in Grittons Burg, die Charaktere haben gerade die Tür aufgestoßen und sehen sich Gritton und seinen Wachen gegenüber. Der Plan ist, den Junker zunächst zur Aufgabe zu überreden.

Damit nicht immer derselbe Spieler eine Runde beginnt, lässt der SL immer denjenigen beginnen, der die letzte Runde aufgehört hat. Ansonsten ist die Reihenfolge immer in Uhrzeigersinn um den Tisch herum. Das kann auch anders geregelt werden; allerdings sollte der SL darauf achten, dass jeder Spieler mal eine Runde beginnt.
\begin{description}
\item[Runde 1:]~
\begin{description}
  \item[Weidener Ritter:] ``Ich trete einen Schritt in den Raum, stütze mich auf mein Schwert und sage: `Wir haben von Eurern dunklen Plän durchschaut, Junker Gritton. So ergebet Euch und lasset Euch abführen, auf dass ihr der Gerichtsbarkeit übergeben werdet.' '' Darauf antwortet der Spielleiter in der Rolle des Junkers: ``Niemals!'' Dann ergänzt der Spieler des Ritters: ``Offensive gegen den Junker.''

  \item[Gauklerin:] ``Ich husche unbemerkt hinter dem Ritter nach links Richtung Tisch. Alle Blicke sind ja auf den Ritter gerichtet, und es ist ja auch relativ dunkel in dem Raum. Also verstecke ich mich dann so hinter dem Tisch und warte mal, was passiert. Offensive gegen die Wachen.''

  \item[Söldnerin:] ``Der Junker sieht, wie meine Söldnerin ihren Säbel langsam aus der Scheide zieht und vorsichtig auf ihn zugeht. Dabei sieht er den Säbel auf sich gerichtet und sagt: `Wachen! Haltet sie von mir fern!'. Daraufhin stellen sich die Wachen vor ihren Herrn, die Hellebarden fest in der Hand. Offensive gegen den Junker.''

  \item[SL:] ``Zum Ritter gewandt: `Ich rate Euch und Euren Gefährten, so schnell wie möglich hier zu verschwinden. Ich möchte kein Blutvergießen, und Ihr habt bestimmt auch nichts dagegen, wenn Eurer Blut in Eurem Körper verbleibt. Also zieht ab, und ihr bleibt unbehelligt. Bleibt, und ihr werdet sterben.' '' Der Spieler des Ritters antwortet: ``Wir werden nicht eher gehen, bis wir Euch Euer Handwerk gelegt haben.'' Zuletzt wieder der SL: ``Offensive gegen den Ritter.''
\end{description}

Jetzt haben alle erzählt und würfeln. Die Spieler würfeln jeweils zwei Offensiv- und drei Defensiv-Würfel. Der Weidener Ritter auf Überreden/Überzeugen, die Gauklerin auf Sich Verstecken und die Söldnerin auf Einhand-Schwerter. Die Ergebnisse sind: Ritter 1 gelungner Offensivwurf und 1 gelungener Defensivwurf (kurz 1/1), Gauklerin 2/1, Söldnerin 1/2 und SL 1/2. Der Nebenkonflikt verursacht automatisch 1/0 auf jeden Charakter.

Damit bekommt der Ritter 1 Konfliktpunkt Verlust (gegen ihn steht ein automatischer gelungener Offensivwurf durch den Nebenkonflikt und ein gelungener Offensivwurf durch den Junker, davon kann er nur einen durch den eigenen Defensiverfolg ausgleichen), Gauklerin und Söldnerin keinen (können den Schaden durch den Nebenkonflikt abwehren), der Junker keinen (er kann die beiden Offensiverfolge des Ritters und der Söldnerin mit seinen beiden gelungenen Defensivwürfen abwehren) und die Wachen 2 durch die Gauklerin. Da bisher niemand verletzt wurde, ist der Schaden geistig.

Zwischenstand der Konfliktpunkte nach der ersten Runde: Ritter 2, Gauklerin 3, Söldnerin 3, Junker 4 und Wachen 8.

\item[Runde 2:]~
\begin{description}
  \item[SL:] ``Junker Gritton sagt: `Wenn Ihr dermaßen unkooperativ seid, dann kann ich Euch auch nicht weiterhelfen.' Und dann, zu den Wachen gewandt: `Packt sie!'. Daraufhin senken die Wachen ihre Hellebarden und gehen bedrohlich in eure Richtung. Offensiv gegen den Ritter.''
  \item[Weidener Ritter:] ``Ich packe mein Schwert fester, die Adern meiner Hände treten bläulich hervor, die Knöchel verfärben sich weiß -- so erwarte ich die Wachen. Da schnellt auch schon die Spitze der ersten Hellebarde nach vorne. Offensiv gegen die Wachen.'' 
  \item[Gauklerin:] ``Ich bin ja noch unter dem Tisch, oder? Ok, dann krieche ich mal etwas nach vorne und ziehe meinen Dolch aus der Stiefelscheide. Ich kann die Beine einer Wache sehen, die vor mir steht und steche den Dolche von hinten in die Kniekehle, so dass die Wache schreiend zusammenbricht. Offensiv gegen die Wachen.''
  \item[Söldnerin:] ``Ha, es tut sich eine Lücke zwischen den Wachen auf, da sie auf uns zustürmen. Mit zwei Schritten und ein paar Paraden gegen die Hellebarden stehe ich vor dem Junker und versuche, ihm meinen Säbel vor die Kehle zu halten um ihn zur Aufgabe zu zwingen. Doch er duckt sich geschickt darunter weg und hält auf einmal zwei Kurzschwerter in den Händen. Offensiv gegen den Junker.''
\end{description}

Wieder wird gewürfelt. Die Ergebnisse: Ritter 2/2, Gauklerin 2/1, Söldnerin 0/1 und SL 2/2.

Zwischenstand der Konfliktpunkte nach der zweiten Runde: Ritter 1, Gauklerin 3, Söldnerin 3, Junker 4 und Wachen 4.

\item[Runde 3:]~
\begin{description}
  \item[Söldnerin:] ``Der Junker geht wie ein Berserker auf die Söldnerin los. Ein Schlag nach dem anderen hagelt auf die Söldnerin nieder, sie kann sich kaum noch wehren. Blut sickert aus kleinen Schnitten, wie verzweifelt sticht sie ab und zu ihren Säbel nach dem Junker. Offensiv gegen den Junker.''
  \item[SL:] ``Die zusammengesackte Wache entdeckt die Gauklerin unter dem Tisch. Sie stößt den Stuhl zur Seite und rammt ihr den Stiel der Hellebarde in den Bauch, was die Gauklerin mit einem `Uffz!' quittiert. Dann versucht sie, sich wieder aufzurichten und ruft: `Hier unter dem Tisch ist auch noch eine!' Offensiv gegen die Söldnerin.''
  \item[Weidener Ritter:] ``Ich liefere mir ein wildes Gefecht gegen zwei der Wachen. Sie schlagen und stechen mit ihren Hellebarden, doch ich kann den Schlägen ausweichen. Dabei weiche ich zwei Schritte zurück, so dass ich kurz hinter dem Türrahmen der Eingangstür stehe. So habe ich nur noch einen Gegner, der jetzt auch einen schweren Körpertreffer einstecken muss. Offensiv gegen die Wachen.''
  \item[Gauklerin:] ``Der Stoß hat mir die Tränen in die Augen getrieben. Doch schon nach wenigen Sekunden sehe ich wieder klar, die Wache zieht sich am Tisch hoch und will die anderen auf mich aufmerksam machen. Ich krieche ein Stück zurück, mache meinen Bogen bereit richte mich kurz auf und schieße der Wache einen Pfeil in den Hals. Offensiv gegen die Wachen.''
\end{description}
Die Würfelergebnisse: Ritter 0/1, Gauklerin 0/1, Söldnerin 2/2 und SL 2/1.

Die Konfliktpunkte: Ritter 2, Gauklerin 3, Söldnerin 3, Junker 3 und Wachen 4.
\end{description}

Und so geht das Beispiel weiter, bis entweder alle Charaktere aufgeben bzw. keine Konfliktpunkte mehr haben oder bis weder Wachen noch Junker Konfliktpunkte übrig haben (oder der Spielleiter aufgibt).

Am Ende muss dann noch Schaden ausgewürfelt werden: Für jede Runde und jeden verlorenen Konflipunkt würfelt jeder Spieler einen W20. Bei einer 5 oder mehr gibts einen Schadenspunkt -- in diesem Fall körperlich, weil es sich um einen Kampf gehandelt hat.


\section{Bonuswürfel}\label{Bonuswuerfel}
Durch Gegenstände, Wissenstalente oder Magie kann ein Spieler Bonuswürfel oder sogar automatische gelungene Würfe erhalten. Je nach Konfliktart können sie auf unterschiedliche Weise eingesetzt werden.

In Kurzkonflikten kann ein Spieler durch Gegenstände, Wissenstalente und Magie Zusatzwürfel bekommen. Sollte er durch Gegenstände oder Magie automatische gelungene Würfe erhalten, so geben diese auch je einen Zusatzwürfel.

Ob eine Probe erfolgreich ist oder nicht, hängt bei mehreren Würfeln vom niedrigsten Würfelergebnis ab. Ist die Probe misslungen (d.\,h. alle Würfel zeigen einen Wert, der den Talent-Gesamtwert übertrifft) \emph{und} zeigt einer der Würfel zuzüglich der Konfliktschwierigkeit eine 20 oder mehr, so muss mit einer erneuten Probe festgestellt werden, ob der Misserfolg sogar kritisch war.

Ist dagegen die Probe gelungen und zeigt der kleinste Würfel (zuzüglich der Schwierigkeit) eine 1 oder weniger, so muss erneut gewürfelt werden, ob der Erfolg kritisch war. 
Hat der Held Sonderfertigkeiten eingesetzt, so kann die Schwelle, ab der auf einen kritischen Erfolg geprüft wird, auf 2 bzw. 4 (bei zwei benutzten Sonderfertigkeiten) steigen.

\medskip

In Neben- und Hauptkonflikten werden Bonuswürfel vor dem Wurf zu den Würfeln fürs Erzählen hinzugefügt und in Offensiv- und Defensivwürfel aufgeteilt. Bonuswürfel oder automatische gelungene Würfe erhält ein Spieler immer dann, wenn aus der Erzählung klar wird, was er für einen Gegenstand verwendet. Magische Effekte wirken auch dann, wenn der Spieler die Wirkung nicht erwähnt.

Bei Gegenständen muss der Charakter meist ein bestimmtes passendes Talent benutzen, um die Bonuswürfel auszunutzen. Jeder Spieler darf pro Runde maximal die Zusatzwürfel eines benutzten Gegenstandes einbringen.

Alleine mit Wissenstalenten kann ein Spieler nicht direkt offensiv oder defensiv in einen Konflikt eingreifen. Die Wissenstalente dienen lediglich dazu, um die Anzahl der Würfel für eines anderes Talent zu erhöhen. Dazu muss allerdings das Thema des Wissenstalentes auch in den Konflikt mit einfließen -- wie bei anderen Talenten auch legt der SL fest, ob ein Wissenstalent relevant ist oder nicht.

Statt ein Wissenstalent selber zu benutzen kann ein Spieler auch entscheiden, durch Zurufe, Zeichen oder auf eine andere Weise einem anderen Spieler zu helfen. Dazu kann er die Bonuswürfel für ein Wissenstalent vor dem Würfeln auch an einen anderen Spieler abgeben. Also kann sich auch ein Spieler, der selbst kein relevantes Basis- oder Spezialtalent hat, an einem Konflikt beteiligen und mit seinem Wissenstalent anderen helfen (auch wenn er selber nichts ausrichten kann).

\begin{optional}
\section{Optional: Nachgewürfelte Bonuswürfel}

Oft haben die Spieler während eines Konfliktes Bonuswürfel zur Verfügung -- diese Würfel werden normalerweise einfach zusätzlich zu den Würfeln gerollt. Wenn nicht genug Würfel zur Verfügung stehen oder die Wichtigkeit von Bonuswürfeln (gerade für Charaktere mit hohen Talentgesamtwerten) senken wollen, können folgender Regel gegriffen werden: Statt zusätzliche Würfel zu bekommen hat der Spieler das Recht, pro Bonuswürfel einen Misserfolg nachzuwürfeln. Die Bonuswürfe werden einzeln hintereinander ausgeführt, so dass es problemlos möglich ist, einen einzelnen Würfel auch mehrfach nachzuwürfeln.
\end{optional}


