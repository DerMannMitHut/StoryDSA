\chapter{Spiel-Leiten}\label{Ch:SpielLeiten}
\lettrine{E}{in} Geheimnis oder Mysterium gibts in diesen Spielregeln nicht, daher schadet es auch nichts, wenn auch Charakterspieler dieses Kapitel lesen. Es soll dem Spielleiter die n"otigen Techniken und auch ein paar Ideen liefern, eine spannende Story zu liefern.

\section{Aufgaben des Spielleiters}
Bei StoryDSA hat der Spielleiter eine recht umfassende und wichtige Aufgabe: Er lenkt die Geschichte. Das bedeutet im Normalfall, dass sich ein Spielleiter auch außerhalb des eigentlichen Spielabends vorbereitet. Das kann er mit Hilfe von fertigen Abenteuern (z.\,B. Kaufabenteuer oder Download-Abenteuer) tun oder sich selbst was ausdenken. Ideen und Hilfen hierzu gibt das Kapitel `Abenteuer vorbereiten'.

Eine zweite wichtige Aufgabe ist, das Spiel am Laufen zu halten. Das ist, neben ein paar allgemeinen Betrachtungen, der Schwerpunkt dieses Kapitels. Insbesondere bekommt der Spielleiter hier ein paar Tipps zur Improvisation. Improvisieren muss der Spielleiter nämlich immer dann, wenn die Spieler mit ihren Helden etwas machen, was der Spielleiter nicht vorbereitet hat. Das kommt sehr häufig vor und endet leider allzuoft damit, dass der Spielleiter die Ideen der Spieler beispielsweise mit einer SL-Erzählphase abwürgen muss und diese sich dann gegängelt fühlen (obwohl dem Spielleiter das nach den \StoryDSA-Regeln natürlich zusteht).

Darüberhinaus kann zur Planung ein gewisse Zeiteinteilung wichtig sein. Der Schwerpunkt bei \StoryDSA liegt auf dem gemeinsamen Spiel, also dem Freien Spiel, Nebenkonflikten und natürlich Hauptkonflikten. Für eine Phase freies Spiel oder einen Nebenkonflikt sollte der Spielleiter daher ca. 10--20 Minuten einplanen, für einen Hauptkonflikt sogar bis zu 30~Minuten. Dagegen sollten Kurzkonflikte in etwa 5~Minuten abgehandelt sein, SL-Erzählphasen sollten etwa 2~Minuten, keinesfalls aber länger als 5~Minuten, dauern.

Dauert ein Spielabend also etwa 4~Stunden, so könnte er 2~Hauptkonflikte, 5~Nebenkonflikte, 5 Phasen mit freiem Spiel und die dazugehörigen SL-Erzählphasen unterbringen. Dafür müssen die Spieler aber in den vier Stunden intensiv spielen und sich nicht lange mit Regelfragen oder Gesprächen abseits des Spieles aufhalten. Je nachdem, wie viele solche Unterbrechungen ein Spiel hat, sind es entsprechend weniger Phasen, die in 4~Stunden geschafft werden.

Die angegebenen Zeiten sind natürlich nur Richtlinien und keine Gesetze. Gerade das Freie Spiel ist in manchen Spielgruppen beliebter als in anderen. Hier sollte der Spielleiter das Spiel so lange laufen lassen, wie es allen Beteiligten Spaß macht. Sobald es anfängt abzuflauen, ist der richtige Zeitpunkt, das Spiel zu unterbrechen und mit einer Erzählphase zu einem Konflikt überzuleiten.


\section{Die Regeln}
Die erz"ahlte Geschichte ist das wichtigste Element in \StoryDSA; es geht darum, dass die Spieler eine interessante Geschichte erleben und diese in einem gewissen Rahmen selber mitgestalten k"onnen. Dazu stellen die Regeln Hilfsmittel zur Verf"ugung: Konflikte, Freies Spiel und SL-Erz"ahlphasen, die ja bereits beschrieben wurden.

Dabei sind die Regeln so angelegt, dass die Charakterspieler im wesentlichen \DEF{Richtungsentscheidungen}\index{Richtungsentscheidungen} und \DEF{Farbe}\index{Farbe} hinzuf"ugen k"onnen. Dabei bedeutet Farbe, dass die genaue Ausgestaltung zwar in der Hand der Spieler liegt, die Ereignisse selbst jedoch in der Hand des Spielleiters liegen. Farbe kommt beispielsweise bei Nebenkonflikten ins Spiel, bei denen der Spielleiter bestimmt, um was es geht, die Charakterspieler jedoch sagen, wie sie ihr Ziel erreichen. Dabei ist die Konfliktende-Regel ganz wichtig, denn kein Spieler darf das Ende vorwegnehmen.

Richtungsentscheidungen k"onnen die Spieler, wenn der Spielleiter es zul"asst, im freien Spiel treffen: Welchem Teil der Geschichte folgen die Charaktere als n"achstes, d.\,h. welcher Handlungsstrang soll weiter verfolgt werden? Ob er Richtungsentscheidungen zulassen m"ochte, sollte er den Spielern in der vorangehenden SL-Erz"ahlphase klar machen, indem er beispielsweise sagt: ``\dots und so sitzt ihr im `Goldenen Krug' und diskutiert "uber die Frage, wie es nun weitergehen soll. Was wollt ihr also tun?''

Dar"uberhinaus k"onnen Hauptkonflikte Wendungen ins Spiel bringen. Bei Nebenkonflikten steht das Ergebnis ja im Wesentlichen fest, so dass auch ein misslungener Nebenkonflikt den erw"unschten Ausgang hat; eventuell mit unerwarteten Nebenwirkungen, jedoch wird das Ziel der Charaktere erreicht. Bei Hauptkonflikten dagegen ist das Ende unklar. Das auch ein Punkt, den der Spielleiter nicht aus dem Auge verlieren darf: Wie geht es weiter, wenn die Charaktere gewinnen? Wie geht es weiter wenn die Charaktere verlieren?

\section{Charaktergeschichten und Geschichtscharaktere}
Es gibt f"ur DSA eine ziemlich gro"se Anzahl an Abenteuern, die sich f"ur \StoryDSA gut eignen, denn die meisten Autoren gehen davon aus, dass der Spielleiter die Story lenkt. So ist es meist problemlos m"oglich, Kauf- oder Downloadabenteuer zu \StoryDSA-Abenteuern umzugestalten. Im Wesentlichen muss der Spielleiter nur dar"uber nachdenken, welche Teile des Abenteuers SL-Erz"ahlphasen, Freies Spiel, Kurz-, Neben- oder Hauptkonflikt werden soll.

Jedoch ist v"ollig klar, dass jede vorgefertigte Geschichte nicht wirklich auf die Bed"urfnisse der Spieler zugeschnitten ist. Andererseits sind vorgefertigte Abenteuer nat"urlich weniger Arbeit als selber gemachte und -- wenn es um das Spielen von Metaplot-Ereignissen geht -- oft auch die einzig sinnvolle M"oglichkeit. Dabei verlangen gerade diese Geschichten besondere Charaktere: So ist es praktisch unm"oglich, das Jahr des Feuers mit einem Haufen Borbaradianer wie vorgesehen zu spielen.

Dieses Dilemma l"asst sich auf verschiedene Arten l"osen, die alle ihre Berechtigung haben. Alle Spieler sollten daher vor dem Spiel dar"uber reden und gemeinsam nach einer L"osung suchen.
\begin{description}
  \item[Zugeschnittene Geschichten] Der SL erkl"art sich dazu bereit, auf die Charaktere zugeschnittene Geschichten zu leiten. Die Gruppe erschafft also zuerst die Charaktere und gibt dem Spielleiter dann etwas Zeit, eine Geschichte auszuarbeiten. 

  \emph{Vorteile:} Die Spieler k"onnen genau den Charakter spielen, den sie m"ochten. Die Abenteuer passen zu den Charakteren. Der SL kann seiner Kreativit"at freien Lauf lassen. Spielleiterwechsel sind m"oglich.

  \emph{Nachteile:} Der SL muss relativ viel Arbeit in das Spiel stecken. Dar"uberhinaus ist es schwierig, dem offiziellen Meta-Plot zu folgen, da vieles in den Kaufabenteuern pr"asentiert wird. Spielleiterwechsel sind nicht unbedingt einfach.

  \item[Zugeschnittene Charaktere] Die Spieler einigen sich auf eine fertige Geschichte (vorzugsweise auf ein Abenteuer, das "uber mehrere Spielabende verl"auft oder eine gr"o"sere Kampagne, wie die Borbarad-Kampagne, das Jahr des Feuers, o.\,"a.) Dann erschaffen sie Charaktere nach Vorgaben der Kampagne, d.\,h. der Spielleiter kann sehr genaue Vorgaben und Einschr"ankungen bzgl. der Charakterwahl machen.

  \emph{Vorteile:} Die Spieler erleben gemeinsam ein St"uck aventurischer Geschichte, der SL hat einen relativ geringen Vorbereitungsaufwand. Das Abenteuer passt gut zu den Charakteren.

  \emph{Nachteile:} Da die Spieler die Geschichte ja vorher nicht kennen, kaufen sie die Katze im Sack und haben "ublicherweise geringere Freiheiten, was Charaktererstellung angeht. Spielleiterwechsel sind w"ahrend der Kampagne nicht m"oglich.

  \item[Ausgew"ahlte Abenteuer] Auch hier erschaffen die Spieler die Charaktere, die Abenteuer sind aber von diesen Charakteren weitgehend unabh"angig. Der Spielleiter benutzt gr"o"stenteils vorgefertigte Abenteuer und sucht sie so aus, dass der Inhalt mit den Spielercharakteren spielbar ist. Diese Spielart f"uhrt meist zu episodenhaftem Spiel mit unzusammenh"angenden Geschichten. Hier kann es sich auch anbieten, dass Spieler mehrere Charaktere erschaffen und jeweils die besser passenden ins Abenteuer f"uhren. Die Charakterwahl sollte dann gemeinsam mit der gesamten Gruppe vorgenommen werden.

  \emph{Vorteile:} Die Spieler haben recht gro"se Freiheiten bei der Charaktererschaffung (sehr exotische Charaktere sollten vermieden werden, da sich daf"ur schlecht fertige Abenteuer finden lassen), der Aufwand f"ur den Spielleiter ist relativ gering. Ein Spielleiterwechsel ist kein Problem.

  \emph{Nachteile:} Die Geschichte ist nicht gut auf die Charaktere abgestimmt. Der Spielleiter hat den zus"atzlichen Aufwand, fertige Abenteuer zu pr"ufen und passende auszuw"ahlen.
\end{description}

Auf Verschiedene Arten von Abenteuern und die Vorbereitung wird im Kapitel~\emph{Abenteuer vorbereiten} ausf"uhrlich eingegangen. Dort wird auch eine Methode zur Erstellung eigener Abenteuer vorgestellt und ein ausführliches Beispiel gegeben.

Im Folgenden soll es um Techniken gehen, die ein Spielleiter während einer Sitzung benutzen kann, um die Regeln optimal auszunutzen.



\section{Wahl der Konfliktart}
Die Wahl der Konfliktart ist entscheidend dafür, welchen Stellenwert bestimmte Situationen haben. Durch Neben- und insbesondere Hauptkonflikte werden Szenen hervorgehoben, Kurzkonflikte  beschreiben nebensächliches. Zunächst noch eine Zusammenfassung der Eigenschaften der verschiedenen Konflikte:
\begin{description}
	\item[Kurzkonflikt:] Der Spielleiter legt das Konfliktziel fest; der Charakterspieler entscheidet, wie sein Charakter das Ziel erreichen will. Dann wird einmal gewürfelt. Der Charakterspieler interpretiert einen gelungen Konflikt (d.\,h. das Ziel wird erreicht), der Spielleiter interpretiert einen misslungenen Konflikt (d.\,h. das Ziel wird nicht erreicht oder es wird zwar erreicht, aber nur unter erschwerten Bedingungen oder zusätzlichen Schwierigkeiten). Üblicherweise ist nur ein Charakter betroffen, andere können helfen.
	
	\item[Nebenkonflikt:] Der Spielleiter legt das Konfliktziel fest; die Charakterspieler erzählen rundenweise, wie das Konfliktende näher rückt. Dabei wird öfter gewürfelt. Die Charakterspieler interpretiert einen gelungen Konflikt (d.\,h. das Ziel wird erreicht), der Spielleiter interpretiert einen misslungenen Konflikt (d.\,h. das Ziel wird nicht erreicht oder es wird zwar erreicht, aber nur unter erschwerten Bedingungen oder zusätzlichen Schwierigkeiten). Üblicherweise sind mehrere, wenn nicht alle Charaktere, beteiligt.
	
	\item[Hauptkonflikt:] Der Spielleiter legt das Konfliktziel fest; die Charakterspieler erzählen zusammen mit dem Spielleiter rundenweise, wie das Konfliktende näher rückt. Dabei wird öfter gewürfelt. Die Charakterspieler interpretiert einen gelungen Konflikt (d.\,h. das Ziel wird erreicht), der Spielleiter interpretiert einen misslungenen Konflikt (d.\,h. das Ziel wird nicht erreicht). Üblicherweise sind mehrere, wenn nicht alle Charaktere, beteiligt.
\end{description}


\subsection{Haupt- oder Nebenkonflikt?}
Ein Konflikt sollte ein Hauptkonflikt sein, wenn die folgenden Fragen mit ja beantwortet werden:
\begin{enumerate}
	\item Sind beide möglichen Ausgänge des Konfliktes interessant (SCs erreichen ihr Ziel oder nicht)?
	\item Ist am Konflikt ein für das Abenteuer oder die Kampagne wichtiger SLC als Konfliktgegener beteiligt?
	\item Berührt das Thema der Konfliktes den Kern des Abenteuers (vgl. Seite~\pageref{subsec:DerKernDesAbenteuers})?
\end{enumerate}
Die Fragen sind der Wichtigkeit nach sortiert, d.\,h. wenn die erste Frage mit nein beantwortet wird, handelt es sich mit Sicherheit nicht um einen Hauptkonflikt. Denn dann steht der Ausgang des Konfliktes ja schon von vorne herein fest und sollte auch so geschehen. Auch die dritte Frage ist immer noch von zentraler Bedeutung, kann jedoch durch einen extrem wichtigen NSC (Frage 2) wettgemacht werden, d.\,h. wenn der Konflikt zwar nicht das Konfliktthema berührt und trotzdem ein sehr zentraler NSC als Konfliktgegner beteiligt ist, dann handelt es sich wahrscheinlich trotzdem um einen Hauptkonflikt.

Ist der Konflikt zwar offen (also Frage 1 mit ja beantwortet), aber berührt das Thema nicht und ist ohne zentralen NSC, so sollte statt eines Hauptkonfliktes besser ein Nebenkonflikt mit offenem Ende ausgetragen werden.

\subsection{Kurz- oder Nebenkonflikt?}
Ist bereits klar, dass es sich bei dem Konflikt nicht um einen Hauptkonflikt handelt, so kommt ein Kurz- oder ein Nebenkonflikt in Frage. Nur wenn die folgenden Fragen mit ja beantwortet werden, sollte ein Nebenkonflikt benutzt werden:
\begin{enumerate}
	\item Ist der Konfliktinhalt so interessant, dass man sich mehrere Minuten Spielzeit damit aufhalten möchte?
	\item Besteht der Konflikt nicht nur in einer Diskussion mit SLCs?
	\item Sind alle SCs am Konflikt beteiligt?
\end{enumerate}
Auch hier sind die Fragen der Wichtigkeit nach geordnet. Nur, wenn der Konfliktinhalt spannend genug ist, sollte überhaupt ein Nebenkonflikt durchgeführt werden. Handelt es sich dabei aber um eine reine Diskussion mit einem oder mehreren SLCs, so ist eine Diskussion am Spieltisch während eines Nebenkonfliktes unvermeidbar. Leider ist aber die Struktur solcher Konflikte zu starr, so dass sich eher ein Kurzkonflikt anbietet, bei dem das Ende der Diskussion nach dem Würfeln entsprechend dem Ergebnis ausgespielt wird. Zuletzt sollten noch alle oder zumindest alle bis auf ein SC am Konflikt beteiligt sein, damit das Spiel für die unbeteiligten Zuschauer nicht zu langweilig wird.









\section{Improvisation}

Oft reicht leider auch nicht die beste Vorbereitung aus. Die Spieler werden
auch bei der besten Vorbereitung immer wieder auf Ideen kommen, die man als
Spielleiter nicht so vorhergesehen und vorbereitet hat. Um nicht in die
Verlegenheit zu kommen, die Spieler mit ``Gewalt'' in die gewünschten Bahnen zu
drücken, sollte der Spielleiter solche Situationen durch Improvisation 
lösen. Um dabei erfolgreich zu sein, gibt es einige bewährte Techniken.

\subsection{Ja-Sager-SL}
Viele Spielleiter neigen dazu, ihren Spielern erstmal alles zu verbieten was
irgendwie möglich ist. Sie leiten das Spiel als \DEF{Nein-Sager-SL}\index{Nein-Sager-SL}
(NSSL)\index{NSSL|see{Nein-Sager-SL}}, d.h. wenn der Spieler fragt, ob es dieses oder jenes gerade gibt, sagen sie erstmal
pauschal nein, es sei denn, es gibt einen Grund der zwingend dafür spricht.
Vorteil dieser Methode ist die Sicherheit, dass der Spielleiter seine
Vorbereitung nicht verlässt. Nachteil dieser Methode ist, dass jegliche
kreative Energie der Spieler abgeblockt wird. Dabei kann man gerade diese oft
zur Improvisation nutzen.

Nach dieser Beschreibung ist es nicht schwierig zu folgern, was ein
\DEF{Ja-Sager-SL}\index{Ja-Sager-SL} (JSSL)\index{JSSL|see{Ja-Sager-SL}} ist,
nämlich das Gegenteil eines NSSL. Das bedeutet, ein
JSSL sagt zu Spieler-Ideen immer ja, es sei denn, es gibt einen Grund der
zwingend dagegen spricht. Also sagt auch ein JSSL manchmal Nein, nur eben
seltener als ein NSSL. Ein JSSL weicht damit zwangsläufig häufiger von seiner
Vorbereitung ab, kann aber auch die Ideen der Spieler nutzen, um im Spiel dahin
zu kommen, wo er hin möchte.

Und wo hilft das bei der Improvisation? Ganz einfach: Es kommt im Spiel häufig
vor, dass die Spieler nicht genau wissen, wo sie jetzt weiter machen wollen,
die Situation ist verfahren. Der Spielleiter hat zwar einen Plan, wie das
Abenteuer weiter gehen soll, jedoch kommen die Spieler einfach nicht drauf
sondern versuchen was anderes. Wenn der SL in dieser Situation einfach so
flexibel ist und zu einer Idee der Spieler ja sagt, dann kann er das Ergebnis
meist so benutzen, dass das Abenteuer dann weiter geht.

Konkretes Beispiel: Der Spielleiter möchte, dass die Spieler im Wald in der
Nähe des Dorfes nach Spuren suchen; sie würden dann Pferdespuren finden und
dann weiter zum Versteck der Räuber kommen. Dummerweise kommen sie nicht drauf
und überlegen hin und her, wie sie das Versteck finden können. Eine Spielerin
kommt auf die Idee, dass die Räuber für ihre Überfälle einen Wagen benutzt
haben müssen, da das Diebesgut recht schwer ist. Der Überfall ist aber schon
eine Woche her, also ist fraglich, ob überhaupt noch Wagenspuren zu finden
sind. Ein NSSL sagt: ``Nein, da gibts keine Spuren'' und wartet darauf, dass die
Spieler auf die Idee kommen, beim Dorf nach Spuren zu suchen. Ein JSSL sagt:
``Ja, da sind gerade noch so Spuren zu erahnen'' und macht weiter, als würden die
Spieler den Pferdespuren folgen.

Eine weitere gute Sache ist, dem einfachen \emph{ja} ein \emph{und} oder ein \emph{aber} 
anzuhängen. Damit kann man die Idee des Spielers aufnehmen und hat mehr Kontrolle über den
weiteren Verlauf nehmen. Als Beispiel wählen wir wieder eine festgefahrene
Situation: Die Charaktere benötigen ein Motiv als Indiz für die Schuld des
mutmaßlichen Mörders. Die Spieler sind auf der richtigen Spur, kommen aber
nicht drauf, bei der Versicherungsgesellschaft nach einer
Lebensversicherungspolice zu fragen. Eine Spielerin kommt aber auf die Idee,
die Büroräume heimlich nach einem Hinweis für ein Motiv zu durchsuchen. Der
JSSL sagt: ``\emph{Ja}, in den Büroräumen kannst du was finden, \emph{aber} dazu musst du
den Safe knacken und dabei besteht das Risiko, dass du den Alarm auslöst.''
Dadurch ist aus einer einfachen Befragung bei der Versicherungsgesellschaft ein
eventuell spannender Konflikt mit ungewissem Ausgang geworden.

Nicht immer fällt einem SL eine geeignete Ergänzung ein. Trotzdem ist
das Aufnehmen von Ideen der Spieler grundsätzlich eine gute Idee, um schneller
in der Geschichte voran zu kommen. Darüberhinaus fühlen sich die Spieler auch
bestägtigt, da viele ihrer Ideen nun Erfolg haben. Auf der anderen Seite darf
auch ein JSSL das Wort \emph{nein} nicht vergessen: Die Ideen der Spieler müssen
plausibel ins Spiel passen und zu einfach sollen die Abenteuer auch nicht
werden. Eine Aussage wie ``Mein Charakter löst das Abeneuer'' ist sicherlich
immer mit einem klaren \emph{nein} zu beantworten.

\subsection{Entscheidungen fällen}
Häufig steht der Spielleiter vor dem Problem, eine nicht vorbereitete Situation
auflösen zu müssen. Die Spieler haben also etwas unerwartetes gemacht und der
Spielleiter steht vor der Frage: Wie reagieren die SLCs? Was passiert sonst
noch? Meistens ist die erste Idee die richtige, denn das, was für einen selber offensichtlich scheint, überrascht die anderen doch ziemlich.

Ist das Problem aber größer und hat der Spielleiter etwas Zeit (z.\,B. aufgrund einer kurzen Spielunterbrechung oder während des freien Spieles), kann es sich lohnen, ein paar mehr Gedanken zu machen. Eine gute Möglichkeit ist es, sich zunächst ein paar Varianten zu
überlegen und dann eine davon auszuwählen. Es hat sich bewährt, folgende
Varianten zu bedenken:

\begin{itemize}
\item wahrscheinlichste (bzw. eine sehr wahrscheinliche) Variante
\item überraschenste (bzw. eine sehr unwahrscheinliche) Variante
\item für die Charaktere schwierigste Variante
\item für die Charaktere einfachste Variante
\end{itemize}

Diese Varianten müssen nicht alle verschieden sein, eventuell kommen auch
nur drei verschiedene Versionen raus (wenn z.B. die für die Charaktere beste
Variante mit der unwahrscheinlichsten zusammenfällt).

\begin{beispiel}
\paragraph{Beispiel:} Die Helden beschließen statt sich auf die eigenen Fähigkeiten zu
verlassen, für die Berge lieber einen Einheimischen aus dem Dorf als Führer zu
gewinnen. Der Spielleiter hatte diese Idee nicht vorausgesehen und sieht keinen
Grund, der gegen einen Führer spricht, also sagt er ja (denn er ist ein JSSL).
Aber wie ist der Führer nun? Die wahrscheinlichste Variante ist sicherlich,
dass die Charaktere als Führer einen einheimischen Jäger finden, der sich recht
gut in den Bergen auskennt. Sehr überraschend wäre, wenn der Führer ein
Mädchen von 9 Jahren ist. Die einfachste Variante für die Charaktere wäre, wenn der
Führer das Ziel der Helden kennt und ohne Probleme hinführt. Die schwierigste
Variante ist, dass der Führer grob weiß, wohin die Charaktere wollen, sie
jedoch absichtlich hintergeht, um selber ohne die Charaktere dahin zu gelangen.

Von diesen Varianten wählt der SL dann die aus, die er gerade für die beste
hält. Eine sehr interessante Möglichkeit ist sicherlich die für die Charaktere
schwierigste Variante, allerdings kommt die Spielgruppe dann in der Geschichte nur
langsam voran. Die für die Charaktere einfachste Variante dagegen beschleunigt das
Spiel und führt die Charaktere schnell zu der Stelle, an der das Dungeon
beginnt, genau wie die wahrscheinliche Variante. Die überraschendste Variante
ist nicht wesentlich schlechter als die wahrscheinlichste, eröffnen in diesem 
Beispiel jedoch interessante Möglichkeiten, die Rollen auszuspielen und kleinere
Schwierigkeiten einzubauen.
\end{beispiel}

\subsection{Zufallswürfe}

Abseits der normalen Regelmechanik können Würfel eine gute Inspirationsquelle
für Improvisation sein. Dabei wirft der Spielleiter einfach einen W20.
Ist das Ergebnis niedrig, dann passiert etwas günstiges für
die Charaktere, ist es hoch, passiert was ungünstiges.

Dabei wird nicht auf einen bestimmten Wert o.ä. gewürfelt, sondern einfach nur
eine allgemeine Tendenz in der gerade anstehenden Situation festgelegt. Ist das
Ergebnis beispielsweise sehr schlecht, werden die Charaktere von 6 statt von 4
Räubern überfallen. Ist es gut, lässt sich die Bardame leichter betören und
rückt mit den Informationen raus.

\subsection{SLC-Reaktionen}
Manchmal wissen die Spieler einfach nicht weiter oder springen nicht auf das Abenteuer, wie es geplant wurde, an. Dann hilft oft nur eine krasse Reaktion eines oder mehrerer SLCs, um die Spieler aus der Reserve zu locken. Typische `gute' SLC-Aktionen sind:
\begin{itemize}
  \item gewaltt"atig werden
  \item ein Geheimnis aufdecken
  \item einen Verrat begehen
  \item einfach ein Arschloch sein
\end{itemize}

Mit solchen Mitteln erzwingt der Spielleiter normalerweise eine Handlung der Spieler, gerade in einem Spiel wie StoryDSA, in dem die Spielercharaktere Helden sind. Denn solche Reaktionen von SLCs sind oft ungerecht und verlangen nach Aufklärung, insbesondere wenn das Motiv nicht wirklich klar ist. Ein Motiv ist für einen solchen Ausbruch auch nicht unbedingt nötig: Entweder ergibt sich dann im Laufe des Spiels eine vernünftige Begründung oder das Motiv bleibt im Unklaren und liegt in der Persönlichkeit des SLC verborgen.


\chapter{Sammlung: Spiel-Leiten}
Bisher sammeln sich hier nur die Ideen, was im Spielleiter-Kapitel alles behandelt werden soll, in Form von Überschriften und Stichpunkten.

Wichtige Quellen:

\verb;http://grofafo.org/index.php/topic,26021.msg580443/topicseen.html#new;

\section{Vorbereitung}
\begin{itemize}
  \item Anzahl Konflikte (vgl. Donjon-Regeln)
  \item Evtl. Ressourcen-Beschränkung (Peng!)
\end{itemize}
