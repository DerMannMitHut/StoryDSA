\BN
\chapter{Liste der Vor- und Nachteile} \label{Ch:VorNachteile} \index{Vorteile} \index{Nachteile}
\begin{description}
\item{Adlig (2 VP):} Der Held erhält einen Bonuswürfel beim Umgang mit anderen Adligen.

\item{Affinität zu (Geistern, Elementaren, Dämonen):} 1 Bonuswürfel im Umgang entweder mit Geistern oder mit Elementaren oder Dämonen. Kosten: 2 Vorteilspunkte

\item{Akademische Ausbildung:} Grundwissen in 2–4 Wissenstalenten und Etikette; Kosten: 2 Wissenstalente: 6 Vorteilspunkte, 3 Wissenstalente 8 Vorteilspunkte, 4 Wissenstalente 10 Vorteilspunkte.

\item{Angenehme Auffälligkeit (2 oder 6 VP):} Der Held fällt seinen Mitmenschen z. B. wegen seiner wohlklingenden Stimme, seines hübschen Gesichtes oder seiner besonders athletischen Figur angenehm auf. In passenden Situationen – in der Regel solchen gesellschaftlicher Art – stehen dem Helden 1 (2 VP) oder 2 (6 VP) Bonuswürfel zur Verfügung.

\item{Astrale Regeneration:} 50~\% Chance (auf W20 1 bis 10), einen zusätzlichen AsP zur normalen Regeneration zu erhalten. Kosten: 1 Vorteilspunkt.

\item{Astrale Regeneration (3 VP)}
Pro Regenerationsphase (vgl. 10.5) hat der Held eine 50~\%-Chance (1-10 auf W20), einen zusätzlichen AsP zur"uckzugewinnen. [Anm.: Dieser Vorteil wurde bereits unter 10.5 der Regelversion 0.12 etwas „versteckt“ aufgef"uhrt. Damit er nicht untergeht, f"uhre ich ihn hier noch einmal auf. Die Kosten wurden aufgrund folgen- der "uberlegungen auf 3 VP festgesetzt: Im Ergebnis stellt dieser Vorteil den SC so, als h"atte er ein (Berufs-)Talent „Astrale Regeneration“, dessen Talentwert auf 10 festgelegt ist, und gibt dem SC dar"uber hinaus einen Bonusw"urfel. In Anwendung der allgemeinen Regeln (vgl. 2.4 ) wurden f"ur die Aktivierung des Talents 1 VP und f"ur den Bonusw"urfel 2 VP berechnet. Wenn man – "ahnlich wie bei DSA4 – verschiedene Stufen des Vorteils w"unscht, k"onnte man das "uber die Zahl der W"urfel regeln. Bei 2 W"urfeln kostete der Vorteil dann 7 Punkte.]

\item{Astralmacht (Hohe Astralenergie) (3, 6, 9 oder max. 12):}
F"ur je 3 investierte VP erwirbt ein magisch begabter Held 1 zus"atzlichen AsP. Ein Held kann auf diese Weise h"ochsten 4 zus"atzliche AsP gewinnen. [Anm.: Da Lebens- und Willenskraft im Verh"altnis 1:2 in Astralenergie umgewandelt werden k"on- nen (vgl. 10.1), wurden hier die Kosten pro AsP auf die H"alfte der Kosten angesetzt, die f"ur einen zus"atzliche Punkt Lebens- oder Willenskraft (s. u. Hohe Lebens- und Willenskraft) gezahlt werden m"ussen. Die Beschr"ankung auf h"ochstens 4 zus"atzliche AsP dient dem Spielgleichgewicht.]

\item{Astralmacht:} 2 / 4 / 6 Astralpunkte zusätzlich. Kosten: 1 / 2 / 3 Vorteilspunkte.

\item{Ausbildung (variable Kosten, s. dazu unten)}
Dieser Talentvorteil gibt dem SC Bonuspunkte auf ausbildungsrelevante Talente. Der SC kann Bonuspunkte f"ur maximal f"unf Talente erhalten. Dabei kostet Grundwissen in einem Wissenstalent jeweils 2 Punkte; mehr als Grundwissen kann ein SC im Rahmen der Ausbildung nicht erwerben. Jeder Punkt in einem anderen Talent kostet einen Punkt, wobei jedoch zu beachten ist, dass Spezial- talente erst noch aktiviert werden m"ussen, was wiederum 1 Punkt kostet. Der SC muss also 2 Punkte investieren, wenn er ein Spezialtalent auf einen Wert von 1 bringen will. Ein Berufstalent kostet 1 Punkt; es darf nur ein Berufstalent pro Ausbildung gew"ahlt werden. Ein SC darf zu Spiel- beginn nur eine Ausbildung durchlaufen haben. Einige typische der zahlreichen Ausbildungs- varianten sind im folgenden aufgef"uhrt:
\begin{itemize}
\item   Akademische Ausbildung (Krieger/Magier/Gelehrte) (6, 8 oder 10 VP) Der SC hat Grundwissen in Etikette sowie in 2 (6 VP), 3 (8 VP) oder 4 (10 VP) weiteren passenden Wissenstalenten erworben. SC mit den Professionen Krieger, Magier oder Gelehrter m"ussen diesen Vorteil w"ahlen.
\item   Schwertgesellen-Ausbildung (6 VP) Der SC hat Grundwissen in Etikette erworben. Auf das Talent seiner Hauptwaffe – welche das ist, h"angt von dem gelernten Kampfstil ab, wird aber "ublicherweise „Einh"ander“, „Zweih"ander“ oder „Fechtwaffen“ sein – erh"alt er einen Bonus von 2 Punkten, von denen einer der
Aktivierung dieses Waffentalents dient. Dar"uber hinaus hat der Held entweder Grundwissen in einer Sprache bzw. einem anderen Wissenstalent oder einen Bonus von 2 Punkten in einem passenden anderen Talent. SC mit der Profession Schwertgeselle m"ussen diesen Vorteil w"ahlen, anderen Professionen steht er zumindest zu Beginn ihrer Laufbahn, also bei der Charaktererstellung, nicht zur Verf"ugung.
\item   Handwerksausbildung (5 VP) Der SC hat eine Handwerksausbildung hinter sich. Er erh"alt das entsprechende Berufstalent mit einem TaW von 10 und dazu 2 passende Talente aus den Bereichen Handwerk und/oder Wissen. Handwerkstalente erhalten einen Bonus von h"ochstens 2 Punkten. In dem/den gew"ahlten Wissenstalent(en) hat der SC Grundwissen.
\end{itemize}

\item{Ausdauernd:} 1 Bonuswürfel, wenn körperliche Ausdauer gefragt ist; Kosten: 2 Vorteilspunkte

\item{Ausdauernd (2 VP):}
Der Held erh"alt 1 Zusatzw"urfel in Situationen, in denen k"orperliche Ausdauer gefragt ist.

\item{Ausrüstungsvorteil / Besonderer Besitz:} 2 / 4 CP nur für Konfliktgegenstände oder Rüstungsgegenstände; Kosten: 1 / 2 Vorteilspunkte.

\item{Ausr"ustungsvorteil (2, 4 oder 6 VP):}
Der Held erwirbt f"ur jeden VP, den er in Ausr"ustungsvorteil investiert, einen zus"atzlichen CP f"ur den Erwerb von Konfliktgegenst"anden (einschließlich R"ustungen). Die so erhaltenen CP gelten bei der Charaktererschaffung zus"atzlich zu den 50 CP, die jedem Helden ohnehin schon zum Aktivieren und Steigern der Talente sowie zum Kauf von Konfliktgegenst"anden zustehen.
[Anm.: Da ein Konfliktgegenstand im Wert von 1 Bonusw"urfel 2 CP kostet (vgl. ), habe ich die o. g. Kosten f"ur diesen Vorteil gew"ahlt, um sicherzustellen, dass ein SC zu Spielbeginn h"ochsten 3 Kon- fliktgegenst"ande oder eine Waffe im Wert von 2 Bonusw"urfeln ohne R"uckgriff auf die ihm zur Heldenerschaffung zur Verf"ugung stehenden 50 CP erwerben kann.]

\item{Balance / Herausragende Balance:} 1 oder 2 Bonuswürfel in Situationen, in denen es um Balance halten, Stürzen oder akrobatische Leistungen geht. Kosten: 2 oder 6 Vorteilspunkte.

\item{Balance (2 VP):}
Der Vorteil Balance verschafft dem Helden 2 Bonuspunkte auf K"orperbeherrschung. Der erste Punkt steht dabei f"ur die Aktivierung des Talents.

\item{Begabung für (magisches Talent):} Für 1 CP erhält man 2 magische Effekte in diesem magischen Talent. Die Anzahl an maximal möglichen magischen Effekten beträgt 1,5 x Talentgesamtwert; Kosten: 12 Vorteilspunkte.

\item{Begabung für (Ritual):} 1 Bonuswürfel beim Einsatz des Rituals. Kosten: 2 Vorteilspunkte.

\item{Begabung für (Talent):} 1 Bonuspunkt auf gewähltes Talent. Kosten: 1 Vorteilspunkt.

\item{Begabung für (Talentgruppe):} Für 1 CP erhält man 2 Talentpunkte bei der Steigerung. Gilt nicht für Steigerungen bei der Charaktererschaffung; Kosten: Nahkampf: 12 Vorteilspunkte; Fernkampf: 12 Vorteilspunkte; Körperliche Talente: 12 Vorteilspunkte; Gesellschaftstalente: 8 Vorteilspunkte; Naturtalente: 6 Vorteilspunkte; Handwerkstalente – 10 Vorteilspunkte.

\item{Begabung für (Zauber):} 1 Bonuswürfel beim Einsatz des Zaubers. Kosten: 2 Vorteilspunkte.

\item{Besonderer Besitz (6 VP):}
Der Besondere Besitz verschafft dem Helden schon zu Beginn seiner Laufbahn einen (passenden) Konfliktgegenstand, der 2 Zusatzw"urfel wert ist.

\item{Breitgefächerte Bildung / Gebildet:} Grundwissen in 2–4 Wissenstalenten; Kosten: 2 Wissenstalente: 4 Vorteilspunkte, 3 Wissenstalente 6 Vorteilspunkte, 4 Wissenstalente 8 Vorteilspunkte.

\item{Dämmerungssicht:} 1 Bonuswürfel im Halbdunkeln, wenn es auf Sicht ankommt; Kosten: 2 Vorteilspunkte

\item{D"ammerungssicht (2 VP):} Der Held erh"alt 1 Bonusw"urfel, wenn er im Halbdunkel etwas ersp"ahen will.

\item{Eidetisches Gedächtnis / Gutes Gedächtnis:} 1 / 2 Bonuswürfel, wenn es um Gedächtnisleistungen und Erinnern geht. Kosten: 2 / 6 Vorteilspunkte.

\item{Eigeboren:} 3 Bonuswürfel im Umgang mit anderen Hexen. Kosten: 12 Vorteilspunkte.

\item{Eisenaffine Aura:} Magisch begabte Charaktere können eiserne Rüstungsgegenstände nutzen und erhalten nur 1 Maluswürfel. Kosten: 4 Vorteilspunkte.

\item{Eisern:} Der Malus auf den Talentgesamtwert durch körperlichen Schaden ist um 2 reduziert, beträgt also nur 1 bzw. 4. Kosten: 4 Vorteilspunkte.

\item{Eisern (1 bzw. 3 VP):}
Der Held erh"alt einen Bonus von +1 (1 VP) bzw. +2 (3 VP) auf seine k"orperliche R"ustung.

\item{Eiserner Wille (1 bzw. 3 VP):} Der Held erh"alt einen Bonus von +1 (1 VP) oder +2 (3 VP) auf seine geistige R"ustung.

\item{Empathie:} Je 1 Bonuspunkt auf Betören, Überreden/Überzeugen, Heilkunde: Seele; Kosten: 3 Vorteilspunkte.

\item{Entfernungssinn:} Du kannst Entfernung abschätzen. Und erhältst +1 Talentpunkt Bonus auf alle Fernkampftalente. Kosten: 6 Vorteilspunkte.

\item{Entfernungssinn (2 VP):} Der Held bekommt 1 Bonusw"urfel, wenn er Entfernungen absch"atzen will oder er Handlungen durchf"uhrt, bei denen es auf das Absch"atzen von Entfernungen ankommt, z. B. im Fernkampf oder bei Zaubern auf weiter entfernte Zielen.

\item{Feenfreund:} 2 Bonuswürfel im Umgang mit Feenwesen. Kosten: 6 Vorteilspunkte.

\item{Feste Matrix:} Ein Patzer tritt nur dann ein, wenn bei der Bestätigungsprobe eine weitere 20 fällt. Kosten: 2 Vorteilspunkte.

\item{Flink (2 VP):} Flink gibt dem Helden 2 Bonuspunkte auf Athletik.

\item{Freund fremder Wesen (2 oder 6 VP):} Der Held erh"alt 1 (2 VP) oder 2 (6 VP) Bonusw"urfel bei Proben, die mit dem Umgang mit Tieren (Variante Tierfreund), Feen (Variante Feenfreund) oder Kobolden (Variante Koboldfreund) zu tun haben.
[Anm.: Dieser Vorteil fasst die DSA4-Vorteile „Tierfreund“, „Feenfreund“ und „Koboldfreund“ zusammen.]

\item{Gebildet (2 , 4 oder 6 VP):} F"ur jeden VP, der in Gebildet investiert wird, erh"alt der Held einen CP, mit dem er Sprachen, Wissens- und Handwerkstalente aktivieren und/oder im Rahmen des f"ur seine Klasse zul"assigen Maximums steigern darf. Die "uber diesen Vorteil erworbenen CP gelten zus"atzlich zu den 50 CP, die jedem Charakter ohnehin schon zur Talentaktivierung und -steigerung zustehen.

\item{Gefahreninstinkt:} 1 bzw. 2 Bonuswürfel bei bislang nicht erkannten Gefahren; Kosten: 2 bzw. 6 Vorteilspunkte

\item{Gefahreninstinkt (2 oder 6 VP):} Gefahreninstinkt verschafft dem Helden 1 (2 VP) oder 2 (6 VP) Zusatzw"urfel, wenn er bisher unerkannte Gefahren zu erkennen sucht. [Anm.: Vgl. 2.4 = S. 17 der Regelversion 0.12.]

\item{Geräuschhexerei:} Du kannst Geräusche erschaffen, die so klingen, als ob sie in ein paar Schritt Entfernung von dir entstanden wären. Dies bringt dir in passenden Situationen 1 Bonuswürfel. Kosten: 2 Vorteilspunkte.

\item{Glück:} Pro Spiel würfelt der Spielleiter mit W3-1. Du kannst eine entsprechenden Anzahl an beliebigen Würfen wiederholen. Kosten: 6 Vorteilspunkte.

\item{Gl"uck (8 VP):} Der Held darf bis zu zweimal pro aventurischen Tag einen W"urfelwurf einmal wiederholen und das f"ur ihn g"unstigere Ergebnis w"ahlen. Alternativ kann er vom Meister verlangen, dass dieser einen Wurf wiederholt.

\item{Glück im Spiel:} 1 Bonuswürfel, wenn es um Glücksspiel geht. Kosten: 2 Vorteilspunkte.

\item{Gl"uck im Spiel (2 oder 6 VP):} Der Held erh"alt 1 (2 VP) oder 2 (6 VP) Bonusw"urfel, wenn er an einem (Gl"ucks-)Spiel teilnimmt oder sich im Falschspiel "ubt.

\item{Gutaussehend / Herausragendes Aussehen:} 1 oder 2 Bonuswürfel in entsprechenden Konflikten und bei entsprechenden Talente (Betören, Überreden). Kosten: 2 oder 6 Vorteilspunkte.

\item{Guter Ruf (2 VP):}
Der SC ist bekannt f"ur seine Freigiebigkeit, seine Gerechtigkeit, Hilfsbereitschaft oder dergleichen und erh"alt 1 Bonusw"urfel in Situationen, die zu seinem Ruf passen. Der Ruf ist auf eine bestimmte Gegend beschr"ankt, die der Spieler angeben muss.

\item{Herausragender Sechster Sinn:} 1 Bonuswürfel bei allen magischen Aktivitäten, die mit Hellsicht, Magiegespür und dem Erkennen von magischen Mustern und Präsenzen zu tun haben. Kosten: 2 Vorteilspunkte.

\item{Herausragender Sinn:} 1 Bonuswürfel, bei Konflikten in denen der entsprechende Sinn sehr hilfreich ist. Evtl. 1 Maluswürfel, wenn der Sinn überreizt wird. Kosten: 2 Vorteilspunkte.

\item{Hitzeresistenz:} Mali durch ungünstige Umstände sprich durch große Hitze auf den Talentgesamtwert werden um 2 reduziert. Kosten: 1 Vorteilspunkt.

\item{Hohe Lebenskraft:} 1/2/3 Punkte Lebenskraft kosten 1/2/3 Vorteilspunkte.

\item{Hohe Lebens- oder Willenskraft (6 oder 12 VP):} F"ur je 6 investierte VP steigt die Lebens- oder die Willenskraft um 1 Punkt. Die Kosten werden f"ur Lebens- und Willenskraft separat berechnet. Maximal d"urfen f"ur diesen Vorteil pro Krafttyp 12 VP investiert werden, so dass jeder Krafttyp um maximal 2 zus"atzliche Punkte steigen kann. [Anm.: Die Beschr"ankung auf maximal 2 zus"atzliche Punkte pro Krafttyp dient v. a. dem Spiel- gleichgewicht. Legt man die unter Ziffer 8 der Regelversion 0.12 aufgef"uhrten Durchschnittswerte f"ur Anf"angercharaktere von 10 Punkten Willenskraft und 12 Punkten Lebenskraft zugrunde, bedeu- ten die hier maximal zugelassenen 2 Punkte immerhin einen Kraftanstieg von rund 20~\%. Die vergleichsweise hohen Kosten sorgen daf"ur, dass die Spieler nicht zu viele Punkte in diesen Vorteil investieren und sich „"ubercharaktere“ schaffen. Zugleich werden die Spieler jeden zus"atzlichen Kraftpunkt umso mehr wertsch"atzen. Zur „Hohen Astralenergie“ s. o. „Astralmacht“.]

\item{Hohe Magieresistenz / Schwer zu verzaubern:} Wirst du durch magische Effekte angegriffen, erhält der Angreifer 1 bzw. 2 Maluswürfel oder einen Talentmalus von 2 bzw. 5 Punkten. Bei schwer zu verzaubern gilt dies auch für positive Effekte. Kosten: 2 bzw. 6 Vorteilspunkte.

\item{Hohe Magieresistenz (1 oder 3 VP):} Ein SLC-Zauberer, der den SC verzaubern will, muss in Hauptkonflikten 1 (2 VP) oder 2 (6 VP) Malusw"urfel hinnehmen. In Kurz- und Nebenkonflikten wirkt dieser Vorteil "ahnlich wie Eiserner Wille, gibt dem SC also +1 oder +2 auf seine Geistige R"ustung, allerdings nur, soweit in dem Kon- flikt Magie gegen die Helden eingesetzt wurde.

\item{Kampfrausch:} 1 Bonuswürfel; allerdings bei Aktivierung maximal einen defensiven Würfel benutzen; Kosten: 2 Vorteilspunkte

\item{Kampfrausch (2 VP):}
Versetzt sich ein SC in den Kampfrausch, erh"alt er im Kampf einen Zusatzw"urfel. Solange er sich im Kampfrausch befindet, darf er allerdings h"ochstens einen defensiven W"urfel benutzen. [Anm.: Vgl. 2.4 = S. 17 der Regelversion 0.12.]

\item{Kälteresistenz:} Mali durch ungünstige Umstände sprich durch große Kälte auf den Talentgesamtwert werden um 2 reduziert. Kosten: 1 Vorteilspunkt.

\item{Koboldfreund:} 2 Bonuswürfel im Umgang mit Kobolden. Kosten: 6 Vorteilspunkte.

\item{Kriegerbrief:} 1 Bonuswürfel bei Verhandlungen mit Gesetzestreuen; Kosten: 2 Vorteilspunkte

\item{Kriegerbrief (2 VP):}
Der Kriegerbrief verschafft dem Helden 1 Zusatzw"urfel beim Umgang mit gesetzestreuen Personen. SC mit der Profession Krieger m"ussen diesen Vorteil w"ahlen. [Anm.: Vgl. 2.4 = S. 17 der Regelversion 0.12.]

\item{Machtvoller Vertrauter:} 2 oder 4 Bonuspunkte auf das Talent „Vertrautentier“. Kosten: 2 oder 4 Vorteilspunkte.

\item{Magiegespür:} Stärkere magische Orte und Quellen werden intuitiv gespürt. Dies äußert sich z.B. durch Frösteln, Beklemmungsgefühle oder durch sphärische Klänge im Kopf. 1 Bonuswürfel für das Lokalisieren von magischen Quellen. Kosten: 2 Vorteilspunkte.

\item{Nachtsicht:} 1 Bonuswürfel im Dunkeln, wenn es auf Sicht ankommt; Kosten: 4 Vorteilspunkte

\item{Natürliche Waffen:} Wird wie ein Konfliktgegenstand gehandhabt. 1 Bonuswürfel kostet 2 Vorteilspunkte.

\item{Ortskenntnis:} 2 Bonuspunkte auf Orientierung im festgelegten Gebiet; Kosten: 1 Vorteilspunkt.

\item{Orientierungssinn (2 bzw. 6 VP):}
Der Held erh"alt 1 (2 VP) oder – sofern er sich orientieren kann, als h"atte er einen Inneren Kompass – 2 (6 VP) Bonusw"urfel, wenn er sich orientieren muss/m"ochte (z. B. bei Anwendung der Talente Orientierung, Wildnisleben oder Gassenwissen. [Anm.: Dieser Vorteil umfasst die beiden DSA4-Vorteile Richtungssinn und Innerer Kompass.]

\item{Natürlicher Rüstungsschutz:} 1 oder 2 Rüstungsschutz Bonus kosten 1 bzw. 3 Vorteilspunkte.

\item{Prophezeien:} Bei Bedarf kann der SL eine Vision geben; Kosten: 2 Vorteilspunkte

\item{Resistenz gegen Gift / Immunität gegen Gift:} Bonuswürfel durch Konfliktgegenstände Gifte werden um 1 reduziert, wenn sie gegen dich eingesetzt werden. Maluswürfel durch Gifte werden um 1 reduziert. Talentmali durch Gifte werden halbiert. Kosten: 6 Vorteilspunkte.

\item{Resistenz gegen Krankheiten:} Auswirkungen durch Krankheiten (Mail auf Talente) werden halbiert. Bei Konflikten gegen Krankheiten erhältst du 1 Bonuswürfel. Kosten: 4 Vorteilspunkte.

\item{Richtungssinn / Innerer Kompass:} 3 oder 6 Bonuspunkte auf Orientierung; Kosten: 3 oder 6 Vorteilspunkte.

\item{Schlangenmensch:} Je 1 Bonuspunkt auf Waffenloser Kampf, Körperbeherrschung, Schleichen, Sich verstecken, Fesseln/Entfesseln; Kosten: 5 Vorteilspunkte

\item{Schlangenmensch (4 VP):} Ein Schlangenmensch erh"alt je 1 Bonuspunkt auf Raufen, K"orperbeherrschung, Sich Verstecken und Fessel/Entfesseln. [Anm.: vgl. Ziffer 2.4]

\item{Schnelle Heilung:} 50~\% Chance (auf W20 1 bis 10), einen zusätzlichen Punkt Lebenskraft zur normalen Regeneration zu erhalten. Kosten: 1 Vorteilspunkt.

\item{Schnelle Heilung (3 VP):} Pro Regenerationsphase (vgl. 8.2) hat der Held eine 50~\%-Chance (1-10 auf W20), einen zus"atz- lichen Punkt Lebens- oder Willenskraft zur"uckzugewinnen. Dieser Vorteil muss f"ur jeden der beiden Krafttypen separat gew"ahlt werden. [Anm.: Dieser Vorteil wurde – auch in Hinblick auf die Kosten – in Anlehnung an „Astrale Rege- neration“ (s. o.) konzipiert. Wenn man – "ahnlich wie bei DSA4 – verschiedene Stufen des Vorteils w"unscht, k"onnte man das "uber die Zahl der W"urfel regeln. Bei 2 W"urfeln kostete der Vorteil dann 7 Punkte.]

\item{Soziale Anpassungsfähigkeit:} Mali durch ungünstige Umstände durch fremde Kulturen bei gesellschaftlichen Talenten auf den Talentgesamtwert werden um 2 reduziert. Kosten: 2 Vorteilspunkte.

\item{Soziale Anpassungsf"ahigkeit (2 VP):} Der Held erh"alt 1 Zusatzw"urfel, er sich in ungewohnter sozialer Umgebung zurechtfinden muss, sei es in Gesellschaftskreisen, mit denen der SC sonst kaum Kontakt hat, sei es im Umfeld fremder Kulturen.

\item{Sprachgefühl:} Die Kosten bei der Steigerung von Sprachen und Schriften sind halbiert. Kosten: 6 Vorteilspunkte.

\item{Tierempathie / Tierfreund:} 1 oder 2 Bonuswürfel im Umgang mit Tieren. Kosten: 2 oder 6 Vorteilspunkte.

\item{Titularadel / Adlige Abstammung:} 1 oder 2 Bonuswürfel bei Verhandlungen und Situationen in denen eine adlige Abstammung hilfreich ist, wie z.B. auf einem Hofball oder in einer diplomatischen Mission. Kosten: 2 oder 6 Vorteilspunkte.

\item{Unbeschwertes Zaubern:} Die Maluswürfel durch den Vorteil Hohe Magieresistenz / schwer zu verzaubern werden um 1 reduziert bzw. der Talentmalus um 2 Punkte. Kosten: 2 Vorteilspunkte.

\item{Verbindungen:} 1 oder 2 Bonuswürfel in Situationen, in denen die Verbindung hilfreich ist. Kosten: 2 oder 6 Vorteilspunkte.

\item{Verhüllte Aura:} Die Magiebegabung eines Charakters kann nicht festgestellt werden. Kosten: 4 Vorteilspunkte.

\item{Veteran:} Ein Berufstalent, dass der Spieler bei der Erschaffung wählt, beginnt auf 14 statt auf 10. Kosten: 4 Vorteilspunkte.

\item{Wesen der Nacht:} 2 Talentpunkte Bonus auf alle Magietalente, wenn diese in der Nacht angewendet werden. Kosten: 2 Vorteilspunkte.

\item{Willensst"arke (2 VP):} Der SC erh"alt einen Bonus von 2 Punkten auf Selbstbeherrschung.

\item{Wohlklang:} 1 Bonuswürfel in entsprechenden Situationen. Kosten: 2 Vorteilspunkte.

\item{Zäher Hund:} Der Charakter ist erst handlungsunfähig, wenn die Lebenskraft um $1/4$ der Lebenskraft überschritten ist. Bei 8 Lebenskraft also bei 10 oder mehr statt 8 oder mehr.

\item{Z"aher Hund (3 VP):}
Der Held ertr"agt die Folgen von Schaden besser als andere Personen. Abweichend von den in Ab- schnitt 8.1 aufgef"uhrten Mali, die der Held als Folge erlittenen Schadens zu tragen hat, gelten die folgenden Abz"uge:
\begin{itemize}
\item Schaden $\ge 1/2$ Kraft: 2 Pkte. Malus auf alle Werte
\item Schaden $\ge 3/4$ Kraft: 4 Pkte. Malus auf alle Werte
\item Schaden $\ge$ Kraft:    handlungsunf"ahig
\end{itemize}
[Anm.: Bei dem Beispiel-Zwergen zu „Eulen im Wald“ gab dieser Vorteil einen R"ustungsbonus von 2 Punkten gegen k"orperlichen Schaden. Da nach vorliegendem Regel-Vorschlag „Eisern“ f"ur einen erh"ohten R"ustungsschutz sorgt (s. o.), habe ich „Z"aher Hund“ umgewandelt und so eine weitere Regelfacette in das Spiel eingef"uhrt. Im Ergebnis ist der Vorteil „Z"aher Hund“, wie er hier pr"asentiert wird, ein verdeckter Talentvorteil, weil der SC als Schadensfolge geringere Mali auf seine Talente hinnehmen muss. Der im Spiel effektiv eingesetzte Gesamt-TaW ist wegen dieses Vorteils n"amlich h"oher, als er bei Anwendung der allgemeinen Regeln zu den Schadensfolgen (vgl. 8.1) eigentlich w"are. Auf dieser Grundlage wurden auch die Kosten f"ur diesen Vorteil berechnet und auf 3 VP festgesetzt. Ein SC erh"alt durch diesen Vorteil – im Vergleich zu dem unter 8.1 aufgef"uhrten Mali-Schema – insgesamt bis zu 3 Talentpunkte zus"atzlich. Die Kosten pro Talentpunkt sind mit 1 VP angegeben (vgl. 2.4).]

\item{Zauberhaar:} 3 zusätzliche Punkte Astralkraft kosten 1 Vorteilspunkt. Das Haar wirkt besonders. Beim Entfernen des Haares vergehen die 3 zusätzlichen Punkte für eine gewisse Zeit.

\item{Zeitgefühl:} Der Charakter kann bis auf eine Viertel Stunde genau die Tageszeit bestimmen. Kosten: 1 Vorteilspunkt.

\item{Zwergennase:} 1 Bonuswürfel beim Aufspüren von Geheimgängen, verborgenen Türen und Hohlräumen in Gemäuern. Kosten: 2 Vorteilspunkte.

\item{Zwergennase (2 VP):} Zwergennase verschafft dem Helden 1 Zusatzw"urfel, wenn er nach Geheimf"achern, verborgenen G"angen und Hohlr"aumen in Gem"auern u. "a. sucht.

\end{description}
\EN
