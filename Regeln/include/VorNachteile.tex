\BN
\chapter{Liste der Vor- und Nachteile} \label{Ch:VorNachteile} \index{Vorteile} \index{Nachteile}

\lettrine{I}{n} der folgenden Liste sind einige Vor- und Nachteile zusammengefasst, die es auch bei DSA~4 gibt. Die Regelvorschläge orientieren sich an der Beschreibung der Vor- und Nachteile im Abschnitt "`Vor- und Nachteile"' auf Seite \pageref{VorUndNachteile}.

\section{Vorteile}

% Regelgrundlage:
% Talentbonus: 1 CP = 1 VP
% Bonuswürfel: 1 BW = 2 VP, 2 BW = 6 VP [3 BW = 12 VP]
% Rüstung: 1 RS = 1 VP, 2 RS = 3 VP, 3 RS = 6 VP, 4 RS = 10 VP
\begin{description}
% 1 BW = 2 VP
\item{Adlig (2 VP):} Der Held erhält einen Bonuswürfel beim Umgang mit anderen Adligen.

% 1 BW = 2 VP
\item{Affinität zu Geistern/Elementaren/Dämonen (2 VP):} Ein Bonuswürfel im Umgang entweder mit Geistern oder mit Elementaren oder Dämonen. Kosten: 2 Vorteilspunkte

% 1/2 BW = 2/6 VP
\item{Angenehme Auffälligkeit (2/6 VP):} Der Held fällt seinen Mitmenschen angenehm auf, z.\,B. wegen seiner wohlklingenden Stimme, seines hübschen Gesichtes oder seiner besonders athletischen Figur. In passenden Situationen -- in der Regel solchen gesellschaftlicher Art -- stehen dem Helden ein Bonuswürfel (2 VP) oder zwei (6 VP) zur Verfügung.

% Sonder-VP
\item{Astrale Regeneration (1/3/6 VP):} Bei jeder normalen Regeneration wirft der Spieler einen W20. Bei 1 bis 10 bekommt er einen zusätzlichen AsP. Kosten: Ein Würfel: 1 VP, zwei Würfel: 3 VP, drei Würfel: 6 VP.

% Sonder-VP
\item{Astralmacht (Hohe Astralenergie) (3/6/9/12 VP):}
Für je 3 investierte VP erwirbt ein magisch begabter Held einen zusätzlichen AsP. Ein Held kann auf diese Weise höchsten vier zusätzliche AsP gewinnen.

% pro CP ein VP
\item{Ausbildung (5 bis 10 VP):} Ausbildung in drei bis fünf Talenten. In Wissenstalenten wird Grundwissen erworben (2 VP). Spezialtalente werden aktiviert und auf TaW~1 gebracht (2 VP). Basistalente werden auf TaW~2 gebracht (2 VP). Berufstalente werden aktiviert (1 VP).

Folgende Ausbildungen dienen als Beispiel:
\begin{itemize}
\item Akademische Ausbildung (Krieger/Magier/Gelehrter): Die Ausbildung umfasst nur Wissenstalente. Eines davon ist Etikette, die anderen Wissenstalente sind je nach Ausbildung zu wählen. (6/8/10 VP)
\item Schwertgeselle: Die Ausbildung umfasst Etikette, ein Waffen-Spezialtalent und ein oder zwei weitere Talente. (6/8 VP)
\item Handwerksausbildung: Die Ausbildung umfasst ein Berufstalent und zwei weitere Talente. (5 VP)
\end{itemize}

% 1 BW = 2 VP
\item{Ausdauernd (2 VP):}
Der Held erhält ein Bonuswürfel in Situationen, in denen körperliche Ausdauer gefragt ist.

% pro CP ein VP
\item{Ausrüstungsvorteil (bis zu 6 VP):}
Der Held erwirbt für jeden VP, den er in Ausrüstungsvorteil investiert, einen zusätzlichen CP für den Erwerb von Konfliktgegenständen (einschließlich Rüstungen).

% 1/2 BW = 2/6 VP
\item{Balance / Herausragende Balance (2/6 VP):} 1 oder 2 Bonuswürfel in Situationen, in denen es um Balance halten, Stürzen oder akrobatische Leistungen geht.

% pro CP 1 CP / 1 BW=2VP
\item{Begabung (variabel):}
\begin{itemize}
\item Talent/Talentgruppe: Talentbegabungen geben 1 TaW auf Basistalente bzw. aktivieren Spezialtalente. Diese kosten pro Talentpunkt ein VP.
\item Ritual/Zauber/magischer Effekt: 1 Bonuswürfel, wenn der Zauber eingesetzt wird. Kosten: 2 VP.
\end{itemize}

% pro CP ein VP
\item{Besonderer Besitz (6 VP):}
Der Besondere Besitz verschafft dem Helden schon zu Beginn seiner Laufbahn einen (passenden) Konfliktgegenstand, der 2 Zusatzwürfel wert ist.

% pro CP ein VP
\item{Breitgefächerte Bildung / Gebildet (4/6/8 VP):} Grundwissen in zwei bis vier Wissenstalenten; Kosten: 2 VP pro Wissenstalent

% 1BW=2VP
\item{Dämmerungssicht (2 VP):} Ein Bonuswürfel im Halbdunkeln, wenn es auf Sicht ankommt.

% 1/2BW=2/6VP
\item{Eidetisches Gedächtnis / Gutes Gedächtnis (2/6 VP):} Ein bzw. zwei Bonuswürfel, wenn es um Gedächtnisleistungen und Erinnern geht.

% 3BW=12VP
\item{Eigeboren:} Drei Bonuswürfel im Umgang mit anderen Hexen. Kosten: 12 Vorteilspunkte.

% spezial
\item{Eisenaffine Aura (4 VP):} Magisch begabte Charaktere können eiserne Rüstungsgegenstände nutzen und erhalten nur 1 Maluswürfel. Kosten: 4 Vorteilspunkte.

\FEHLT{Wie ist es denn normalerweise?}

% spezial
\item{Eisern (6 VP):} Der Malus auf den Talentgesamtwert durch körperlichen Schaden ist um 3 reduziert, beträgt also nur 3, falls der Schaden größer als 3/4 der Kraft ist.

\FEHLT{Ab hier überarbeiten!}

\item{Eiserner Wille (1 bzw. 3 VP):} Der Held erhält einen Bonus von +1 (1 VP) oder +2 (3 VP) auf seine geistige Rüstung.

\item{Empathie:} Je 1 Bonuspunkt auf Betören, Überreden/Überzeugen, Heilkunde: Seele; Kosten: 3 Vorteilspunkte.

\item{Entfernungssinn:} Du kannst Entfernung abschätzen. Und erhältst +1 Talentpunkt Bonus auf alle Fernkampftalente. Kosten: 6 Vorteilspunkte.

\item{Entfernungssinn (2 VP):} Der Held bekommt 1 Bonuswürfel, wenn er Entfernungen abschätzen will oder er Handlungen durchführt, bei denen es auf das Abschätzen von Entfernungen ankommt, z. B. im Fernkampf oder bei Zaubern auf weiter entfernte Zielen.

\item{Feenfreund:} 2 Bonuswürfel im Umgang mit Feenwesen. Kosten: 6 Vorteilspunkte.

\item{Feste Matrix:} Ein Patzer tritt nur dann ein, wenn bei der Bestätigungsprobe eine weitere 20 fällt. Kosten: 2 Vorteilspunkte.

\item{Flink (2 VP):} Flink gibt dem Helden 2 Bonuspunkte auf Athletik.

\item{Freund fremder Wesen (2 oder 6 VP):} Der Held erhält 1 (2 VP) oder 2 (6 VP) Bonuswürfel bei Proben, die mit dem Umgang mit Tieren (Variante Tierfreund), Feen (Variante Feenfreund) oder Kobolden (Variante Koboldfreund) zu tun haben.
[Anm.: Dieser Vorteil fasst die DSA4-Vorteile „Tierfreund“, „Feenfreund“ und „Koboldfreund“ zusammen.]

\item{Gebildet (2 , 4 oder 6 VP):} Für jeden VP, der in Gebildet investiert wird, erhält der Held einen CP, mit dem er Sprachen, Wissens- und Handwerkstalente aktivieren und/oder im Rahmen des für seine Klasse zulässigen Maximums steigern darf. Die über diesen Vorteil erworbenen CP gelten zusätzlich zu den 50 CP, die jedem Charakter ohnehin schon zur Talentaktivierung und -steigerung zustehen.

\item{Gefahreninstinkt:} 1 bzw. 2 Bonuswürfel bei bislang nicht erkannten Gefahren; Kosten: 2 bzw. 6 Vorteilspunkte

\item{Gefahreninstinkt (2 oder 6 VP):} Gefahreninstinkt verschafft dem Helden 1 (2 VP) oder 2 (6 VP) Zusatzwürfel, wenn er bisher unerkannte Gefahren zu erkennen sucht. [Anm.: Vgl. 2.4 = S. 17 der Regelversion 0.12.]

\item{Geräuschhexerei:} Du kannst Geräusche erschaffen, die so klingen, als ob sie in ein paar Schritt Entfernung von dir entstanden wären. Dies bringt dir in passenden Situationen 1 Bonuswürfel. Kosten: 2 Vorteilspunkte.

\item{Glück:} Pro Spiel würfelt der Spielleiter mit W3-1. Du kannst eine entsprechenden Anzahl an beliebigen Würfen wiederholen. Kosten: 6 Vorteilspunkte.

\item{Glück (8 VP):} Der Held darf bis zu zweimal pro aventurischen Tag einen Würfelwurf einmal wiederholen und das für ihn günstigere Ergebnis wählen. Alternativ kann er vom Meister verlangen, dass dieser einen Wurf wiederholt.

\item{Glück im Spiel:} 1 Bonuswürfel, wenn es um Glücksspiel geht. Kosten: 2 Vorteilspunkte.

\item{Glück im Spiel (2 oder 6 VP):} Der Held erhält 1 (2 VP) oder 2 (6 VP) Bonuswürfel, wenn er an einem (Glücks-)Spiel teilnimmt oder sich im Falschspiel übt.

\item{Gutaussehend / Herausragendes Aussehen:} 1 oder 2 Bonuswürfel in entsprechenden Konflikten und bei entsprechenden Talente (Betören, Überreden). Kosten: 2 oder 6 Vorteilspunkte.

\item{Guter Ruf (2 VP):}
Der SC ist bekannt für seine Freigiebigkeit, seine Gerechtigkeit, Hilfsbereitschaft oder dergleichen und erhält 1 Bonuswürfel in Situationen, die zu seinem Ruf passen. Der Ruf ist auf eine bestimmte Gegend beschränkt, die der Spieler angeben muss.

\item{Herausragender Sechster Sinn:} 1 Bonuswürfel bei allen magischen Aktivitäten, die mit Hellsicht, Magiegespür und dem Erkennen von magischen Mustern und Präsenzen zu tun haben. Kosten: 2 Vorteilspunkte.

\item{Herausragender Sinn:} 1 Bonuswürfel, bei Konflikten in denen der entsprechende Sinn sehr hilfreich ist. Evtl. 1 Maluswürfel, wenn der Sinn überreizt wird. Kosten: 2 Vorteilspunkte.

\item{Hitzeresistenz:} Mali durch ungünstige Umstände sprich durch große Hitze auf den Talentgesamtwert werden um 2 reduziert. Kosten: 1 Vorteilspunkt.

\item{Hohe Lebenskraft:} 1/2/3 Punkte Lebenskraft kosten 1/2/3 Vorteilspunkte.

\item{Hohe Lebens- oder Willenskraft (6 oder 12 VP):} Für je 6 investierte VP steigt die Lebens- oder die Willenskraft um 1 Punkt. Die Kosten werden für Lebens- und Willenskraft separat berechnet. Maximal dürfen für diesen Vorteil pro Krafttyp 12 VP investiert werden, so dass jeder Krafttyp um maximal 2 zusätzliche Punkte steigen kann. [Anm.: Die Beschränkung auf maximal 2 zusätzliche Punkte pro Krafttyp dient v. a. dem Spiel- gleichgewicht. Legt man die unter Ziffer 8 der Regelversion 0.12 aufgeführten Durchschnittswerte für Anfängercharaktere von 10 Punkten Willenskraft und 12 Punkten Lebenskraft zugrunde, bedeu- ten die hier maximal zugelassenen 2 Punkte immerhin einen Kraftanstieg von rund 20~\%. Die vergleichsweise hohen Kosten sorgen dafür, dass die Spieler nicht zu viele Punkte in diesen Vorteil investieren und sich „übercharaktere“ schaffen. Zugleich werden die Spieler jeden zusätzlichen Kraftpunkt umso mehr wertschätzen. Zur „Hohen Astralenergie“ s. o. „Astralmacht“.]

\item{Hohe Magieresistenz / Schwer zu verzaubern:} Wirst du durch magische Effekte angegriffen, erhält der Angreifer 1 bzw. 2 Maluswürfel oder einen Talentmalus von 2 bzw. 5 Punkten. Bei schwer zu verzaubern gilt dies auch für positive Effekte. Kosten: 2 bzw. 6 Vorteilspunkte.

\item{Hohe Magieresistenz (1 oder 3 VP):} Ein SLC-Zauberer, der den SC verzaubern will, muss in Hauptkonflikten 1 (2 VP) oder 2 (6 VP) Maluswürfel hinnehmen. In Kurz- und Nebenkonflikten wirkt dieser Vorteil ähnlich wie Eiserner Wille, gibt dem SC also +1 oder +2 auf seine Geistige Rüstung, allerdings nur, soweit in dem Kon- flikt Magie gegen die Helden eingesetzt wurde.

\item{Kampfrausch:} 1 Bonuswürfel; allerdings bei Aktivierung maximal einen defensiven Würfel benutzen; Kosten: 2 Vorteilspunkte

\item{Kampfrausch (2 VP):}
Versetzt sich ein SC in den Kampfrausch, erhält er im Kampf einen Zusatzwürfel. Solange er sich im Kampfrausch befindet, darf er allerdings höchstens einen defensiven Würfel benutzen. [Anm.: Vgl. 2.4 = S. 17 der Regelversion 0.12.]

\item{Kälteresistenz:} Mali durch ungünstige Umstände sprich durch große Kälte auf den Talentgesamtwert werden um 2 reduziert. Kosten: 1 Vorteilspunkt.

\item{Koboldfreund:} 2 Bonuswürfel im Umgang mit Kobolden. Kosten: 6 Vorteilspunkte.

\item{Kriegerbrief:} 1 Bonuswürfel bei Verhandlungen mit Gesetzestreuen; Kosten: 2 Vorteilspunkte

\item{Kriegerbrief (2 VP):}
Der Kriegerbrief verschafft dem Helden 1 Zusatzwürfel beim Umgang mit gesetzestreuen Personen. SC mit der Profession Krieger müssen diesen Vorteil wählen. [Anm.: Vgl. 2.4 = S. 17 der Regelversion 0.12.]

\item{Machtvoller Vertrauter:} 2 oder 4 Bonuspunkte auf das Talent „Vertrautentier“. Kosten: 2 oder 4 Vorteilspunkte.

\item{Magiegespür:} Stärkere magische Orte und Quellen werden intuitiv gespürt. Dies äußert sich z.B. durch Frösteln, Beklemmungsgefühle oder durch sphärische Klänge im Kopf. 1 Bonuswürfel für das Lokalisieren von magischen Quellen. Kosten: 2 Vorteilspunkte.

\item{Nachtsicht:} 1 Bonuswürfel im Dunkeln, wenn es auf Sicht ankommt; Kosten: 4 Vorteilspunkte

\item{Natürliche Waffen:} Wird wie ein Konfliktgegenstand gehandhabt. 1 Bonuswürfel kostet 2 Vorteilspunkte.

\item{Ortskenntnis:} 2 Bonuspunkte auf Orientierung im festgelegten Gebiet; Kosten: 1 Vorteilspunkt.

\item{Orientierungssinn (2 bzw. 6 VP):}
Der Held erhält 1 (2 VP) oder – sofern er sich orientieren kann, als hätte er einen Inneren Kompass – 2 (6 VP) Bonuswürfel, wenn er sich orientieren muss/möchte (z. B. bei Anwendung der Talente Orientierung, Wildnisleben oder Gassenwissen. [Anm.: Dieser Vorteil umfasst die beiden DSA4-Vorteile Richtungssinn und Innerer Kompass.]

\item{Natürlicher Rüstungsschutz:} 1 oder 2 Rüstungsschutz Bonus kosten 1 bzw. 3 Vorteilspunkte.

\item{Prophezeien:} Bei Bedarf kann der SL eine Vision geben; Kosten: 2 Vorteilspunkte

\item{Resistenz gegen Gift / Immunität gegen Gift:} Bonuswürfel durch Konfliktgegenstände Gifte werden um 1 reduziert, wenn sie gegen dich eingesetzt werden. Maluswürfel durch Gifte werden um 1 reduziert. Talentmali durch Gifte werden halbiert. Kosten: 6 Vorteilspunkte.

\item{Resistenz gegen Krankheiten:} Auswirkungen durch Krankheiten (Mail auf Talente) werden halbiert. Bei Konflikten gegen Krankheiten erhältst du 1 Bonuswürfel. Kosten: 4 Vorteilspunkte.

\item{Richtungssinn / Innerer Kompass:} 3 oder 6 Bonuspunkte auf Orientierung; Kosten: 3 oder 6 Vorteilspunkte.

\item{Schlangenmensch:} Je 1 Bonuspunkt auf Waffenloser Kampf, Körperbeherrschung, Schleichen, Sich verstecken, Fesseln/Entfesseln; Kosten: 5 Vorteilspunkte

\item{Schlangenmensch (4 VP):} Ein Schlangenmensch erhält je 1 Bonuspunkt auf Raufen, Körperbeherrschung, Sich Verstecken und Fessel/Entfesseln. [Anm.: vgl. Ziffer 2.4]

\item{Schnelle Heilung:} 50~\% Chance (auf W20 1 bis 10), einen zusätzlichen Punkt Lebenskraft zur normalen Regeneration zu erhalten. Kosten: 1 Vorteilspunkt.

\item{Schnelle Heilung (3 VP):} Pro Regenerationsphase (vgl. 8.2) hat der Held eine 50~\%-Chance (1-10 auf W20), einen zusätz- lichen Punkt Lebens- oder Willenskraft zurückzugewinnen. Dieser Vorteil muss für jeden der beiden Krafttypen separat gewählt werden. [Anm.: Dieser Vorteil wurde – auch in Hinblick auf die Kosten – in Anlehnung an „Astrale Rege- neration“ (s. o.) konzipiert. Wenn man – ähnlich wie bei DSA4 – verschiedene Stufen des Vorteils wünscht, könnte man das über die Zahl der Würfel regeln. Bei 2 Würfeln kostete der Vorteil dann 7 Punkte.]

\item{Soziale Anpassungsfähigkeit:} Mali durch ungünstige Umstände durch fremde Kulturen bei gesellschaftlichen Talenten auf den Talentgesamtwert werden um 2 reduziert. Kosten: 2 Vorteilspunkte.

\item{Soziale Anpassungsfähigkeit (2 VP):} Der Held erhält 1 Zusatzwürfel, er sich in ungewohnter sozialer Umgebung zurechtfinden muss, sei es in Gesellschaftskreisen, mit denen der SC sonst kaum Kontakt hat, sei es im Umfeld fremder Kulturen.

\item{Sprachgefühl:} Die Kosten bei der Steigerung von Sprachen und Schriften sind halbiert. Kosten: 6 Vorteilspunkte.

\item{Tierempathie / Tierfreund:} 1 oder 2 Bonuswürfel im Umgang mit Tieren. Kosten: 2 oder 6 Vorteilspunkte.

\item{Titularadel / Adlige Abstammung:} 1 oder 2 Bonuswürfel bei Verhandlungen und Situationen in denen eine adlige Abstammung hilfreich ist, wie z.B. auf einem Hofball oder in einer diplomatischen Mission. Kosten: 2 oder 6 Vorteilspunkte.

\item{Unbeschwertes Zaubern:} Die Maluswürfel durch den Vorteil Hohe Magieresistenz / schwer zu verzaubern werden um 1 reduziert bzw. der Talentmalus um 2 Punkte. Kosten: 2 Vorteilspunkte.

\item{Verbindungen:} 1 oder 2 Bonuswürfel in Situationen, in denen die Verbindung hilfreich ist. Kosten: 2 oder 6 Vorteilspunkte.

\item{Verhüllte Aura:} Die Magiebegabung eines Charakters kann nicht festgestellt werden. Kosten: 4 Vorteilspunkte.

\item{Veteran:} Ein Berufstalent, dass der Spieler bei der Erschaffung wählt, beginnt auf 14 statt auf 10. Kosten: 4 Vorteilspunkte.

\item{Wesen der Nacht:} 2 Talentpunkte Bonus auf alle Magietalente, wenn diese in der Nacht angewendet werden. Kosten: 2 Vorteilspunkte.

\item{Willensstärke (2 VP):} Der SC erhält einen Bonus von 2 Punkten auf Selbstbeherrschung.

\item{Wohlklang:} 1 Bonuswürfel in entsprechenden Situationen. Kosten: 2 Vorteilspunkte.

\item{Zäher Hund:} Der Charakter ist erst handlungsunfähig, wenn die Lebenskraft um $1/4$ der Lebenskraft überschritten ist. Bei 8 Lebenskraft also bei 10 oder mehr statt 8 oder mehr.

\item{Zäher Hund (3 VP):}
Der Held erträgt die Folgen von Schaden besser als andere Personen. Abweichend von den in Ab- schnitt 8.1 aufgeführten Mali, die der Held als Folge erlittenen Schadens zu tragen hat, gelten die folgenden Abzüge:
\begin{itemize}
\item Schaden $\ge 1/2$ Kraft: 2 Pkte. Malus auf alle Werte
\item Schaden $\ge 3/4$ Kraft: 4 Pkte. Malus auf alle Werte
\item Schaden $\ge$ Kraft:    handlungsunfähig
\end{itemize}
[Anm.: Bei dem Beispiel-Zwergen zu „Eulen im Wald“ gab dieser Vorteil einen Rüstungsbonus von 2 Punkten gegen körperlichen Schaden. Da nach vorliegendem Regel-Vorschlag „Eisern“ für einen erhöhten Rüstungsschutz sorgt (s. o.), habe ich „Zäher Hund“ umgewandelt und so eine weitere Regelfacette in das Spiel eingeführt. Im Ergebnis ist der Vorteil „Zäher Hund“, wie er hier präsentiert wird, ein verdeckter Talentvorteil, weil der SC als Schadensfolge geringere Mali auf seine Talente hinnehmen muss. Der im Spiel effektiv eingesetzte Gesamt-TaW ist wegen dieses Vorteils nämlich höher, als er bei Anwendung der allgemeinen Regeln zu den Schadensfolgen (vgl. 8.1) eigentlich wäre. Auf dieser Grundlage wurden auch die Kosten für diesen Vorteil berechnet und auf 3 VP festgesetzt. Ein SC erhält durch diesen Vorteil – im Vergleich zu dem unter 8.1 aufgeführten Mali-Schema – insgesamt bis zu 3 Talentpunkte zusätzlich. Die Kosten pro Talentpunkt sind mit 1 VP angegeben (vgl. 2.4).]

\item{Zauberhaar:} 3 zusätzliche Punkte Astralkraft kosten 1 Vorteilspunkt. Das Haar wirkt besonders. Beim Entfernen des Haares vergehen die 3 zusätzlichen Punkte für eine gewisse Zeit.

\item{Zeitgefühl:} Der Charakter kann bis auf eine Viertel Stunde genau die Tageszeit bestimmen. Kosten: 1 Vorteilspunkt.

\item{Zwergennase:} 1 Bonuswürfel beim Aufspüren von Geheimgängen, verborgenen Türen und Hohlräumen in Gemäuern. Kosten: 2 Vorteilspunkte.

\item{Zwergennase (2 VP):} Zwergennase verschafft dem Helden 1 Zusatzwürfel, wenn er nach Geheimfächern, verborgenen Gängen und Hohlräumen in Gemäuern u. ä. sucht.

\end{description}

\section{Nachteile}

\EN
