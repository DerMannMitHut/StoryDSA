\BN
\chapter{Grundregeln}\label{Ch:Grundregeln}\index{Grundregeln}
\lettrine{W}{ie} in jedem anderen Rollenspiel gibt es auch bei \StoryDSA einige Regeln, auf denen das gesamte System beruht. In diesem Kapitel möchte ich diese Regelnideen vorab erklären, damit die späteren Kapitel -- wie die Charaktererschaffung -- leichter verständlich sind.

\section{Charakterwerte}
Jeder Spielercharakter besteht aus einer Reihe von Werten. Das können sowohl Zahlenwerte sein, aber auch Besitztümer oder andere beschreibenden Werte gehören hierzu. Diese Werte repräsentieren sogenannte Knusperstückchen, also Fähigkeiten und Möglichkeiten des Charakters. Im Spiel erlauben sie dem Charakterspieler, Fakten zur gemeinsamen Erzählung beizutragen. So kann ein Spieler, dessen Held ein Schwert besitzt, erzählen, wie der Held dieses Schwert benutzt. Ohne das Schwert dürfte er davon natürlich nicht sprechen.

Was gibt es alles für Werte? Es gibt tatsächlich eine ganze Reihe von Werten, angelehnt an die Werte, die es auch bei DSA gibt:

\begin{itemize}
\item Eigenschaften: Mut, Klugheit, Intuition, Charisma, Gewandtheit, Fingerfertigkeit, Konstitution, Körperkraft. Eigenschaften hat jeder Charakter und stellen die grundlegenden geistigen und körperlichen Fähigkeiten dar; sie werden mit Zahlenwerten beschrieben. Die Werte liegen im Bereich $-1$ bis $4$. Dabei bedeutet $-1, 0$: schlecht, $1$: normal, $2$: gut, $3$: sehr gut und $4$: herausragend. 

\item Talente: Dies sind die gelernten Fähigkeiten, die Training und spezielles Wissen voraussetzen. Ein Teil der Talente (Basistalente) hat jeder Held, alle anderen werden zusätzlich erworben. Jedes Talent ist abhängig von drei Eigenschaftswerten und einem Talentwert. Aus der Hälfte der Summe der Eigenschaftswerte und dem Talentwert wird der sogenannte Talentgesamtwert berechnet. Ist der Held in einem Talent untrainiert, so beträgt der Talentgesamtwert trotzdem mindestens 5. Ein guter Wert für einen Anfängercharakter ist 10, später kann der Wert bis auf 18 steigen.

\item Sonderfertigkeiten: Hiermit kann ein Spieler seinen Charakter auf bestimmte Talente spezialisieren. Eine Sonderfertigkeit gibt in mindestens drei Talenten in bestimmten Situationen einen Vorteil. Sonderfertigkeiten können erst gewählt werden, wenn der Talentgesamtwert der beiteiligten Talente mindestens 15 beträgt.

\item Vor- und Nachteile: Aus besonderen körperliche, geistige oder soziale Voraussetzungen heraus ergeben sich für Helden besondere Vor- oder Nachteile. Jeder Vor- und Nachteil ist individuell geregelt. Häufig sind es einfach nur besondere Boni oder Mali auf Talente, manchmal gelten aber auch besondere Regeln.

\item Energien: Als Energie wird jeder Wert bezeichnet, der einen bestimmten maximalen Füllstand aufweist, von dem der Held Punkte verlieren kann. Diese Punkte regenerieren sich auf bestimmte Art und Weise wieder. Sollten die Energien unter bestimmte Werte fallen (unter die Hälfte, unter ein Viertel und unter 1), so treten besondere Regeln in Kraft. Als Energien kommen bei \StoryDSA Konfliktpunkte, Willenskraft, Lebenskraft, Astralenergie und Karmaenergie zum tragen.

\item Gegenstände: Normale Gegenstände, die keine regeltechnische Auswirkung haben (z.B. Kleidung oder Geld) kann sich der Spieler beliebig notieren und damit seinen Charakter ausschmücken. Ein besonderer Gegenstand, der regeltechnische Vorteile im Spiel gibt, muss wie eine Fähigkeit mit Hilfe von Steigerungen erworben werden. Gegenstände sind von den Regeln her mit Vorteilen vergleichbar: häufig geben sie Boni auf Talente, manchmal gelten aber auch besondere Regeln.

\item Abenteuerpunkte, Steigerungen und Stufe: Die Abenteuerpunkte geben an, wie viel Erfahrung der Held in seinen Abenteuern gesammelt hat. Für jeweils 100~Abenteuerpunkte bekommt der Spieler eine Steigerung. Um Fähigkeiten zu erwerben oder zu verbessern, muss der Spieler diese Steigerungen ausgeben.
Außerdem kann aus diesen Abenteuerpunkten die Stufe abgelesen werden; die Umrechnung erfolgt nach der traditionellen quadratischen Formel (Stufen sind bei 100, 300, 600, 1000 usw.). Je höher die Stufe ist, umso bessere Fähigkeiten sind für den Charakter freigeschaltet.
\end{itemize}         

\section{Spielablauf}

Im Gegensatz zu vielen anderen Rollenspielen gilt bei StoryDSA das (eventuell aus Wushu bekannte) Prinzip der erzählten Wahrheit: Alles das, was ein Spieler erzählt gilt so, wie er es erzählt. Das bedeutet, es gibt keine Frage an den Spielleiter "`Kann mein Held dieses oder jenes machen?"' Wenn der Spieler das erzählt, ist es so. Dieses Prinzip gilt immer, auch in Konflikten. Um konkret zu werden: Beschreibt ein Spieler, wie sein Held auf einen Baum klettert und Ausschau hält, so ist der Charakter oben. Dafür muss er \emph{niemals} eine Probe o.ä. würfeln! Allerdings gibt es Situationen, in denen die Regeln dem Spieler verbieten, dies einfach so zu erzählen.

Damit niemand völligen Unsinn erzählt und die Regeln zum Erzählen eingehalten werden, haben die anderen Spieler die Möglichkeit, ein Veto einzulegen. Es gibt zwei verschiedene Arten von Veto: Das allgemeine Veto kann jeder aussprechen, wenn er denkt, dass eine Erzählung gegen Genrekonventionen, die Welt oder Regeln verstößt. Darüber wird dann in der Gruppe diskutiert und die Erzählung entsprechend abgeändert. Das persönliche Veto kann jeder aussprechen, um zu verhindern, dass ein anderer Spieler etwas über den eigenen Helden erzählt, was unerwünscht ist. Das kann jeder Spieler für seinen eigenen Charakter einlegen, der Spielleiter kann ein persönliches Veto für alle anderen Spielfiguren einlegen.

Im Spiel gibt es drei verschiedene Spielphasen, in denen unterschiedliche Regeln gelten.
\begin{enumerate}
\item Freies Spiel. Jeder Charakterspieler stellt seinen Helden dar; der Spielleiter legt die Gegebenheiten fest und übernimmt die Meisterpersonen. Im freien Spiel werden keine Würfel gebraucht, es gibt keine mechanischen Regeln.
\item Konflikte. Konflikte sind durch Würfeln bestimmt. Sie laufen in Runden ab und dienen dazu, dass die Heldengruppe bestimmte Schwierigkeiten überwindet. Spannung und Action stehen im Vordergrund.
\item SL-Erzählphase. Der Spielleiter erzählt alleine, wie die Geschichte weitergeht. Zumeist sind dies Übergänge zwischen freiem Spiel und Konflikten. In diesen SL-Erzählphasen haben die Spieler kein persönliches Vetorecht.
\end{enumerate}

Damit jeder Spieler weiß, in welcher Spielphase sich die Gruppe befindet, gibt es einige Schlüsselsätze und Handzeichen, damit der Spielfluss durch den Phasenwechsel nicht unterbrochen wird.

\section{Würfelei}

Gewürfelt wird natürlich auch, und zwar in Konflikten. Jeder Spieler benötigt am Anfang 7W20, später noch ein paar mehr. Andere Würfel werden bei StoryDSA nicht geworfen. Beim Würfeln kommt es immer darauf an, höchstens einen bestimmten Talentgesamtwert zu erreichen. Also: Wer eine 8 in Athletik hat sollte, wenn er klettert, auch höchstens eine 8 würfeln.

Was ist jetzt ein Konflikt? Zumeist handelt es sich tatsächlich um eine Auseinandersetzung zwischen den Helden und Meisterfiguren, oft sind es aber auch Probleme, die die Umwelt den Charakteren zu schaffen macht. Immer haben die Spielfiguren ein Ziel, dass sie erreichen möchten und irgendetwas steht dagegen. Wie weit die Gruppe von dem Erreichen des Zieles entfernt ist, wird vom Spielleiter durch eine Anzahl Konfliktpunkte festgelegt (z.B. 20 Konfliktpunkte).

Zusammen mit dem Prinzip der erzählten Wahrheit ergeben sich hieraus auch die Regeln: Was immer ein Spieler erzählt, passiert auch genau so, wie er es erzählt. Beim Erzählen sollte sich der Spieler aber an den Fähigkeiten seines Charakters orientieren. Dann wird auf dieses Talent auch gewürfelt. Gelingt ein Wurf, so hat die Tat des Charakters etwas zum Erreichen des Ziels beigetragen, die Konfliktpunkte sinken um 1.

Bei Verstößen gegen die erlaubten Erzählinhalte sollte die Erzählung durch die anderen Spieler gestoppt und korrigiert werden, indem sie einfach ein allgemeines Veto einlegen. Wenn also jemand eine Niete in Athletik ist (Talentgesamtwert von 2), so sollte der Spieler nicht erzählen, dass sich sein Held wie ein Affe von Baum zu Baum schwingt. Er könnte genausogut erzählen, wie sein Held vom Baum herunterfällt und dürfte dann genauso Würfeln und den Konflikt näher zum Ende treiben -- das Ergebnis des Würfelwurfes ist völlig unabhängig von der Erzählung. Nur die Frage, auf welches Talent gewürfelt wird, wird durch die Fiktion bestimmt.

Um die Spieler zum Erzählen anzuregen gibt es je mehr Würfel, umso mehr der Spieler erzählt. Mehr als 5 Würfel gibt es aber nicht. Bonuswürfel kann es durch Magie, besondere Gegenstände oder Vorteile geben, so dass am Anfang maximal 6 oder 7 Würfel möglich sind. In höheren Stufen können es natürlich mehr werden.

Nun ist es aber so, dass auch jeder Spielercharakter nur eine bestimmte Menge an Konfliktpunkten hat, zu Beginn sind es 3. Pro Runde sinken diese Konfliktpunkte -- bei Anfängern um 1, später auch schneller. Sind die verbraucht, so scheidet der Charakter und mit ihm sein Spieler aus dem Konflikt aus. Das kann der Spieler hinauszögern, indem er seine Würfel aufteilt: Einerseits in den Fortschritt (offensiv), andererseits um zu verhindern, dass sein Charakter nicht mehr am Konflikt teilnehmen kann (defensiv).

\section{Weiteres}
Verletzungen können durch jeden Konflikt entstehen -- geistiger oder körperlicher Schaden ist allgegenwärtig. Sterben kann ein Charakter dadurch allerdings nicht. Das ist nur dann möglich, wenn das explizit in einem Konflikt auf dem Spiel steht. Allerdings begrenzt die Lebenskraft und Willenskraft die Anzahl an Konflikten, die ohne Regeneration hintereinander bewältigt werden kann. Und zu große Verletzungen behindern einen Charakter auch.

Astral- und Karmaenergie eröffnen zusätzliche Möglichkeiten, im Konflikt die Würfelergebnisse zu manipulieren: Es können Würfel weitergegeben werden, automatische Erfolge erzielt werden und einiges mehr. Dabei ist Karmaenergie etwa doppelt so mächtig wie Astralenergie, ist aber auch schwieriger zu erwerben und zu regenerieren. Insgesamt wurde darauf geachtet, dass Magie nicht übermächtig wird und sich gut in die Erzählung einfügt.

Genug des Überblicks, jetzt sollte der Charaktererschaffung nichts mehr im Wege stehen.
\EN                      
